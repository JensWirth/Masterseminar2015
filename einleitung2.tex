% !TEX root = main.tex
\chapter{Einleitung}
Hier sollen die Grundlagen dafür gelegt werden, auch Operatoren mit variablen Koeffizienten richtig behandeln zu können. Die Darstellung basiert auf \cite{Hormander:1965}, \cite{Hormander:1966}, sowie \cite[Kapitel 18]{Hormander:1985}. Zuerst verallgemeinern wir den Begriff des Differentialoperators soweit, daß wir auch Inverse und allgemeinere Funktionen von solchen Operatoren behandeln können. 

\section{Operatoren und Symbole}
Sei $\Omega$ eine Mannigfaltigkeit. Differentialoperatoren auf $\Omega$ können dann in lokalen Karten definiert werden oder global durch ihre Eigenschaften charakterisiert werden. Die bekannteste davon ist, daß eine stetige lineare Abbildung $P:\rmC_0^\infty(\Omega)\to\rmC^\infty(\Omega)$ genau dann Differentialoperator ist, wenn $P$ lokal ist, also wenn
\begin{equation}
  \forall f\in\rmC_0^\infty(\Omega)\quad:\quad \supp Pf \subseteq \supp f
\end{equation}
gilt. Für Verallgemeinerungen brauchbarer ist folgende Charakterisierung:

\begin{lem}
Eine  lineare Abbildung   $P:\rmC_0^\infty(\Omega)\to\rmC^\infty(\Omega)$ ist genau dann ein Differentialoperator der Ordnung $m$, wenn für alle $f\in\rmC_0^\infty(\Omega)$ und alle $g\in\rmC^\infty(\Omega)$ die Funktion 
\begin{equation}\label{eq:6:6.2}
\e^{-\i\lambda g} P(f\e^{\i\lambda g}) = \sum_{j=0}^m P_j(f,g) \lambda^j
\end{equation}
ein Polynom vom Grad $m$ in $\lambda$ ist.
\end{lem}
\begin{proof}
Die Hinrichtung ist klar. Zum Beweis der Rückrichtung sei $x'\in\Omega$ ein Punkt und $f\in\rmC_0^\infty(\Omega)$ mit $f=1$  in einer (in einem Kartengebiet liegenden) Umgebung $\omega$ von $x'$. Sei weiter $\xi\in\R^n$ und $g(x) = x\cdot\xi$. Dann folgt aus \eqref{eq:6:6.2}
\begin{equation}
    \e^{-\i \lambda x\cdot\xi} P ( f\e^{\i \lambda x\cdot\xi}) = \sum_{j=0}^m p_j(f;x,\xi) \lambda^j
\end{equation}
mit $\rmC^\infty$-Funktionen auf $\omega\times\R^n$ als Koeffizienten $p_j(f;x,\xi)$. Da auch
$p_j(f;x,\lambda\xi)=p_j(f;x,\xi)\lambda^j$ gilt, ist $p_j(f;x,\xi)$ homogen vom Grad $j$. Die einzigen auf ganz $\R^n$ glatten homogenen Funktionen sind Polynome,
$p_j(f;x,\xi) = \sum_{|\alpha|=j} a_\alpha(f;x) \xi^\alpha$. Sei nun $u\in\rmC_0^\infty(\omega)$. Dann gilt mit der Fourierschen Inversionsformel
\begin{equation}
    u(x) = (2\pi)^{-n/2} \int \e^{\i x\cdot\xi} \widehat u(\xi)\d\xi
\end{equation}
und da nach Konstruktion $u=fu$ gilt, folgt insbesondere 
\begin{align}
    Pu(x) &= P(fu)(x) = (2\pi)^{-n/2} \int P(f \e^{\i x\cdot\xi}) \widehat u(\xi)\d\xi \notag \\&= \sum_{j=0}^m (2\pi)^{-n/2} \int \e^{\i x\cdot\xi} p_j(f;x,\xi) \widehat u(\xi)\d\xi
    = \sum_{|\alpha|\le m} a_\alpha(f;x)\D^\alpha u.
\end{align}
Mit Linearität folgt die Behauptung.
\end{proof}

Sei nun $\Omega$ ein Gebiet.
Wir betrachten im folgenden allgemeiner Operatoren $P : \rmC_0^\infty(\Omega) \to \rmC^\infty(\Omega)$ und deren Beschreibung durch ein \eIndex[Pseudodifferentialoperator]{Symbol}
\begin{equation}\label{eq:6:6.6}
    \sigma_P(f; x,\xi) = \e^{-\i x\cdot\xi} P( f(x) \e^{\i x\cdot\xi}).
\end{equation}  
Dabei sei $f\in\rmC_0^\infty(\Omega)$ kompakt getragen mit $f(x)=1$ auf einer kompakten Teilmenge $\omega\Subset\Omega$. 
Dann gilt wiederum für jedes $u\in\rmC_0^\infty(\omega)$ 
\begin{equation}
   P u (x) = (2\pi)^{-n/2} \int \e^{\i x\cdot\xi} \sigma_P(f; x,\xi) \widehat u(\xi) \d\xi.
\end{equation}
Der Operator $P$ wird als \eIndex{Pseudodifferentialoperator} der \eIndex[Pseudodifferentialoperator]{Ordnung} $m$ bezeichnet, falls jedes so entstehende Symbol 
$\sigma_P(f;\cdot,\cdot)$ zur nachfolgend definierten \eIndex{Symbolklasse} $S^m_{\rm loc}(\Omega\times\R^n)$ gehört.

\begin{df}
Eine Funktion $\sigma\in\rmC^\infty(\Omega\times\R^n)$ gehört zur Symbolklasse $S^m_{\rm loc}(\Omega\times\R^n)$ falls
\begin{equation} 
  \sup_{x\in K}  |  \partial_x^\beta\partial_\xi^\alpha \sigma(x,\xi) | \le C_{K,\alpha,\beta} \langle\xi\rangle^{m-|\alpha|}
\end{equation}
für alle Kompakta $K\Subset\Omega$ und alle Multiindices $\alpha,\beta\in\N^n_0$ gilt. Dabei bezeichnet $\langle\xi\rangle = \sqrt{1+|\xi|^2}$.
\end{df}

Ist nun $\sigma\in S^m_{\rm loc}(\Omega\times\R^n)$, so konvergiert für jedes $u\in\rmC_0^\infty(\Omega)$ das Integral
\begin{equation}\label{eq:6:6.9}
   P u (x) = (2\pi)^{-n/2} \int \e^{\i x\cdot\xi} \sigma(x,\xi) \widehat u(\xi) \d\xi
\end{equation}
und definiert eine glatte Funktion $Pu\in\rmC^\infty(\Omega)$. Es gilt sogar mehr

\begin{lem}\label{lem:5.3}
Sei $\sigma\in S^m_{\rm loc}(\Omega\times\R^n)$ und $Pu$ für $u\in\rmC_0^\infty(\Omega)$ durch \eqref{eq:6:6.9} definiert. Dann gilt für jedes $f\in\rmC_0^\infty(\Omega)$
und das durch \eqref{eq:6:6.6} zugeordnete Symbol
\begin{equation}
    \sigma_P(f;\cdot,\cdot)\in S^m_{\rm loc}(\Omega\times\R^n). 
\end{equation}
Weiterhin gilt
\begin{equation}
   \sigma_P(f;x,\xi) \sim \sum_{\alpha}  \frac{1}{\alpha!} \big( \partial_\xi^\alpha \sigma(x,\xi) \big) \big(\D_x^\alpha f(x)\big)
\end{equation}
und, falls $f(x)=1$ auf $\omega\Subset\Omega$ gilt, auch $\sigma_P(f;x,\xi)=\sigma(x,\xi) \mod\mathscr S(\overline{\omega}\times\R^n)$.\footnote{Der Schwartzraum $\mathscr S(\overline\omega\times\R^n)$ auf dem Vollzylinder $\overline\omega\times\R^n$ ist so definiert, wie man es sich vorstellt. Eine Funktion $g$ gehört zu $\mathscr S(\overline\omega\times\R^n)$, falls für alle Multiindices $\alpha,\beta,\gamma\in\N_0^n$ die Abschätzung
\begin{equation}
  \sup_{x\in\overline\omega}\sup_{\xi\in\R^n} | \xi^\alpha \D_x^\beta \D_\xi^\gamma g(x,\xi) | < \infty
\end{equation}
gilt.}
\end{lem}
\begin{proof}
Nach Definition gilt
\begin{align}
   \sigma_P(f;x,\xi) &=(2\pi)^{-n}  \e^{-\i x\cdot\xi} \int \e^{\i x\cdot\eta} \sigma(x,\eta) \int_\Omega \e^{-\i y\cdot(\eta-\xi)} f(y) \d y \d\eta \notag\\
   & = (2\pi)^{-n/2}  \int \e^{\i x\cdot(\eta-\xi)} \sigma(x,\eta) \widehat f(\eta-\xi)\d\eta \notag\\
   & = (2\pi)^{-n/2}  \int \e^{\i x\cdot\eta} \sigma(x,\eta+\xi) \widehat f(\eta)\d\eta \notag\\
   & = (2\pi)^{-n/2}  \int \e^{\i x\cdot\eta} \left( \sum_{|\alpha|<N} \frac{\eta^\alpha}{\alpha!} \partial_\xi^\alpha \sigma(x,\xi) + R_N(x,\xi,\eta) \right) \widehat f(\eta)\d\eta \notag\\  
   & =  \sum_{|\alpha|<N} \frac{1}{\alpha!} \big( \partial_\xi^\alpha \sigma(x,\xi) \big) \big(\D_x^\alpha f(x)\big)  +
   \rho_N(x,\xi)
\end{align}
unter Ausnutzung des taylorschen Satzes mit Entwicklungspunkt $\eta=0$. Weiter gilt unter Nutzung der Integraldarstellung des Restgliedes $R_N(x,\xi,\eta)$ 
\begin{equation}
   R_N(x,\xi,\eta) = N \int_0^1 \sum_{|\alpha|=N} \frac{(1-t)^N \eta^\alpha}{\alpha!} \partial_\xi^\alpha \sigma(x,\xi+t\eta) \d t 
\end{equation}
und somit für beliebige Kompakta $K\Subset\Omega$ und $N\ge m$
\begin{align}
   |\langle\xi\rangle^{N-m} \rho_N(x,\xi)|& \lesssim 
   \left|\langle\xi\rangle^{N-m}  \int \e^{\i x\cdot\eta} \int_0^1 \sum_{|\alpha|=N} \frac{(1-t)^N \eta^\alpha}{\alpha!} \partial_\xi^\alpha \sigma(x,\xi+t\eta) \d t  \widehat f(\eta) \d\eta   \right| \notag\\
&\lesssim  \int \int_0^1 \langle\eta\rangle^N \langle\xi+t\eta\rangle^{m-N} \langle\xi\rangle^{N-m} \d t | \widehat f(\eta) |\d\eta   
\lesssim \int \langle\eta\rangle^{2N} |\widehat f(\eta)|\d\eta
\end{align}
gleichmäßig in $x\in K$ (und unter Ausnutzung der Ungleichung von Peetre $\langle \xi\rangle^s \lesssim \langle\eta\rangle^s \langle\xi-\eta\rangle^s$ für $s\ge0$).
Weiter gilt für Multiindices $\beta,\gamma\in\N_0^n$
\begin{multline}
   \partial_x^\gamma \partial_\xi^\beta \rho_N(x,\xi) \\
   =  (2\pi)^{-n/2} N \sum_{\delta\le\gamma} \binom{\gamma}{\delta} \i^{|\delta|} \int \e^{\i x\cdot\eta} \int_0^1 \sum_{|\alpha|=N} \frac{(1-t)^N \eta^{\alpha+\delta}}{\alpha!} \partial_x^{\gamma-\delta} \partial_\xi^{\alpha+\beta} \sigma(x,\xi+t\eta) \d t  \widehat f(\eta) \d\eta 
\end{multline}
und obige Abschätzung liefert $\rho_N\in S^{m-N}_{\rm loc}(\Omega\times\R^n)$. Das impliziert die Behauptung.
\end{proof}

Pseudodifferentialoperatoren auf Gebieten kann man nicht notwendigerweise verketten. Wir bezeichnen einen Pseudodifferentialoperator $P$ als \eIndex[Pseudodifferentialoperator]{eigentlich getragen}, falls er $\rmC_0^\infty(\Omega)$ nach $\rmC_0^\infty(\Omega)$ abbildet. Dies gilt zum Beispiel dann, wenn das Symbol in $x$ kompakt getragen ist. Allgemeiner gibt es für eigentlich getragene Operatoren zu jedem Kompaktum $K\Subset\Omega$ ein Kompaktum $L\Subset\Omega$, so daß für $u\in\rmC_0^\infty(K)$ stets $\supp Pu \subset L$ gilt. Sei nun $f\in\rmC_0^\infty(K)$. Dann gilt für einen zweiten Pseudodifferentialoperator $Q$ definiert durch das Symbol $\tau\in S^{m_2}_{\rm loc}(\Omega\times\R^n)$
\begin{equation}
    \sigma_{Q\circ P} (f;x,\xi) = (2\pi)^{-n} \e^{-\i x\cdot\xi} \int \e^{\i x\cdot\eta}  \tau(x,\eta) 
      \int_{\Omega} \e^{-\i y\cdot(\eta-\xi)} \sigma_P(f;y,\xi) \d y \d\eta 
\end{equation}
und damit nach der Argumentation des letzten Beweises
\begin{equation}
    \sigma_{Q\circ P}(f;x,\xi) \sim \sum_\alpha \frac1{\alpha!} \big(\partial_\xi^\alpha \tau(x,\xi)\big) \big( \D_x^\alpha \sigma_P(f;x,\xi)\big).
\end{equation}
Das ist eine erste Form der Kompositionsformel für Pseudodifferentialoperatoren. Ist $f(x)=1$ auf einer kleineren Menge $\omega\Subset K$, so ergibt sich das Symbol der Komposition $P\circ Q$ für $x\in\omega$ und modulo $\mathscr S(\overline\omega\times\R^n)$ als asymptotische Summe
\begin{equation}
    \sum_{\alpha}  \frac1{\alpha!} \big(\partial_\xi^\alpha \tau(x,\xi)\big) \big( \D_x^\alpha \sigma(x,\xi)\big).
\end{equation}
Analog kann man den formal Adjungierten eines eigentlich getragenen Pseudodifferentialoperators bestimmen. So gilt (vorerst formal, also distributionell)
\begin{equation}
   P^* u (y) =\int \e^{\i x\cdot\xi}  \int_\Omega \e^{-\i y\cdot\xi} \overline{\sigma(y,\xi)} u(y)\d y\d\xi
\end{equation}
und damit wiederum
\begin{equation}
   \sigma_{P^*}(f;x,\xi) \sim \sum_\alpha \frac1{\alpha!} \partial_\xi^\alpha\D_x^\alpha \big(\overline{\sigma(x,\xi)} f(x)\big).
\end{equation}
Auf einer Menge $\omega\Subset\Omega$ auf welcher $f(x)=1$ gilt, vereinfacht sich dies modulo $\mathscr S(\overline\omega\times\R^n)$  zu
\begin{equation}
   \sum_\alpha \frac1{\alpha!} \partial_\xi^\alpha\D_x^\alpha \overline{\sigma(x,\xi)}.
\end{equation}
Pseudodifferentialoperatoren mit Symbolen aus $S^0_{\rm loc}(\Omega\times\R^n)$ sind $\rmL^2_{\rm comp}$-$\rmL^2_{\rm loc}$-beschränkt. Das liefert nachfolgendes Lemma. Zusammen mit obigen asymptotischen Entwicklungen erhält man die $\rmH^s_{\rm comp}$-$\rmH^{s-m}_{\rm loc}$-Beschränktheit für Operatoren mit Symbolen aus $S^m_{\rm loc}(\Omega\times\R^n)$ für alle $s$.
\begin{lem}
Sei $\sigma\in S^0_{\rm loc}(\Omega\times\R^n)$ und $u,v\in\rmC_0^\infty(\Omega)$. Dann gilt für den durch \eqref{eq:6:6.9} definierten Operator $P$
und jede Funktion $f\in\rmC_0^\infty(\Omega)$
\begin{equation}
   | \spro{Pu}{f v} | \le C \|u\|\|v\|.
\end{equation}
Die Konstante $C$ hängt dabei nur von endlich vielen Seminormen des Symbols und der Wahl von $f$ ab.
\end{lem}
\begin{proof}
Wir betrachten die Hilfsfunktion
\begin{equation}\label{eq:6:h-def}
  h(\eta,\xi) =  (2\pi)^{-n/2} \int \e^{-\i x\cdot\eta} \overline{f(x)} \sigma(x,\xi) \d x.
\end{equation}
Da $f$ kompakt getragen ist, liefert partielle Integration Abschätzungen
\begin{equation}
   |h(\eta,\xi)|  \le C_N \langle\eta\rangle^{-N}
\end{equation}
für alle $N$, so dass für die quadratische Form $\spro{Pu}{fv}$
\begin{equation}
\begin{split}
   \spro{Pu}{fv} &= (2\pi)^{-n/2} \iint \e^{\i x\cdot \xi}\overline{ v(x) f(x)} \sigma(x,\xi) \widehat u(\xi)\d\xi\d x\\
   &= (2\pi)^{-n/2} \iint \overline{ \widehat v(\eta) }h(\eta-\xi,\xi) \widehat u(\xi)\d\xi\d\eta
\end{split}
\end{equation}
gilt. Mit dem Schurtest folgt die Behauptung.
\end{proof}

Wir bezeichnen mit $S^m_{\rm comp}(\Omega\times\R^n)$ die Menge der Symbole $\sigma\in S^m_{\rm loc}(\Omega\times\R^n)$, für welche ein $K\Subset\Omega$ mit $\sigma(x,\xi)=0$ für $x\in\Omega\setminus K$ (und alle $\xi$) existiert. Für solche Symbole gilt obige Aussage natürlich ohne die Hilfsfunktion $f$; für jedes $\sigma\in S^0_{\rm comp}(\Omega\times\R^n)$ und alle $u,v \in\rmC_0^\infty(\Omega)$ existiert zum durch \eqref{eq:6:6.9} definierten Operator $P$ eine Konstante $C$ mit
\begin{equation}
   | \spro{Pu}{v} | \le C \|u\|\|v\|.
\end{equation}
Der Beweis ist analog.


\section{Hörmanders globale Pseudodifferentialoperatoren}

Im folgenden betrachten wir globale Symbole $\sigma(x,\xi)$, welche die Bedingung
\begin{equation}
    \sup_{x\in\R^n} |\partial_x^\beta\partial_\xi^\alpha \sigma(x,\xi)| \le C_{\alpha,\beta} \langle\xi\rangle^{m-|\alpha|}
\end{equation}
für alle Multiindices $\alpha,\beta\in\N_0^n$ erfüllen. Die Symbolabschätzungen sind also global gleichmäßig in $x$.
Wir bezeichnen die Menge dieser Symbole mit $S^m_{\rm unif}(\R^n\times\R^n)$. Gilt $\sigma\in S^m_{\rm unif}(\R^n\times\R^n)$
und $u\in\mathscr S(\R^n)$ eine Schwartzfunktion, so definiert
\begin{equation}\label{eq:6:6.17}
   Pu (x) = (2\pi)^{-n/2} \int \e^{\i x\cdot\xi} \sigma(x,\xi) \widehat u(\xi)\d\xi
\end{equation}
wiederum eine Schwartzfunktion. Das folgt durch geeignetes partielles integrieren. Interessanter für uns ist folgender Satz; Operatoren der Ordnung $0$ sind $\rmL^2$--$\rmL^2$-beschränkt.
 
\begin{thm}
Sei $\sigma\in S^0_{\rm unif}(\R^n\times\R^n)$ und $P$ durch \eqref{eq:6:6.17} definiert. Dann gibt es eine Konstante $C>0$, so daß für alle $u\in\mathscr S(\R^n)$
\begin{equation}
   \| Pu\| \le C \|u\|
\end{equation}
gilt.
\end{thm}
\begin{proof}
Wir nehmen in einem ersten Schritt an, daß das Symbol $\sigma(x,\xi)$ kompakt in $x$ getragen ist und bezeichnen seine Fouriertransformierte 
als $\widehat \sigma(\zeta,\xi)$. Nach Voraussetzung ist diese gleichmäßig in $\xi$ schnell fallend in $\zeta$, insbesondere gilt also
\begin{equation}
   |\widehat \sigma(\zeta,\xi)| \le C_N \langle\zeta\rangle^{-N}, \qquad N>n. 
\end{equation}
Damit gilt mit der Fourierschen Inversionsformel und dem Satz von Fubini
\begin{align}
    P u(x) &= (2\pi)^{-n} \int \e^{\i x\cdot\xi} \left(\int \e^{\i x\cdot\zeta} \widehat \sigma(\zeta,\xi) \d\zeta\right) \widehat u(\xi)\d\xi \notag\\
    &= (2\pi)^{-n} \int \e^{\i x\cdot\zeta} \left( \int \e^{\i x\cdot\xi}\widehat \sigma(\zeta,\xi) \widehat u(\xi) \d\xi \right) \d\zeta
\end{align}
die Normabschätzung
\begin{equation}
  \| Pu \| \le (2\pi)^{-n/2} \int  \sup_{\xi\in\R^n} |\widehat \sigma(\zeta,\xi)| \d\zeta\, \|u\| \le C \|u\|. 
\end{equation}
Ersetzt man nun allgemein das Symbol $\sigma(x,\xi)$ durch $\chi(x)\sigma(x,\xi)$ für eine Abschneidefunktion $\chi\in\rmC_0^\infty(\R^n)$ 
mit $|\chi(x)|\le 1$, so hängt die Konstante $C$ in der Normabschätzung $\|\chi Pu\|\le C\|u\|$ nur von den Konstanten in den Symbolabschätzungen von $\sigma$ und der Größe des Trägers von $\chi$ (aber nicht seiner Position) ab.

Für allgemeine Symbole nutzen wir obiges Argument in der Nähe beliebiger Punkte $x_0$ und betrachten dazu das Symbol $\sigma(x,\xi)\chi(x-x_0)$ 
für ein fest gewähltes $\chi\in\rmC_0^\infty(\R^n)$ getragen in einer Kugel vom Radius $1$ und gleich $1$ in der Nähe des Ursprungs.
Zeigt man nun die Hilfsaussage
\begin{equation}\label{eq:5.34}
\int |\chi(x-x_0) P u(x) |^2\d x\le C_N \int \frac{|u(x)|^2}{(1+|x-x_0|)^N} \d x,
\end{equation}
so folgt durch Integration in $x_0$ und mit dem Satz von Fubini
\begin{multline}
  \|\chi\|_2^2 \| Pu\|_2^2 = \iint  |\chi(x-x_0) P u(x) |^2\d x \d x_0 \\
  \le C_N \int |u(x)|^2 \int (1+|x-x_0|)^{-N} \d x_0 \d x = C_N \|u\|_2^2
\end{multline}
und damit für hinreichend große $N$ die Behauptung. Zum Beweis von \eqref{eq:5.34} zerlegen wir $u$ in zwei Teile, $u=u_1+u_2$, wobei $u_2$ außerhalb einer Kugel vom Radius $2$ um $x_0$ getragen sei und $u_1$ Träger in einer Kugel vom Radius $3$ um $x_0$ habe. Weiter nehmen wir an, $|u_j(x)|\le |u(x)|$. Für $u_1$ folgt aus dem ersten Teil
\begin{equation}
    \int |\chi(x-x_0) P u(x) |^2\d x \le C \int_{|x-x_0|\le 3} |u_1(x)|^2\d x \le C_N \int \int \frac{|u(x)|^2}{(1+|x-x_0|)^N} \d x
\end{equation}
für alle $N$ mit geeigneten Konstanten $C_N$. Für den zweiten Teil nutzt man partielle Integration
\begin{equation}
\begin{split}
   \chi(x-x_0) Pu_2(x) &= (2\pi)^{-n}  \iint \e^{\i (x-y)\cdot\xi} \chi(x-x_0) \sigma(x,\xi) u_2(y)\d y\d\xi \\
   &= \iint \e^{\i(x-y)\cdot\xi} \big( \Delta_\xi^\ell \sigma(x,\xi) \big) \frac{\chi(x-x_0) u_2(y)}{|x-y|^{2\ell}} \d y\d\xi
\end{split}
\end{equation}
und mit Cauchy--Schwarz folgt 
\begin{equation}
   |\chi(x-x_0) Pu_2(x)|^2 \le C \int \frac{|u_2(x)|^2}{1+|x-y|^{2\ell}} \d y\; \int \frac{\d y}{1+|x-y|^{2\ell}}
\end{equation}
für hinreichend große $\ell$. Nach nochmaliger Integration in $x$ folgt \eqref{eq:5.34} und damit die Behauptung.
\end{proof}


\begin{thm}
Seien $A$ und $B$ globale Pseudodifferentialoperatoren mit Symbolen $\sigma_A\in S^{m_1}_{\rm unif}(\R^n\times\R^n)$ und $\sigma_B\in S^{m_2}_{\rm unif}(\R^n\times\R^n)$. Dann gilt
\begin{enumerate}
\item die Verkettung $A\circ B$ ist ein globaler Pseudodifferentialoperator der Ordnung $m_1+m_2$ und für das zugehörige Symbol gilt
\begin{equation}
   \sigma_{A\circ B}(x,\xi) \sim \sum_\alpha \frac1{\alpha!} \big(\partial_\xi^\alpha \sigma_A(x,\xi) \big) \big(\D_x^\alpha \sigma_B(x,\xi)\big);
\end{equation}
\item der formal adjungierte Operator $A^*$ ist ein Pseudodifferentialoperator der Ordnung $m_1$ und für das zugehörige Symbol gilt
\begin{equation}
   \sigma_{A^*}(x,\xi) = \sum_{\alpha} \frac1{\alpha!} \partial_\xi^\alpha \D_x^\alpha \overline{\sigma_A(x,\xi)}.
\end{equation}
\end{enumerate}
\end{thm}
\begin{proof}[Beweisidee] Die Idee ist wiederum ähnlich zum Beweis von Lemma~\ref{lem:5.3}. Der durch Verkettung entstehende Operator hat formal das Symbol
\begin{equation}
    \sigma_{A\circ B}(x,\xi) = \lim_{\epsilon\to0}  (2\pi)^{-n} \iint \e^{-\i y\cdot\eta  } \sigma_A(x,\eta+\xi) \sigma_B(y+x,\xi) \chi(\epsilon y,\epsilon\eta) \d\eta\d y
\end{equation}
für eine Abschneidefunktion\footnote{Die Funktion dient nur dazu die Integrale konvergent und damit Fubini anwendbar zu machen.} $\chi\in\rmC_0^\infty(\R^{n}\times\R^n)$, $\chi(x,\xi)=1$ in der Nähe von $(0,0)$. Nun schreibt man $\sigma_A(x,\eta+\xi)$ als Taylorreihe bezüglich $\eta$ im Punkt $\eta=0$ 
\begin{equation}
   \sigma_A(x,\eta+\xi) = \sum_{|\alpha|\le N} \frac1{\alpha!} \big(\partial_\xi^\alpha \sigma_A(x,\xi)\big) \eta^\alpha + r_N(x,\xi,\eta)
\end{equation}
und erhält
\begin{equation}
\begin{split}
  \sigma_{A\circ B}(x,\xi) =& \lim_{\epsilon\to0}  (2\pi)^{-n}  \sum_{|\alpha|\le N}  \frac1{\alpha!} \iint \e^{-\i y\cdot\eta  }  \big(\partial_\xi^\alpha \sigma_A(x,\xi)\big) \sigma_B(x+y,\xi)  \eta^\alpha \chi(\epsilon y, \epsilon \eta)\d\eta\d y \\
  &+ \lim_{\epsilon\to0}  (2\pi)^{-n}  \iint \e^{-\i y\cdot\eta  }  r_N(x,\xi,\eta) \sigma_B(x+y,\xi)  \chi(\epsilon y, \epsilon \eta)\d\eta\d y.
\end{split}
\end{equation}
Der zweite Term wird durch partielle Integration und mit Hilfe des Integralrestgliedes abgeschätzt. Der erste Term liefert nach partieller Integration und unter Ausnutzung der Fourierschen Inversionsformel
\begin{equation}
  \sum_{|\alpha|\le N}  \frac1{\alpha!}  \big(\partial_\xi^\alpha \sigma_A(x,\xi)\big) \big(\D_x^\alpha \sigma_B(x,\xi)  \big).
\end{equation}
Die Formel für das Symbol des adjungierten Operators erhält man ähnlich.
\end{proof}


\begin{df}[Sobolevnormen]\label{df:sob-norm}
Für $u\in\rmC_0^\infty(\Omega)$, $\Omega \subset\R^n$ ein Gebiet, definieren wir die Sobolevnorm der Ordnung $s\in\R$,
\begin{equation}
     \| u \|_{(s)} = \bigg(\int \langle\xi\rangle^{2s} |\widehat u(\xi)|^2\d\xi\bigg)^{1/2}. 
\end{equation}
Weiter bezeichne $\rmH^s_0(\Omega)$ den Abschluß von $\rmC_0^\infty(\Omega)$ bezüglich dieser Norm. Die bisher für $s\in\N$ genutzten Sobolevnormen sind zu dieser äquivalent. Speziell für $\Omega=\R^n$ schreiben wir $\rmH^s(\R^n)$ für den Abschluß von $\rmC_0^\infty(\R^n)$ in der $\|\cdot\|_{(s)}$-Norm.
\end{df}

\begin{cor}
Sei $A$ ein Pseudodifferentialoperator mit Symbol $\sigma_A\in S^m_{\rm unif}(\R^n\times\R^n)$ für ein $m\in\R$. Dann gilt für alle $s\in\R$ und geeignete Konstanten $C_s>0$ die Abschätzung
\begin{equation}
  \forall u\in\rmC_0^\infty(\R^n)\quad:\quad \|Pu \|_{(s)} \le C_s \|u\|_{(s+m)}.
\end{equation}
\end{cor}

%
%
%\section{Lokalisierungen}
%Sei im folgenden $\Theta\in\rmC_0^\infty(\R^n)$ mit $\supp\Theta\subset [-3/4,3/4]^n$ sowie $\Theta>0$ auf $[-1/2,1/2]^n$. Dann sind die Funktionen
%\begin{equation}
%\varphi_k(x) = \Theta(x-k) \bigg(\sum_{j\in\Z^n} \Theta(x-j)^2\bigg)^{-1/2}
%\end{equation}
%Elemente von $\rmC_0^\infty(\R^n)$ mit $\sum_{k} \varphi_k(x)^2=1$ und es existiert für jeden Multiindex $\alpha\in\N_0^n$ eine Konstante $C_\alpha$ mit
%\begin{equation}
%    \sum_{k\in\Z^n} | \D^\alpha \varphi_k(x)|^2 \le C_\alpha.
%\end{equation}
%Weiterhin impliziert $x,y\in\supp\varphi_k$ stets $|x-y|\le 3\sqrt n/2$. Ausgehend von dieser Partition der Eins definieren wir
%\begin{equation}
%    \psi_k(\xi)= \varphi_k(\xi / \sqrt{|\xi|}), 
%\end{equation}
%so daß
%\begin{equation}\label{eq:6:6.35} 
%    \sum_{k\in\Z^n} \psi_k(\xi)^2 = 1\qquad\text{und}\qquad |\xi|^\alpha \sum_{k\in\Z^n} |\partial_\xi^\alpha \psi_k(\xi)|^2 \le C_\alpha
%\end{equation}
%gilt. Weiter impliziert $\xi,\eta\in\supp\psi_k$ stets $|\sqrt{|\xi|}-\sqrt{|\eta|}|\le C$ und damit 
%\begin{equation}
% |\xi-\eta|\le C \langle\xi\rangle^{1/2} \qquad\text{für alle $\xi,\eta\in\supp\psi_k$}.
%\end{equation}
%Weiterhin nützlich ist
%\begin{equation}
%  \sum_{k\in\Z^n} |\psi_k(\xi)-\psi_k(\eta)|^2 \le C \frac{|\xi-\eta|^2}{ \langle\xi\rangle^{1/2} \langle\eta\rangle^{1/2}},
%\end{equation} 
%was nur für $|\xi-\eta|\le \langle\xi\rangle^{1/2}$ nichttrivial ist und dann aus \eqref{eq:6:6.35} für $|\alpha|=1$ per Integration über die Strecke von $\xi$ zu $\eta$ folgt.
%
%
%Als eine Anwendung wollen wir zu einem Operator $P$ zum Symbol $\sigma\in S^m_{\rm comp} (\Omega\times\R^n)$ die sesquilineare Form
%\begin{equation}
%   \spro{Pu}{v} = (2\pi)^{-n/2} \iint \e^{\i x\cdot\xi} \sigma(x,\xi) \widehat u(\xi) \d\xi \overline{v(x)}\d x 
%   = (2\pi)^{-n/2} \iint \widehat u(\xi) \overline{\widehat v(\eta)} \widehat\sigma(\eta-\xi,\xi) \d\xi\d\eta
%\end{equation} 
%betrachten. Wir setzen $u_k(x) = \psi_k(\D) u(x)$, also $\widehat u_k(\xi) = \psi_k(\xi) \widehat u(\xi)$ und untersuchen die Differenz 
%$\spro{Pu}{v}-\sum_k \spro{Pu_k}{v_k}$. Für diese gilt
%\begin{align}
%       \spro{Pu}{v} -\sum_{k\in\Z^n} \spro{Pu_k}{v_k}  &= (2\pi)^{-n/2}  \iint \widehat\sigma (\eta-\xi,\xi) \widehat u(\xi) \overline{\widehat v(\eta)} \bigg( 1 - \sum_{k\in\mathbb Z^n} \psi_k(\xi)\psi_k(\eta)\bigg) \d\xi\d\eta  \notag\\
%       &= \frac12 (2\pi)^{-n/2} \iint \widehat\sigma (\eta-\xi,\xi) \widehat u(\xi) \overline{\widehat v(\eta)}  \sum_{k\in\mathbb Z^n} | \psi_k(\xi)-\psi_k(\eta)|^2 \d\xi\d\eta 
%\end{align}
%und damit
%\begin{equation}
%   \bigg|\spro{Pu}{v} - \sum_{k\in\Z^n}\spro{Pu_k}{v_k}\bigg| \le C \iint | \widehat \sigma(\eta-\xi,\xi) \widehat u(\xi) \overline{\widehat u(\eta)}| |\eta-\xi|^2 \langle\xi\rangle^{-1/2} \langle\eta\rangle^{-1/2} \d\xi\d\eta, 
%\end{equation}
%zusammen mit den Symbolabschätzungen also
%\begin{equation}
%    \bigg|\spro{Pu}{v} - \sum_{k\in\Z^n}\spro{Pu_k}{v_k}\bigg| \le C \|u\|_{(m-1/2)} \|v\|_{(-1/2)}.
%\end{equation}
% Andererseits gilt für jeden einzelnen Summanden
% \begin{align}
%     |\spro{Pu_k}{v_k} | &= \bigg|\iint \widehat u(\xi) \overline{\widehat v(\eta)} \psi_k(\xi)\psi_k(\eta) \widehat\sigma(\eta-\xi,\xi) \d\xi\d\eta\bigg|
%     \notag\\
%     &\le \|u\|_{(m)}  \|v\|_{(0)} \iint | \langle\xi\rangle^{-m}\psi_k(\xi)\psi_k(\eta) \widehat\sigma(\eta-\xi,\xi)|^2 \d\xi\d\eta.
% \end{align}
%Das verbleibende Integral ist beschränkt, 
%\begin{align}
% \iint | \langle\xi\rangle^{-m}\psi_k(\xi)\psi_k(\eta) \widehat\sigma(\eta-\xi,\xi)|^2 \d\xi\d\eta 
% \le C_N  \iint | \psi_k(\xi)\psi_k(\eta) \langle \xi-\eta\rangle^{-N} |^2 \d\xi\d\eta \le C
%\end{align}
%gleichmäßig in $k$. Insgesamt haben wir also gezeigt, dass
%\begin{equation}
%   |\spro{Pu}{v}| \le C \|u\|_{(m)} \|v\|_{(0)}
%\end{equation}
%gilt, der Operator $P$ also von $\rm H^m_0(\Omega) \to \rmL^2(\Omega)$ beschränkt ist. Das haben wir vorher schon aus dem Kalkül und der $\rmL^2$-Beschränktheit der Operatoren der Ordnung Null geschlossen, der hier dargestellte Beweis hat aber den Vorteil mit weniger Voraussetzungen an das Symbol auszukommen. Die Beweisidee wird uns in den folgenden Abschnitten noch mehrmals begegnen.
%



