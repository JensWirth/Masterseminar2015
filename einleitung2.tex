% !TEX root = main.tex
\chapter{Einleitung}
Hier sollen die Grundlagen dafür gelegt werden, auch Operatoren mit variablen Koeffizienten richtig behandeln zu können. Die Darstellung basiert auf \cite{Hormander:1965}, \cite{Hormander:1966}, sowie \cite[Kapitel 18]{Hormander:1985}. Zuerst verallgemeinern wir den Begriff des Differentialoperators soweit, daß wir auch Inverse und allgemeinere Funktionen von solchen Operatoren behandeln können. 

\section{Operatoren und Symbole}
Sei $\Omega$ eine Mannigfaltigkeit. Differentialoperatoren auf $\Omega$ können dann in lokalen Karten definiert werden oder global durch ihre Eigenschaften charakterisiert werden. Die bekannteste davon ist, daß eine stetige lineare Abbildung $P:\rmC_0^\infty(\Omega)\to\rmC^\infty(\Omega)$ genau dann Differentialoperator ist, wenn $P$ lokal ist, also wenn
\begin{equation}
  \forall f\in\rmC_0^\infty(\Omega)\quad:\quad \supp Pf \subseteq \supp f
\end{equation}
gilt. Für Verallgemeinerungen brauchbarer ist folgende Charakterisierung:

\begin{lem}
Eine  lineare Abbildung   $P:\rmC_0^\infty(\Omega)\to\rmC^\infty(\Omega)$ ist genau dann ein Differentialoperator der Ordnung $m$, wenn für alle $f\in\rmC_0^\infty(\Omega)$ und alle $g\in\rmC^\infty(\Omega)$ die Funktion 
\begin{equation}\label{eq:6:6.2}
\e^{-\i\lambda g} P(f\e^{\i\lambda g}) = \sum_{j=0}^m P_j(f,g) \lambda^j
\end{equation}
ein Polynom vom Grad $m$ in $\lambda$ ist.
\end{lem}
\begin{proof}
Die Hinrichtung ist klar. Zum Beweis der Rückrichtung sei $x'\in\Omega$ ein Punkt und $f\in\rmC_0^\infty(\Omega)$ mit $f=1$  in einer Umgebung $\omega$ von $x'$. Sei weiter $\xi\in\R^n$ und $g(x) = x\cdot\xi$. Dann folgt aus \eqref{eq:6:6.2}
\begin{equation}
    \e^{-\i \lambda x\cdot\xi} P ( f\e^{\i \lambda x\cdot\xi}) = \sum_{j=0}^m p_j(f;x,\xi) \lambda^j
\end{equation}
mit $\rmC^\infty$-Funktionen auf $\omega\times\R^n$ als Koeffizienten $p_j(f;x,\xi)$. Da auch
$p_j(f;x,\lambda\xi)=p_j(f;x,\xi)\lambda^j$ gilt, ist $p_j(f;x,\xi)$ homogen vom Grad $j$. Die einzigen auf ganz $\R^n$ glatten homogenen Funktionen sind Polynome,
$p_j(f;x,\xi) = \sum_{|\alpha|=j} a_\alpha(f;x) \xi^\alpha$. Sei nun $u\in\rmC_0^\infty(\omega)$. Dann gilt mit der Fourierschen Inversionsformel
\begin{equation}
    u(x) = (2\pi)^{-n/2} \int \e^{\i x\cdot\xi} \widehat u(\xi)\d\xi
\end{equation}
und da $u=fu$ folgt insbesondere 
\begin{align}
    Pu(x) &= P(fu)(x) = (2\pi)^{-n/2} \int P(f \e^{\i x\cdot\xi}) \widehat u(\xi)\d\xi \notag \\&= \sum_{j=0}^m (2\pi)^{-n/2} \int \e^{\i x\cdot\xi} p_j(f;x,\xi) \widehat u(\xi)\d\xi
    = \sum_{|\alpha|\le m} a_\alpha(f;x)\D^\alpha u.
\end{align}
Mit Linearität folgt die Behauptung.
\end{proof}

Wir betrachten im folgenden allgemeiner Operatoren $P : \rmC_0^\infty(\Omega) \to \rmC^\infty(\Omega)$ und deren Beschreibung durch ein \eIndex[Pseudodifferentialoperator]{Symbol}
\begin{equation}\label{eq:6:6.6}
    \sigma_P(f; x,\xi) = \e^{-\i x\cdot\xi} P( f(x) \e^{\i x\cdot\xi}).
\end{equation}  
Dabei sei $f\in\rmC_0^\infty(\Omega)$ kompakt getragen mit $f(x)=1$ auf einer kompakten Teilmenge $\omega\Subset\Omega$. 
Dann gilt für jedes $u\in\rmC_0^\infty(\omega)$ 
\begin{equation}
   P u (x) = (2\pi)^{-n/2} \int \e^{\i x\cdot\xi} \sigma_P(f; x,\xi) \widehat u(\xi) \d\xi.
\end{equation}
Der Operator $P$ wird als \eIndex{Pseudodifferentialoperator} der \eIndex[Pseudodifferentialoperator]{Ordnung} $m$ bezeichnet, falls jedes so entstehende Symbol 
$\sigma_P(f;\cdot,\cdot)$ zur nachfolgend definierten \eIndex{Symbolklasse} $S^m(\Omega\times\R^n)$ gehört.

\begin{df}
Eine Funktion $\sigma\in\rmC^\infty(\Omega\times\R^n)$ gehört zur Symbolklasse $S^m(\Omega\times\R^n)$ falls
\begin{equation} 
  \sup_{x\in K}  |  \partial_x^\beta\partial_\xi^\alpha \sigma(x,\xi) | \le C_{K,\alpha,\beta} \langle\xi\rangle^{m-|\alpha|}
\end{equation}
für alle Kompakta $K\Subset\Omega$ und alle Multiindices $\alpha,\beta\in\N^n_0$ gilt. Dabei bezeichnet $\langle\xi\rangle = \sqrt{1+|\xi|^2}$.
\end{df}

Ist nun $\sigma\in S^m(\Omega\times\R^n)$, so konvergiert für jedes $u\in\rmC_0^\infty(\Omega)$ das Integral
\begin{equation}\label{eq:6:6.9}
   P u (x) = (2\pi)^{-n/2} \int \e^{\i x\cdot\xi} \sigma(x,\xi) \widehat u(\xi) \d\xi
\end{equation}
und definiert eine glatte Funktion $Pu\in\rmC^\infty(\Omega)$. Es gilt sogar mehr

\begin{lem}
Sei $\sigma\in S^m(\Omega\times\R^n)$ und $Pu$ für $u\in\rmC_0^\infty(\Omega)$ durch \eqref{eq:6:6.9} definiert. Dann gilt für jedes $f\in\rmC_0^\infty(\Omega)$
und das durch \eqref{eq:6:6.6} zugeordnete Symbol
\begin{equation}
    \sigma_P(f;\cdot,\cdot)\in S^m(\Omega\times\R^n). 
\end{equation}
Weiterhin gilt
\begin{equation}
   \sigma_P(f;x,\xi) \sim \sum_{\alpha}  \frac{1}{\alpha!} \big( \partial_\xi^\alpha \sigma(x,\xi) \big) \big(\D_x^\alpha f(x)\big)
\end{equation}
\end{lem}
\begin{proof}
Nach Definition gilt
\begin{align}
   \sigma_P(f;x,\xi) &=(2\pi)^{-n}  \e^{-\i x\cdot\xi} \int \e^{\i x\cdot\eta} \sigma(x,\eta) \int_\Omega \e^{-\i y\cdot(\eta-\xi)} f(y) \d y \d\eta \notag\\
   & = (2\pi)^{-n/2}  \int \e^{\i x\cdot(\eta-\xi)} \sigma(x,\eta) \widehat f(\eta-\xi)\d\eta \notag\\
   & = (2\pi)^{-n/2}  \int \e^{\i x\cdot\eta} \sigma(x,\eta+\xi) \widehat f(\eta)\d\eta \notag\\
   & = (2\pi)^{-n/2}  \int \e^{\i x\cdot\eta} \left( \sum_{|\alpha|<N} \frac{\eta^\alpha}{\alpha!} \partial_\xi^\alpha \sigma(x,\xi) + R_N(x,\xi,\eta) \right) \widehat f(\eta)\d\eta \notag\\  
   & =  \sum_{|\alpha|<N} \frac{1}{\alpha!} \big( \partial_\xi^\alpha \sigma(x,\xi) \big) \big(\D_x^\alpha f(x)\big)  +
   \rho_N(x,\xi)
\end{align}
unter Ausnutzung des taylorschen Satzes mit Entwicklungspunkt $\eta=0$. Weiter gilt unter Nutzung der Integraldarstellung des Restgliedes $R_N(x,\xi,\eta)$ 
\begin{equation}
   R_N(x,\xi,\eta) = N \int_0^1 \sum_{|\alpha|=N} \frac{(1-t)^N \eta^\alpha}{\alpha!} \partial_\xi^\alpha \sigma(x,\xi+t\eta) \d t 
\end{equation}
und somit für beliebige Kompakta $K\Subset\Omega$ und $N\ge m$
\begin{align}
   |\langle\xi\rangle^{N-m} \rho_N(x,\xi)|& \lesssim 
   \left|\langle\xi\rangle^{N-m}  \int \e^{\i x\cdot\eta} \int_0^1 \sum_{|\alpha|=N} \frac{(1-t)^N \eta^\alpha}{\alpha!} \partial_\xi^\alpha \sigma(x,\xi+t\eta) \d t  \widehat f(\eta) \d\eta   \right| \notag\\
&\lesssim  \int \int_0^1 \langle\eta\rangle^N \langle\xi+t\eta\rangle^{m-N} \langle\xi\rangle^{N-m} \d t | \widehat f(\eta) |\d\eta   
\lesssim \int \langle\eta\rangle^{2N} |\widehat f(\eta)|\d\eta
\end{align}
gleichmäßig in $x\in K$ (und unter Ausnutzung der Ungleichung von Peetre $\langle \xi\rangle^s \lesssim \langle\eta\rangle^s \langle\xi-\eta\rangle^s$ für $s\ge0$).
Weiter gilt für Multiindices $\beta,\gamma\in\N_0^n$
\begin{multline}
   \partial_x^\gamma \partial_\xi^\beta \rho_N(x,\xi) \\
   =  (2\pi)^{-n/2} N \sum_{\delta\le\gamma} \binom{\gamma}{\delta} \i^{|\delta|} \int \e^{\i x\cdot\eta} \int_0^1 \sum_{|\alpha|=N} \frac{(1-t)^N \eta^{\alpha+\delta}}{\alpha!} \partial_x^{\gamma-\delta} \partial_\xi^{\alpha+\beta} \sigma(x,\xi+t\eta) \d t  \widehat f(\eta) \d\eta 
\end{multline}
und obige Abschätzung liefert $\rho_N\in S^{m-N}(\Omega\times\R^n)$. Das impliziert die Behauptung.
\end{proof}
