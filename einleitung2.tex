% !TEX root = main.tex
\chapter{Einleitung}
Hier sollen die Grundlagen dafür gelegt werden, auch Operatoren mit variablen Koeffizienten richtig behandeln zu können. Die Darstellung basiert auf \cite{Hormander:1965}, \cite{Hormander:1966}, sowie \cite[Kapitel 18]{Hormander:1985}. Zuerst verallgemeinern wir den Begriff des Differentialoperators soweit, daß wir auch Inverse und allgemeinere Funktionen von solchen Operatoren behandeln können. 

\section{Operatoren und Symbole}
Sei $\Omega$ eine Mannigfaltigkeit. Differentialoperatoren auf $\Omega$ können dann in lokalen Karten definiert werden oder global durch ihre Eigenschaften charakterisiert werden. Die bekannteste davon ist, daß eine stetige lineare Abbildung $P:\rmC_0^\infty(\Omega)\to\rmC^\infty(\Omega)$ genau dann Differentialoperator ist, wenn $P$ lokal ist, also wenn
\begin{equation}
  \forall f\in\rmC_0^\infty(\Omega)\quad:\quad \supp Pf \subseteq \supp f
\end{equation}
gilt. Für Verallgemeinerungen brauchbarer ist folgende Charakterisierung:

\begin{lem}
Eine  lineare Abbildung   $P:\rmC_0^\infty(\Omega)\to\rmC^\infty(\Omega)$ ist genau dann ein Differentialoperator der Ordnung $m$, wenn für alle $f\in\rmC_0^\infty(\Omega)$ und alle $g\in\rmC^\infty(\Omega)$ die Funktion 
\begin{equation}\label{eq:6:6.2}
\e^{-\i\lambda g} P(f\e^{\i\lambda g}) = \sum_{j=0}^m P_j(f,g) \lambda^j
\end{equation}
ein Polynom vom Grad $m$ in $\lambda$ ist.
\end{lem}
\begin{proof}
Die Hinrichtung ist klar. Zum Beweis der Rückrichtung sei $x'\in\Omega$ ein Punkt und $f\in\rmC_0^\infty(\Omega)$ mit $f=1$  in einer Umgebung $\omega$ von $x'$. Sei weiter $\xi\in\R^n$ und $g(x) = x\cdot\xi$. Dann folgt aus \eqref{eq:6:6.2}
\begin{equation}
    \e^{-\i \lambda x\cdot\xi} P ( f\e^{\i \lambda x\cdot\xi}) = \sum_{j=0}^m p_j(f;x,\xi) \lambda^j
\end{equation}
mit $\rmC^\infty$-Funktionen auf $\omega\times\R^n$ als Koeffizienten $p_j(f;x,\xi)$. Da auch
$p_j(f;x,\lambda\xi)=p_j(f;x,\xi)\lambda^j$ gilt, ist $p_j(f;x,\xi)$ homogen vom Grad $j$. Die einzigen auf ganz $\R^n$ glatten homogenen Funktionen sind Polynome,
$p_j(f;x,\xi) = \sum_{|\alpha|=j} a_\alpha(f;x) \xi^\alpha$. Sei nun $u\in\rmC_0^\infty(\omega)$. Dann gilt mit der Fourierschen Inversionsformel
\begin{equation}
    u(x) = (2\pi)^{-n/2} \int \e^{\i x\cdot\xi} \widehat u(\xi)\d\xi
\end{equation}
und da $u=fu$ folgt insbesondere 
\begin{align}
    Pu(x) &= P(fu)(x) = (2\pi)^{-n/2} \int P(f \e^{\i x\cdot\xi}) \widehat u(\xi)\d\xi \notag \\&= \sum_{j=0}^m (2\pi)^{-n/2} \int \e^{\i x\cdot\xi} p_j(f;x,\xi) \widehat u(\xi)\d\xi
    = \sum_{|\alpha|\le m} a_\alpha(f;x)\D^\alpha u.
\end{align}
Mit Linearität folgt die Behauptung.
\end{proof}

Im folgenden sollen die Polynome in $\lambda$ durch asymptotische Entwicklungen ersetzt werden. Die folgende Definition geht auf \cite{Hormander:1965} zurück.

\begin{df}
Sei $P:\rmC_0^\infty(\Omega)\to\rmC^\infty(\Omega)$  linear. Dann heißt $P$ (klassischer) \eIndex{Pseudodifferentialoperator} der Ordnung $m\in\R$, wenn für jedes $f\in\rmC_0^\infty(\Omega)$ und alle $g\in K\Subset \rmC^\infty(\Omega)$ mit $\d g\ne0$ auf $\supp f$
\begin{equation}
    \e^{-\i\lambda g} P(f\e^{\i\lambda g}) \sim \sum_{j=0}^\infty P_j(f,g) \lambda^{m-j}
\end{equation}
in $\rmC^\infty(G)$ gleichmäßig in $K$  gilt.\footnote{D.h., für jedes $N$ ist
\begin{equation}\label{eq:6:6.7}
   \lambda^{N-m} \left( \e^{-\i\lambda g} P(f\e^{\i\lambda g}) - \sum_{j=0}^{N-1} P_j(f,g) \lambda^{m-j}\right)
\end{equation}
gleichmäßig in $\lambda>1$ und $g\in K$ in $\rmC^\infty(\Omega)$ beschränkt.} 
\end{df}

\begin{thm}
Sei $P$ ein Pseudodifferentialoperator der Ordnung $m$ auf $\Omega$ und $\omega\Subset\Omega$. Dann existiert eine Funktion
$p\in\rmC^\infty(\omega\times\R^n)$ mit
\begin{equation}\label{eq:6:6.8}
     \sup_{x\in\omega} \sup_{\xi\in\R^n} \left| \langle\xi\rangle^{|\alpha|-m} \D_x^\beta \D_\xi^\alpha p(x,\xi) \right| < \infty
\end{equation}
für alle Multiindices $\alpha,\beta\in\mathbb N_0^n$, so daß für jede $u\in\rmC_0^\infty(\omega)$
\begin{equation}\label{eq:6:6.9}
   Pu(x) = (2\pi)^{-n/2} \int \e^{\i x\cdot\xi} p(x,\xi) \widehat u(\xi) \d\xi
\end{equation}
gilt.
\end{thm}
\begin{proof}
Sei $f\in\rmC_0^\infty(\Omega)$ mit $f=1$ auf $\omega$. Dann gilt $u=uf$ für $u\in\rmC_0^\infty(\omega)$ und mit 
\begin{equation}
   p(x,\xi) = p(f;x,\xi) =   \e^{-\i x\cdot\xi} P(f\e^{\i x\cdot\xi})
\end{equation}
folgt die Darstellung \eqref{eq:6:6.9}. Damit genügt es, \eqref{eq:6:6.8} zu zeigen. Wir betrachten zuerst den Fall $\alpha=0$. Dann besagt \eqref{eq:6:6.7} mit $N=0$
\begin{equation}
  \sup_{\lambda>1} \sup_{|\xi|=1} \sup_{x\in\omega} \left| \lambda^{-m} \D_x^\beta p(x,\lambda\xi) \right|<\infty
\end{equation}
für jedes $\beta\in\mathbb N_0^n$. Die Abschätzungen für $\alpha\ne0$ folgen nun per Induktion. Die Funktion $p(x,\xi)$ ist glatt und Differenzieren liefert
\begin{equation}
    \D_{\xi_j} p(f;x,\xi) = p(x_jf; x,\xi)- x_j p(f;x,\xi) ,
\end{equation}
beide Terme haben nach Voraussetzungen asymptotische Entwicklungen. Also folgt, daß
\begin{equation}
   \lambda^{-m}  \left( \D_\xi^\alpha p(x,\lambda \xi) - \D_\xi^\alpha \sum_{j=0}^{|\alpha|} p_j(x,\lambda\xi) \right)
\end{equation}
in $\rmC^\infty(\Omega)$ beschränkt ist.

 \fbox{\bf ZU ERGÄNZEN}
\end{proof}

\begin{df}
Eine Funktion $p\in\rmC^\infty(\Omega\times\R^n)$, welche \eqref{eq:6:6.7} für jedes $\omega\Subset\Omega$ erfüllt, wird als \eIndex[Pseudodifferentialoperator]{Symbol} der Ordnung $m$ bezeichnet. Die Menge aller solchen Symbole sei $\mathcal S^m(\Omega\times\R^n)$.

Ein Symbol $p\in\mathcal S^m(\Omega\times\R^n)$ heißt \eIndex[Pseudodifferentialoperator!Symbol]{klassisch}, falls es eine Folge homogener Funktionen $p_j(x,\xi)$ vom Grad $m-j$, also mit $p_j(x,\lambda\xi)=\lambda^{m-j}p_j(x,\xi)$ für $\lambda>0$, gibt, so daß
\begin{equation}
     p(x,\xi) \sim \sum_j p_j(x,\xi),\qquad |\xi|\to\infty,
\end{equation}
gilt.\footnote{D.h., mit einer Abschneidefunktion $\chi$ gilt $p-\sum_{j=1}^{N-1} \chi p_j\in\mathcal S^{m-N}(\Omega\times\R^n)$ für jedes $N\in\mathbb N$.}
\end{df}

%Sei $P$ ein Operator auf $\rmL^2(\Omega)$ mit Definitionsbereich $\mathcal D_P$ und gelte $\rmC_0^\infty(\Omega)\subset\mathcal D_P$. Der Operator wird als (klassischer) \eIndex{Pseudodifferentialoperator} bezeichnet, wenn für jedes $\varphi\in\rmC_0^\infty(\Omega)$ und mit 
%$\varphi_\xi (x) = \varphi(x) \e^{\i x\cdot\xi}$ die Funktion
%\begin{equation}
%   \sigma_\varphi(x,\xi) = \e^{-\i x\cdot\xi}  P \varphi_\xi (x)
%\end{equation}
%eine asymptotische Entwicklung für $|\xi|\to\infty$ besitzt.

%Wir betrachten im folgenden Differentialoperatoren auf $\R^n$, analoge Resultate gelten auch jeweils für minimale Operatoren auf Gebieten. Im folgenden bezeichne
%$\mathscr B^\infty(\R^n)$ die Menge der glatten Funktionen mit beschränkten Ableitungen,
%\begin{equation}
%   \mathscr B^\infty(\R^n) = \{ f\in\rmC^\infty(\R^n) : \D^\alpha f\in\rmL^\infty(\R^n)\}.
%\end{equation}
%Ein Differentialoperator
%\begin{equation}
%   \mathcal P = \sum_{|\alpha|\le m} a_\alpha(x) \D^\alpha
%\end{equation}
%mit Koeffizienten $a_\alpha\in\mathscr B^\infty(\R^n)$ besitzt das Symbol
%\begin{equation}
%  \mathcal P(x,\xi) = \sum_{|\alpha|\le m} a_\alpha(x)\xi^\alpha.
%\end{equation}
%Operator und Symbol hängen dabei eng zusammen. Einerseits gilt
%\begin{equation}
%    \mathcal P(x,\xi) = \e^{-\i x\cdot \xi} \mathcal P (\e^{\i x\cdot\xi}),
%\end{equation}
%andererseits kann die Anwendung von $\mathcal P$ auf Schwartzfunktionen $u\in\mathscr S(\R^n)$ als iteriertes Integral
%\begin{equation}
%    \mathcal P u(x) = (2\pi)^{-n} \iint \e^{\i (x-y)\cdot \xi} \mathcal P(x,\xi) u(y) \d y\d\xi 
%\end{equation}
%geschrieben werden.

\section{Kalkül}

\section{Littlewood--Paley-Zerlegung}

