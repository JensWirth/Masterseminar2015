% !TEX root = main.tex
%\chapter{Ein unlösbarer Operator}
%\cite{Lewy:1957}
%\cite{Hormander:1960b}


\section{Das Beispiel von Lewy}


Bis Mitte des 20. Jahrhunderts ging man davon aus, dass die lokale Existenz glatter Lösungen für lineare Differentialgleichungen stets gegeben sei. Das Beispiel von Hans Lewy zeigt eindrucksvoll, dass dies nicht zwingend der Fall sein muss.  In diesem Abschnitt soll eine lineare, partielle Differentialgleichung erster Ordnung in drei Variablen mit komplexwertigen $\rmC^\infty$-Koeffizienten vorgestellt werden, welche in \emph{keiner} offenen Menge eine glatte/distributionelle Lösung besitzt. Hierfür betrachtet man den durch
\begin{equation}\label{lewy:differentialausdruck}
\mathscr L := -\frac{\partial}{\partial x_1} -\i\frac{\partial}{\partial x_2} +2\i(x_1+\i x_2)\frac{\partial}{\partial y_1}
\end{equation}
definierten Differentialausdruck auf $\R^3$. Das erste überraschende Resultat von Lewy ist in folgendem Lemma enthalten:
\begin{lem}[{\cite{Lewy:1957}}]\label{thm:1_lewy}
Zu einer reellwertigen Funktion $\psi\in \rmC^1(\mathbb{R})$ besitze das Problem
\begin{equation}\label{eq:1_lewy:gleichung}
\mathscr Lu=\psi'(y_1)
\end{equation}
in einer Umgebung $\Omega\subset\R^3$ von $(0,0,y_1^0)$ eine $\rmC^1$-Lösung $u$. Dann ist $\psi$ analytisch in $y_1=y_1^0$.
\end{lem}

\begin{proof}
Wir integrieren $(\partial_1+\i\partial_2) u$ für eine Lösung $u$ von \eqref{eq:1_lewy:gleichung} über einen Kreis in der $x_1$-$x_2$ Ebene um den Punkt $(0,0,y_1^0)$. Der Radius wird dabei so klein gewählt, dass der Kreis in $\Omega$ liegt. Sei dazu
\begin{equation}
x_1^2+x_2^2=y_2=\mathrm{const},\qquad y_1=\mathrm{const},
\end{equation}
$t=\log\sqrt{y_2}=\log\sqrt{x_1^2+x_2^2}$ und $\theta$ derjenige Winkel gegeben durch
\begin{equation}
x_1+\i x_2=\sqrt{y_2} \, \e^{\i\theta} =\e^{t+\i\theta}.
\end{equation}
Dann erhält man durch einfaches Nachrechnen $\overline x \overline\partial_x = \overline\partial_t$, also
\begin{equation}\label{thm:1_lewy:proof1}
\frac{\partial}{\partial x_1} +\i\frac{\partial}{\partial x_2}=
\e^{-t+\i\theta} \left(\frac{\partial}{\partial t}
+\i\frac{\partial}{\partial \theta}\right).
\end{equation}
Diese Identität zusammen mit partieller Integration liefert
\begin{align}\label{thm:1_lewy:abl_gleich}
\begin{split}
\int_0^{2\pi} \left(\frac{\partial}{\partial x_1} +\i\frac{\partial}{\partial x_2}\right)u\d\theta 
&= \int_0^{2\pi} \e^{-t+\i\theta} \left(\frac{\partial}{\partial \,t}+\i\frac{\partial}{\partial \theta}\right)u \d\theta \\
&= \int_0^{2\pi} \e^{-t+\i\theta} \left(\frac{\partial u}{\partial \,t} +\i\frac{\partial u}{\partial \theta}\right)\d\theta \\
&= \int_0^{2\pi} \e^{-t+\i\theta} \left(\frac{\partial u}{\partial \,t} +u\right)\d\theta .
\end{split}
\end{align}
Weiter impliziert $\sqrt{y_2}=\e^t$ für jede differenzierbare Funktion $w$
\begin{align}\label{thm:1_lewy:int_gleichheit1}
\begin{split}
\frac{\partial w}{\partial\, t}+ w  = 2\sqrt{y_2} \frac{\partial}{\partial y_2} \left( \sqrt{y_2} w(\log\sqrt{y_2})\right).
\end{split}
\end{align}
Eingesetzt in das letzte Integral aus \eqref{thm:1_lewy:abl_gleich} ergibt sich
\begin{align}\label{thm:1_lewy:abl_gleich_final}
\begin{split}
\int_0^{2\pi} \left(\frac{\partial}{\partial x_1} +\i\frac{\partial}{\partial x_2}\right)u\d\theta 
= 2\left(\frac{\partial}{\partial y_2}\right)\int_0^{2\pi} \e^{\i\theta}\sqrt{y_2}u\d\theta.
\end{split}
\end{align}
Setzen wir nun 
\begin{equation}
I(y_1,y_2)=\i\int_0^{2\pi} \e^{\i\theta} \sqrt{y_2} u\d\theta,
\end{equation}
so liefert \eqref{eq:1_lewy:gleichung} und danach \eqref{thm:1_lewy:abl_gleich_final}
\begin{align}
\begin{split}
\frac{\partial I}{\partial y_1} +\i\frac{\partial I}{\partial y_2} 
&= \i\int_{0}^{2\pi} \frac{\partial}{\partial y_1}\left(\e^{\i\theta}\sqrt{y_2}u\right)+\i\frac{\partial}{\partial y_2}\left(\e^{\i\theta}\sqrt{y_2}u\right)\d\theta \\
&= \i\int_{0}^{2\pi} \e^{\i\theta}\sqrt{y_2}\left(\frac{\partial}{\partial y_1}u\right) +\i\e^{\i\theta}\frac{\partial}{\partial y_2}(\sqrt{y_2}u)\d\theta\\
&= \i\int_{0}^{2\pi} \e^{\i\theta}\sqrt{y_2} \left(
	\frac{\psi'(y_1)}{2\i(x_1+\i x_2)}
	+ \frac{\left(\frac{\partial}{\partial x_1} + \i\frac{\partial}{\partial x_2}\right)u}{2\i(x_1+\i x_2)} 
	+ \frac{\i}{\sqrt{y_2}} \frac{\partial}{\partial y_2}(\sqrt{y_2} u)
\right)\d\theta \\
&= \i\int_{0}^{2\pi} \e^{\i\theta}\sqrt{y_2} \left( 
	\frac{\psi'(y_1)}{2\i\sqrt{y_2}\e^{\i\theta}} 
	+ \frac{\left(\frac{\partial}{\partial x_1} + \i\frac{\partial}{\partial x_2}\right)u}{2\i\sqrt{y_2}\e^{\i\theta}}
	+ \frac{\i}{\sqrt{y_2}} \frac{\partial}{\partial y_2}(\sqrt{y_2} u)
\right)\d\theta	 \\
&= \int_0^{2\pi} \frac{\psi'(y_1)}{2}\d\theta 
	+ \frac{1}{2}\int_0^{2\pi} \left(\frac{\partial}{\partial x_1} +\i \frac{\partial}{\partial x_2}\right)u\d\theta
	- \int_0^{2\pi} \e^{\i\theta}\frac{\partial}{\partial y_2}(\sqrt{y_2} u) \d\theta \\
&= 2\pi\frac{\psi'(y_1)}{2}
	+  \left(\frac{\partial}{\partial y_2}\right)\int_0^{2\pi} \e^{\i\theta}\sqrt{y_2}u\d\theta
	- \left(\frac{\partial}{\partial y_2}\right) \int_0^{2\pi} \e^{\i\theta}\sqrt{y_2}u\d\theta\\
&= \pi\psi'(y_1).
\end{split}
\end{align}
Ferner ist
\begin{equation}
J(y):=J(y_1,y_2):=I(y_1,y_2)-\pi\psi(y_1),
\end{equation}
eine $\rmC^1$-Funktion. Nach Konstruktion erfüllt diese die Cauchy-Riemannschen Differentialgleichungen
\begin{equation}
\frac{\partial J}{\partial y_1} +\i\frac{\partial J}{\partial y_2} = 0
\end{equation}
 und ist somit analytisch in $y=y_1+\i y_2$. Ihr Definitionsbereich enthält alle $y=(y_1,y_2)$ mit $y_2>0$ hinreichend klein und $y_1$ nahe $y_1^0$.
 Da weiter für $y_2=0$ nach Konstruktion $I(y_1,0)=0$ gilt, folgt
\begin{align*}
 J(y_1,0)=-\pi\psi'(y_1)
\end{align*}
mit nach Voraussetzung reellem $\psi$ und $J$ kann mithilfe des Spiegelungsprinzips in die untere komplexe Halbebene analytisch fortgesetzt werden. Somit ist $\psi'(y_1)$ 
und damit auch $\psi(y_1)$ analytisch in einer Umgebung des Punktes $y_1=y_1^0$ und die Behauptung ist gezeigt.
\end{proof}

Betrachtet man nun \eqref{eq:1_lewy:gleichung} mit reellwertigem $\psi\in \rmC^\infty(\R)$, welches \textit{nicht} analytisch in $y_1=y_1^0$ ist, so liefert die Kontraposition des obigen Lemmas, dass \eqref{eq:1_lewy:gleichung} in keiner Umgebung $\Omega$ von $(0,0,y_1^0)$ eine Lösung $u\in \rmC^1(\Omega)$ besitzen kann. Lewy nutzte obiges Lemma um eine Funktion $f\in\rmC^\infty(\R)$ zu konstruieren, so dass
\begin{equation} 
   \mathscr L u = f(y_1)
\end{equation}
in {\em keinem} Gebiet $\Omega\subset\R^3$ eine Lösung in einem Hölderraum $\rmC^{1,\alpha}(\Omega)$ besitzen kann.




\section{Hörmanders Unlösbarkeitskriterium}
Hörmander zeigte in \cite{Hormander:1960a}, dass für jedes Gebiet $\Omega\subset\R^n$ eine glatte Funktion $f\in\rmC^\infty(\R^n)$ existiert, für welche keine Distribution $u\in\mathscr{D}'(\Omega)$ mit $\mathscr Lu=f$ existiert. Dazu zeigte er eine notwendige Bedingung für die Lösbarkeit eines Differentialausdrucks erster Ordnung. In \cite{Hormander:1960b} verallgemeinerte er diese noch auf Operatoren höherer Ordnung.

Zuerst zeigen wir ein Analogon zu Satz~\ref{thm:4:Umkehrung}, welches es erlaubt von Lösbarkeitsaussagen auf das Verschwinden der Poissonklammer 
von $p$ und $\overline p$ zu schließen. 

\begin{thm}[{\cite[Theorem 1]{Hormander:1960b}}]\label{thm:3.1_hoer}
Sei $\Omega\subset\mathbb{R}^n$ ein Gebiet und $P(x,\D)$ ein Differentialausdruck der Ordnung $m$ mit Hauptsymbol $p(x,\xi)$.
Angenommen, für jedes $f\in \rmC_0^\infty(\Omega)$  existiert eine distributionelle Lösung $u\in\mathscr D'(\Omega)$ zu
\begin{equation}\label{eq:3.1_hoer}
P(x,\D)u=f.
\end{equation}
Dann gilt für alle $x\in\Omega$ und $\xi\in\mathbb{R}^n$ mit $p(x,\xi)=0$
\begin{equation}\label{eq:3.1_hoer_aussage}
 \{p,\overline{p}\}(x,\xi)=0.
\end{equation}
\end{thm}
\begin{proof}
Wir zeigen dies indirekt. {\sl Schritt 1.} Angenommen, die Behauptung gilt nicht. Sei also ohne Beschränkung der Allgemeinheit $0\in\Omega$ und gelte für ein $\xi\in\R^n$ sowohl $p(0,\xi)=0$ als auch\footnote{Da $\{p,\overline p\}$ reellwertig und ungerade in $\xi$ ist, erfüllt für $\{p,\overline p\}(0,\xi)\ne0$ entweder $\xi$ oder $-\xi$ die Bedingung $\{p,\overline p\}(0,\xi)<0$.}   $\{p,\overline p\}(0,\xi)<0$. Analog zu Lemma~\ref{thm:4:lem2} folgt die Existenz einer Funktion $u\in\rmC^\infty(\Omega)$ mit 
\begin{equation}
    p(x,\nabla u(x)) = \mathcal O(|x|^q),\qquad x\to0
\end{equation}
sowie
\begin{equation}
   u(x) = \i \xi\cdot x + x\cdot Ax + \mathcal O(|x|^3),\qquad x\to0 
\end{equation}
zu vorgegebener Ordnung $q$, obigem Vektor $\xi$ und geeignet (mit Lemma~\ref{thm:4:lem3}) gewählter symmetrischer Matrix $A$ mit negativ definitem Realteil. 
$\bullet$\qquad {\it Schritt 2.} Wir definieren die Hilfsfunktionen
\begin{equation}
   f_{\tau, k}(x) = \tau^{-k} \chi(\tau x),\qquad \text{wobei $\chi\in\rmC_0^\infty(\R)$ mit}\quad \widehat \chi(-\xi) = (2\pi)^{-n/2}  \int \e^{\i x\cdot\xi} \chi(x) \d x \ne 0,
\end{equation}
sowie für $q=2(r+1)$, $r=n+k+m+N$ und dem oben konstruierten $u(x)$ die Hilfsfunktionen
\begin{equation}
   v_{\tau,k,N} (x) = \tau^{n+1+k} \e^{\tau u(x)} \sum_{j=0}^{r-1} \tau^{-j} \varphi_j(x) 
\end{equation}
mit noch zu bestimmenden Koeffizienten $\varphi_j\in\rmC_0^\infty(\Omega)$. Nach Konstruktion gilt für die Sobolevnormen
\begin{equation}\label{eq:cond1}
    \limsup_{\tau\to\infty} \| f_{k,\tau}\|_{(k)} < \infty,
\end{equation}
und für $\varphi_0(0)=1$ folgt
\begin{equation}\label{eq:cond3}
   \frac1\tau \int f_{\tau,k}(x) v_{\tau,k,N}(x)\d x = \int \e^{\tau u(\frac x\tau)} \chi(x) \sum_{j=0}^{r-1} \varphi_j(\frac x\tau) \tau^{-j} \d x \longrightarrow \int \e^{\i x\cdot\xi}\chi(x)\d x\ne0
\end{equation}
für $\tau\to\infty$. Die verbleibenden $\varphi_j$ werden so gewählt, dass 
\begin{equation}\label{eq:cond2}
   \limsup_{\tau\to\infty} \|{}^tP(x,\D) v_{\tau,k,N}(x)\|_{(N)} <\infty 
\end{equation}
gilt. Dazu nutzt man, dass 
\begin{equation}
{}^tP(x,\D) v_{\tau,k,N}(x)=\tau^{n+1+k+m}  \e^{\tau u(x)} \sum_{j=0}^m \tau^{-j} a_j(x) 
\end{equation}
mit Koeffizienten $a_j\in\rmC_0^\infty(\Omega)$ gilt und wählt $\varphi_j$ so, dass $a_j(x)=\mathcal O(|x|^{q-2j})$ (was auf eine erneute Anwendung des Satzes von Cauchy--Kowalewskaja hinausläuft). Danach zeigt eine einfache Rechnung, dass für jedes $\psi\in\rmC_0^\infty(\omega)$ mit $\psi(x)=\mathcal O(|x|^{2s})$, $x\to0$,
und $\omega\Subset\Omega$ hinreichend klein stets die Normabschätzung $\limsup_{\tau\to\infty} \|\tau^{s-N}\psi \e^{\tau u}\|_{(N)}<\infty$ gilt und \eqref{eq:cond2} folgt. 
$\bullet$\qquad {\it Schritt~3.}  Sei $B^\infty_0(\omega)=\{f\in\rmC^\infty(\Omega) \mid \supp f\subseteq\overline\omega\}$ und 
\begin{equation}\label{eq:M_N}
   M_N = \{ f\in B^\infty_0(\omega)\mid \exists_{u\in\mathscr D'(\omega)} \; \forall_{\psi\in\rmC_0^\infty(\omega)}\; |\langle u,\psi\rangle | \le N \|\psi\|_{(N)} \quad\text{und}\quad P(x,\D)u=f \}.
\end{equation}
Für jedes $u\in\mathscr D'(\Omega)$ und $\omega\Subset\Omega$ gilt $ |\langle u,\psi\rangle | \le N \|\psi\|_{(N)}$ für alle $\psi\in\rmC_0^\infty(\omega)$ und ein hinreichend großes $N$. Die Voraussetzung von Satz~\ref{thm:3.1_hoer} 
impliziert also
\begin{equation}
   \bigcup_{N=1}^\infty M_N = B^\infty_0(\omega).
\end{equation} 
Die Mengen $M_N$ sind abgeschlossen (siehe nachfolgendes Lemma \ref{lm:closed}),  konvex und symmetrisch. Weiter ist $B^\infty_0(\omega)$ metrisierbar, der Bairesche Kategoriensatz impliziert also, dass mindestens eine der Mengen $M_N$ einen inneren Punkt (wegen Symmetrie den Ursprung) besitzt. Es gibt also ein $k$ und $\epsilon>0$, so dass
\begin{equation}\label{eq:fepsInMN}
    \{ f\in B^\infty_0(\omega)\mid \|f\|_{(k)}<\epsilon\} \subset M_N
\end{equation}
für ein $N$ gilt. $\bullet$ \qquad {\em Schritt 4.} Wir zeigen, dass \eqref{eq:fepsInMN} im Widerspruch zur Existenz der Hilfsfunktionen aus Schritt 2 steht. 
Zum Einen impliziert \eqref{eq:fepsInMN} die Existenz einer Distribution $u$ mit $P(x,\D)u=f$ für $f\in B^\infty_0(\omega)$ mit $\|f\|_{(k)}<\epsilon$. Damit gilt für jede Testfunktion $v\in\rmC_0^\infty(\omega)$ die Abschätzung
\begin{equation}\label{eq:4.74}
   \bigg| \int f(x) v(x) \d x\bigg| = |\langle u, {}^t P(x,\D) v \rangle| \le N \| {}^t P(x,\D) v\|_{(N)}.
\end{equation}
Speziell mit $f=c f_{k,\tau}$ und $v=v_{k,N,\tau}$ folgt für $\tau\to\infty$ ein Widerspruch: Für $c>0$ klein genug impliziert \eqref{eq:fepsInMN} zusammen mit \eqref{eq:cond1}, dass $f\in M_N$. Also gilt \eqref{eq:4.74}. Nun ist die rechte Seite aber wegen \eqref{eq:cond2} gleichmäßig in $\tau$ beschränkt, die linke strebt mit \eqref{eq:cond3} für $\tau\to\infty$  gegen Unendlich. Widerspruch.
\end{proof}


\begin{rem}
Die im obigen Beweis verwendete Menge $B^\infty_0(\omega)$ ist gerade der Abschluss von $\rmC_0^\infty(\omega)$ in $\rmC^\infty(\Omega)$
(also auch in $\rmC^\infty(\R^n)$). Damit kann für jedes Gebiet $\Omega\subset\R^n$ den entsprechenden Raum auch als
\begin{equation}
B_0^\infty(\Omega) =\{f\in \rmC^\infty(\Omega) \mid \forall_{\alpha\in\N_0^n}\;\forall_{\epsilon >0} \;\exists_{K_{\alpha,\epsilon}\Subset\Omega} \;:\; \sup\nolimits_{x\in \Omega\setminus K_{\alpha,\epsilon}}|\D^\alpha f(x)|<\epsilon \}
\end{equation}
charakterisieren. Das nachfolgende Lemma liefert damit insbesondere die im Beweis verwendete Abgeschlossenheit der Mengen $M_N$.
\end{rem}


\begin{lem}\label{lm:closed}
Sei $\omega\Subset\Omega$ und
\begin{equation}
M_N:= \{f\in B_0^\infty(\omega)\mid \exists_{u\in\mathscr D'(\omega)}\; \forall_{\psi\in\rmC_0^\infty(\omega)}\; |\langle u,\psi\rangle | \le N \|\psi\|_{(N)} \quad\text{und}\quad P(x,\mathrm{D})u=f\quad \text{in $\omega$} \}.
\end{equation}
Dann ist $M_N\subset B_0^\infty(\omega)$ abgeschlossen.
\end{lem}

\begin{proof}
Die Menge der Distributionen $u\in\mathscr D'(\omega)$, welche die  Bedingung 
\begin{equation}\label{lewy:uglSchwartz}
|\langle u,\psi\rangle|\leq N\|\psi\|_{(N)} 
\end{equation}
für alle $\psi\in\rmC_0^\infty(\omega)$ erfüllen, ist folgenkompakt. Wir zeigen dies mit einem Diagonalfolgenargument.
Sei dazu $(u_n)_{n\in\N}$ eine Folge in $\mathscr D'(\omega)$ mit dieser Schranke. Sei weiter $(\psi_j)_{j\in\N}$ eine Folge
von $\rmC_0\infty(\omega)$ Funktionen, die in $\rmH^N_0(\omega)$ (also dem Abschluss von $\rmC_0^\infty(\omega)$ in der $\|\cdot\|_{(N)}$-Norm) dicht ist.
Diese Existiert wegen der Separabilität von $\rmH_0^N(\omega)$.

Da die Folge $\langle u_n,\psi_1\rangle$ in $\C$ beschränkt ist, existiert eine Teilfolge $(u_{n_k}^{(1)})_{k\in\mathbb{N}}$ derart, dass
$\langle u_{n_k}^{(1)},\psi_1\rangle$ konvergiert. Weiter finden wir auch eine Teilfolge $(u_{n_k}^{(2)})_{k\in\mathbb{N}}$ von $u_{n_k}^{(1)}$, so dass $\langle u_{n_k}^{(2)},\psi_2\rangle$ konvergiert, etc. Wählen wir nun die Diagonalfolge $v_k = u_{n_k}^{(k)}$, so konvergiert $\langle v_k,\psi_j\rangle$ nach Konstruktion für alle $j$. 
Also konvergiert wegen der vorausgesetzten Schranke \eqref{lewy:uglSchwartz} die Folge   $\langle v_k,\psi\rangle$  für alle $\psi\in\rmH^N_0(\omega)$ und da $\rmC_0^\infty(\omega)\subset \rmH_0^N(\omega)$ gilt in $\mathscr D'(\omega)$. Der Grenzwert erfüllt offenbar ebenfalls \eqref{lewy:uglSchwartz}.

Sei also $(f_n)_{n\in\mathbb{N}}$ eine gegen $f\in B_0^\infty(\omega)$ konvergente Folge aus $M_N$. Dann gilt  $f_n=P(x,\D)u_n$ für gewisse $u_n\in\mathscr{D}'(\omega)$ mit  $|\langle u,\psi\rangle | \le N \|\psi\|_{(N)}$ für alle $\psi\in\rmC_0^\infty(\omega)$. Auf Grund der gerade gezeigten Folgenkompaktheit existiert eine in $\mathscr D'(\omega)$  konvergente Teilfolge $(u_{n_k})_{k\in\N}$. Sei nun $u=\lim_{k\rightarrow\infty}u_{n_k}$. Dann erfüllt $u$ ebenfalls  
 \eqref{lewy:uglSchwartz} und da $P(x,\D)u_{n_k}\to P(x,\D)u$ in $\mathscr D'(\omega)$ konvergiert, folgt $P(x,\D)u=f$.
\end{proof}




\begin{thm}[{\cite[Theorem 2]{Hormander:1960b}}]\label{thm:2_hoer}
Sei $P(x,\D)$ ein Differentialausdruck der Ordnung $m$ mit  Hauptsymbol $p(x,\xi)$ und existiere zu jedem $\omega\Subset\Omega$ 
ein $x\in\omega$ und ein $\xi\in\R^n$ mit
\begin{equation}
 \{p,\overline p\}(x,\xi)\ne 0.
\end{equation}
Dann existieren Funktionen $f\in B_0^\infty(\Omega)$, so dass
\begin{equation}\label{lewy:pxDu=f}
P(x,\mathrm D)u=f
\end{equation}
in keiner der Mengen $\omega\subseteq\Omega$ eine Lösung $u\in\mathscr D'(\omega)$ besitzt. Die Menge dieser Funktionen $f$ ist von zweiter Kategorie\footnote{Eine Teilmenge $A$ eines topologischen Raumes $B$ heißt von erster Kategorie, falls eine abzählbare Menge nirgends dichter Teilmengen aus $B$ existiert, deren Vereinigung $A$ ergibt. Ist dies nicht der Fall, so heißt $A$ von zweiter Kategorie.}.
\end{thm}

\begin{proof} Der Beweis folgt {\cite[Theorem 3.2]{Hormander:1960a}}.
Sei zunächst $\omega\subseteq\Omega$ eine feste, nichtleere Menge und $M$ gegeben durch
\begin{equation}
M:=\{f\in B_0^\infty(\Omega)\mid \exists\,u\in\mathscr{D}'(\omega) : P(x,\mathrm{D})u=f\}.
\end{equation}
\textit{Schritt 1}\\
Wir wollen zeigen, dass $M$ von erster Kategorie ist. Sei hierzu $\omega_1\Subset\omega$ offen und nichtleer, d.h. insbesondere ist $\overline{\omega_1}\subset\omega$ kompakt. Dann existiert für jede Distribution $u\in\mathscr{D}'(\omega)$ ein $N\in\mathbb{N}$, so dass $u$ die Ungleichung
\begin{equation}\label{lewy:uglSchwartz}
|\langle u,\phi\rangle|\leq N\sum_{|\alpha|\leq N}\sup\limits_{x\in\omega_1} |\mathrm{D}^\alpha \phi(x)| 
\end{equation}
für alle $\phi\in \rmC_0^\infty(\omega_1)$ erfüllt ist. Die Menge $M_N$, gegeben durch
\begin{equation}
M_N:= \{f\in B_0^\infty(\Omega)\mid \exists\,u\in\mathscr D'(\omega_1): P(x,\mathrm{D})u=f\;\mathrm{in}\;\omega_1\;\wedge\; u\mathrm{\;erf"ullt\;}\eqref{lewy:uglSchwartz}\},
\end{equation}
ist abgeschlossen (vergleiche vorigies Lemma \eqref{lm:closed}), konvex und symmetrisch. \\\\
\textit{Schritt 2}\\
Wir zeigen nun per Kontraposition, dass $M_N$ keine inneren Punkte besitzt. Wenn wir in Satz \ref{thm:3.1_hoer} die Menge $\Omega$ durch $\omega_1$ ersetzen, so existiert nach Kontraposition ein $g\in\mathscr{D}(\omega_1)$ mit
\begin{equation}\label{lewy:tgM}
t\neq 0\quad\Rightarrow\quad tg\notin M_N.
\end{equation}
Angenommen, es existiere ein innerer Punkt $f$ von $M_N$, so wäre auch $f+tg\in M_N$ für $t$ klein genug. Aufgrund der Symmetrie und der Konvexität von $M_N$ ist dann aber auch (für alle $\lambda\in[0,1]$, also insbesondere für $\lambda=1/2$)
\begin{equation}
\lambda(f+tg)+(1-\lambda)(-f) = \frac{f+tg-f}{2}=\frac{tg}{2}\in M_N,
\end{equation}
was einen Widerspruch zu \eqref{lewy:tgM} liefert.\\\\
Also ist $M_N$ abgeschlossen und ohne innere Punkte, folglich also $\bigcup M_N$ nach Definition von erster Kategorie. Wegen $M\subset\bigcup M_N$ folgt also auch, dass $M$ von erster Kategorie ist.\\\\
\textit{Schritt 3}\\
Wir zeigen nun, dass \eqref{lewy:pxDu=f} tatsächlich keine Lösung in $\omega$ besitzt: Sei $(\omega_j)_{j\in\mathbb{N}}$ eine abzählbare Basis aus offenen Teilmengen von $\Omega$, wobei $\omega_i\neq\emptyset$ für alle $i\in\mathbb{N}$ ist. Sei weiter
\begin{equation}
M^{(j)}:=\{f\in B_0^\infty(\Omega)\mid \exists u\in\mathscr{D}'(\omega_j): P(x,\mathrm{D})u=f\;\mathrm{auf\;}\omega_j\},
\end{equation}
dann folgt aus \textit{Schritt 1}, dass $M^{(j)}$ und folglich auch $\bigcup M^{(j)}$ von erster Kategorie ist. Für $f\notin \bigcup M^{(j)}$ kann somit $P(x,\mathrm{D})u=f$ auf keinem $\omega_j$ gelöst werden. Da für jede beliebige, offene, nichtleere Menge $\omega\subseteq\Omega$ ein Index $j_0$ existiert, so dass $\omega_{j_0}\subset\omega$ ist, besitzt $P(x,\mathrm{D})u=f$ keine Lösung auf $\omega$.
\end{proof}



\begin{exa}
Für Lewys Beispiel 
\begin{equation}
p(x,\xi)=-i\xi_1+\xi_2-2(x_1+ix_2)\xi_3
\end{equation}
in $n=3$ Dimensionen aus dem ersten Teil dieses Kapitels erhalten wir $\{p,\overline{p}\}(x,\xi)=-8\xi_3$. Wegen
\begin{equation}
\xi_3 =1,\quad \xi_1=-2x_2,\quad \xi_2=2x_1\qquad\Rightarrow\qquad p(x,\xi)=0
\end{equation}
ist $\{p,\overline{p}\}(x,\xi)=-8\neq 0$ und \eqref{eq:3.1_hoer_aussage} gilt nicht für alle $x\in\mathbb{R}^n$ womit folglich die Voraussetzungen von Satz \ref{thm:2_hoer} erfüllt sind.
\end{exa}