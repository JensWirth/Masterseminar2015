% !TEX root = main.tex
%\chapter{Ein unlösbarer Operator}
%\cite{Lewy:1957}
%\cite{Hormander:1960b}


\section{Das Beispiel von Lewy}


Bis Mitte des 20. Jahrhunderts ging man davon aus, dass die lokale Existenz glatter Lösungen für lineare Differentialgleichungen stets gegeben sei. Das Beispiel von Hans Lewy zeigt eindrucksvoll, dass dies nicht zwingend der Fall sein muss.  In diesem Abschnitt soll eine lineare, partielle Differentialgleichung erster Ordnung in drei Variablen mit komplexwertigen $\rmC^\infty$-Koeffizienten vorgestellt werden, welche in \emph{keiner} offenen Menge eine glatte/distributionelle Lösung besitzt. Hierfür betrachtet man den durch
\begin{equation}\label{lewy:differentialausdruck}
\mathscr L := -\frac{\partial}{\partial x_1} -\i\frac{\partial}{\partial x_2} +2\i(x_1+\i x_2)\frac{\partial}{\partial y_1}
\end{equation}
definierten Differentialausdruck auf $\R^3$. Das erste überraschende Resultat von Lewy ist in folgendem Lemma enthalten:
\begin{lem}[{\cite{Lewy:1957}}]\label{thm:1_lewy}
Zu einer reellwertigen Funktion $\psi\in \rmC^1(\mathbb{R})$ besitze das Problem
\begin{equation}\label{eq:1_lewy:gleichung}
\mathscr Lu=\psi'(y_1)
\end{equation}
in einer Umgebung $\Omega\subset\R^3$ von $(0,0,y_1^0)$ eine $\rmC^1$-Lösung $u$. Dann ist $\psi$ analytisch in $y_1=y_1^0$.
\end{lem}

\begin{proof}
Wir integrieren $(\partial_1+\i\partial_2) u$ für eine Lösung $u$ von \eqref{eq:1_lewy:gleichung} über einen Kreis in der $x_1$-$x_2$ Ebene um den Punkt $(0,0,y_1^0)$. Der Radius wird dabei so klein gewählt, dass der Kreis in $\Omega$ liegt. Sei dazu
\begin{equation}
x_1^2+x_2^2=y_2=\mathrm{const},\qquad y_1=\mathrm{const},
\end{equation}
$t=\log\sqrt{y_2}=\log\sqrt{x_1^2+x_2^2}$ und $\theta$ derjenige Winkel gegeben durch
\begin{equation}
x_1+\i x_2=\sqrt{y_2} \, \e^{\i\theta} =\e^{t+\i\theta}.
\end{equation}
Dann erhält man durch einfaches Nachrechnen $\overline x \overline\partial_x = \overline\partial_t$, also
\begin{equation}\label{thm:1_lewy:proof1}
\frac{\partial}{\partial x_1} +\i\frac{\partial}{\partial x_2}=
\e^{-t+\i\theta} \left(\frac{\partial}{\partial t}
+\i\frac{\partial}{\partial \theta}\right).
\end{equation}
Diese Identität zusammen mit partieller Integration liefert
\begin{align}\label{thm:1_lewy:abl_gleich}
\begin{split}
\int_0^{2\pi} \left(\frac{\partial}{\partial x_1} +\i\frac{\partial}{\partial x_2}\right)u\d\theta 
&= \int_0^{2\pi} \e^{-t+\i\theta} \left(\frac{\partial}{\partial \,t}+\i\frac{\partial}{\partial \theta}\right)u \d\theta \\
&= \int_0^{2\pi} \e^{-t+\i\theta} \left(\frac{\partial u}{\partial \,t} +\i\frac{\partial u}{\partial \theta}\right)\d\theta \\
&= \int_0^{2\pi} \e^{-t+\i\theta} \left(\frac{\partial u}{\partial \,t} +u\right)\d\theta .
\end{split}
\end{align}
Weiter impliziert $\sqrt{y_2}=\e^t$ für jede differenzierbare Funktion $w$
\begin{align}\label{thm:1_lewy:int_gleichheit1}
\begin{split}
\frac{\partial w}{\partial\, t}+ w  = 2\sqrt{y_2} \frac{\partial}{\partial y_2} \left( \sqrt{y_2} w(\log\sqrt{y_2})\right).
\end{split}
\end{align}
Eingesetzt in das letzte Integral aus \eqref{thm:1_lewy:abl_gleich} ergibt sich
\begin{align}\label{thm:1_lewy:abl_gleich_final}
\begin{split}
\int_0^{2\pi} \left(\frac{\partial}{\partial x_1} +\i\frac{\partial}{\partial x_2}\right)u\d\theta 
= 2\left(\frac{\partial}{\partial y_2}\right)\int_0^{2\pi} \e^{\i\theta}\sqrt{y_2}u\d\theta.
\end{split}
\end{align}
Setzen wir nun 
\begin{equation}
I(y_1,y_2)=\i\int_0^{2\pi} \e^{\i\theta} \sqrt{y_2} u\d\theta,
\end{equation}
so liefert \eqref{eq:1_lewy:gleichung} und danach \eqref{thm:1_lewy:abl_gleich_final}
\begin{align}
\begin{split}
\frac{\partial I}{\partial y_1} +\i\frac{\partial I}{\partial y_2} 
&= \i\int_{0}^{2\pi} \frac{\partial}{\partial y_1}\left(\e^{\i\theta}\sqrt{y_2}u\right)+\i\frac{\partial}{\partial y_2}\left(\e^{\i\theta}\sqrt{y_2}u\right)\d\theta \\
&= \i\int_{0}^{2\pi} \e^{\i\theta}\sqrt{y_2}\left(\frac{\partial}{\partial y_1}u\right) +\i\e^{\i\theta}\frac{\partial}{\partial y_2}(\sqrt{y_2}u)\d\theta\\
&= \i\int_{0}^{2\pi} \e^{\i\theta}\sqrt{y_2} \left(
	\frac{\psi'(y_1)}{2\i(x_1+\i x_2)}
	+ \frac{\left(\frac{\partial}{\partial x_1} + \i\frac{\partial}{\partial x_2}\right)u}{2\i(x_1+\i x_2)} 
	+ \frac{\i}{\sqrt{y_2}} \frac{\partial}{\partial y_2}(\sqrt{y_2} u)
\right)\d\theta \\
&= \i\int_{0}^{2\pi} \e^{\i\theta}\sqrt{y_2} \left( 
	\frac{\psi'(y_1)}{2\i\sqrt{y_2}\e^{\i\theta}} 
	+ \frac{\left(\frac{\partial}{\partial x_1} + \i\frac{\partial}{\partial x_2}\right)u}{2\i\sqrt{y_2}\e^{\i\theta}}
	+ \frac{\i}{\sqrt{y_2}} \frac{\partial}{\partial y_2}(\sqrt{y_2} u)
\right)\d\theta	 \\
&= \int_0^{2\pi} \frac{\psi'(y_1)}{2}\d\theta 
	+ \frac{1}{2}\int_0^{2\pi} \left(\frac{\partial}{\partial x_1} +\i \frac{\partial}{\partial x_2}\right)u\d\theta
	- \int_0^{2\pi} \e^{\i\theta}\frac{\partial}{\partial y_2}(\sqrt{y_2} u) \d\theta \\
&= 2\pi\frac{\psi'(y_1)}{2}
	+  \left(\frac{\partial}{\partial y_2}\right)\int_0^{2\pi} \e^{\i\theta}\sqrt{y_2}u\d\theta
	- \left(\frac{\partial}{\partial y_2}\right) \int_0^{2\pi} \e^{\i\theta}\sqrt{y_2}u\d\theta\\
&= \pi\psi'(y_1).
\end{split}
\end{align}
Ferner ist
\begin{equation}
J(y):=J(y_1,y_2):=I(y_1,y_2)-\pi\psi(y_1),
\end{equation}
eine $\rmC^1$-Funktion. Nach Konstruktion erfüllt diese die Cauchy-Riemannschen Differentialgleichungen
\begin{equation}
\frac{\partial J}{\partial y_1} +\i\frac{\partial J}{\partial y_2} = 0
\end{equation}
 und ist somit analytisch in $y=y_1+\i y_2$. Ihr Definitionsbereich enthält alle $y=(y_1,y_2)$ mit $y_2>0$ hinreichend klein und $y_1$ nahe $y_1^0$.
 Da weiter für $y_2=0$ nach Konstruktion $I(y_1,0)=0$ gilt, folgt
\begin{align*}
 J(y_1,0)=-\pi\psi'(y_1)
\end{align*}
mit nach Voraussetzung reellem $\psi$ und $J$ kann mithilfe des Spiegelungsprinzips in die untere komplexe Halbebene analytisch fortgesetzt werden. Somit ist $\psi'(y_1)$ 
und damit auch $\psi(y_1)$ analytisch in einer Umgebung des Punktes $y_1=y_1^0$ und die Behauptung ist gezeigt.
\end{proof}

Betrachtet man nun \eqref{eq:1_lewy:gleichung} mit reellwertigem $\psi\in \rmC^\infty(\R)$, welches \textit{nicht} analytisch in $y_1=y_1^0$ ist, so liefert die Kontraposition des obigen Lemmas, dass \eqref{eq:1_lewy:gleichung} in keiner Umgebung $\Omega$ von $(0,0,y_1^0)$ eine Lösung $u\in \rmC^1(\Omega)$ besitzen kann. Lewy nutzte obiges Lemma um eine Funktion $f\in\rmC^\infty(\R)$ zu konstruieren, so dass
\begin{equation} 
   \mathscr L u = f(y_1)
\end{equation}
in {\em keinem} Gebiet $\Omega\subset\R^3$ eine Lösung in einem Hölderraum $\rmC^{1,\alpha}(\Omega)$ besitzen kann.




\section{Hörmander's Unlösbarkeitskriterium}
Hörmander zeigte in \cite{Hormander:1960a}, dass für jedes Gebiet $\Omega\subset\R^n$ eine glatte Funktion $f\in\rmC^\infty(\R^n)$ existiert, für welche keine Distribution $u\in\mathscr{D}'(\Omega)$ mit $\mathscr Lu=f$ existiert. Dazu zeigte er eine notwendige Bedingung für die Lösbarkeit eines Differentialausdrucks erster Ordnung. In \cite{Hormander:1960b} verallgemeinerte er diese noch auf Operatoren höherer Ordnung.

Zunächst benötigen wir ein Analogon zu Satz~\ref{thm:4:Umkehrung}, welches es erlaubt von Lösbarkeitsaussagen auf das Verschwinden der Poissonklammer 
von $p$ und $\overline p$ zu schließen. 

\begin{thm}[{\cite[Theorem 1]{Hormander:1960b}}]\label{thm:3.1_hoer}
Sei $\Omega\subset\mathbb{R}^n$ ein Gebiet und $P(x,\D)$ ein Differentialausdruck der Ordnung $m$ mit Hauptsymbol $p(x,\xi)$.
Angenommen, für jedes $f\in \rmC_0^\infty(\Omega)$  existiert eine distributionelle Lösung $u\in\mathscr D'(\Omega)$ zu
\begin{equation}\label{eq:3.1_hoer}
P(x,\D)u=f.
\end{equation}
Dann gilt für alle $x\in\Omega$ und $\xi\in\mathbb{R}^n$ mit $p(x,\xi)=0$
\begin{equation}\label{eq:3.1_hoer_aussage}
 \{p,\overline{p}\}(x,\xi)=0.
\end{equation}
\end{thm}
\begin{proof}[Beweisskizze] 
Der Beweis beruht wiederum auf der Konstruktion geeigneter Testfunktionen zum Einsetzen in den Operator, allerdings kombiniert mit einer Anwendung des Satzes von Baire. {\bf TODO}
\end{proof}




Zur Formulierung des nächsten Satzes benötigen wir noch folgende Definition.

\begin{df}
Sei $\Omega\subseteq\mathbb{R}^n$ ein Gebiet. Wir bezeichnen mit $B_0^\infty(\Omega)$ den Raum 
\begin{equation}
B_0^\infty(\Omega) :=\{f\in \rmC^\infty(\Omega) \mid \forall_{\alpha\in\N_0^n}\;\forall_{\epsilon >0} \;\exists_{K_{\alpha,\epsilon}\Subset\Omega} \;:\; \sup\nolimits_{x\in \Omega\setminus K_{\alpha,\epsilon}}|\D^\alpha f(x)|<\epsilon \}
\end{equation}
versehen mit der durch die Seminormen $p_\alpha(f)=\sup_{x\in\Omega} |\D^\alpha f|$ erzeugten lokalkonvexen Struktur.
\end{df}



\begin{thm}[{\cite[Theorem 2]{Hormander:1960b}}]\label{thm:2_hoer}
Sei $P(x,\D)$ ein Differentialausdruck der Ordnung $m$ mit  Hauptsymbol $p(x,\xi)$ und existiere zu jedem $\omega\Subset\Omega$ 
ein $x\in\omega$ und ein $\xi\in\R^n$ mit
\begin{equation}
 \{p,\overline p\}(x,\xi)\ne 0.
\end{equation}
Dann existieren Funktionen $f\in B_0^\infty(\Omega)$, so dass
\begin{equation}\label{lewy:pxDu=f}
P(x,\mathrm D)u=f
\end{equation}
in keiner der Mengen $\omega\subseteq\Omega$ eine Lösung $u\in\mathscr D'(\omega)$ besitzt. Die Menge dieser Funktionen $f$ ist von zweiter Kategorie\footnote{Eine Teilmenge $A$ eines topologischen Raumes $B$ heißt von erster Kategorie, falls eine abzählbare Menge nirgends dichter Teilmengen aus $B$ existiert, deren Vereinigung $A$ ergibt. Ist dies nicht der Fall, so heißt $A$ von zweiter Kategorie.}.
\end{thm}

\begin{proof} Der Beweis folgt {\cite[Theorem 3.2]{Hormander:1960a}}.
Sei zunächst $\omega\subseteq\Omega$ eine feste, nichtleere Menge und $M$ gegeben durch
\begin{equation}
M:=\{f\in B_0^\infty(\Omega)\mid \exists\,u\in\mathscr{D}'(\omega) : P(x,\mathrm{D})u=f\}.
\end{equation}
\textit{Schritt 1}\\
Wir wollen zeigen, dass $M$ von erster Kategorie ist. Sei hierzu $\omega_1\Subset\omega$ offen und nichtleer, d.h. insbesondere ist $\overline{\omega_1}\subset\omega$ kompakt. Dann existiert für jede Distribution $u\in\mathscr{D}'(\omega)$ ein $N\in\mathbb{N}$, so dass $u$ die Ungleichung
\begin{equation}\label{lewy:uglSchwartz}
|\langle u,\phi\rangle|\leq N\sum_{|\alpha|\leq N}\sup\limits_{x\in\omega_1} |\mathrm{D}^\alpha \phi(x)| 
\end{equation}
für alle $\phi\in \rmC_0^\infty(\omega_1)$ erfüllt ist. Die Menge $M_N$, gegeben durch
\begin{equation}
M_N:= \{f\in B_0^\infty(\Omega)\mid \exists\,u\in\mathscr D'(\omega_1): P(x,\mathrm{D})u=f\;\mathrm{in}\;\omega_1\;\wedge\; u\mathrm{\;erf"ullt\;}(\ref{lewy:uglSchwartz})\},
\end{equation}
ist abgeschlossen, konvex und symmetrisch. Während die Konvexität (Konvexkombinationen von Lösungen sind wieder Lösungen) und die Symmetrie klar sind, bleibt noch die Abgeschlossenheit zu zeigen:\\\\
\textit{Schritt 2}\\
Bezüglich der schwachen Topologie auf $\mathscr{D}'(\omega_1)$ ist die Menge der Distributionen, welche (\ref{lewy:uglSchwartz}) erfüllen, (folgen-)kompakt: Dies folgt aus einem Diagonalfolgenargument. Zu zeigen ist, dass
\begin{equation}
\{u\in\mathscr D'(\omega)\mid \forall\,\phi\in \rmC_0^\infty(\omega_1)\quad |\langle u,\phi\rangle|\leq N \|\phi\|_N\}\subset\mathscr D'(\omega)
\end{equation}
kompakt ist. Hierbei bezeichnet $\|\,\cdot\,\|_N$ die Sobolev-Norm zum Differentationsgrad $N$. 
Aufgrund der Darstellung
\begin{equation}
\rmC_0^\infty(\omega_1)=\bigcup\limits_{\omega_2\Subset\omega_1}\bigcap\limits_N H_0^N(\omega_2)
\end{equation}
und der Separabilität der Sobolevräume $H_0^N(\omega_2)$, finden wir eine abzählbare, dichte Teilmenge $\Phi=\{\phi_1,\phi_2,\ldots\}$ in $\rmC_0^\infty(\omega_1)$.
Für eine Folge $(u_n)_{n\in\mathbb{N}}$ in $\mathscr D'(\omega)$ finden wir dann eine Teilfolge $(u_{n_k}^{(1)})_{k\in\mathbb{N}}$, so dass
$\langle u_{n_k}^{(1)},\phi_1\rangle$ als Folge auf einem Kompaktum (vgl. (\ref{lewy:uglSchwartz})) komplexer Zahlen konvergiert. Weiter finden wir auch eine Teilfolge $(u_{n_k}^{(2)})_{k\in\mathbb{N}}$ zu $\phi_2$, so dass $\langle u_{n_k}^{(2)},\phi_2\rangle$ konvergiert, etc. Wählen wir nun eine Diagonalfolge $(v_n)_{n\in\mathbb{N}}$, so konvergiert $\langle v_n,\phi_j\rangle$ für alle $j$, was jedoch gerade die obige Behauptung bezüglich der Kompaktheit beweist.\\\\
Sei also $(f_j)_{j\in\mathbb{N}}$ eine gegen $f\in B_0^\infty(\Omega)$ stark konvergente Folge in $M_N$, d.h. $f_j=Pu_j$ auf $\omega_1$ für gewisse $u_j\in\mathscr{D}'(\omega_1)$, welche (\ref{lewy:uglSchwartz}) erfüllen. Dann existiert ein schwacher Grenzwert $u=\lim_{j\rightarrow\infty}u_j$, welcher aus Gründen der Separabilität von $H_0^N(\omega)$ ebenfalls (\ref{lewy:uglSchwartz}) erfüllt. Dann ist aber wegen $Pu_j\rightarrow Pu$ in $\mathscr D'(\omega_1)$ auch $f=Pu$ in $\omega_1$ und somit auch $f\in M_N$, was die Abgeschlossenheit liefert.\\\\
\textit{Schritt 3}\\
Wir zeigen nun per Kontraposition, dass $M_N$ keine inneren Punkte besitzt. Wenn wir in Satz \ref{thm:3.1_hoer} die Menge $\Omega$ durch $\omega_1$ ersetzen, so existiert nach Kontraposition ein $g\in\mathscr{D}(\omega_1)$ mit
\begin{equation}\label{lewy:tgM}
t\neq 0\quad\Rightarrow\quad tg\notin M_N.
\end{equation}
Angenommen, es existiere ein innerer Punkt $f$ von $M_N$, so wäre auch $f+tg\in M_N$ für $t$ klein genug. Aufgrund der Symmetrie und der Konvexität von $M_N$ ist dann aber auch (für alle $\lambda\in[0,1]$, also insbesondere für $\lambda=1/2$)
\begin{equation}
\lambda(f+tg)+(1-\lambda)(-f) = \frac{f+tg-f}{2}=\frac{tg}{2}\in M_N,
\end{equation}
was einen Widerspruch zu (\ref{lewy:tgM}) liefert.\\\\
Also ist $M_N$ abgeschlossen und ohne innere Punkte, folglich also $\bigcup M_N$ nach Definition von erster Kategorie. Wegen $M\subset\bigcup M_N$ folgt also auch, dass $M$ von erster Kategorie ist.\\\\
\textit{Schritt 4}\\
Wir zeigen nun, dass (\ref{lewy:pxDu=f}) tatsächlich keine Lösung in $\omega$ besitzt: Sei $(\omega_j)_{j\in\mathbb{N}}$ eine abzählbare Basis aus offenen Teilmengen von $\Omega$, wobei $\omega_i\neq\emptyset$ für alle $i\in\mathbb{N}$ ist. Sei weiter
\begin{equation}
M^{(j)}:=\{f\in B_0^\infty(\Omega)\mid \exists u\in\mathscr{D}'(\omega_j): P(x,\mathrm{D})u=f\;\mathrm{auf\;}\omega_j\},
\end{equation}
dann folgt aus \textit{Schritt 1}, dass $M^{(j)}$ und folglich auch $\bigcup M^{(j)}$ von erster Kategorie ist. Für $f\notin \bigcup M^{(j)}$ kann somit $P(x,\mathrm{D})u=f$ auf keinem $\omega_j$ gelöst werden. Da für jede beliebige, offene, nichtleere Menge $\omega\subseteq\Omega$ ein Index $j_0$ existiert, so dass $\omega_{j_0}\subset\omega$ ist, besitzt $P(x,\mathrm{D})u=f$ keine Lösung auf $\omega$.
\end{proof}

\begin{exa}
Für Lewys Beispiel 
\begin{equation}
p(x,\xi)=-i\xi_1+\xi_2-2(x_1+ix_2)\xi_3
\end{equation}
in $n=3$ Dimensionen aus dem ersten Teil dieses Kapitels erhalten wir $\{p,\overline{p}\}(x,\xi)=-8\xi_3$. Wegen
\begin{equation}
\xi_3 =1,\quad \xi_1=-2x_2,\quad \xi_2=2x_1\qquad\Rightarrow\qquad p(x,\xi)=0
\end{equation}
ist $\{p,\overline{p}\}(x,\xi)=-8\neq 0$ und \eqref{eq:3.1_hoer_aussage} gilt nicht für alle $x\in\mathbb{R}^n$ womit folglich die Voraussetzungen von Satz \ref{thm:2_hoer} erfüllt sind.
\end{exa}