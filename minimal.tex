% !TEX root = main.tex
\chapter{Minimale Operatoren}

Ein erstes Ziel in dieses Kapitels ist es, den Vergleich der Stärke (Definition \ref{df:1:1.2})
zweier Differentialausdrücke mit konstanten Koeffizienten auf einen Vergleich ihrer Symbole zurückzuführen.

\section{Die Stärke von Differentialausdrücken mit konstanten Koeffizienten}

Seien $\mcP=P(\D)$ und $\mcQ=Q(\D)$ zwei Differentialausdrücke mit konstanten Koeffizienten und zugehörigen Symbolen $P(\xi)$ und $Q(\xi)$. Wir nehmen zuerst an, es gibt eine Konstante $C>0$ mit $|Q(\xi)|\le C|P(\xi)|$ für alle $\xi\in\R^n$. Dann ist $\mcQ$ schwächer als $\mcP$, da mit der Formel von Plancherel 
\begin{equation}
    \|Q(\D) u\|^2 = \int |Q(\xi)\widehat u(\xi)|^2 \d\xi \le C^2 \int |P(\xi)\widehat u(\xi)|^2 \d\xi = C^2 \|P(\D)u \|^2 
\end{equation}
für alle $u\in\rmC_0^\infty(\R^n)$ gilt. Man beachte, daß in diesem Falle jede reelle Nullstelle von $P$ auch Nullstelle von $Q$ sein muß. Das soll im folgenden verallgemeinert und die genutzte Abschätzung $C=\sup_{\xi\in\R^n} |Q(\xi)|/|P(\xi)|$ für Operatoren auf beschränkten Gebieten  durch eine Bedingung der Form
\begin{equation}\label{eq:2:1}
\sup_{\x\in\R^n}\frac{\til{Q}(\x)}{\til{P}(\x)}<\infi
\end{equation}
ersetzt werden, wobei $\til{P}$ dem Polynom $P$ durch
\begin{equation}\label{eq:2:2}
\til{P}(\z)^2:=\sum_\al\abs{P^{(\al)}(\z)}^2,\qquad \zeta=\xi+\i\eta \in\C^n
\end{equation}
zugeordnet wird und $\widetilde Q$ analog definiert sei. Dabei bezeichne 
\begin{equation}
P^{(\al)}(\x):=\partial^\alpha P(\x)
\end{equation}
die $\alpha$-te Ableitung von $P$. Es wird sich zeigen, daß die Bedingung \eqref{eq:2:1} genau dann gilt, wenn $\mcQ$ schwächer als $\mcP$ ist.

In den folgenden Betrachtungen benötigt man häufig die leibnizsche Formel
\begin{equation}\label{eq:2:leib}
\forall u,v\in \rmC^\infi(\Omega) \quad:\quad P(\D)(uv)=\sum_{\al}(P^{(\al)}(\D)v)\left(\frac{\D^\al u}{\al!}\right).
\end{equation} 
Für $P(\D)=\D^\al$ ist dies die gewöhnliche Leibnizregel und somit folgt \eqref{eq:2:leib} aus Linearitätsgründen. Wir treffen weiter folgende Konvention:
Im Falle griechischer Indizes
ist die Ableitung nach einem Multiindex $\al\in\N^n_0$ gemeint,
im Fall lateinischer Indizes oder Zahlen definieren wir
\begin{equation}
P^{(k_1,\dots,k_l)}(\x):=\pd_{k_1}\cdots\pd_{k_l}P(\x).
\end{equation}

\begin{thm}\label{thm:2:2.1}
Sei $\Om$ ein beschränktes Gebiet.
Der Differentialausdruck $\mcQ$ ist genau dann schwächer als $\mcP$,
wenn
\begin{equation}\label{eq:thm:2:2.1}
\sup_{\x\in\R^n}\frac{\til{Q}(\x)}{\til{P}(\x)}<\infi
\end{equation}
erfüllt ist.
\end{thm}

Wir zeigen zunächst, wie in beiden Richtungen abgeschätzt wird.
Für die Rückrichtung benötigen wir ein Lemma,
das wir später beweisen.

\begin{proof}
{\em Hinrichtung.}
Sei $\mcQ$ schwächer als $\mcP$.
Dann gilt für ein $C>0$ und alle $u\in\rmC_0^\infty(\Omega)$ die Ungleichung
\begin{equation}\label{eq:2:4}
\norm{\mcQ u}^2\leq C(\norm{\mcP u}^2+\norm{u}^2).
\end{equation}
Für jedes $\x\in\R^n$ und ein festes $\psi\in \rmC^\infi_0(\Om)$, $\psi\neq0$,
definieren wir $\psi_\x(x):=\e^{\i x\cdot\xi}\psi(x)$.
Dies bringt durch die Formel von Leibniz die Ableitungen von $P(\x)$ ins Spiel
\begin{equation}\label{eq:2.5}
P(\D)\psi_\x(x)=\e^{\i x\cdot\xi}\sum_\al P^{(\al)}(\x)\frac{\D^\alpha\psi(x)}{\al!},
\end{equation}
analoges gilt für $Q(\D)$.
Schreiben wir kurz
\begin{equation}\label{eq:2.6}
\Psi_{\al\be}:=\frac1{\al!\be!}\spro{\D^\alpha\psi}{\D^\beta\psi},
\end{equation}
so liest sich Ungleichung \eqref{eq:2:4} mit $u=\psi_\x$ als
\begin{equation}\label{eq:2.7}
\sum_{\al,\be}Q^{(\al)}(\x)\overline{{Q}^{(\be)}(\x)}\Psi_{\al\be}
\leq C\bigg(\sum_{\al,\be}P^{(\al)}(\x)\overline{{P}^{(\be)}(\x)}\Psi_{\al\be}+\Psi_{00}\bigg).
\end{equation}
Bei $\left(\Psi_{\al\be}\right)_{\al,\be}$ handelt es sich um die gramsche Matrix
eines Skalarproduktes auf einem $\C^M$, da für alle $t=(t_\alpha)_{|\alpha|\le m}\in\C^M\setminus\{0\}$
\begin{equation}
  \sum_{\alpha,\beta} t_\alpha \overline{t_\beta} \Psi_{\alpha\beta} = \int_\Omega \left|\sum_\alpha \frac{t_\alpha \D^\alpha\psi (x)}{\alpha! }\right|^2 \d x 
  = \int \left|\sum_\alpha \frac{t_\alpha \xi^\alpha}{\alpha!} \right|^2 |\widehat\psi(\xi)|^2 \d\xi \ne0
\end{equation}
gilt.
Dies ist gleichbedeutend damit,
dass die Vektoren $D^\al\psi\in L^2$, $\abs{\al}\leq m$, linear unabhängig sind.
Es gibt also Konstanten $\Psi_-,\Psi_+\in\R_+$, so dass
\begin{equation}\label{eq:2.8}
\abs{t}^2\Psi_-\leq\sum_{\al,\be}t_\al \overline{t_\be}\Psi_{\al\be}\leq\abs{t}^2\Psi_+.
\end{equation}
Die Ungleichungen \eqref{eq:2.7} und \eqref{eq:2.8} kombinieren für $t_\alpha=Q^{(\alpha)}(\xi)$ und damit $\widetilde Q(\xi)^2=|t|^2$ 
und unter Ausnutzung von $\widetilde P(\xi)^2\ge c^{-1}>0$ zu
\begin{equation}\label{eq:2.9}
\til{Q}(\x)^2 \Psi_- \le C  \bigg(\sum_{\al,\be}P^{(\al)}(\x)\overline{{P}^{(\be)}(\x)}\Psi_{\al\be}+\Psi_{00}\bigg) \leq  C(1+c) \til{P}(\x)^2 \Psi_+
\end{equation}
für alle $\xi\in\R^n$ und ergeben damit gerade die Ungleichung \eqref{eq:thm:2:2.1}.
%
$\bullet$\qquad {\em Rückrichtung.}
Sei die Ungleichung \eqref{eq:thm:2:2.1} erfüllt, das heißt gelte
\begin{equation}\label{eq:2.10}
\til{Q}(\x)\leq C\til{P}(\x),
\end{equation}
für alle $\xi\in\R^n$ und mit einer Konstanten $C>0$,
also auch $\abs{Q(\x)}\leq C\til{P}(\x)$.
Für $u\in\rmC_0^\infty(\Omega)$ gilt
\begin{align}
\|Q(\D) u\|^2 &= \int |Q(\xi) \widehat{u}(\xi)|^2 \d\xi
\leq C \int |\widetilde P(\xi) \widehat{u}(\xi)|^2 \d\xi \notag\\&  = C\sum_\al\int | P^{(\al)}(\xi)\widehat{u}(\xi)|^2\d\xi = C\sum_\al\norm{P^{(\al)}(\D)u}^2.\label{eq:2.11}
\end{align}
Um den Beweis abzuschließen,
fehlt also nur noch eine Ungleichung der Form
\begin{equation}
\norm{P^{(\al)}(\D)u}\leq C_{\mcP,\Om}\norm{P(\D)u}, \label{eq:2:wantedineq}
\end{equation}
mit $C_{\mcP,\Om}>0$ unabhängig von $u\in\rmC_0^\infty(\Omega)$ und $\alpha\in\N_0^n$. Diese werden wir mit der Methode der Energieintegrale herleiten (Korollar \ref{cor:2:2.5}).
\end{proof}

\begin{rem}
Der Beweis zeigt insbesondere, daß in Formel \eqref{eq:thm:2:2.1} die Funktion $\til{Q}(\x)$ durch $|Q(\x)|$ ersetzt werden kann.
Definieren wir weiterhin
\begin{equation}
W(\mcP):=\{\mcQ=Q(\D):\mcQ~\text{ist schwächer als}~\mcP\},
\end{equation}
so ist wegen $|Q_1(\x)+Q_2(\x)|^2\leq2(|Q_1(\x)|^2+|Q_2(\x)|^2)$
die Menge $W(\mcP)$ ein $\C$-Vektorraum.
Dies folgt auch im ganz allgemeinen Fall aus Formel \eqref{eq:1:1.21}.
\end{rem}

\section{Energieintegrale}

Im folgenden entwickeln wir eine Methode, um sesquilineare Differentialausdrücke abzuschätzen.
Wir betrachten dazu für $u,v\in\rmC_0^\infty(\Omega)$
\begin{equation}
F(\D,\overline{\D})u\bar{v}
:=\sum_{\al,\be}a_{\al\be}(\D^\alpha u)(\cc{\D^\beta v}),
\end{equation}
mit Koeffizienten $a_{\al\be}\in\C$. Uns interessieren insbesondere die Terme
\begin{equation}\label{eq:2:qf}
\Phi_F(u):=\int F(\D,\overline{\D})u(x)\bar{u}(x)\d x
=\int F(\x,\x)\abs{\hat{u}(\x)}^2\d \x,
\end{equation}
welche quadratische Formen $\Phi_F$ auf $\rmC^\infi_0(\Om)\subs \rmL^2(\Om)$
definieren.
Es besteht eine eins-zu-eins Korrespondenz
zwischen sesquilinearen Differentialausdrücken $F(\D,\cc{\D})$
und den Symbolen
\begin{equation}
F(\z,\cc{\z}):=\sum_{\al,\be}a_{\al\be}\z^\al\cc{\z}^\be, \z\in\C^n,
\end{equation}
analog zu den Symbolen von Differentialausdrücken.

Die quadratischen Formen $\Phi_F$
sind schon durch die Werte $F(\x,\x)$, $\x\in\R^n$ bestimmt.
Verschiedene quadratische Differentialausdrücke bzw.~Symbole
können also die gleiche quadratische Form definieren.
Folgendes Lemma zeigt, dass es eine eins-zu-eins
Korrespondenz zwischen den auf reelle Werte eingeschränkten
Symbolen $F(\x,\x)$, $\x\in\R^n$,
und den quadratischen Formen $\Phi_F$ gibt.

\begin{lem}\label{lem:2:qfsym}
Sei $\Om$ ein beliebiges Gebiet.
Dann gilt 
\begin{equation}\label{eq:lem:2:qfsym}
\forall u\in\rmC_0^\infty(\Omega) \quad:\quad \int F(\D,\overline{\D})u\bar{u}\d x=0,
\end{equation}
genau dann, wenn $F(\x,\x)=0$ für alle $\x\in\R^n$.
\end{lem}
\begin{proof}
Es ist nur die Hinrichtung zu beweisen. Gelte also \eqref{eq:lem:2:qfsym}.
Sei $u_\y(x):=u(x)\e^{\i {x}\cdot{\y}}$ für ein festes $u\in \rmC^\infi_0(\Om)$, $u\neq0$.
Dann ist
\begin{align}
\int F(\D,\overline{\D})u_\y(x)\overline{u_\y(x)}\d x
&=\int F(\x,\x)\abs{\widehat{u}(\x-\y)}^2\d \y\notag\\&
=\int F(\x+\y,\x+\y)\abs{\widehat{u}(\x)}^2\d \x=:q(\y)
\end{align}
für alle $\y\in\R^n$.
Auf der rechten Seite steht wieder ein Polynom $q(\y)$ in $\y\in\R^n$
und nach Voraussetzung müssen dessen Koeffizienten verschwinden.
Wegen
\begin{equation}
(\x+\y)^\al=\sum_{\be+\ga=\al}\begin{pmatrix}\al\\\be\end{pmatrix}\x^\be\y^\ga=\y^\al+\dots
\end{equation}
gilt für die homogenen Hauptteile $\pi_q(\x)$ bzw.~$\pi_F(\x)$
von $p(\x)$ bzw.~$F(\x,\x)$ die Gleichung $\pi_q(\x)=\norm{u}^2\pi_F(\x)$.
Also würde $F(\x,\x)\neq0$ zu einem Widerspruch führen.
\end{proof}

Unser Ziel ist es, Ableitungen von Differentialausdrücken zu kontrollieren.
Dazu betrachten wir zugeordnete Ableitungen sesquilinearer Differentialausdrücke. Wir beginnen mit
\begin{equation}
\frac{\pd}{\partial x_k}
\left[F(\D,\overline{\D})u\bar{u}\right]
=\i\left[(\D_k-\overline{\D}_k)F(\D,\overline{\D})\right]u\bar{u}.
\end{equation}
Für einen Vektor $\ul{G}=(G_k)_{k=1,\dots,n}$ sesquilinearer Differentialausdrücke $G_k$ ergibt sich damit formal als Divergenz die Formel
\begin{subequations}
\begin{equation}\label{eq:2.15a}
\div(\ul{G}(\D,\overline{\D})u\bar{u})=\sum_{k=1}^n\frac{\pd}{\partial x_k}\left[G_k(\D,\overline{\D})u\bar{u}\right]=F(\D,\overline{\D})u\bar{u},
\end{equation}
wobei
\begin{equation}\label{eq:2.15b}
F(\z,\bar{\z})=\i\sum_{k=1}^n(\z_k-\bar{\z}_k)G_k(\z,\bar{\z})=-2\sum_{k=1}^n\y_kG_k(\z,\bar{\z})
\end{equation}
\end{subequations}
für $\z=\x+\i\y$ mit $\x,\y\in\R^n$.
\begin{lem}\label{lem:2:2.2}
Ein Polynom $F(\z,\overline{\z})$ in $\z=\x+\i\y$ und $\overline{\z}=\x-\i\y$
kann genau dann in der Form \eqref{eq:2.15b} dargestellt werden,
wenn $F(\x,\x)=0$ für alle $\x\in\R^n$ gilt.

In diesem Fall gilt
\begin{equation}\label{eq:2:2.16}
G_k(\x,\x)=-\frac12\left.\frac{\partial F(\x+\i\y,\x-\i\y)}{\partial \y_k}\right\rvert_{\y=0}
\end{equation}
für alle $\x\in\R^n$.
\end{lem}
\begin{proof}
Die Hinrichtung ist offensichtlich.
Die Rückrichtung und Formel \eqref{eq:2:2.16}
folgen aus der (reellen) Taylorformel von $F(\x+\i\y,\x-\i\y)$
mit den Entwicklungspunkten $(\x,0)\in\R^{2n}$.
\end{proof}

Die Polynome $G_k(\z,\overline{\z})$ sind nicht eindeutig durch $F(\zeta,\overline\zeta)$ bestimmt.
%Doch aufgrund von Lemma \ref{lem:2:qfsym} stört das nicht.
Mit Hilfe einer speziellen quadratischen Differentialform und obiger Formel
gewinnen wir folgende Ungleichung,
mit der wir die Ableitungen der Differentialausdrücke kontrollieren können.

\begin{lem}
Sei $\Om$ ein Gebiet, so dass für ein $k\in\{1,\ldots,n\}$ der $k$-Durchmesser $B_k:=\sup_{x,y\in\Om}\abs{x_k-y_k}$ endlich ist. Dann gilt
für Differentialausdrücke $P(\D)$ und $Q(\D)$, sowie alle Funktionen $u\in\rmC_0^\infty(\Omega)$ die Ungleichung
\begin{equation}\label{eq:lem:2:2.4}
\abs{\spro{P^{(k)}(\D)u}{\overline{Q}(\D)u}}\leq\norm{P(\D)u}\left(\norm{\overline{Q}^{(k)}(\D)u}+B_k\norm{\overline{Q}(\D)u}\right).
\end{equation}
\end{lem}

\begin{proof}
Wir betrachten den quadratischen Differentialausdruck mit dem Symbol
\begin{equation}\label{eq:2:spsym}
F(\z,\bar{\z}):=P(\z)Q(\bar{\z})-Q(\z)P(\bar{\z}).
\end{equation}
Dieser erfüllt offensichtlich $F(\x,\x)=0$ für alle $\x\in\R^n$.
Nach Lemma \ref{lem:2:2.2} erhalten wir,
nachdem wir mit $-\i x_k$ multipliziert haben
\begin{equation}\label{eq:2:2.22}
-\i x_kF(\D,\overline{\D})u\bar{u}=-\i x_k\sum_{j=1}^n\frac{\pd}{\partial x_j}\left[G_j(\D,\overline{\D})u\bar{u}\right],
\end{equation}
wobei nach Formel \eqref{eq:2:2.16} das Symbol
\begin{equation}
G_k(\z,\z)=-\i\left(P^{(k)}(\z)Q(\cc{\z})-Q^{(k)}(\z)P(\cc{\z})\right)\label{eq:2:Gk},
\quad\z\in\C^n
\end{equation}
gewählt werden kann.
Man beachte bei der Berechnung von \eqref{eq:2:Gk},
dass $P(\z),Q(\z)$ holomorph
und $P(\overline{\z}),Q(\overline{\z})$ anti-holomorph sind.
Integrieren wir Gleichung \eqref{eq:2:2.22},
so liefert eine partielle Integration
in der Variablen $x_k$ auf der rechten Seite
\begin{equation}\label{eq:2:ints}
\int-\i x_kF(\D,\cc{\D})u(x)\cc{u(x)}\d x=\i\int G_k(\D,\cc{\D})u(x)\cc{u(x)}\d x.
\end{equation}
Wir setzen nun \eqref{eq:2:spsym} und \eqref{eq:2:2.22} ein,
schreiben die rechte Seite von \eqref{eq:2:ints}
in Skalarprodukte um und bringen einen Term auf die andere Seite.
Es resultiert die Gleichung
\begin{equation}
\spro{P^{(k)}(\D)u}{\overline{Q}(\D)u}
=\spro{P(\D)u}{\overline{Q}^{(k)}(\D)u}-\i\int x_k\left(P(\D)u\cc{\overline{Q}(\D)u}-Q(\D)u\cc{\overline{P}(\D)u}\right)\d x.
\end{equation}
Ohne Beschränkung der Allgemeinheit können wir das Gebiet $\Om$ so legen,
dass $\abs{x_k}\leq B_k/2$ für alle $x\in\Om$ gilt.
Also erhalten wir mit Cauchy-Schwarz
und der Dreiecksungleichung die Ungleichung \eqref{eq:lem:2:2.4}.
\end{proof}

\begin{cor}\label{cor:2:2.5}
Ist $P(\x)$ vom Grad $m$ in $\x_k$, so gilt
\begin{equation}\label{eq:cor:2:2.5}
\norm{P^{(k)}(\D)u}\leq mB_k\norm{P(\D)u}
\end{equation}
für alle $u\in\rmC_0^\infty(\Omega)$.
\end{cor}
\begin{proof}
Setzen wir $\overline{Q}(\x)=P^{(k)}(\x)$, so erhalten wir aus \eqref{eq:lem:2:2.4} die Ungleichung
\begin{equation}\label{eq:2.28}
\norm{P^{(k)}(\D)u}^2\leq\norm{P(\D)u}\left(\norm{P^{(kk)}(\D)u}+B_k\norm{P^{(k)}(\D)u}\right)
\end{equation}
für alle $u\in\rmC_0^\infty(\Omega)$. Wir gehen nun Induktiv in $m$ vor.
Für $m=1$ ist $P^{(kk)}=0$, $P^{(k)}\neq0$ und somit entspricht die Gleichung \eqref{eq:cor:2:2.5}
gerade der Gleichung \eqref{eq:lem:2:2.4} nach Kürzen eines Faktors.
Angenommen das Korollar ist für $m-1$ erfüllt,
dann erhalten wir durch Kombination von \eqref{eq:2.28} und \eqref{eq:cor:2:2.5}
\begin{equation}
\norm{P^{(k)}(\D)u}^2\leq\norm{P(\D)u}\left((m-1)B_k\norm{P^{(k)}(\D)u}+B_k\norm{P^{(k)}(\D)u}\right)
\end{equation}
und der Induktionsschritt  ist gezeigt.
\end{proof}

\begin{cor}\label{cor:2:2.6}
Für festes beschränktes Gebiet $\Om$ und beliebiges $\al\in\N^n_0$ gilt
\begin{equation}
\forall u\in\rmC_0^\infty(\Omega)\quad:\quad \norm{P^{(\al)}(\D)u}\leq C_{m,\al,\Om}\norm{P(\D)u},
\end{equation}
mit $C_{m,\al,\Om}=m(m-1)\cdots(m-\abs{\al})B^\al$, wobei $B^\al=\prod_{l=1}^n B^{\al_l}_l$.
Die Konstante $C_{m,\al,\Om}$ hängt also nur vom Grad $m$ des Polynoms $P(\x)$,
der Ableitungsordnung $\abs{\al}$ und der Ausdehnung des Gebiets $\Om$ ab.
\end{cor}

\begin{proof}
Mehrfache Anwendung von Korollar \ref{cor:2:2.5}.
\end{proof}

Das letzte Korollar ist gerade der im Beweis
von Satz \ref{thm:2:2.1} gebrauchte Baustein.
Dabei können wir $C_{\mcP,\Om}:=\max_{|\al|\le m} C_{m,\al,\Om}$
für die Konstante in \eqref{eq:2:wantedineq} wählen.

\section{Beispiele und spezielle Symbole}

Wir betrachten als instruktive Beispiele
einige klassische Differentialausdrücke zweiter Ordnung.
Zur Vereinfachung der Notation schreiben wir
\begin{align}
f(\x)\asymp g(\x)\quad&\gdw&
\exists_{C_-,C_+>0}\forall_{\x\in\R^n}\quad&:\quad C_-g(\x)\leq f(\x)\leq C_+g(\x),\\
f(\x)\apprle g(\x)\quad&\gdw&
\exists_{C>0}\forall_{\x\in\R^n}~\quad&:\quad f(\x)\leq Cg(\x).
\end{align}

\begin{exa}\label{exa:2:lap}
Wir betrachten das zum \eIndex[Differentialoperator]{Laplaceoperator} $\Lap$
korrespondierende Symbol $P(\x)=\x_1^2+\dots+\x_n^2=\abs{\x}^2$.
Das regularisierte Symbol ist gegeben durch
\begin{equation}
\til{P}(\x)^2=\abs{\x}^4+4\abs{\x}^2+4n.
\end{equation}
Beachten wir, dass stets
\begin{equation}\label{eq:2:sqrest}
a^2+b^2\leq(a+b)^2\leq2(a^2+b^2),\quad\text{für}~a,b\geq0,
\end{equation}
so sehen wir, dass
\begin{equation}
\til{P}(\x)^2\asymp\abs{\x}^4+1.
\end{equation}
Also ist der Laplaceoperator stärker als alle Differentialausdrücke von gleichem oder kleinerem Grad.
Diese sind alle von der Form
\begin{equation}
Q(\x)=\x^\tr A\x+\ska{b}{\x}+c,
\end{equation}
mit einer Matrix $A\in\C^{n\ti n}$, $b\in\C^n$ und $c\in\C$,
wobei wir $A^\tr=A$ annehmen können.
Wir untersuchen, welche davon gleich stark wie der Laplaceoperator sind.
Es ist
\begin{equation}\label{eq:2:rpQLap1}
\til{Q}(\x)=\abs{\x^\tr A\x+\ska{b}{\x}+c}^2+\sum_{k=1}^n\abs{2e^\tr_kA\x+b_k}^2
+4\sum_{1\leq k\leq l\leq n}\abs{a_{kl}}^2.
\end{equation}
Da die Ausdrücke $\til{Q}(\x)$ und $1+\abs{\x^\tr A\x}^2$
auf kompakten Mengen positiv und beschränkt sind (falls $\mcQ\neq0$) genügt es,
diese für $\abs{\x}\geq R$ mit festem $R>0$ groß genug zu betrachten.
Ist $\abs{\x^\tr A\x}\neq0$ falls $\x\neq0$,
so gilt $\abs{\x^\tr A\x}\asymp\abs{\x}^2$.
In diesem Fall kann man, für $R$ groß genug,
die linearen Terme in \eqref{eq:2:rpQLap1} vernachlässigen
und erhält insgesamt
\begin{equation}
\til{Q}(\x)\asymp\abs{\x^\tr A\x}^2+1\asymp\abs{\x}^4+1\asymp\til{P}(\x).
\end{equation}
Im anderen Fall betrachtet man \eqref{eq:2:rpQLap1}
für alle $\x\in\R^n$ mit $\x^\tr A\x=0$,
und stellt fest, dass $\til{Q}(\x)$ für diese $\x$ nur linear wächst,
$Q(\D)$ also echt schwächer als $P(\D)$ ist.

Im Fall einer reellen Matrix gilt $\x^\tr A\x\neq0$ genau dann für alle $\x\neq0$,
wenn entweder alle Eigenwerte positiv oder alle negativ sind.
Im Fall einer komplexen Matrix ist
\begin{equation}
\abs{\x^\tr A\x}^2=\abs{\x^\tr A_\Re\x}^2+\abs{\x^\tr A_\Im\x}^2,
\end{equation}
mit $A=A_\Re+\i A_\Im$, $A_\Re,A_\Im\in\R^{n\ti n}$.
Die weitere Untersuchung gestaltet sich allerdings recht aufwendig
und soll nicht weiter vertieft werden.
\end{exa}

\begin{exa}\label{exa:2:schroe}
Der Schrödingergleichung für ein freies Teilchen\index{Differentialoperator!Schrödinger-}
\begin{equation}
\i\frac{\pd}{\partial t}\psi(x,t)=-\Lap\psi(x,t),\end{equation}
lässt sich das Symbol $P(\x)=\x^2_1+\dots+\x^2_{n-1}-\x_n$ zuordnen.

Wir bestimmen alle Operatoren gleicher Stärke.
Das Symbol $Q(\x)$ ist genau dann schwächer wie $P(\x)$,
wenn
\begin{equation}\label{eq:2:subschroe}
Q(\x)^2\apprle(\x^2_1+\dots+\x^2_{n-1}-\x_n)^2+\x^2_1+\dots+\x^2_{n-1}+1.
\end{equation}
Also hat $Q(\x)$ maximalen Grad 2 in $\x_1,\dots,\x_{n-1}$,
Grad 1 in $\x_n$ und ist somit von der Form
\begin{equation}
Q(\x)=a_0+\sum^n_{k=1}a_k\x_k+\sum^n_{k,l=1}a_{kl}\x_k\x_l,
\end{equation}
mit $a_{nn}=0$.
Wir betrachten die Gleichung \eqref{eq:2:subschroe}
für spezielle Vektoren:
\begin{equation}\label{eq:2:subschroespec}
Q(\x)^2\apprle(\x^2_1+\dots+\x^2_{n-1}+1),
\quad\text{für alle}~\x\in\R^n~\text{mit}~\x_n=\x^2_1+\dots+\x^2_{n-1}.
\end{equation}
Da wir $\x_1,\dots,\x_{n-1}$ in \eqref{eq:2:subschroespec} frei wählen können folgt, dass
\begin{equation}
Q(\x)=a_0+\sum_{k=1}^{n-1}a_k\x_k+a_n(\x_n-\x^2_1-\dots-\x^2_{n-1}),
\end{equation}
mit beliebigen $a_0,a_1,\dots,a_n\in\C$.
Es ist $Q(\x)$ genau dann gleich Stark wie $P(\x)$,
wenn $a_n\neq0$.
\end{exa}

\begin{exa}\label{exa:2:heat}
Der Wärmeleitungsgleichung\index{Differentialoperator!parabolisch}
\begin{equation}
\frac{\pd}{\pd t}T(x,t)=\Lap T(x,t),
\end{equation}
entspricht das Symbol $P(\x)=\x^2_1+\dots+\x^2_{n-1}+\i\x_n$.

Ein Symbol $Q(\x)$ ist genau dann schwächer als $P(\x)$, wenn
\begin{equation}
Q(\x)^2\apprle(\x^2_1+\dots+\x^2_{n-1})^2+\x^2_1+\dots+\x^2_{n-1}+\x^2_n+1.
\end{equation}
Diese Ungleichung ist genau dann erfüllt, wenn
\begin{equation}
Q(\x)=a_0+\sum_{k=1}^na_k\x_k+\sum_{k,l=1}^{n-1}a_{kl}\x_k\x_l,
\end{equation}
mit beliebigen $a_0,a_k,a_{kl}\in\C$.
Das Symbol $Q(\x)$ ist genau dann gleich stark wie $P(\x)$,
wenn $a_n\neq0$ und $\sum_{k,l=1}^{n-1}a_{kl}\x_k\x_l\neq0$ falls $\x\notin e_n\R$
mit dem $n$-ten Standartbasisvektor $e_n$.
\end{exa}

\begin{exa}\label{exa:2:hyper}
Als letztes Beispiel betrachten wir die ultrahyperbolische Gleichung\index{Differentialoperator!(ultra)hyperbolisc}
\begin{equation}
\Lap_1u=\Lap_2u,
\end{equation}
mit $\Lap_1:=\pd_1^2+\dots+\pd_m^2$, $\Lap_2:=\pd^2_{m+1}+\dots+\pd^2_n$.
Das korrespondierende Symbol ist $P(\x)=\x^2_1+\dots+\x^2_m-(\x^2_{m+1}+\dots+\x^2_n)$.

Ein Symbol $Q(\x)$ ist genau dann gleich stark wie $P(\x)$,
wenn
\begin{equation}
Q(\x)^2\apprle(\x^2_1+\dots+\x^2_m-(\x^2_{m+1}+\dots+\x^2_n))^2+\x^2_1+\dots+\x^2_n+1.
\end{equation}
Setzen wir wie in Beispiel \ref{exa:2:schroe} $\x^2_1+\dots+\x^2_m=\x^2_{m+1}+\dots+\x^2_n$,
so erhalten wir für diese $\x$:
\begin{equation}\label{eq:2:rpQhyp}
Q(\x)^2\apprle\x^2_1+\dots+\x^2_n+1.
\end{equation}
Da wir in \eqref{eq:2:rpQhyp} noch genug Wahlfreiheit haben folgt,
dass $Q(\x)$ von der Form
\begin{equation}
Q(\x)=a_0+\sum_{k=1}^na_k\x^k+b(\x_1^2+\dots+\x^2_m-(\x^2_{m+1}+\dots+\x^2_n)),
\end{equation}
mit beliebigen $a_0,a_1,\dots,a_n,b\in\C$ sein muss
und $Q(\x)$ ist genau dann gleich stark wie $P(\x)$,
wenn $b\neq0$.
\end{exa}

Wir wollen einige Unterschiede der betrachteten Beispiele festhalten.
Vergleichen wir die Beispiele \ref{exa:2:lap}, \ref{exa:2:hyper}
mit den Beispielen \ref{exa:2:schroe}, \ref{exa:2:heat}.
In den ersten beiden Fällen hängt der Vergleich der Stärke nur von den Hauptteilen,
d.h.~den Potenzen größter Ordnung ab.
In den zweiten zwei Fällen hängt die Vergleich auch von niederen Termen ab.
Dies hängt damit zusammen, dass sich die Symbole $P(\x)$ in
\ref{exa:2:schroe} und \ref{exa:2:heat} nicht auf allen (nicht-trivialen)
Unterräumen des $\R^n$ wie die höchste im Symbol vorkommende Potenz verhalten.

Auch die Paare \ref{exa:2:lap}, \ref{exa:2:heat}
und \ref{exa:2:schroe}, \ref{exa:2:hyper}
weisen einen große Unterschied auf.
Bei den ersten beiden Operatoren hat man viel Wahlfreiheit
für gleich starke Operatoren.
Bei den zweiten zwei Operatoren hat man im Hauptteil stets
nur einen frei wählbaren Parameter.
Dies hängt damit zusammen, dass in den Symbolen $P(\x)$
in \ref{exa:2:schroe} und \ref{exa:2:hyper} Differenzterme auftreten.

\begin{df}
\begin{enumerate}
\item
Ein Differentialausdruck $P(\D)$ heißt \eIndex[Differentialoperator]{vom Haupttyp},
falls er gleich stark ist wie jeder Differentialausdruck mit gleichem Hauptteil.
\item
Ein Differentialausdruck $P(\D)$ heißt \eIndex[Differentialoperator]{elliptisch},
falls er stärker ist als jeder Differentialausdruck von kleiner oder gleichem Grad.
\end{enumerate}
\end{df}

Wie eben diskutiert sind also der Laplaceoperator
und der Operator aus Beispiel \ref{exa:2:hyper} vom Haupttyp,
die anderen beiden nicht. Der Laplaceoperator ist als einziger der angegebenen Operatoren elliptisch.
Wir geben nun Klassifikationen für diese Operatortypen.

\begin{thm}\label{thm:2:2.9}
Ein Symbol $P(\x)$ ist genau dann vom Haupttyp,
wenn die partiellen Ableitungen $p^{(k)}(\x)=\partial_k p(\xi)$
des Hauptteils $p(\x)$ für kein $\x\in\R^n\setminus\{0\}$
alle gleichzeitig Null sind.
\end{thm}

\begin{proof}
{\it Hinrichtung.} Ist $P(\x)$ vom Haupttyp und vom Grad $m>0$,
dann gilt das gleiche für $p(\x)$.
Dann ist $p(\x)$ auch stärker als $p(\x)+\x^\al$
mit $\abs{\al}=m-1$.
Nach Theorem \ref{thm:2:2.1} und der Linearität von $W(\mcP)$
folgt dann
\begin{equation}\label{eq:2:homquot}
\sup_{\x\in\R^n}\frac{(\x^2_1+\dots+\x^2_n)^{m-1}}{\sum\abs{p^{(\al)}(\x)}^2}<\infi.
\end{equation}
Angenommen es gibt ein $\x_0\in\R^n$,
so dass $p^{(i)}(\x_0)=0$ für alle $i=1,\dots,n$.
Dann gilt auch $p(\x_0)=0$, da nach Eulers Formel für homogene Polynome folgt
\begin{equation}
mp(\x)=\sum_{k=1}^n\x_kp^{(k)}(\x)
\end{equation}
Wir setzen $\x=t\x_0$, $t\in\R$, in Gleichung \eqref{eq:2:homquot},
und erhalten einen Quotienten aus Polynomen in der Variablen $t$.
Aufgrund der Homogenität gilt
\begin{equation}
p(t\x_0)=t^mp(\x_0)=0,~p^{(1)}(t\x_0)=t^{m-1}p^{(1)}(\x_0)=0,
\dots,~p^{(n)}(t\x_0)=t^{m-1}p^{(n)}(\x_0)=0,
\end{equation}
und somit hat das Polynom im Zähler echt kleineren Grad als $2(m-1)$.
Dies liefert einen Widerspruch für $t\to\infi$. $\bullet$\qquad {\it Rückrichtung.}
Nehmen wir nun umgekehrt an, $P(\x)$ erfüllt die Bedingung aus Theorem \ref{thm:2:2.9}
und $Q(\x)$ hat den gleichen Hauptteil $p(\x)$ wie $P(\x)$.
Durch Weglassen positiver Term erhalten wir 
\begin{equation}
\til{P}(\x)^2>\sum_{k=1}^n\abs{P^{(k)}(\x)}^2=\pi(\x)+r(\x).
\end{equation}
Dabei fassen wir die höchsten Terme des Ausdrucks
$\sum_{k=1}^n\abs{P^{(k)}(\x)}^2$ zu $\pi(\x)$ zusammen.
Diese sind dann gegeben durch
\begin{equation}
\pi(\x)=\sum_{k=1}^n\abs{p^{(k)}(\x)}^2,
\end{equation}
was direkt aus der binomischen Formel
und der Additivität des Grads von Polynomen folgt.
Also hat $r(\x)$ Grad echt kleiner als $2(m-1)$.
Nach Annahme ist $\pi(\x)\neq0$ falls $\x\neq0$
and somit $\abs{r(\x)}/\pi(\x)\leq\tfrac12$,
für $\abs{\x}$ groß genug.
Für diese $\x$ ist dann auch $\til{P}(\x)^2>\pi(\x)/2$.
Mit der Dreiecksungleichung erhalten wir nun
\begin{equation}
\frac{\abs{Q(\x)^2}}{\til{P}(\x)^2}
\leq\frac{\abs{P(\x)}^2}{\til{P}(\x)^2}+\frac{\abs{Q(\x)-P(\x)}^2}{\til{P}(\x)^2}
\leq1+2\frac{\abs{Q(\x)-P(\x)}^2}{\pi(\x)},
\end{equation}
für $\abs{\x}$ groß genug.
Das Polynom im Zähler des zweiten Terms hat maximal den Grad $2(m-1)$.
Insgesamt ist damit $\sup_{\x\in\R^n}\abs{Q(\x)}/\til{P}(\x)<\infi$,
also $Q(\D)$ schwächer als $P(\D)$.
\end{proof}

Der Satz und sein Beweis bestätigen also die Anmerkungen nach den Beispielen.
Elliptische Operatoren lassen ähnlich sich klassifizieren.

\begin{thm}
Ein Differentialausdruck $P(\D)$ ist genau dann elliptisch,
wenn der homogene Hauptteil $p(\xi)$ von $P(\xi)$ für alle $\xi\in\R^n\setminus\{0\}$ die Bedingung $p(\xi)\ne0$ erfüllt.
\end{thm}


Zum Abschluss betrachten wir Produktoperatoren auf Produkträumen.
Dies sind Differentialoperatoren mit Symbolen von der Form
\begin{equation}\label{eq:2:prodpol}
P(\x)=P_a(\x_a)P_b(\x_b),
\end{equation}
mit $\x_a=(\x_1,\dots,\x_m)$ und $\x_b=(\x_{m+1},\dots,\x_n)$,
wobei $0<m<n$ fest gewählt ist
und $P_a(\x)$ ein Polynom in $\R^m$
und $P_b(\x)$ ein Polynom in $\R^{n-m}$ ist.
Wir erhalten:

\begin{thm}
Für Produktpolynome wie in \eqref{eq:2:prodpol}
gilt die Gleichung
\begin{equation}
W(\mcP)=\Lin\left(W(\mcP_a) W(\mcP_b)\right),
\end{equation}
$W(\mcP)$ ist also das Tensorprodukt von $W(\mcP_a)$ und $W(\mcP_b)$.
\end{thm}
\begin{proof}
Wir betrachten zunächst einen Multiindex $\alpha\in\mathbb N_0^n$ und setzen $\al_a=(\al_1,\dots,\al_m,0,\dots,0)$
und $\al_b=(0,\dots,0,\al_{m+1},\dots,\al_n)$. Dann gilt
\begin{equation}
P^{(\al)}(\x)
=\partial^\alpha\left(P_a(\x_a)P_b(\x_b)\right)
=P^{(\al_a)}_a(\x_a)P^{(\al_b)}_b(\x_b).
\end{equation} 
Also folgt, dass
\begin{equation}\label{eq:2:prodeq}
\til{P}(\x)=\sum_\al\abs{P^{(\al)}(\x)}^2
=\sum_{\al_a,\al_b}\abs{P^{(\al_a)}_a(\x_a)}^2\abs{P^{(\al_b)}_b(\x_b)}^2
=\til{P}_a(\x_a)\til{P}_b(\x_b).
\end{equation}
Dies zeigt $\Lin\left(W(\mcP_a) W(\mcP_b)\right)\subs W(\mcP)$. 

Nehmen wir nun an, dass $Q(\x)\in W(\mcP)$.
Betrachten wir $Q(\x_a,\x_{b})$ mit variablem $\x_a$ und festen $\x_{b}$,
so folgt aus Gleichung \eqref{eq:2:prodeq},
dass $Q(\x_a,\x_{b})\in W(\mcP_a)$.
Analoges gilt bei vertauschten Rollen von $a$ und $b$.
Wir wählen eine Vektoraumbasis $p_1(\x_a),\dots,p_l(\x_a)$ von $W(\mcP_a)$.
Es gilt
\begin{equation}
\sum_{k=1}^lp_k(\x_l)a_k(\x_b),
\end{equation}
mit Polynomen $a_k(\x_b)$.
Wir zeigen $a_k(\x_b)\in W(\mcP_b)$ für alle $k=1,\dots,K$.
Da die $p_1(\x_a),\dots,p_K(\x_a)$ linear unabhängig sind,
gibt es $\x_{a,1},\dots,\x_{a,K}$,
so dass $\left(p_k(a,\x_l)\right)_{k,l=1,\dots,K}$ invertierbar ist.
Das Gleichungssystem
\begin{equation}
Q(\x_{a,l},\x_b)=\sum_{k=1}^lp_k(\x_{a,l})a_k(\x_b)\quad\text{für}~k=1,\dots,K,
\end{equation}
kann also nach den $a_k(\x_b)$ aufgelöst werden kann.
Dies impliziert nach obigem $a_k(\x_b)\in W(\mcP_b)$.

\end{proof}

\section{Kompaktheitskriterien minimaler Operatoren}
Wir wollen nun Satz \ref{thm:2:2.1} benutzen um Kriterien zu stellen, so dass die Abbildung
\begin{equation}\label{eq:2:map}
\mcR_{P_0}\to\mcR_{Q_0},\quad P(D)u\mto Q(D)u
\end{equation}
nicht nur stetig, sondern auch kompakt ist, wobei wir dazu ebenfalls die regularisierten Symbole $\til{P}$, $\til{Q}$ heranziehen werden. Dazu definieren wir zunächst die Vollstetigkeit eines Operators.
\begin{df}
Seien $V,W$ zwei Banachräume und $T: V\supset \mathcal D_T \rightarrow W$ ein Operator. Der Operator $T$ heißt \eIndex[Operator]{vollstetig}, falls für jede schwach konvergente Folge $(x_n)_{n \in \mathbb{N}}$ in $V$ die Folge $(Tx_n)_{n \in \mathbb{N}}$ normkonvergent in $W$ ist.
\end{df}
Wir sagen im Folgenden $Q(D)$ ist vollstetig zu $P(D)$, wenn die Abbildung aus \eqref{eq:2:map} vollstetig ist. Setzt man zusätzlich voraus, dass es sich um einen reflexiven Raum handelt, dann impliziert die Vollstetigkeit der Abbildung auch die Kompaktheit. In unserem Fall handelt es sich bei den Bildräumen $\mcR_{P_0}$ und $\mcR_{Q_0}$ um Hilberträume, welche insbesondere reflexiv sind, weshalb hier immer von Kompaktheit die Rede sein wird.
\begin{thm}\label{Abbildung kompakt}
Der Operator $Q(D)$ ist genau dann vollstetig zu $P(D)$,
wenn folgendes gilt:
\begin{equation}\label{eq:2:tozero}
\frac{\til{Q}(\x)}{\til{P}(\x)}\to0 \quad\text{für}\quad\x\to\infi.
\end{equation}
\end{thm}
\begin{proof}
\item Wir nehmen zunächst an, dass \eqref{eq:2:tozero} erfüllt ist und betrachten eine Folge ${(u_n)}_{n \in \mathbb{N}} $ in $C^\infi_0(\Om)$ mit
\begin{equation}\label{eq:2:PDb}
\norm{P(D)u_n}\leq1 \ \ \forall n \in \mathbb{N}.
\end{equation}
Wir zeigen, dass dann eine Teilfolge ${(u_{n_k})}_{k \in \mathbb{N}}$ in $C^\infi_0(\Om)$ existiert,
so dass die Folge $\left(Q(D)u_{n_k}\right)_{k \in \mathbb{N}}$ konvergiert.
Nach Voraussetzung gilt 
\begin{equation}
\frac{\til{Q}(\x)}{\til{P}(\x)} = \frac{\left( \sum_{|\alpha| \leq k} | Q^{(\alpha)} (\x)|^2 \right)^{\frac{1}{2}} }{ \left(\sum_{|\alpha| \leq k} | P^{(\alpha)} (\x)|^2\right)^{\frac{1}{2}}} \rightarrow 0\ \ \text{für} \ \ \x \rightarrow \infty
\end{equation}
und da hierbei über alle Ableitungen summiert wird, ist der Ausdruck $|Q(\x)|/\til{P}(\x)$ beschränkt. Also existiert eine Konstante $C>0$, so dass
\begin{equation}
|Q(\x)|^2 \leq C \left(\sum_{|\alpha| \leq k} | P^{(\alpha)} (\x)|^2 \right)\label{eq:1.5}
\end{equation}
gilt. Für die Fouriertransformierte der Funktion $Q(D)u$ mit $u \in C^\infi_0(\Om)$ gilt 
\begin{equation}
\widehat{Q(D)u} = Q(\x) \hat{u}.
\end{equation}
Damit folgt nun mit dem Satz von Plancherel und der Ungleichung\eqref{eq:1.5}:
\begin{equation}
\int |Q(D)u|^2 \mathrm{d}x = \int |Q(\x)|^2 |\hat{u}|^2 \mathrm{d}\x \leq  \sum_{|\alpha| \leq k}C\int | P^{\alpha} (\x)|^2 \mathrm{d}\x =\sum_{|\alpha| \leq k} C\int |P^{\alpha}(D)u|^2 \mathrm{d}x.
\end{equation}
Satz \ref{thm:2:2.1} liefert uns eine Konstante $\widetilde{C}>0$ mit
\begin{equation}
\Vert P^{(\alpha)}(D)u \Vert \leq \widetilde{C} \Vert P(D)u \Vert, \label{eq:ableitung leq PD}
\end{equation}
woraus sich dann mit \eqref{eq:2:PDb} sofort
\begin{equation}
\norm{Q(D)u_n}\leq C' \norm{P(D)u_n} \leq C' \ \ \forall n \in \mathbb{N}, C' >0
\end{equation}
ergibt. Damit können wir zeigen, dass die Funktionen $\widehat{Q(D)}u_n = Q(\x) \hat{u}_n$ gleichgradig stetig und gleichmäßig beschränkt sind. Mit der Ungleichung von Cauchy-Schwarz gilt nämlich
\begin{align}
| Q(\x_1)\hat{u}_n(\x_1) - Q(\x_2)\hat{u}_n(\x_2)| & \leq \int_{\Omega}  \underbrace{\left| e^{i \ska{x}{\x_1}} - e^{i \ska{x}{\x_2}} \right| }_{\leq \varepsilon} | Q(D) u_n(x)| \mathrm{d}x 
\\ & \leq \varepsilon |\Omega|^{\frac{1}{2}} \Vert Q(D)u_n \Vert \leq \varepsilon \tilde{C}.
\end{align}
Also können wir nach dem Satz von Arzel\`a-Ascoli eine lokal gleichmäßig konvergente Teilfolge $Q(\x)\hat{u}_{n_k}$ auswählen, d.h. $Q(\x)\hat{u}_{n_k}$ konvergiert auf jeder kompakten Menge $K \subseteq \Omega$ gleichmäßig.
Nach Voraussetzung \eqref{eq:2:tozero} lässt sich für jedes $\varepsilon > 0$ eine kompakte Menge $K \subseteq \Omega$ finden, so dass 
\begin{equation}
\frac{\abs{Q(\x)}}{\til{P}(\x)} <\varepsilon \ \ \forall \x \in K^{\mathrm{c}}
\end{equation}
gilt. Damit folgt nun
\begin{align}
\int_{K^{\mathrm c}}\abs{Q(\x)}^2\abs{\hat{u}_{n_k}(\x)-\hat{u}_{n_l}(\x)}^2\mathrm{d}\x
&\leq\varepsilon^2\int\til{P}(\x)^2\abs{\hat{u}_{n_k}(\x)-\hat{u}_{n_l}(\x)}^2\mathrm{d}\x\\
&=\varepsilon^2\sum_{|\alpha| \leq k}\norm{P^{(\al)}(D)(u_{n_k}-u_{n_l})}^2.\label{eq:2:Kcsum}
\end{align}
und diese Summe ist nach \eqref{eq:ableitung leq PD}
und \eqref{eq:2:PDb} beschränkt.
Mit der lokalen gleichmäßigen Konvergenz gilt außerdem
\begin{equation}
\int_K\abs{Q(\x)}^2\abs{\hat{u}_{n_k}(\x)-\hat{u}_{n_l}(\x)}^2\mathrm{d}\x\longrightarrow0
\quad\text{für}~ k,l \rightarrow \infi,
\end{equation}
was die Konvergenz der Teilfolge $\left(Q(D)u_{n_k} \right)_{k \in \mathbb{N}}$ impliziert. Da die Funktionen $P(D)u$ für $u\in C^\infi_0(\Om)$ dicht in $\mcR_{P_0}$ liegen, ist $Q(D)$ damit vollstetig zu $P(D)$.

Wir nehmen nun an, dass $Q(D)$ vollstetig zu $P(D)$ ist und zeigen, dass \eqref{eq:2:tozero} erfüllt ist.
 Ist $(\x)_{n \in \mathbb{N}}$ eine Folge mit $\x_n \rightarrow \infty$, dann genügt es zu zeigen, dass $\til{Q}(\x_{n})/\til{P}(\x_{n})\to0$ gilt.
Wir nehmen dazu ein beliebiges $u\in C^\infi_0(\Om)$, $u\neq0$ und definieren die Folge $(u_n)_{n \in \mathbb{N}}$ durch
\begin{equation}
u_n(x):=u(x)\frac{\e^{\i\ska{x}{\x_n}}}{\til{P}(\x_n)}\ \ \forall x \in \Omega, n \in \mathbb{N}.
\end{equation}
Analog wie im Beweis von Satz \ref{thm:2:2.1} liefert die Produktregel
\begin{equation}
P(D)u_n(x)=\e^{\i\ska{x}{\x_n}}\sum_\al\frac{P^{(\al)}(\x_n)}{\til{P}(\x_n)}\frac{D^\al u(x)}{|\al|!}.
\end{equation}
und damit eine Konstante $C>0$ mit
\begin{equation}\label{eq:2:PDunb}
\norm{P(D)u_n}\leq C.
\end{equation}
Wir zerlegen nun
\begin{equation}\label{eq:2:kz}
\norm{Q(D)u_n-Q(D)u_m}^2=\norm{Q(D)u_n}^2+\norm{Q(D)u_m}^2-\de_{nm},
\end{equation}
wobei
\begin{equation}
\de_{nm}=2\Re\left\{\sum_{\al,\be}\frac{Q^{(\al)}(\x_n)}{\til{P}(\x_n)}\frac{\cc{Q^{(\be)}(\x_n)}}{\til{P}(\x_n)}
\frac1{\al!\be!}\int D^\al u\cc{D^\be u}\e^{\i\ska{x}{\x_n-\x_m}}\mathrm{d} x\right\}.
\end{equation}
Um $\de_{nm}\to0$ zu erreichen setzen wir voraus, dass $\x_n-\x_m\to\infi$, für $n,m\to\infi$, $n\neq m$ gilt. Ansonsten kann man ebenfalls eine entsprechende Teilfolge betrachten.
Da die Faktoren $Q^{(\alpha)}(\x_n)/\til{P}(\x_n)$ nach Satz \ref{thm:2:2.1} beschränkt und $D^\al u\cc{D^\be u}$ integrierbar sind, folgt mit dem Riemannschen Lemma $\de_{nm}\to0$ für $n,m\to\infi$, $n\neq m$.
Aufgrund der Kompaktheit finden wir eine konvergente Teilfolge $(Q(D)u_{n_k})_{k \in \mathbb{N}}$, sodass aus \eqref{eq:2:kz} und Satz \ref{thm:2:2.1}
\begin{equation}
\norm{Q(D)u_{n_k}}^2=\sum_{\al,\be}\frac{\abs{Q(\x_{n_k})}^2}{\til{P}(\x_{n_k})^2} \underbrace{\frac{1}{|\alpha|! |\beta|!}\int D_\alpha u \overline{D_\beta u} \mathrm{d}x}_{=:\Psi_{\alpha \beta}} \longrightarrow 0
\end{equation}
und damit auch 
\begin{equation}
\frac{\til{Q}(\x_{n_k})^2}{\til{P}(\x_{n_k})^2} \longrightarrow 0
\end{equation}
folgt.
\end{proof}
Wir betrachten nun den minimalen Operator $P_0$ und untersuchen die Kompaktheit der Inversen $P_0^{-1}$. Dazu benötigen wir zunächst folgende Definition:
\begin{df}
\item Sei $P(\x)$ das Symbol eines Differentialausdrucks $P(D)$. Wir sagen dass $P(\x)$ vollständig ist, wenn folgendes gilt:
\begin{align*}
\Lambda(P):=\{ \nu \in \mathbb{R}^n| \ \ \forall \xi \in \mathbb{C}^n, \forall t \in \mathbb{R}: P(\xi + t\nu) = P(\xi)  \} = \{0\}.
\end{align*}
\end{df}
\begin{lem} Sei $Q(\x)$ ein homogenes Polynom vom Grad $m \in \mathbb{N}$ und  $\nu \in \mathbb{R}^n$ mit
\begin{align*}
D^{\alpha}Q(\nu)=0 \ \forall \alpha \in \mathbb{N}_0^{n}, \ |\alpha| = m-1,
\end{align*}
dann gilt $\nu \in \Lambda(Q)$.
\end{lem}
\begin{proof}
\item Für $m=1$ gilt offensichtlich $\nu = 0 \in \Lambda(Q)$. Wir nehmen also an, dass die Behauptung für Polynome vom Grad kleiner $m>1$ erfüllt ist. Wir leiten das Polynom $Q(\xi)$ partiell nach einer Variable $\x_i, \ i \in \{1,\ldots,n\}$ ab und benutzen die Induktionsvoraussetzung, d.h. es gilt
\begin{equation}
\partial Q(\xi + t \nu)/ \partial \xi_i = \partial Q(\xi)/\partial \xi_i \ \ \forall i \in \{1,\ldots,n\}.
\end{equation}
Also ist der Ausdruck $Q(\xi + t \nu) - Q(\xi)$ unabhängig von $\xi$ und wir erhalten
\begin{equation}
Q(\xi + t \nu) - Q(\xi) = Q(t \nu).
\end{equation}
Mit $t=1$ und $\xi=\nu$ folgt
\begin{equation}
Q(\xi + t\nu) - Q(\xi) = Q(2 \nu) - Q(\nu)= Q(\nu) \Leftrightarrow 2^m Q(\nu) = 2 Q(\nu),
\end{equation}
d.h. $Q(\nu)=0$ und damit 
\begin{equation}
Q(\xi + t\nu) = Q(\xi).
\end{equation}
Also gilt $\nu \in \Lambda(Q)$.
\end{proof}
Mit diesem Lemma und Satz \ref{thm:2:2.1} können wir nun das Kompaktheitskriterium von $P_0^{-1}$ angeben.
\begin{thm}
Die Inverse des minimalen Operators $P_0$ ist genau dann kompakt, wenn das zu $P(D)$ dazugehörige Symbol $P(\xi)$ vollständig ist.
\end{thm}
\begin{proof}

\end{proof}
