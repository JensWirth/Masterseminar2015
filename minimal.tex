% !TEX root = main.tex
\chapter{Minimale Operatoren}\label{chap2}

Ein erstes Ziel dieses Kapitels ist es, den Vergleich der Stärke
zweier Differentialausdrücke mit konstanten Koeffizienten  (Definition \ref{df:1:1.2})  auf einen Vergleich ihrer Symbole zurückzuführen. 

\section{Die Stärke von Differentialausdrücken mit konstanten Koeffizienten}

Seien $\mcP=P(\D)$ und $\mcQ=Q(\D)$ zwei Differentialausdrücke mit konstanten Koeffizienten und zugehörigen Symbolen $P(\xi)$ und $Q(\xi)$. Wir nehmen zuerst an, es gibt eine Konstante $C>0$ mit $|Q(\xi)|\le C|P(\xi)|$ für alle $\xi\in\R^n$. Dann ist $\mcQ$ schwächer als $\mcP$, da mit der Formel von Plancherel 
\begin{equation}
    \|Q(\D) u\|^2 = \int |Q(\xi)\widehat u(\xi)|^2 \d\xi \le C^2 \int |P(\xi)\widehat u(\xi)|^2 \d\xi = C^2 \|P(\D)u \|^2 
\end{equation}
für alle $u\in\rmC_0^\infty(\R^n)$ gilt. Man beachte, daß in diesem Falle jede reelle Nullstelle von $P$ auch Nullstelle von $Q$ sein muß. Das soll im folgenden verallgemeinert und die genutzte Abschätzung $C=\sup_{\xi\in\R^n} |Q(\xi)|/|P(\xi)|$ für Operatoren auf beschränkten Gebieten  durch eine Bedingung der Form
\begin{equation}\label{eq:2:1}
\sup_{\xi\in\R^n}\frac{\widetilde{Q}(\xi)}{\widetilde{P}(\xi)}<\infty
\end{equation}
ersetzt werden, wobei $\widetilde{P}$ dem Polynom $P$ durch
\begin{equation}\label{eq:2:2}
\widetilde{P}(\zeta)^2=\sum_\alpha\abs{P^{(\alpha)}(\zeta)}^2,\qquad \zeta=\xi+\i\eta \in\C^n
\end{equation}
zugeordnet wird und $\widetilde Q$ analog definiert sei. Dabei bezeichne 
\begin{equation}
P^{(\alpha)}(\xi)=\partial^\alpha P(\xi)
\end{equation}
die $\alpha$-te Ableitung des Polynoms $P(\xi)$. Es wird sich zeigen, daß die Bedingung \eqref{eq:2:1} genau dann gilt, wenn $\mcQ$ schwächer als $\mcP$ ist.

In den folgenden Betrachtungen benötigt man häufig die leibnizsche Formel
\begin{equation}\label{eq:2:leib}
\forall u,v\in \rmC^\infty(\Omega) \quad:\quad P(\D)(uv)=\sum_{\alpha}(P^{(\alpha)}(\D)v)\left(\frac{\D^\alpha u}{\alpha!}\right).
\end{equation} 
Für $P(\D)=\D^\alpha$ ist dies die gewöhnliche Leibnizregel, allgemein folgt \eqref{eq:2:leib} unter Ausnutzung der Linearität. Wir treffen weiter folgende Konvention:
Im Falle griechischer Indizes
ist die Ableitung nach einem Multiindex $\alpha\in\N^n_0$ gemeint,
im Fall lateinischer Indizes oder Zahlen definieren wir
\begin{equation}
P^{(k_1,\dots,k_l)}(\xi)=\partial_{k_1}\cdots\partial_{k_l}P(\xi).
\end{equation}

\begin{thm}[{\cite[Theorem~2.1]{Hormander:1955}}]\label{thm:2:2.1}
Sei $\Omega$ ein beschränktes Gebiet.
Der Differentialausdruck $\mcQ$ ist genau dann schwächer als $\mcP$,
wenn
\begin{equation}\label{eq:thm:2:2.1}
\sup_{\xi\in\R^n}\frac{\widetilde{Q}(\xi)}{\widetilde{P}(\xi)}<\infty
\end{equation}
erfüllt ist.
\end{thm}

Wir zeigen zunächst, wie in beiden Richtungen abgeschätzt wird.
Für die Rückrichtung benötigen wir ein Lemma,
das wir später beweisen.

\begin{proof}
{\em Hinrichtung.}
Sei $\mcQ$ schwächer als $\mcP$.
Dann gilt für ein $C>0$ und alle $u\in\rmC_0^\infty(\Omega)$ die Ungleichung
\begin{equation}\label{eq:2:4}
\norm{\mcQ u}^2\leq C(\norm{\mcP u}^2+\norm{u}^2).
\end{equation}
Für jedes $\xi\in\R^n$ und ein festes $\psi\in \rmC^\infty_0(\Omega)$, $\psi\neq0$,
definieren wir $\psi_\xi(x)=\e^{\i x\cdot\xi}\psi(x)$.
Dies bringt durch die Formel von Leibniz die Ableitungen von $P(\xi)$ ins Spiel
\begin{equation}\label{eq:2.5}
P(\D)\psi_\xi(x)=\e^{\i x\cdot\xi}\sum_\alpha P^{(\alpha)}(\xi)\frac{\D^\alpha\psi(x)}{\alpha!},
\end{equation}
analoges gilt für $Q(\D)$.
Schreiben wir kurz
\begin{equation}\label{eq:2.6}
\Psi_{\alpha\beta}=\frac1{\alpha!\beta!}\spro{\D^\alpha\psi}{\D^\beta\psi},
\end{equation}
so liest sich Ungleichung \eqref{eq:2:4} mit $u=\psi_\xi$ als
\begin{equation}\label{eq:2.7}
\sum_{\alpha,\beta}Q^{(\alpha)}(\xi)\overline{{Q}^{(\beta)}(\xi)}\Psi_{\alpha\beta}
\leq C\bigg(\sum_{\alpha,\beta}P^{(\alpha)}(\xi)\overline{{P}^{(\beta)}(\xi)}\Psi_{\alpha\beta}+\Psi_{00}\bigg).
\end{equation}
Bei $\left(\Psi_{\alpha\beta}\right)_{\alpha,\beta}$ handelt es sich um die gramsche Matrix
eines Skalarproduktes auf einem $\C^M$, da für alle $t=(t_\alpha)_{|\alpha|\le m}\in\C^M\setminus\{0\}$
\begin{equation}
  \sum_{\alpha,\beta} t_\alpha \overline{t_\beta} \Psi_{\alpha\beta} = \int_\Omega \left|\sum_\alpha \frac{t_\alpha \D^\alpha\psi (x)}{\alpha! }\right|^2 \d x 
  = \int \left|\sum_\alpha \frac{t_\alpha \xi^\alpha}{\alpha!} \right|^2 |\widehat\psi(\xi)|^2 \d\xi \ne0
\end{equation}
gilt.
Dies ist gleichbedeutend damit,
dass die Vektoren $D^\alpha\psi\in L^2$, $\abs{\alpha}\leq m$, linear unabhängig sind.
Es gibt also Konstanten $\Psi_-,\Psi_+\in\R_+$, so dass
\begin{equation}\label{eq:2.8}
\abs{t}^2\Psi_-\leq\sum_{\alpha,\beta}t_\alpha \overline{t_\beta}\Psi_{\alpha\beta}\leq\abs{t}^2\Psi_+.
\end{equation}
Die Ungleichungen \eqref{eq:2.7} und \eqref{eq:2.8} kombinieren für $t_\alpha=Q^{(\alpha)}(\xi)$ und damit $\widetilde Q(\xi)^2=|t|^2$ 
und unter Ausnutzung von $\widetilde P(\xi)^2\ge c^{-1}>0$ zu
\begin{equation}\label{eq:2.9}
\widetilde{Q}(\xi)^2 \Psi_- \le C  \bigg(\sum_{\alpha,\beta}P^{(\alpha)}(\xi)\overline{{P}^{(\beta)}(\xi)}\Psi_{\alpha\beta}+\Psi_{00}\bigg) \leq  C(1+c) \widetilde{P}(\xi)^2 \Psi_+
\end{equation}
für alle $\xi\in\R^n$ und ergeben damit gerade die Ungleichung \eqref{eq:thm:2:2.1}.
%
$\bullet$\qquad {\em Rückrichtung.}
Sei die Ungleichung \eqref{eq:thm:2:2.1} erfüllt, das heißt gelte
\begin{equation}\label{eq:2.10}
\widetilde{Q}(\xi)\leq C\widetilde{P}(\xi),
\end{equation}
für alle $\xi\in\R^n$ und mit einer Konstanten $C>0$,
also auch $\abs{Q(\xi)}\leq C\widetilde{P}(\xi)$.
Für $u\in\rmC_0^\infty(\Omega)$ gilt
\begin{align}
\|Q(\D) u\|^2 &= \int |Q(\xi) \widehat{u}(\xi)|^2 \d\xi
\leq C \int |\widetilde P(\xi) \widehat{u}(\xi)|^2 \d\xi \notag\\&  = C\sum_\alpha\int | P^{(\alpha)}(\xi)\widehat{u}(\xi)|^2\d\xi = C\sum_\alpha\norm{P^{(\alpha)}(\D)u}^2.\label{eq:2.11}
\end{align}
Um den Beweis abzuschließen,
fehlt also nur noch eine Ungleichung der Form
\begin{equation}
\norm{P^{(\alpha)}(\D)u}\leq C_{\mcP,\Omega}\norm{P(\D)u}, \label{eq:2:wantedineq}
\end{equation}
mit $C_{\mcP,\Omega}>0$ unabhängig von $u\in\rmC_0^\infty(\Omega)$ und $\alpha\in\N_0^n$. Diese werden wir mit der Methode der Energieintegrale herleiten (Korollar \ref{cor:2:2.5}).
\end{proof}

\begin{rem}
Der Beweis zeigt insbesondere, daß in Formel \eqref{eq:thm:2:2.1} die Funktion $\widetilde{Q}(\xi)$ durch $|Q(\xi)|$ ersetzt werden kann.
Definieren wir weiterhin
\begin{equation}
W(\mcP)=\{\mcQ=Q(\D):\mcQ~\text{ist schwächer als}~\mcP\},
\end{equation}
so ist wegen $|Q_1(\xi)+Q_2(\xi)|^2\leq2(|Q_1(\xi)|^2+|Q_2(\xi)|^2)$
die Menge $W(\mcP)$ ein $\C$-Vektorraum.
Dies folgt auch im ganz allgemeinen Fall aus Formel \eqref{eq:1:1.21}.
\end{rem}

\section{Energieintegrale}

Im folgenden entwickeln wir eine Methode, um sesquilineare Differentialausdrücke abzuschätzen.
Wir betrachten dazu für $u,v\in\rmC_0^\infty(\Omega)$
\begin{equation}
F(\D,\overline{\D})u\bar{v}
=\sum_{\alpha,\beta}a_{\alpha\beta}(\D^\alpha u)(\cc{\D^\beta v}),
\end{equation}
mit Koeffizienten $a_{\alpha\beta}\in\C$. Uns interessieren insbesondere die Terme
\begin{equation}\label{eq:2:qf}
\Phi_F(u)=\int F(\D,\overline{\D})u(x)\bar{u}(x)\d x
=\int F(\xi,\xi)\abs{\hat{u}(\xi)}^2\d \xi,
\end{equation}
welche quadratische Formen $\Phi_F$ auf $\rmC^\infty_0(\Omega)\subseteq \rmL^2(\Omega)$
definieren.
Es besteht eine eins-zu-eins Korrespondenz
zwischen sesquilinearen Differentialausdrücken $F(\D,\cc{\D})$
und den Symbolen
\begin{equation}
F(\zeta,\cc{\zeta})=\sum_{\alpha,\beta}a_{\alpha\beta}\zeta^\alpha\cc{\zeta}^\beta, \zeta\in\C^n,
\end{equation}
analog zu den Symbolen von Differentialausdrücken.

Die quadratischen Formen $\Phi_F$
sind schon durch die Werte $F(\xi,\xi)$, $\xi\in\R^n$ bestimmt.
Verschiedene quadratische Differentialausdrücke bzw.~Symbole
können also die gleiche quadratische Form definieren.
Folgendes Lemma zeigt, dass es eine eins-zu-eins
Korrespondenz zwischen den auf reelle Werte eingeschränkten
Symbolen $F(\xi,\xi)$, $\xi\in\R^n$,
und den quadratischen Formen $\Phi_F$ gibt.

\begin{lem}[{\cite[Lemma~2.5]{Hormander:1955}}]\label{lem:2:qfsym}
Sei $\Omega$ ein beliebiges Gebiet.
Dann gilt 
\begin{equation}\label{eq:lem:2:qfsym}
\forall u\in\rmC_0^\infty(\Omega) \quad:\quad \int F(\D,\overline{\D})u\bar{u}\d x=0,
\end{equation}
genau dann, wenn $F(\xi,\xi)=0$ für alle $\xi\in\R^n$.
\end{lem}
\begin{proof}
Es ist nur die Hinrichtung zu beweisen. Gelte also \eqref{eq:lem:2:qfsym}.
Sei $u_\eta(x)=u(x)\e^{\i {x}\cdot{\eta}}$ für ein festes $u\in \rmC^\infty_0(\Omega)$, $u\neq0$.
Dann ist
\begin{align}
\int F(\D,\overline{\D})u_\eta(x)\overline{u_\eta(x)}\d x
&=\int F(\xi,\xi)\abs{\widehat{u}(\xi-\eta)}^2\d \eta\notag\\&
=\int F(\xi+\eta,\xi+\eta)\abs{\widehat{u}(\xi)}^2\d \xi=q(\eta)
\end{align}
für alle $\eta\in\R^n$.
Auf der rechten Seite steht wieder ein Polynom $q(\eta)$ in $\eta\in\R^n$
und nach Voraussetzung müssen dessen Koeffizienten verschwinden.
Wegen
\begin{equation}
(\xi+\eta)^\alpha=\sum_{\beta+\gamma=\alpha}\begin{pmatrix}\alpha\\\beta\end{pmatrix}\xi^\beta\eta^\gamma=\eta^\alpha+\dots
\end{equation}
gilt für die homogenen Hauptteile $\pi_q(\xi)$ bzw.~$\pi_F(\xi)$
von $p(\xi)$ bzw.~$F(\xi,\xi)$ die Gleichung $\pi_q(\xi)=\norm{u}^2\pi_F(\xi)$.
Also würde $F(\xi,\xi)\neq0$ zu einem Widerspruch führen.
\end{proof}

Unser Ziel ist es, Ableitungen von Differentialausdrücken zu kontrollieren.
Dazu betrachten wir zugeordnete Ableitungen sesquilinearer Differentialausdrücke. Wir beginnen mit
\begin{equation}
\frac{\partial}{\partial x_k}
\left[F(\D,\overline{\D})u\bar{u}\right]
=\i\left[(\D_k-\overline{\D}_k)F(\D,\overline{\D})\right]u\bar{u}.
\end{equation}
Für einen Vektor $\ul{G}=(G_k)_{k=1,\dots,n}$ sesquilinearer Differentialausdrücke $G_k$ ergibt sich damit formal als Divergenz die Formel
\begin{subequations}
\begin{equation}\label{eq:2.15a}
\div(\ul{G}(\D,\overline{\D})u\bar{u})=\sum_{k=1}^n\frac{\partial}{\partial x_k}\left[G_k(\D,\overline{\D})u\bar{u}\right]=F(\D,\overline{\D})u\bar{u},
\end{equation}
wobei
\begin{equation}\label{eq:2.15b}
F(\zeta,\bar{\zeta})=\i\sum_{k=1}^n(\zeta_k-\bar{\zeta}_k)G_k(\zeta,\bar{\zeta})=-2\sum_{k=1}^n\eta_kG_k(\zeta,\bar{\zeta})
\end{equation}
\end{subequations}
für $\zeta=\xi+\i\eta$ mit $\xi,\eta\in\R^n$.
\begin{lem}[{\cite[Lemma~2.2]{Hormander:1955}}]\label{lem:2:2.2}
Ein Polynom $F(\zeta,\overline{\zeta})$ in $\zeta=\xi+\i\eta$ und $\overline{\zeta}=\xi-\i\eta$
kann genau dann in der Form \eqref{eq:2.15b} dargestellt werden,
wenn $F(\xi,\xi)=0$ für alle $\xi\in\R^n$ gilt.

In diesem Fall gilt
\begin{equation}\label{eq:2:2.16}
G_k(\xi,\xi)=-\frac12\left.\frac{\partial F(\xi+\i\eta,\xi-\i\eta)}{\partial \eta_k}\right\rvert_{\eta=0}
\end{equation}
für alle $\xi\in\R^n$.
\end{lem}
\begin{proof}
Die Hinrichtung ist offensichtlich.
Die Rückrichtung und Formel \eqref{eq:2:2.16}
folgen aus der (reellen) Taylorformel von $F(\xi+\i\eta,\xi-\i\eta)$
mit den Entwicklungspunkten $(\xi,0)\in\R^{2n}$.
\end{proof}

Die Polynome $G_k(\zeta,\overline{\zeta})$ sind nicht eindeutig durch $F(\zeta,\overline\zeta)$ bestimmt.
Mit Hilfe einer speziellen quadratischen Differentialform und obiger Formel
gewinnen wir folgende Ungleichung,
mit der wir die Ableitungen der Differentialausdrücke kontrollieren können.

\begin{lem}
Sei $\Omega$ ein Gebiet, so dass für ein $k\in\{1,\ldots,n\}$ der $k$-Durchmesser $B_k=\sup_{x,y\in\Omega}\abs{x_k-y_k}$ endlich ist. Dann gilt
für Differentialausdrücke $P(\D)$ und $Q(\D)$, sowie alle Funktionen $u\in\rmC_0^\infty(\Omega)$ die Ungleichung
\begin{equation}\label{eq:lem:2:2.4}
\abs{\spro{P^{(k)}(\D)u}{\overline{Q}(\D)u}}\leq\norm{P(\D)u}\left(\norm{\overline{Q}^{(k)}(\D)u}+B_k\norm{\overline{Q}(\D)u}\right).
\end{equation}
\end{lem}

\begin{proof}
Wir betrachten den quadratischen Differentialausdruck mit dem Symbol
\begin{equation}\label{eq:2:spsym}
F(\zeta,\bar{\zeta})=P(\zeta)Q(\bar{\zeta})-Q(\zeta)P(\bar{\zeta}).
\end{equation}
Dieser erfüllt offensichtlich $F(\xi,\xi)=0$ für alle $\xi\in\R^n$.
Nach Lemma \ref{lem:2:2.2} erhalten wir,
nachdem wir mit $-\i x_k$ multipliziert haben
\begin{equation}\label{eq:2:2.22}
-\i x_kF(\D,\overline{\D})u\bar{u}=-\i x_k\sum_{j=1}^n\frac{\partial}{\partial x_j}\left[G_j(\D,\overline{\D})u\bar{u}\right],
\end{equation}
wobei nach Formel \eqref{eq:2:2.16} das Symbol
\begin{equation}
G_k(\zeta,\zeta)=-\i\left(P^{(k)}(\zeta)Q(\cc{\zeta})-Q^{(k)}(\zeta)P(\cc{\zeta})\right)\label{eq:2:Gk},
\quad\zeta\in\C^n
\end{equation}
gewählt werden kann.
Man beachte bei der Berechnung von \eqref{eq:2:Gk},
dass $P(\zeta),Q(\zeta)$ holomorph
und $P(\overline{\zeta}),Q(\overline{\zeta})$ anti-holomorph sind.
Integrieren wir Gleichung \eqref{eq:2:2.22},
so liefert eine partielle Integration
in der Variablen $x_k$ auf der rechten Seite
\begin{equation}\label{eq:2:ints}
\int-\i x_kF(\D,\cc{\D})u(x)\cc{u(x)}\d x=\i\int G_k(\D,\cc{\D})u(x)\cc{u(x)}\d x.
\end{equation}
Wir setzen nun \eqref{eq:2:spsym} und \eqref{eq:2:2.22} ein,
schreiben die rechte Seite von \eqref{eq:2:ints}
in Skalarprodukte um und bringen einen Term auf die andere Seite.
Es resultiert die Gleichung
\begin{equation}
\spro{P^{(k)}(\D)u}{\overline{Q}(\D)u}
=\spro{P(\D)u}{\overline{Q}^{(k)}(\D)u}-\i\int x_k\left(P(\D)u\cc{\overline{Q}(\D)u}-Q(\D)u\cc{\overline{P}(\D)u}\right)\d x.
\end{equation}
Ohne Beschränkung der Allgemeinheit können wir das Gebiet $\Omega$ so legen,
dass $\abs{x_k}\leq B_k/2$ für alle $x\in\Omega$ gilt.
Also erhalten wir mit Cauchy-Schwarz
und der Dreiecksungleichung die Ungleichung \eqref{eq:lem:2:2.4}.
\end{proof}

\begin{cor}[{\cite[Lemma~2.7]{Hormander:1955}}]\label{cor:2:2.5}
Ist $P(\xi)$ vom Grad $m$ in $\xi_k$, so gilt
\begin{equation}\label{eq:cor:2:2.5}
\norm{P^{(k)}(\D)u}\leq mB_k\norm{P(\D)u}
\end{equation}
für alle $u\in\rmC_0^\infty(\Omega)$.
\end{cor}
\begin{proof}
Setzen wir $\overline{Q}(\xi)=P^{(k)}(\xi)$, so erhalten wir aus \eqref{eq:lem:2:2.4} die Ungleichung
\begin{equation}\label{eq:2.28}
\norm{P^{(k)}(\D)u}^2\leq\norm{P(\D)u}\left(\norm{P^{(kk)}(\D)u}+B_k\norm{P^{(k)}(\D)u}\right)
\end{equation}
für alle $u\in\rmC_0^\infty(\Omega)$. Wir gehen nun Induktiv in $m$ vor.
Für $m=1$ ist $P^{(kk)}=0$, $P^{(k)}\neq0$ und somit entspricht die Gleichung \eqref{eq:cor:2:2.5}
gerade der Gleichung \eqref{eq:lem:2:2.4} nach Kürzen eines Faktors.
Angenommen das Korollar ist für $m-1$ erfüllt,
dann erhalten wir durch Kombination von \eqref{eq:2.28} und \eqref{eq:cor:2:2.5}
\begin{equation}
\norm{P^{(k)}(\D)u}^2\leq\norm{P(\D)u}\left((m-1)B_k\norm{P^{(k)}(\D)u}+B_k\norm{P^{(k)}(\D)u}\right)
\end{equation}
und der Induktionsschritt  ist gezeigt.
\end{proof}

\begin{cor}[{\cite[Lemma~2.8]{Hormander:1955}}]\label{cor:2:2.6}
Für festes beschränktes Gebiet $\Omega$ und beliebiges $\alpha\in\N^n_0$ gilt
\begin{equation}
\forall u\in\rmC_0^\infty(\Omega)\quad:\quad \norm{P^{(\alpha)}(\D)u}\leq C_{m,\alpha,\Omega}\norm{P(\D)u},
\end{equation}
mit $C_{m,\alpha,\Omega}=m(m-1)\cdots(m-\abs{\alpha})B^\alpha$, wobei $B^\alpha=\prod_{l=1}^n B^{\alpha_l}_l$.
Die Konstante $C_{m,\alpha,\Omega}$ hängt also nur vom Grad $m$ des Polynoms $P(\xi)$,
der Ableitungsordnung $\abs{\alpha}$ und der Ausdehnung des Gebiets $\Omega$ ab.
\end{cor}

\begin{proof}
Mehrfache Anwendung von Korollar \ref{cor:2:2.5}.
\end{proof}

Das letzte Korollar ist gerade der im Beweis
von Satz \ref{thm:2:2.1} gebrauchte Baustein.
Dabei können wir $C_{\mcP,\Omega}=\max_{|\alpha|\le m} C_{m,\alpha,\Omega}$
für die Konstante in \eqref{eq:2:wantedineq} wählen.

\section{Beispiele und spezielle Symbole}

Wir betrachten als instruktive Beispiele
einige klassische Differentialausdrücke zweiter Ordnung.
Zur Vereinfachung der Notation schreiben wir
\begin{subequations}
\begin{align}
f(\xi)\asymp g(\xi)\quad&\gdw&
\exists_{C_-,C_+>0}\forall_{\xi\in\R^n}\quad&:\quad C_-g(\xi)\leq f(\xi)\leq C_+g(\xi),\\
f(\xi)\apprle g(\xi)\quad&\gdw&
\exists_{C>0}\forall_{\xi\in\R^n}~\quad&:\quad f(\xi)\leq Cg(\xi).
\end{align}
\end{subequations}

Die Vergleichseigenschaften der Differentialoperatoren
beziehen sich stets auf das Verhalten der Lösungen $u$ der Gleichung
\begin{equation}
P(D)u=f
\end{equation}
wobei $f\in C_0(\R^n)$.

\begin{exa}\label{exa:2:lap}
Wir betrachten das zum \eIndex[Differentialoperator]{Laplaceoperator} $\Delta$
korrespondierende Symbol $P(\xi)=\xi_1^2+\dots+\xi_n^2=\abs{\xi}^2$.
Das regularisierte Symbol ist gegeben durch
\begin{equation}
\widetilde{P}(\xi)^2=\abs{\xi}^4+4\abs{\xi}^2+4n.
\end{equation}
Beachten wir, dass stets
\begin{equation}\label{eq:2:sqrest}
a^2+b^2\leq(a+b)^2\leq2(a^2+b^2),\quad\text{für}~a,b\geq0,
\end{equation}
so sehen wir, dass
\begin{equation}
\widetilde{P}(\xi)^2\asymp\abs{\xi}^4+1.
\end{equation}
Also ist der Laplaceoperator stärker als alle Differentialausdrücke von gleichem oder kleinerem Grad.
Diese sind alle von der Form
\begin{equation}
Q(\xi)=\xi^\top A\xi+\ska{b}{\xi}+c,
\end{equation}
mit einer Matrix $A\in\C^{n\times n}$, $b\in\C^n$ und $c\in\C$,
wobei wir $A^\top=A$ annehmen können.
Wir untersuchen, welche davon gleich stark wie der Laplaceoperator sind.
Es ist
\begin{equation}\label{eq:2:rpQLap1}
\widetilde{Q}(\xi)=\abs{\xi^\top A\xi+\ska{b}{\xi}+c}^2+\sum_{k=1}^n\abs{2e^\top_kA\xi+b_k}^2
+4\sum_{1\leq k\leq l\leq n}\abs{a_{kl}}^2.
\end{equation}
Da die Ausdrücke $\widetilde{Q}(\xi)$ und $1+\abs{\xi^\top A\xi}^2$
auf kompakten Mengen positiv und beschränkt sind (falls $\mcQ\neq0$) genügt es,
diese für $\abs{\xi}\geq R$ mit festem $R>0$ groß genug zu betrachten.
Ist $\abs{\xi^\top A\xi}\neq0$ falls $\xi\neq0$,
so gilt $\abs{\xi^\top A\xi}\asymp\abs{\xi}^2$.
In diesem Fall kann man, für $R$ groß genug,
die linearen Terme in \eqref{eq:2:rpQLap1} vernachlässigen
und erhält insgesamt
\begin{equation}
\widetilde{Q}(\xi)\asymp\abs{\xi^\top A\xi}^2+1\asymp\abs{\xi}^4+1\asymp\widetilde{P}(\xi).
\end{equation}
Im anderen Fall betrachtet man \eqref{eq:2:rpQLap1}
für alle $\xi\in\R^n$ mit $\xi^\top A\xi=0$,
und stellt fest, dass $\widetilde{Q}(\xi)$ für diese $\xi$
maximal linear mit dem Betrag von $\abs{\xi}$ wächst.
In diesem Fall ist $Q(\D)$ also echt schwächer als $P(\D)$ ist.

Im Fall einer reellen Matrix gilt $\xi^\top A\xi\neq0$ genau dann für alle $\xi\neq0$,
wenn entweder alle Eigenwerte positiv oder alle negativ sind.
Im allgemeinen Fall ist die Bedingung $\xi^\top A\xi\neq0$
äquivalent dazu, daß die Menge $\{\xi^\top A\xi:\xi\in\R^n,~\xi\neq0\}$
in einem offenen Halbraum $H_w\subset\C$
von der Form $H_w:=\{z\in\C:\Re(zw)>0\}$, $w\in\C$, $w\neq0$ enthalten ist.
%Im Fall einer komplexen Matrix ist
%\begin{equation}
%\abs{\xi^\top A\xi}^2=\abs{\xi^\top A_\Re\xi}^2+\abs{\xi^\top A_\Im\xi}^2,
%\end{equation}
%mit $A=A_\Re+\i A_\Im$, $A_\Re,A_\Im\in\R^{n\times n}$.
%Die weitere Untersuchung gestaltet sich allerdings recht aufwendig
%und soll nicht weiter vertieft werden.
\end{exa}

\begin{exa}\label{exa:2:schroe}
Der Schrödingergleichung für ein freies Teilchen\index{Differentialoperator!Schrödinger-}
\begin{equation}
\i\frac{\partial}{\partial t}\psi(x,t)=-\Delta\psi(x,t),\end{equation}
lässt sich das Symbol $P(\xi)=\xi^2_1+\dots+\xi^2_{n-1}-\xi_n$ zuordnen.

Wir bestimmen alle Operatoren gleicher Stärke.
Das Symbol $Q(\xi)$ ist genau dann schwächer wie $P(\xi)$,
wenn
\begin{equation}\label{eq:2:subschroe}
Q(\xi)^2\apprle(\xi^2_1+\dots+\xi^2_{n-1}-\xi_n)^2+\xi^2_1+\dots+\xi^2_{n-1}+1.
\end{equation}
Also hat $Q(\xi)$ maximalen Grad 2 in $\xi_1,\dots,\xi_{n-1}$,
Grad 1 in $\xi_n$ und ist somit von der Form
\begin{equation}
Q(\xi)=\sum^n_{k,l=1}a_{kl}\xi_k\xi_l+\sum^n_{k=1}b_k\xi_k+c,
\end{equation}
mit $a_{nn}=0$.
Wir betrachten die Gleichung \eqref{eq:2:subschroe}
für spezielle Vektoren:
\begin{equation}\label{eq:2:subschroespec}
Q(\xi)^2\apprle(\xi^2_1+\dots+\xi^2_{n-1}+1),
\quad\text{für alle}~\xi\in\R^n~\text{mit}~\xi_n=\xi^2_1+\dots+\xi^2_{n-1}.
\end{equation}
Da wir $\xi_1,\dots,\xi_{n-1}$ in \eqref{eq:2:subschroespec} frei wählen können folgt, dass
\begin{equation}
Q(\xi)=a(\xi^2_1+\dots+\xi^2_{n-1}-\xi_n)+\sum_{k=1}^{n-1}b_k\xi_k+c
=aP(\xi)+\sum_{k=1}^{n-1}b_k\xi_k+c,
\end{equation}
mit beliebigen $a,b_1,\dots,b_{n-1},c\in\C$.
Es ist $Q(\xi)$ genau dann gleich Stark wie $P(\xi)$,
wenn $a\neq0$.
\end{exa}

\begin{exa}\label{exa:2:heat}
Der Wärmeleitungsgleichung\index{Differentialoperator!parabolisch}
\begin{equation}
\frac{\partial}{\partial t}T(x,t)=\Delta T(x,t),
\end{equation}
entspricht das Symbol $P(\xi)=\xi^2_1+\dots+\xi^2_{n-1}+\i\xi_n$.

Ein Symbol $Q(\xi)$ ist genau dann schwächer als $P(\xi)$, wenn
\begin{equation}
Q(\xi)^2\apprle(\xi^2_1+\dots+\xi^2_{n-1})^2+\xi^2_1+\dots+\xi^2_{n-1}+\xi^2_n+1.
\end{equation}
Diese Ungleichung ist genau dann erfüllt, wenn
\begin{equation}
Q(\xi)=\sum_{k,l=1}^{n-1}a_{kl}\xi_k\xi_l+\sum_{k=1}^nb_k\xi_k+c,
\end{equation}
mit beliebigen $a_{kl},b_k,c\in\C$.

Betrachtet man die Unterräume $U:=\{\xi\in\R^n:\xi_n=0\}$
und $U^\perp:=\{(0,\dots,0,\xi_n):\xi_n\in\R\}$ separat,
so erhält man notwendige Bedingungen dafür,
daß $Q(\xi)$ gleich stark wie $P(\xi)$.
Zum einen folgt $\sum^{n-1}_{k,l=1}a_{kl}\xi_k\xi_l\neq0$ für $\xi\in U\setminus\{0\}$,
d.h.~das auf $U$ eingeschränkte Symbol muss gleich stark
wie das Symbol des Laplaceoperators auf dem $\R^{n-1}$ sein.
Zum anderen folgt $b_n\neq0$.

Unter Annahme dieser notwendigen Bedingungen lässt sich
die Betrachtung, wie in Beispiel \ref{exa:2:lap},
auf die für große $|\xi|$ dominierenden Terme reduzieren.
So sieht man, daß das Symbol $Q(\xi)$ genau dann gleich stark wie $P(\xi)$ ist,
wenn zusätzlich
\begin{equation}
\left|\textstyle\sum^{n-1}_{k,l=1}a_{kl}\xi_k\xi_l+b_n\zeta_n\right|^2
\asymp(\xi_1^2+\dots+\xi_{n-1}^2)^2+\xi_n^2
\end{equation}
für $|\xi|\geq R>0$ mit $R$ groß genug.
Dies ist genau dann der Fall, wenn
\begin{equation}
\left\{\textstyle\sum^{n-1}_{k,l=1}a_{kl}\xi_k\xi_l:\xi\in\R^{n-1}\setminus\{0\}\right\}
\cap b_n\R=\emptyset
\end{equation}
erfüllt ist, was eine Bedingung an Teilmengen von $\C$ ist.
Im Fall reeller $a_{kl}$ ist dies genau dann der Fall,
wenn $b_n\notin\R$.
\end{exa}

\begin{exa}\label{exa:2:hyper}
Als letztes Beispiel betrachten wir die ultrahyperbolische Gleichung\index{Differentialoperator!(ultra)hyperbolisch}
\begin{equation}
\Delta_1u=\Delta_2u,
\end{equation}
mit $\Delta_1=\partial_1^2+\dots+\partial_m^2$, $\Delta_2=\partial^2_{m+1}+\dots+\partial^2_n$.
Das korrespondierende Symbol ist $P(\xi)=\xi^2_1+\dots+\xi^2_m-(\xi^2_{m+1}+\dots+\xi^2_n)$.

Ein Symbol $Q(\xi)$ ist genau dann schwächer wie $P(\xi)$,
wenn
\begin{equation}
Q(\xi)^2\apprle(\xi^2_1+\dots+\xi^2_m-(\xi^2_{m+1}+\dots+\xi^2_n))^2+\xi^2_1+\dots+\xi^2_n+1.
\end{equation}
Setzen wir wie in Beispiel \ref{exa:2:schroe} $\xi^2_1+\dots+\xi^2_m=\xi^2_{m+1}+\dots+\xi^2_n$,
so erhalten wir für diese $\xi$:
\begin{equation}\label{eq:2:rpQhyp}
Q(\xi)^2\apprle\xi^2_1+\dots+\xi^2_n+1.
\end{equation}
Da wir in \eqref{eq:2:rpQhyp} noch genug Wahlfreiheit haben folgt,
dass $Q(\xi)$ von der Form
\begin{equation}
Q(\xi)=a(\xi_1^2+\dots+\xi^2_m-(\xi^2_{m+1}+\dots+\xi^2_n))+\sum_{k=1}^na_k\xi_k+c
=aP(\xi)+\sum_{k=1}^na_k\xi_k+c,
\end{equation}
mit beliebigen $a,b_1,\dots,b_n,c\in\C$ sein muss
und $Q(\xi)$ ist genau dann gleich stark wie $P(\xi)$,
wenn $a\neq0$.
\end{exa}

\begin{table}
\begin{center}
\begin{tabular}{|c|c|}
\hline
{\em Laplaceoperator (\ref{exa:2:lap})}
&{\em Wärmeleitungsoperator (\ref{exa:2:heat})}\\
\hline
\hline
$\begin{array}{c}
P(\zeta)=\zeta^2\vphantom{\displaystyle\frac12}\\[3pt]
Q(\zeta)=\zeta^{\mathrm{t}}A\zeta+\langle b,\zeta\rangle+c\\[3pt]
\forall_{\zeta\neq0}:\zeta A\zeta\neq0\vphantom{\displaystyle\frac12}
\end{array}$
&
$\begin{array}{c}
P(\zeta)=\zeta^2_1+\dots+\zeta^2_{n-1}+\i\zeta_n\vphantom{\displaystyle\frac12}\\[3pt]
Q(\zeta)=\sum_{k,l=1}^{n-1}a_{kl}\zeta_k\zeta_l+\sum^n_{k=1}b_k\zeta_k+c\\[3pt]
\left\{\textstyle\sum^{n-1}_{k,l=1}a_{kl}\xi_k\xi_l:\xi\in\R^{n-1}\setminus\{0\}\right\}
\cap b_n\R=\emptyset,~b_n\neq0\vphantom{\displaystyle\frac12}
\end{array}$\\
\hline\hline
{\em Schrödingeroperator (\ref{exa:2:schroe})}
&{\em Ultrahyperbolischer Operator (\ref{exa:2:hyper})}\\
\hline\hline
$\begin{array}{c}
P(\zeta)=\zeta^2_1+\dots+\zeta^2_{n-1}-\zeta_n\vphantom{\displaystyle\frac12}\\[3pt]
Q(\zeta)=aP(\zeta)+\sum^{n-1}_{k=1}b_k\zeta_k+c\\[3pt]
a\neq0\vphantom{\displaystyle\frac12}
\end{array}$
&
$\begin{array}{c}
P(\zeta)=\zeta^2_1+\dots+\zeta^2_m-(\zeta^2_{m+1}+\dots+\zeta^2_n)
\vphantom{\displaystyle\frac12}\\[3pt]
Q(\zeta)=aP(\zeta)+\sum_{k=1}^nb_k\zeta_k+c\\[3pt]
a\neq0\vphantom{\displaystyle\frac12}
\end{array}$\\
\hline
\end{tabular}
\end{center}
\caption{Zusammenfassung der Ergebnisse aus der Beispieldiskussion.
Erste Zeile: Symbol $P(\zeta)$ des untersuchten Operators $P(\D)$.
Zweite Zeile: Symbole $Q(\zeta)$ der Operatoren aus $W(P(\D))$.
Dabei sind die Koeffizienten stets beliebig aus $\C$.
Dritte Zeile: Notwendige und Hinreichende Zusatzbedingungen an $Q(\zeta)$,
daß der Operator mit dem Symbol $Q(\zeta)$
gleich Stark wie $P(\zeta)$ ist.}
\end{table}

Wir wollen einige Unterschiede der betrachteten Beispiele festhalten.
Vergleichen wir die Beispiele \ref{exa:2:lap}, \ref{exa:2:hyper}
mit den Beispielen \ref{exa:2:schroe}, \ref{exa:2:heat}.
In den ersten beiden Fällen hängt der Vergleich der Stärke nur von den Hauptteilen,
d.h.~den Potenzen größter Ordnung ab.
In den zweiten zwei Fällen hängt der Vergleich auch von niedrigeren Termen ab.
Dies hängt damit zusammen, dass die Symbole $P(\xi)$ aus
\ref{exa:2:schroe} und \ref{exa:2:heat}
auf gewissen eindimensionalen Unterräumen des $\R^n$ verschwinden.

Auch die Paare \ref{exa:2:lap}, \ref{exa:2:heat}
und \ref{exa:2:schroe}, \ref{exa:2:hyper}
weisen einen große Unterschied auf.
Bei den ersten beiden Operatoren hat man relativ viel Wahlfreiheit
für gleich starke Operatoren.
Bei den zweiten zwei Operatoren hat man im Hauptteil stets
nur einen frei wählbaren Parameter.
Dies hängt damit zusammen, dass in den Symbolen $P(\xi)$
in \ref{exa:2:schroe} und \ref{exa:2:hyper} Differenzterme auftreten.

\begin{df}
\begin{enumerate}
\item
Ein Differentialausdruck $P(\D)$ heißt \eIndex[Differentialoperator]{vom Haupttyp},
falls er gleich stark ist wie jeder Differentialausdruck mit gleichem Hauptteil.
\item
Ein Differentialausdruck $P(\D)$ heißt \eIndex[Differentialoperator]{elliptisch},
falls er stärker ist als jeder Differentialausdruck von kleiner oder gleichem Grad.
\end{enumerate}
\end{df}

Wie eben diskutiert sind also der Laplaceoperator
und der Operator aus Beispiel \ref{exa:2:hyper} vom Haupttyp,
die anderen beiden nicht. Der Laplaceoperator
ist als einziger der angegebenen Operatoren elliptisch.
Wir geben nun Klassifikationen für diese Operatortypen.

\begin{thm}[{\cite[Theorem~2.3]{Hormander:1955}}]\label{thm:2:2.9}
Ein Symbol $P(\xi)$ ist genau dann vom Haupttyp,
wenn die partiellen Ableitungen $p^{(k)}(\xi)=\partial_k p(\xi)$
des Hauptteils $p(\xi)$ für kein $\xi\in\R^n\setminus\{0\}$
alle gleichzeitig Null sind.
\end{thm}

\begin{proof}
{\it Hinrichtung.} Ist $P(\xi)$ vom Haupttyp und vom Grad $m>0$,
dann gilt das gleiche für $p(\xi)$.
Dann ist $p(\xi)$ auch stärker als $p(\xi)+\xi^\alpha$
mit $\abs{\alpha}=m-1$.
Nach Theorem \ref{thm:2:2.1} und der Linearität von $W(\mcP)$
folgt dann
\begin{equation}\label{eq:2:homquot}
\sup_{\xi\in\R^n}\frac{(\xi^2_1+\dots+\xi^2_n)^{m-1}}{\sum\abs{p^{(\alpha)}(\xi)}^2}<\infty.
\end{equation}
Angenommen es gibt ein $\xi_0\in\R^n$,
so dass $p^{(i)}(\xi_0)=0$ für alle $i=1,\dots,n$.
Dann gilt auch $p(\xi_0)=0$, da nach Eulers Formel für homogene Polynome folgt
\begin{equation}
mp(\xi)=\sum_{k=1}^n\xi_kp^{(k)}(\xi)
\end{equation}
Wir setzen $\xi=t\xi_0$, $t\in\R$, in Gleichung \eqref{eq:2:homquot},
und erhalten einen Quotienten aus Polynomen in der Variablen $t$.
Aufgrund der Homogenität gilt
\begin{equation}
p(t\xi_0)=t^mp(\xi_0)=0,~p^{(1)}(t\xi_0)=t^{m-1}p^{(1)}(\xi_0)=0,
\dots,~p^{(n)}(t\xi_0)=t^{m-1}p^{(n)}(\xi_0)=0,
\end{equation}
und somit hat das Polynom im Zähler echt kleineren Grad als $2(m-1)$.
Dies liefert einen Widerspruch für $t\to\infty$. $\bullet$\qquad {\it Rückrichtung.}
Nehmen wir nun umgekehrt an, $P(\xi)$ erfüllt die Bedingung aus Theorem \ref{thm:2:2.9}
und $Q(\xi)$ hat den gleichen Hauptteil $p(\xi)$ wie $P(\xi)$.
Durch Weglassen positiver Term erhalten wir 
\begin{equation}
\widetilde{P}(\xi)^2>\sum_{k=1}^n\abs{P^{(k)}(\xi)}^2=\pi(\xi)+r(\xi).
\end{equation}
Dabei fassen wir die höchsten Terme des Ausdrucks
$\sum_{k=1}^n\abs{P^{(k)}(\xi)}^2$ zu $\pi(\xi)$ zusammen.
Diese sind dann gegeben durch
\begin{equation}
\pi(\xi)=\sum_{k=1}^n\abs{p^{(k)}(\xi)}^2,
\end{equation}
was direkt aus der binomischen Formel
und der Additivität des Grads von Polynomen folgt.
Also hat $r(\xi)$ Grad echt kleiner als $2(m-1)$.
Nach Annahme ist $\pi(\xi)\neq0$ falls $\xi\neq0$
and somit $\abs{r(\xi)}/\pi(\xi)\leq\tfrac12$,
für $\abs{\xi}$ groß genug.
Für diese $\xi$ ist dann auch $\widetilde{P}(\xi)^2>\pi(\xi)/2$.
Mit der Dreiecksungleichung erhalten wir nun
\begin{equation}
\frac{\abs{Q(\xi)^2}}{\widetilde{P}(\xi)^2}
\leq\frac{\abs{P(\xi)}^2}{\widetilde{P}(\xi)^2}+\frac{\abs{Q(\xi)-P(\xi)}^2}{\widetilde{P}(\xi)^2}
\leq1+2\frac{\abs{Q(\xi)-P(\xi)}^2}{\pi(\xi)},
\end{equation}
für $\abs{\xi}$ groß genug.
Das Polynom im Zähler des zweiten Terms hat maximal den Grad $2(m-1)$.
Insgesamt ist damit $\sup_{\xi\in\R^n}\abs{Q(\xi)}/\widetilde{P}(\xi)<\infty$,
also $Q(\D)$ schwächer als $P(\D)$.
\end{proof}

Der Satz und sein Beweis bestätigen also die Anmerkungen nach den Beispielen.
Elliptische Operatoren lassen ähnlich sich klassifizieren.

\begin{thm}
Ein Differentialausdruck $P(\D)$ ist genau dann elliptisch,
wenn der homogene Hauptteil $p(\xi)$ von $P(\xi)$ für alle $\xi\in\R^n\setminus\{0\}$ die Bedingung $p(\xi)\ne0$ erfüllt.
\end{thm}


Zum Abschluss betrachten wir Produktoperatoren auf Produkträumen.
Dies sind Differentialoperatoren mit Symbolen von der Form
\begin{equation}\label{eq:2:prodpol}
P(\xi)=P_a(\xi_a)P_b(\xi_b),
\end{equation}
mit $\xi_a=(\xi_1,\dots,\xi_m)$ und $\xi_b=(\xi_{m+1},\dots,\xi_n)$,
wobei $0<m<n$ fest gewählt ist
und $P_a(\xi)$ ein Polynom in $\R^m$
und $P_b(\xi)$ ein Polynom in $\R^{n-m}$ ist.
Wir erhalten:

\begin{thm}
Für Produktpolynome wie in \eqref{eq:2:prodpol}
gilt die Gleichung
\begin{equation}
W(\mcP)=\spann\left(W(\mcP_a) W(\mcP_b)\right),
\end{equation}
$W(\mcP)$ ist also das Tensorprodukt von $W(\mcP_a)$ und $W(\mcP_b)$.
\end{thm}
\begin{proof}
Wir betrachten zunächst einen Multiindex $\alpha\in\mathbb N_0^n$ und setzen $\alpha_a=(\alpha_1,\dots,\alpha_m,0,\dots,0)$
und $\alpha_b=(0,\dots,0,\alpha_{m+1},\dots,\alpha_n)$. Dann gilt
\begin{equation}
P^{(\alpha)}(\xi)
=\partial^\alpha\left(P_a(\xi_a)P_b(\xi_b)\right)
=P^{(\alpha_a)}_a(\xi_a)P^{(\alpha_b)}_b(\xi_b).
\end{equation} 
Also folgt, dass
\begin{equation}\label{eq:2:prodeq}
\widetilde{P}(\xi)=\sum_\alpha\abs{P^{(\alpha)}(\xi)}^2
=\sum_{\alpha_a,\alpha_b}\abs{P^{(\alpha_a)}_a(\xi_a)}^2\abs{P^{(\alpha_b)}_b(\xi_b)}^2
=\widetilde{P}_a(\xi_a)\widetilde{P}_b(\xi_b).
\end{equation}
Dies zeigt $\spann\left(W(\mcP_a) W(\mcP_b)\right)\subseteq W(\mcP)$. 

Nehmen wir nun an, dass $Q(\xi)\in W(\mcP)$.
Betrachten wir $Q(\xi_a,\xi_{b})$ mit variablem $\xi_a$ und festen $\xi_{b}$,
so folgt aus Gleichung \eqref{eq:2:prodeq},
dass $Q(\xi_a,\xi_{b})\in W(\mcP_a)$.
Analoges gilt bei vertauschten Rollen von $a$ und $b$.
Wir wählen eine Vektoraumbasis $p_1(\xi_a),\dots,p_l(\xi_a)$ von $W(\mcP_a)$.
Es gilt
\begin{equation}
\sum_{k=1}^lp_k(\xi_l)a_k(\xi_b),
\end{equation}
mit Polynomen $a_k(\xi_b)$.
Wir zeigen $a_k(\xi_b)\in W(\mcP_b)$ für alle $k=1,\dots,K$.
Da die $p_1(\xi_a),\dots,p_K(\xi_a)$ linear unabhängig sind,
gibt es $\xi_{a,1},\dots,\xi_{a,K}$,
so dass $\left(p_k(a,\xi_l)\right)_{k,l=1,\dots,K}$ invertierbar ist.
Das Gleichungssystem
\begin{equation}
Q(\xi_{a,l},\xi_b)=\sum_{k=1}^lp_k(\xi_{a,l})a_k(\xi_b)\quad\text{für}~k=1,\dots,K,
\end{equation}
kann also nach den $a_k(\xi_b)$ aufgelöst werden kann.
Dies impliziert nach obigem $a_k(\xi_b)\in W(\mcP_b)$.
\end{proof}

\section{Ein weiteres Vergleichsresultat}
Wir wollen ein weiteres Vergleichsresultat angeben, welches ebenfalls aus dem Quotienten der regularisierten Symbole $\widetilde{P}$ und $ \widetilde{Q}$ hervorgeht. Es handelt sich hierbei um die Stetigkeit von $Q(\D) u$ für $u$ aus dem Definitionsbereich des minimalen Operators $P_0$. Das Resultat verallgemeinert den sobolevschen Einbettungssatz.

\begin{thm}[{\cite[Theorem~2.6]{Hormander:1955}}]
Sei $Q(D)$ schwächer als $P(D)$. Angenommen, $Q(\D)u$ ist für jedes $u\in\mathcal D_{P_0}$ beschränkt. Dann gilt
\begin{equation}\label{satz: bedingung stetigkeit}
\int \frac{|Q(\xi)|^2}{\widetilde{P}(\xi)^2} \d \xi < \infty.
\end{equation}
Ist andererseits \eqref{satz: bedingung stetigkeit} erfüllt, dann ist $Q(D)u$ stetig und es existiert für jedes $\varepsilon >0$ eine kompakte Menge $K \subseteq \Omega$ mit
\begin{align*}
 \forall x \in \Omega \backslash K\quad:\quad |Q(\D)u(x)| < \varepsilon,
\end{align*}
d.h. $Q(\D)u$ verschwindet auf  $\partial\Omega$.
\end{thm}
\begin{proof}{\it Hinrichtung.}
Angenommen für jedes $u \in \mathcal{D}_{P_0}$ gilt $Q(\D)u\in\rmL^\infty(\Omega)$. 
Damit folgt aus Lemma \ref{lm: Qu stetig}  für alle $u \in \rmC_0^{\infty}(\Omega)$
\begin{equation}
\sup_{x \in \Omega} | Q(\D)u(x)|^2 \leq C ( \Vert  P(\D)u\Vert^2 + \Vert u\Vert^2 ),
\end{equation}
also insbesondere
\begin{equation}\label{eq:abschneide null}
|Q(\D)u(0)|^2 \leq C( \Vert  P(\D)u\Vert^2 + \Vert u\Vert^2 ) 
\end{equation}
mit einer von $u\in\rmC_0^\infty(\Omega)$ unabhängigen Konstanten $C>0$. Für eine schnell fallende Funktion $ \varphi \in \mathscr S(\R^n)$ aus dem Schwartzraum definieren wir
\begin{equation}
v(x)= (2 \pi)^{-\frac{n}{2}} \int_{\Omega}\e^{\i x \cdot \xi} \frac{\varphi (\xi)}{\widetilde{P}(\xi)} \d \xi,
\end{equation}
so daß mit dem Satz von Plancherel
\begin{equation}\label{eq:abschneide phi}
\Vert P^{(\alpha)}(\D)v \Vert^2 = \int |\widehat{v}(\xi)|^2 |P^{(\alpha)}(\xi)|^2 \d \xi = \int \frac{|\varphi(\xi)|^2}{\widetilde{P}(\xi)^2} | P^{(\alpha)} (\xi)|^2 \d \xi \leq \int |\varphi(\xi)|^2 \d \xi= \Vert \varphi \Vert^2
\end{equation}
folgt.
Wir wählen eine geeignete Abschneidefunktion $\psi \in \rmC_0^{\infty}(\Omega)$ mit $\psi(x) \leq 1$ auf $\Omega$ und $\psi(x)=1$ in einer Umgebung von $x=0$. Dann gilt $\tilde{u} = \psi v \in \rmC_0^{\infty}(\Omega)$ und mit der leibnizschen Produktformel folgt
\begin{equation}
P(\D)\tilde{u} = P(\D)(\psi v) = \sum_{\alpha} \frac{\D^{\alpha} \psi}{\alpha !} P^{(\alpha)}(\D) v.
\end{equation}
Mithilfe von \eqref{eq:abschneide phi} ergibt sich also
\begin{equation}\label{eq:abschatz phi}
\Vert P(\D) \tilde{u} \Vert \leq C \Vert \varphi \Vert
\end{equation}
mit einer geeigneten von $\varphi$ unabhängigen Konstanten $C>0$.
Wegen
\begin{equation}
v = (2 \pi)^{-\frac{n}{2}} \int \e^{\i x \cdot \xi} \widehat{v}(\xi) \d \xi, \qquad Q(\D)\e^{\i x \cdot \xi} = \e^{\i x \cdot \xi} Q(\xi), \qquad Q(\D)\tilde{u}(0) = Q(\D)v(0),
\end{equation}
und \eqref{eq:abschneide null}, \eqref{eq:abschatz phi} folgt schließlich
\begin{equation}
|Q(\D)\tilde{u}(0)|^2 = |Q(\D)v(0)|^2 = \left|(2 \pi)^{-\frac{n}{2}} \int \frac{Q(\xi)}{\widetilde{P}(\xi)} \varphi(\xi) \d \xi \right|^2 \leq {C} \Vert \varphi \Vert^2
\end{equation}
und da $\varphi\in\mathscr S(\R^n)$ beliebig war folgt mit dem Darstellungssatz von Fréchet-Riesz
\begin{equation}\label{eq:behauptung L2}
\int \frac{|Q(\xi)|^2}{\widetilde{P}(\xi)^2} \d \xi < \infty
\end{equation}
und somit die Behauptung. $\bullet$\qquad {\it Rückrichtung.}
Gelte nun \eqref{eq:behauptung L2}. Dann folgt für $u \in \rmC_0^{\infty}(\Omega)$ mit der Ungleichung von Cauchy--Schwarz
\begin{align}
|Q(\D)u(x)|^2 &=  \left|(2 \pi)^{-\frac{n}{2}} \int \e^{\i x \cdot \xi} Q(\xi) \widehat{u}(\xi)  \d \xi \right|^2 \notag
\\ & \leq (2 \pi)^{-\frac{n}{2}} \int \frac{|Q(\xi)|^2}{\widetilde{P}(\xi)^2} \d \xi \int \widetilde{P}(\xi)^2 | \widehat{u}(\xi)|^2 \d \xi = C \sum_{\alpha} \Vert P^{(\alpha)}(\D)u\Vert^2.
\end{align}
Mit Korollar \ref{cor:2:2.6} folgt also für alle $u \in \rmC_0^{\infty}(\Omega)$
\begin{equation}
|Q(\D)u(x)|^2 \leq C \Vert P(\D)u\Vert^2
\end{equation}
gleichmäßig in $x\in\Omega$ und damit mit Lemma \ref{lm: Qu stetig} die Behauptung.
\end{proof}

\section{Kompaktheitskriterien}
Satz \ref{thm:2:2.1} gibt ein Kriterium dafür, daß der Operator
\begin{equation}\label{eq:2:map}
\mcR_{P_0}\ni P(\D)u\mapsto Q(\D)u\in  \mcR_{Q_0}
\end{equation}
$\rmL^2$--$\rmL^2$-beschränkt ist. Im folgenden fragen wir nach der Kompaktheit dieses Operators. 

\begin{df}
Seien $V$ und $W$ zwei Banachräume und $T: V \supset \mathcal{D}_T \rightarrow W$ ein linearer Operator. $T$ heißt \eIndex[Operator]{kompakt}, wenn für jede Folge $(u_n)_{n \in \mathbb{N}}$ in $\mathcal{D}_T$ mit $\Vert u_n \Vert \leq 1$ für alle $n \in \mathbb{N}$ eine Teilfolge $(u_{n_k})_{k \in \mathbb{N}}$ existiert, so daß deren Bildfolge $(Tu_{n_k})_{k\in \mathbb{N}}$ in $W$ konvergiert.
\end{df}

Wir sagen im folgenden der Differentialausdruck $P(\D)$ \eIndex[Differentialoperator]{dominiert} den Differentialausdruck $Q(\D)$, wenn die Abbildung aus \eqref{eq:2:map} kompakt ist. Die Eigenschaft dominierend zu sein ist wiederum unabhängig vom beschränkten Gebiet $\Omega$.

\begin{thm}[{\cite[Theorem~2.15]{Hormander:1955}}]\label{thm:2:Abbildung kompakt}
Der Differentialausdruck $P(\D)$ dominiert genau dann den Differentialausdruck $Q(\D)$,
wenn
\begin{equation}\label{eq:2:tozero}
\lim_{|\xi|\to\infty} \frac{\widetilde{Q}(\xi)}{\widetilde{P}(\xi)} = 0
\end{equation}
gilt.
\end{thm}
\begin{proof} {\it Rückrichtung.}
Wir nehmen zunächst an, dass \eqref{eq:2:tozero} erfüllt ist und betrachten eine Folge ${(u_n)}_{n \in \mathbb{N}} $ in $\rmC^\infty_0(\Omega)$ mit
\begin{equation}\label{eq:2:PDb}
\norm{P(\D)u_n}\leq 1.
\end{equation}
Wir zeigen, dass dann eine Teilfolge ${(u_{n_k})}_{k \in \mathbb{N}}$ in $\rmC^\infty_0(\Omega)$ existiert,
so dass  $\left(Q(\D)u_{n_k}\right)_{k \in \mathbb{N}}$ konvergent ist. Nach Voraussetzung gilt 
\begin{equation}
\frac{\widetilde{Q}(\xi)^2}{\widetilde{P}(\xi)^2} = \frac{\sum_{\alpha} | Q^{(\alpha)} (\xi)|^2  }{ \sum_{\alpha} | P^{(\alpha)} (\xi)|^2} \rightarrow 0\qquad  \text{für} \ \ \xi \rightarrow \infty;
\end{equation}
insbesondere ist $\widetilde Q(\xi)/\widetilde P(\xi)$ beschränkt. Also existiert eine Konstante $C>0$, so dass
\begin{equation}
|Q(\xi)|^2 \leq C \sum_{\alpha} | P^{(\alpha)} (\xi)|^2 \label{eq:1.5}
\end{equation}
gilt. Damit folgt mit dem Satz von Plancherel für beliebiges $u\in\rmC_0^\infty(\Omega)$
\begin{multline}
\int |Q(\D)u(x)|^2 \d x = \int |Q(\xi) \widehat{u}(\xi)|^2 \d \xi \\
\leq  \sum_{\alpha}C\int | P^{(\alpha)} (\xi) \widehat u(\xi)|^2 \d \xi =\sum_{\alpha} C\int |P^{(\alpha)}(\D)u(x)|^2 \d x.
\end{multline}
Weiter liefert uns  Satz \ref{thm:2:2.1} eine Konstante $\widetilde{C}>0$ mit
\begin{equation}
\Vert P^{(\alpha)}(\D)u \Vert \leq \widetilde{C} \Vert P(\D)u \Vert, \label{eq:ableitung leq PD}
\end{equation}
woraus sich dann mit \eqref{eq:2:PDb} sofort
\begin{equation}
\norm{Q(\D)u_n}\leq C' \norm{P(\D)u_n} \leq C' 
\end{equation}
für die gegebene Folge $(u_n)_{n\in\N}$ ergibt. Da $\Omega$ beschränkt ist, ist auch die $\rmL^1$-Norm $\|Q(\D)u_n\|_1$ gleichmäßig beschränkt und die 
Fouriertransformierten $Q(\xi) \widehat{u}_n(\xi)$ sind gleichmäßig in $\xi$ und $n$ beschränkt. Wir zeigen, daß sie auch gleichgradig stetig sind. Sei dazu $\varepsilon>0$ und $\delta>0$ so gewählt, dass für $|\xi_1 - \xi_2| < \delta$ stets $|\e^{\i x \cdot \xi_1} - \e^{\i x \cdot \xi_2}|\le\epsilon$ gilt. Dann folgt mit Ungleichung von Cauchy--Schwarz 
\begin{align}
| Q(\xi_1)\widehat{u}_n(\xi_1) - Q(\xi_2)\widehat{u}_n(\xi_2)| & \leq (2\pi)^{-n/2} \int_{\Omega} \left| \e^{\i x \cdot \xi_1} - \e^{\i x \cdot \xi_2} \right| | Q(\D) u_n(x)| \d x \notag
\\ & \leq \varepsilon |\Omega|^{\frac{1}{2}} \Vert Q(\D)u_n \Vert \leq \varepsilon \widetilde{C}.
\end{align}
Also können wir nach dem Satz von Arzel\`a-Ascoli eine lokal gleichmäßig konvergente Teilfolge $(Q(\xi)\widehat{u}_{n_k})_{k\in\N}$ auswählen, d.h. für jedes Kompaktum $K\Subset\R^n$ konvergiert $Q(\xi)\widehat{u}_{n_k}(\xi)$ gleichmäßig.
Nach Voraussetzung \eqref{eq:2:tozero} lässt sich für jedes $\varepsilon > 0$ eine kompakte Menge $K \Subset \R^n$ finden, so dass 
\begin{equation}
\forall \xi\in\R^n\setminus K \quad:\quad \frac{\abs{Q(\xi)}}{\widetilde{P}(\xi)} <\varepsilon
\end{equation}
gilt. Damit folgt nun
\begin{align}
\int_{\R^n\setminus K}\abs{Q(\xi)}^2\abs{\widehat{u}_{n_k}(\xi)-\widehat{u}_{n_l}(\xi)}^2\d \xi
&\leq\varepsilon^2\int\widetilde{P}(\xi)^2\abs{\widehat{u}_{n_k}(\xi)-\widehat{u}_{n_l}(\xi)}^2\d \xi\notag\\
&=\varepsilon^2\sum_\alpha \int  \abs{ P^{(\alpha)}(\xi) (\widehat{u}_{n_k}(\xi)-\widehat{u}_{n_l}(\xi) ) }^2\d \xi\notag\\
&=\varepsilon^2\sum_\alpha \norm{P^{(\alpha)}(\D)(u_{n_k}-u_{n_l})}^2\label{eq:2:Kcsum}
\end{align}
und diese Summe ist nach \eqref{eq:ableitung leq PD} und \eqref{eq:2:PDb} beschränkt.
Da $Q(\xi)\widehat u_{n_k}(\xi)$ auf $K$ gleichmäßig konvergiert, gilt außerdem
\begin{equation}
\int_K\abs{Q(\xi)}^2\abs{\widehat{u}_{n_k}(\xi)-\widehat{u}_{n_l}(\xi)}^2\d \xi\longrightarrow0
\quad\text{für}~ k,l \rightarrow \infty,
\end{equation}
und damit die Konvergenz der Teilfolge $\left(Q(\D)u_{n_k} \right)_{k \in \mathbb{N}}$ in $\rmL^2(\Omega)$. 
Da die Funktionen $P(\D)u$ für $u\in \rmC^\infty_0(\Omega)$ dicht in $\mcR_{P_0}$ liegen, dominiert $P(\D)$ den Operator $Q(\D)$.
$\bullet$\qquad{\it Hinrichtung.}
Wir nehmen nun an, dass $P(\D)$ den Operator $Q(\D)$ dominiert und zeigen, dass \eqref{eq:2:tozero} erfüllt ist.
 Sei dazu $(\xi_n)_{n \in \mathbb{N}}$ eine beliebige Folge in $\mathbb{R}^n$ mit $\xi_n \rightarrow \infty$. Wir zeigen, dass $\widetilde{Q}(\xi_{n})/\widetilde{P}(\xi_{n})\to0$ für $n \rightarrow \infty$ gilt. Da nach Satz~\ref{thm:2:2.1} der Quotient $\widetilde Q(\xi)/\widetilde P(\xi)$ beschränkt ist, reicht es zu zeigen, daß für jede Folge $(\xi_n)_{n \in \mathbb{N}}$, für welche der Quotient $\widetilde Q(\xi_n)/\widetilde P(\xi_n)$ konvergiert, dessen Grenzwert Null sein muß. Durch Wahl einer Teilfolge kann man weiter annehmen, daß $\xi_n-\xi_m\to\infty$ für $n,m\to\infty$, $n\ne m$ gilt. Für eine solche Folge konstruieren wir nun eine Folge von Testfunktionen.
Sei dazu  $u\in \rmC^\infty_0(\Omega)$, $u\neq0$, und für $n\in\N$ definieren wir
\begin{equation}
u_n(x)=u(x)\frac{\e^{\i x \cdot \xi_n}}{\widetilde{P}(\xi_n)}.
\end{equation}
Dann liefert die leibnizsche Produktregel
\begin{equation}
P(\D)u_n(x)=\e^{\i x \cdot \xi_n}\sum_\alpha\frac{P^{(\alpha)}(\xi_n)}{\widetilde{P}(\xi_n)}\frac{\D^\alpha u(x)}{\alpha!}
\end{equation}
und wir finden Konstante $C>0$ mit
\begin{equation}\label{eq:2:PDunb}
\norm{P(\D)u_n}\leq C.
\end{equation}
Ausmultiplizieren liefert nun
\begin{equation}\label{eq:2:kz}
\norm{Q(\D)u_n-Q(\D)u_m}^2=\norm{Q(\D)u_n}^2+\norm{Q(\D)u_m}^2-2\Re\spro{Q(\D)u_n}{Q(\D)u_m}, 
\end{equation}
wobei das Innenprodukt durch
\begin{equation}
\spro{Q(\D)u_n}{Q(\D)u_m}=
\sum_{\alpha,\beta}\frac{Q^{(\alpha)}(\xi_n)}{\widetilde{P}(\xi_n)}\frac{\cc{Q^{(\beta)}(\xi_n)}}{\widetilde{P}(\xi_n)}
\frac1{\alpha!\beta!}\int_\Omega \e^{\i x \cdot(\xi_n-\xi_m)} \D^\alpha u(x) \cc{\D^\beta u(x) }\d  x
\end{equation}
gegeben ist. Da die Faktoren $Q^{(\alpha)}(\xi_n)/\widetilde{P}(\xi_n)$ nach Satz \ref{thm:2:2.1} beschränkt und $\D^\alpha u\cc{\D^\beta u}$ integrierbar sind, folgt mit dem Riemann--Lebesue-Lemma $\spro{Q(\D)u_n}{Q(\D)u_m}\to0$ für $n,m\to\infty$, $n\neq m$. Aufgrund der Kompaktheit finden wir eine Teilfolge $(u_{n_k})_{k\in\N}$ für die $(Q(\D)u_{n_k})_{k \in \mathbb{N}}$ konvergiert. Damit muß nach \eqref{eq:2:kz}  aber $\|Q(\D)u_{n_k}\|\to0$, $k\to\infty$ gelten, d.h.
\begin{equation}
\norm{Q(\D)u_{n_k}}^2=\sum_{\alpha,\beta}\frac{Q^{(\alpha)}(\xi_{n_k}) \overline{Q^{(\beta)}(\xi_{n_k})}}{\widetilde{P}(\xi_{n_k})^2} \underbrace{\frac{1}{\alpha! \beta!}\int \D^{\alpha} u \overline{\D^{\beta} u} \d x}_{=\Psi_{\alpha \beta}} \longrightarrow 0
\end{equation}
und damit wegen der Äquivalenz der Innenprodukte in \eqref{eq:2.8} auch 
\begin{equation}
\frac{\widetilde{Q}(\xi_{n_k})}{\widetilde{P}(\xi_{n_k})} \longrightarrow 0.
\end{equation}
\end{proof}
\begin{exa}
Wir betrachten einen Differentialausdruck $P(\D)$, welcher durch
\begin{equation}
P(\D) = -\i\partial_1 + \partial_2^2
\end{equation}
gegeben ist. Für das dazugehörige Symbol $P(\xi)$ ergibt sich dann
\begin{equation}
P(\xi) = \xi_1 - \xi_2^2.
\end{equation}
Für jedes $\xi \not = 0$ gilt somit
\begin{align}
|P(\xi t)| \rightarrow \infty \qquad \text{für} \ t \rightarrow \infty,
\end{align}
aber nicht zwingend
\begin{align}
|P(\xi)| \rightarrow \infty \qquad \text{für} \ \xi \rightarrow \infty,
\end{align}
da die Nullstellenmenge
\begin{equation}
N= \{ (\xi_1,\xi_2)\in \mathbb{R}^2 : \ \xi_1 = \xi_2^2 \}
\end{equation}
parabelförmig ins Unendliche läuft. Betrachtet man jedoch das regularisierte Symbol
\begin{equation}
\widetilde{P}(\xi)^2 = \sum_{|\alpha| \leq 2} |P^{\alpha}(\xi)|^2 = |\xi_1 - \xi_2^2|^2 + |2\xi_2|^2 + 2,
\end{equation}
dann gilt $\widetilde{P}(\xi) \rightarrow \infty$ für $\xi \rightarrow \infty$. Speziell mit $Q(\D)=I$ impliziert obiger Satz somit aus
$1/\widetilde P(\xi)\to0$ die Kompaktheit von $P_0^{-1}$.
\end{exa}
Wir werden im Folgenden zeigen, dass der Operator $P_0^{-1}$ genau dann kompakt ist, wenn das dazugehörige Symbol $P(\xi)$ von allen Variablen abhängt. Dazu vorbereitend geben wir zunächst folgende Definition.
\begin{df}
Sei $P(\xi)$ das Symbol eines Differentialausdrucks $P(\D)$. Dann bezeichnen wir mit
\begin{equation}
\Lambda(P)=\{ \nu \in \mathbb{R}^n| \ \ \forall \xi \in \mathbb{R}^n, \forall t \in \mathbb{R} \;:\; P(\xi + t\nu) = P(\xi)  \}
\end{equation}
den \eIndex[Polynom]{Linienraum}\footnote{Dieser wurde von G\r{a}rding eingeführt und beschreibt den größten Unterraum des $\R^n$, so daß das Polynom $P$ auf den Quotienten $\R^n/\Lambda(P)$ projiziert werden kann.} von $P$. Wir nennen $P(\xi)$ \eIndex[Differentialoperator]{vollständig}, wenn $\Lambda(P)=\{0\}$ gilt.
\end{df}
\begin{lem}[{\cite[Lemma~2.13]{Hormander:1955}}]\label{Lema:homogene Polynome} 
Sei $Q(\xi)$ ein homogenes Polynom vom Grad $m \in \mathbb{N}$ und  $\nu \in \mathbb{R}^n$ mit
\begin{equation}
Q^{(\alpha)}(\nu)=0
\end{equation}
für alle $\alpha\in\N_0^n$ mit $|\alpha|=m-1$. Dann gilt $\nu \in \Lambda(Q)$.
\end{lem}
\begin{proof}
Für $m=1$ gilt offensichtlich $\alpha = 0$ und es ist nichts zu zeigen. Sei also $m>1$ und wir nehmen an, dass die Behauptung schon für Polynome vom Grad kleiner $m$ gezeigt ist. Die partielle Ableitung $\partial Q/\partial_{\xi_j}$ verschwindet entweder identisch oder ist homogen vom Grad $m-1$. Da sie ebenso die Voraussetzungen des Lemmas erfüllt gilt also nach Induktionsvoraussetzung
\begin{equation}
\partial Q(\xi + t \nu)/ \partial \xi_j = \partial Q(\xi)/\partial \xi_j.
\end{equation}
Da dies für alle $j\in\{1,\ldots,n\}$ gilt, ist der Ausdruck $Q(\xi + t \nu) - Q(\xi)$ unabhängig von $\xi$ und wir erhalten
\begin{equation}
Q(\xi + t \nu) - Q(\xi) = Q(t \nu).
\end{equation}
Speziell mit $t=1$ und $\xi=\nu$ folgt
\begin{equation}
Q(\xi + t\nu) - Q(\xi) = Q(2 \nu) - Q(\nu)= Q(\nu), 
\end{equation}
also $2^mQ(\nu)=Q(2\nu) = 2 Q(\nu)$ und damit $Q(\nu)=0$, also insbesondere
\begin{equation}
Q(\xi + t\nu) = Q(\xi)
\end{equation}
und damit $\nu \in \Lambda(Q)$.
\end{proof}
Mit diesem Lemma und Satz \ref{thm:2:Abbildung kompakt} können wir nun das versprochene Kompaktheitskriterium für $P_0^{-1}$ angeben.
\begin{thm}[{\cite[Theorem~2.17]{Hormander:1955}}]\label{thm:2.17}
Der Operator $P_0^{-1}$ ist genau dann kompakt, wenn das zu $P(\D)$ gehörende Symbol $P(\xi)$ ein vollständiges Polynom ist.
\end{thm}
\begin{proof}{\em Hinrichtung.}
Angenommen $P(\xi)$ ist nicht vollständig, es existiert also ein von Null verschiedener Vektor $0 \not = \nu \in \mathbb{R}^n$ mit
\begin{equation}
P(\xi + t \nu)  = P(\xi)
\end{equation}
für alle $\xi\in\R^n$ und $t\in\R$. Differenzieren dieser Identität liefert
\begin{equation}
\widetilde{P}(\xi + t \nu) = \widetilde{P}(\xi)
\end{equation}
für alle $\xi\in\R^n$ und $t\in\R$ und $\widetilde{P}(\xi) \not \rightarrow \infty$ für $\xi \rightarrow \infty$. Also impliziert Satz \ref{thm:2:Abbildung kompakt} mit $Q(\xi)=1$, dass $P_0^{-1}$ nicht kompakt sein kann. $\bullet$\qquad {\em Rückrichtung.} Sei das Symbol $P(\xi)$ vollständig, d.h. es gilt $\Lambda (P)= \{0\}$. Wir zerlegen das Symbol $P(\xi)$ in seine homogenen Komponenten $P_k(\xi)$, 
\begin{equation}\label{eq:zerlegung des polynoms}
P(\xi) = \sum_{k=0}^m P_k(\xi),\qquad \deg P_k=k.
\end{equation}
Gilt nun $\nu \in \Lambda (P)$, dann folgt für beliebiges $\xi\in\R^n$ und $t\in\R$
\begin{equation}
\sum_{k=0}^m P_k (\xi + t \nu) = \sum_{k=0}^m P_k (\xi)
\end{equation}
und damit unter Ausnutzung der Homogenität auch
\begin{equation}
P_k (\xi + t \nu) = P_k (\xi) 
\end{equation}
für jeden einzelnen Summanden. Also gilt $\Lambda(P) \subseteq \bigcap_{k \leq m} \Lambda (P_k)$. Die umgekehrte Inklusion ergibt sich sofort aus \eqref{eq:zerlegung des polynoms}, woraus wir dann schließlich
\begin{equation}\label{eq:schnitt der homogenen teile}
\Lambda(P) = \bigcap_{k \leq m} \Lambda (P_k) = \{0\}
\end{equation}
folgern. Wir müssen also zeigen, dass $\widetilde{P}(\xi) \rightarrow \infty$ für $\xi \rightarrow \infty$ gilt um Satz \ref{thm:2:Abbildung kompakt} mit $Q(\xi) =1$ erneut anwenden zu können. Dafür zeigen wir, dass die Menge
\begin{equation}
M_{C,P} := \{\xi \in \R^n | \ \widetilde{P}(\xi) \leq C\}
\end{equation}
für jedes $C$ beschränkt ist. {\em Dazu zeigen wir, daß für beliebige Polynome $P$ und Konstanten $C$ die zugeordnete Menge $M_{C,P}$ modulo $\Lambda(P)$ (also auf dem Quotientenraum $\R^n/\Lambda(P)$) beschränkt ist.} Der Beweis erfolgt per Induktion über $m$.

\noindent{\em Induktionsanfang:} Wir betrachten Polynome vom Grad $1$. Diese sind von der Form 
$P(\xi) =  a \cdot \xi + b$ mit $ a  \in \mathbb{C}^n$, $b \in \mathbb{C}$. Für diese gilt $\Lambda(P)=\{\Re a\}^{\perp}\cap\{\Im a\}^\perp$ und $M_{C,P}$ ist wegen $\widetilde P(\xi)^2 = (\Re a\cdot \xi)^2+(\Im a\cdot \xi)^2 + C$ ein zu 
$(\spann\{\Re a,\Im a\})^\perp$ paralleler Streifen. Letzterer ist als Teilmenge des Quotientenraumes $\mathbb{R}^n/(\spann\{\Re a,\Im a\})^{\perp}\simeq \spann\{\Re a,\Im a\}$ beschränkt. $\bullet$

\noindent{\em Induktionsschritt:} Angenommen, die Aussage wurde schon für alle Polynome vom Grad kleiner $m$ gezeigt. Sei weiter $P$ vom Grad $m$.
Nach Definition gilt $\widetilde{P}(\xi) \leq C$ für alle $\xi \in M_{C,P}$, insbesondere also auch $|P^{(\alpha)}(\xi)|\leq C$ für jeden Multiindex $\alpha$. Für $|\alpha|=m-1$ unterscheiden sich $P^{(\alpha)}(\xi)$ und $P_m^{(\alpha)}(\xi)$ nur um eine Konstante, d.h. es gilt $P^{(\alpha)}-P_m^{(\alpha)}=p_\alpha \in\mathbb{C}$ und damit $|P_m^{(\alpha)}(\xi)| \leq C'$ für eine neue Konstante $C'>0$ und alle $\xi \in M_{C,P}$. Gilt nun  $P_m^{(\alpha)}(\xi)=0$ für $\xi\in\R^n$ und alle $|\alpha|=m-1$, 
so folgt mit Lemma \ref{Lema:homogene Polynome}, daß $\xi \in \Lambda(P_m)$. 
Damit ist der Nullraum der linearen Abbildung $\xi \mapsto ( P^{(\alpha)}_m(\xi))_{|\alpha|=m-1} \in\C^M$, $M=\#\{\alpha : |\alpha|=m-1\}$, enthalten in $\Lambda(P_m)$
und somit gilt
\begin{equation}
    \widetilde {P}(\xi)^2 \ge \sum_{|\alpha|=m-1} |P^{(\alpha)}(\xi)|^2 =\sum_{|\alpha|=m-1} |P^{(\alpha)}_m(\xi)+p_\alpha |^2  \ge C'' \dist( \xi,\Lambda(P_m) )^2 - C'''
\end{equation}
mit geeigneten Konstanten $C''>0$ und $C'''$. Damit folgt die Beschränktheit von $M_{C,P}$ modulo $\Lambda(P_m)$ in $\mathbb{R}^n/\Lambda(P_m)$. 

Da $P_m^{(\alpha)}(\xi+\eta) = P_m^{(\alpha)}(\xi)$ für $\eta\in\Lambda(P_m)$ und alle $\alpha$ gilt, ist $\widetilde{P_m}(\xi)$ definiert auf $\R^n/\Lambda(P_m)$ und beschränkt auf Kompakta in diesem Raum. Insbesondere ist $\widetilde {P_m}(\xi)\le C''''$ auf $M_{C,P}$. 
Betrachten wir nun das Polynom
\begin{equation}
R(\xi)=P(\xi) - P_m(\xi) = \sum_{k=0}^{m-1} P_k(\xi),
\end{equation}
so folgt mit der Dreiecksungleichung
\begin{equation}
\widetilde{R}(\xi) \leq \widetilde{P}(\xi) + \widetilde{P}_m(\xi).
\end{equation}
Da das Polynom $R$ vom Grad $m-1$ ist, impliziert die Induktionsvoraussetzung, dass $M_{C,R}$ modulo $\Lambda(R)=\bigcap_{k=1}^{m-1} \Lambda (P_k)$ beschränkt ist. Aus $\widetilde{P}_m(\xi) \leq C''''$ in $M_{C,P}$ ergibt sich somit für alle $\xi\in M_{C,P}$
\begin{equation}
\widetilde{R}(\xi) \leq \widetilde{P}(\xi) + \widetilde{P}_m(\xi) \leq C + C'''',
\end{equation}
also $M_{C,P} \subseteq M_{C+C'''',R}$, womit die Beschränktheit von $M_{C,P}$ modulo $\Lambda(R)$ folgt. Zusammen mit der Beschränktheit modulo $\Lambda(P_m)$ folgt die Behauptung. $\bullet$

Da nach Voraussetzung $P$ vollständig und somit $\Lambda(P)$ trivial ist, ist die Menge $M_{C,P}\subset\R^n$ für jedes $C$ beschränkt und $\widetilde P(\xi)\to \infty$ für $\xi\to\infty$.
\end{proof}



