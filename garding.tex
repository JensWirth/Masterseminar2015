% !TEX root = main.tex
\chapter{Die Ungleichung von G\r{a}rding}

\section{Dirichletformen}
Im folgenden sollen Dirichletformen 
\begin{equation}
   P(x,\D,\overline\D)[u,v] = \sum_{|\alpha|,|\beta|\le m} \int_\Omega p_{\alpha,\beta}(x) \big(\D^\alpha u(x)\big) \overline{\big(\D^\beta v(x)\big)} \d x,\qquad u,v\in\rmC_0^\infty(\Omega)
\end{equation}
zu einem gegebenen Gebiet $\Omega\subset\R^n$ und für Koeffizienten $p_{\alpha,\beta}\in\rmC^\infty(\overline{\Omega})$ mit $p_{\beta,\alpha}(x)=\overline{p_{\alpha,\beta}(x)}$  betrachtet werden. Wir schreiben kurz $P(x,\D,\overline\D)[u]=P(x,\D,\overline\D)[u,u]$ und wenn klar ist, welche Form wir betrachten, nur $P[u,v]$ beziehungsweise $P[u]$. Zugeordnet zu einer solchen Form betrachten wir das Symbol
\begin{equation}
   P(x,\zeta,\overline\zeta) = \sum_{\alpha,\beta} p_{\alpha,\beta}(x) \zeta^\alpha\overline\zeta^\beta,\qquad \zeta\in\C^n,
\end{equation}
und das zugeordnete Hauptsymbol
\begin{equation}
   p(x,\zeta,\overline\zeta) = \sum_{|\alpha|=|\beta|=m} p_{\alpha,\beta}(x) \zeta^\alpha\overline\zeta^\beta,\qquad \zeta\in\C^n.
\end{equation}
Auf Grund der Symmetriebedingung ist das Symbol $P(x,\xi,\xi)$ reellwertig f\"ur alle $\xi\in\R^n$, ebenso ist 
$P(x,\D,\overline \D)[u]$ reell. Spezialfälle solcher Dirichletformen sind die Innenprodukte des $\rmH^m_0(\Omega)$. Für diese schreiben wir kurz
\begin{equation}
  \spro{u}{v}_{(m)} = \sum_{|\alpha|\le m} \spro{\D^\alpha u}{\D^\alpha v} = \sum_{|\alpha|\le m} \int_\Omega   \big(D^\alpha u(x)\big) \overline{\big(\D^\alpha v(x)\big)} \d x.
\end{equation}
Es stellt sich die Frage, unter welchen Voraussetzungen eine Dirichletform äquivalent zu einem solchen Innenprodukt ist. Gilt $a_{\alpha,\beta}\in \rmC^\infty(\overline\Omega)$, so ergibt sich stets eine obere Schranke der Form
\begin{equation}
    P[u] \le C \spro{u}{u}_{(m)}.
\end{equation}
Untere Schranken sind komplizierter. Wir nennen die Form (gleichmäßig) elliptisch, falls 
\begin{equation}
  \inf_{x\in\Omega}  \inf_{|\xi|=1} |p(x,\xi,\xi)| > 0.
\end{equation}
Unter dieser Voraussetzung zeigen wir, dass es Konstanten $a,b\in\R$ gibt, für welche
\begin{equation} 
   \spro{u}{u}_{(m)} \le a\, P[u] + b\, \spro{u}{u}
\end{equation}
gilt. Dies ist dazu äquivalent, dass der Quotient $P [u]  / \spro{u}{u}_{(m)}$ nach unten beschränkt ist.

\begin{thm}[{\cite[Theorem 2.1]{Garding:1953}}]
Sei $P[u]$ eine gleichmäßig elliptische Dirichletform der Ordnung $m$ mit positivem Hauptsymbol auf einem beschränkten Gebiet $\Omega\Subset\R^n$. Dann gilt
\begin{equation}
  \inf_{u\in\rmH_0^m(\Omega)} \frac{P[u]}{\spro{u}{u}_{(m)}} > -\infty.
\end{equation}
\end{thm}
\begin{proof}
{\sl Schritt 1.} Wir beginnen mit dem Spezialfall, dass $P(x,\xi,\xi) = P(\xi)$ unabhängig von $x$ ist. Dann folgt mit dem Satz von Plancherel
\begin{equation}
 P[u] =  \sum_{|\alpha|,|\beta|\le m} \int_\Omega  p_{\alpha,\beta} \big(\D^\alpha u(x)\big)\overline{\big(\D^\beta u(x)\big)} \d x
 =   \sum_{|\alpha|,|\beta|\le m} \int  p_{\alpha,\beta} \xi^\alpha \xi^\beta  |\widehat u(\xi)|^2 \d\xi 
\end{equation}
mit $p_{\alpha,\beta}\in\C$ den Koeffizienten der Form und für beliebiges $u\in\rmC_0^\infty(\Omega)$. Da weiterhin die Elliptizitätsbedingung $p(\xi)\ge \tilde c |\xi|^{2m}$ 
vorausgesetzt ist und $\Omega$ beschränkt ist, gilt
\begin{equation}
  P[u] \ge \tilde c \int |\xi|^{2m} |\widehat u(\xi)|^2 \d\xi  -  \tilde c' \int \langle \xi\rangle^{2m-1} |\widehat u(\xi)|^2 \d\xi  \ge  \tilde c'\; \spro{u}{u}_{(m)}
\end{equation}
mit einer geeigneten Konstanten $\tilde c'$. $\bullet$\qquad{\sl Schritt 2.}  Wir führen den allgemeinen Fall auf den gerade gezeigten Spezialfall zurück. Dazu bezeichne im Folgenden
\begin{equation}
	w(\rho) = \sup\left\{| p_{\alpha,\beta}(x)-p_{\alpha,\beta}(y)| : |\alpha|, |\beta| \le m, | x-y| \le \rho\right\}.
\end{equation} 
Da nach Voraussetzung die Koeffizienten aus $\rmC^\infty(\overline\Omega)$ sind, sind sie insbesondere gleichmäßig stetig und somit gilt
$\lim_{\rho\to0} w(\rho)=0$.  Sei nun $x_0\in \Omega$ beliebig, $u\in \rmH_0^m(\Omega\cap \partial B_\rho(x_0))$ und bezeichne $P_0(x,\xi, \xi):=P(x_0,\xi,\xi)$ die Form mit in $x_0$ eingefrorenen Koeffizienten. Dann gilt
\begin{equation}
\begin{split}
\big| P[u]-P_0[u] \big| &=\left| \sum\limits_{|\alpha|, |\beta| \le m} \int_\Omega \left(p_{\alpha,\beta}(x)-p_{\alpha,\beta}(x_0)\right){\D^\alpha(u(x)}\overline{\D^\beta u(x))}\mathrm{d}x\right|\\
	&\le w(\rho) \sum\limits_{|\alpha|,|\beta| \le m} \spro{\D^\alpha u}{\D^\beta u} \le  w(\rho) \sum\limits_{|\alpha|,|\beta| \le m}\|\D^\alpha u\|\, \|\D^\beta u\| \le
	 w(\rho) \|u\|_{(m)}^2
\end{split}
\end{equation}
unter zweifacher Ausnutzung der Ungleichung von Cauchy--Schwarz. Also folgt unter Verwendung des oben betrachteten Spezialfalls
\begin{equation}
	P[u] \ge P_0[u] - \big| P[u] - P_0[u] \big| \ge \big( \tilde c' - w(\rho) \big) \|u\|_{(m)}^2 \ge  c \| u\|_{(m)}^2
\end{equation} 
für ein $c\in\R$ und alle $u\in \rmH_0^m(\Omega \cap \partial B_\rho(x_0))$,  $\rho$ hinreichend klein. Die Abschätzung ist gleichmäßig in $x_0\in\Omega$.
$\bullet$\qquad {\sl Schritt 3.} Da $\Omega$ beschränkt ist, finden wir zu jedem $\epsilon>0$ eine endliche Überdeckung $\Omega\subset\bigcup_{k=1}^N B_{{\epsilon}/{2}}(x_k)$ und eine dieser Überdeckung untergeordnete Zerlegung der Eins $\psi_k\in\rmC_0^\infty(B_{\epsilon/2}(x_k))$ mit
\begin{equation}
 \sum\limits_{k=1}^N |\psi_k(x)|^2 = 1,\qquad \ x\in \Omega.
\end{equation}
Sei nun $u\in\rmH_0^m(\Omega)$ beliebig. Dann gilt
\begin{equation}
\begin{split}
	P[u]&= \sum_{k=1}^N \sum\limits_{|\alpha|,|\beta|\le m} \int_\Omega |\psi_k(x)|^2p_{\alpha\beta}(x)\big(\D^\alpha u(x)\big)\overline{\big(\D^\beta u(x)\big)}\d x\\
	&=  \sum_{k=1}^N \bigg( \sum\limits_{|\alpha|,|\beta|\le m} \int_\Omega  p_{\alpha\beta}(x)\big(\D^\alpha \psi_k(x) u(x)\big)\overline{\big(\D^\beta \psi_k(x) u(x)\big)}\d x + R_k[u]\bigg),
\end{split}
\end{equation}
wobei der Restterm $R_k[u]$ eine Form der Ordnung $m$ mit verschwindendem Hauptsymbol\footnote{Zur Erinnerung: $r_k(x,\xi,\xi)=0$ heißt, daß im Integral nie ein Produkt von zwei Ableitungen der Ordnung $m$ steht.} ist. Es gilt also mit Cauchy--Schwarz $|R_k[u]|\le c_k \|u\|_{(m)} \|u\|_{(m-1)}$, während jeder Summand der äußeren Summe mit Schritt 2 abgeschätzt werden kann. Also folgt 
\begin{equation}\label{eq:6.16}
P[u] \ge \sum_{k=1}^N  \bigg( c \|\psi_k u\|_{(m)}^2 - c_k \|u\|_{(m)} \|u\|_{(m-1)} \bigg) .
\end{equation}
Weiter gilt 
\begin{equation}
   \|\psi_k u\|_{(m)}^2 =  \sum_{|\alpha|\le m} \int_\Omega |\D^\alpha (\psi_k(x) u(x)) |^2 \d x
   =   \sum_{|\alpha|\le m} \int_\Omega  |\psi_k(x)|^2  |\D^\alpha u(x) |^2 \d x + \tilde R_k[u]
\end{equation}
mit einer weiteren Form $\tilde R_k[u]$  der Ordnung $m$ und verschwindendem Hauptsymbol. Also gilt wiederum $|\tilde R_k[u]|\le b_k \|u\|_{(m)}\|u\|_{(m-1)}$ und  \eqref{eq:6.16} liefert mit $b= \sum_{k=1}^N (c_k + c b_k  )$
\begin{equation}\label{eq:6.18}
P[u] \ge c \|u\|_{(m)}^2 - b \|u\|_{(m)}\|u\|_{(m-1)} .
\end{equation}
Mit $\|u\|_{(m-1)}\le \|u\|_{(m)}$ folgt die Behauptung.
\end{proof}

\begin{rem} 
Durch Interpolation folgt aus \eqref{eq:6.18}
\begin{equation}
P[u] \ge c \|u\|_{(m)}^2 - \tilde b \|u\|_{(m-1/2)}^2
\end{equation}
mit den in Definiton~\ref{df:sob-norm} definierten Sobolevnormen. 
Da $p(x,\xi,\xi)\ge0$ gilt und im wesentlichen nur der Hauptteil durch den ersten Term abgeschätzt wird, folgt aus obigem Beweis insbesondere für jedes $\epsilon>0$
die Existenz einer Konstanten $K(\epsilon)>0$ mit
\begin{equation}
P[u] \ge -\epsilon \|u\|_{(m)}^2 - K(\epsilon) \|u\|_{(m-1/2)}^2.
\end{equation}
Es stellt sich die Frage, was mit $K(\epsilon)$ für $\epsilon\to0$ passiert. 
\end{rem}

\section{Die scharfe G\r{a}rdingungleichung}

Nun soll die gerade aufgeworfene Frage beantwortet werden. Wir betrachten dazu Pseudodifferentialoperatoren $A$ zu Symbolen $\sigma_A(x,\xi)$.

\begin{thm}[Hörmander, {\cite[Theorem 1.3.3]{Hormander:1966}}]
Sei $A\in {\rm OP}S^{2m}_{\rm loc}(\Omega\times\R^n)$ Pseudodifferentialoperator mit Hauptsymbol\footnote{Es gilt also $\sigma_A-a\in S^{2m-1}_{\rm loc}(\Omega\times\R^n)$.} $a(x,\xi)\ge0$. Dann gibt es  für alle $\omega\Subset \Omega$ ein $K>0$ mit
 \begin{equation}
    \Re \spro{Au}{u} \ge - K \|u\|_{(m-1/2)}^2 
 \end{equation}
 für alle $u\in\rmC_0^\infty(\omega)$.
\end{thm}

Es genügt, eine solche Aussage für Operatoren der Ordnung Null zu beweisen. Operatoren anderer Ordnungen können mit dem Pseudodifferentialkalkül auf diesen Fall zurückgeführt werden. Im folgenden beziehen wir uns auf  Lax und Nirenberg \cite{Lax:1966} und zeigen eine entsprechende Aussage für (matrixwertige) Operatoren. 

Zuerst benötigen wir einige Definitionen und Hilfsmittel.

\medskip\noindent
{\em Symbolklassen:} Für $ a\in S^0_{\rm comp}(\R^n\times\R^n)$ definieren wir die Normen
\begin{equation}
\begin{split}&	\vert \widehat{a}(\eta,\cdot)\vert_k := \sum\limits_{\vert \beta\vert \le k} \sup\limits_{\xi} \langle \xi\rangle^{\vert \beta\vert}\vert\D_\xi^\beta  \widehat{a}(\eta,\xi)\vert,\\
&   | a|_{k,\ell} := (2\pi)^{-n/2} \int \langle\eta\rangle^\ell   \vert \widehat{a}(\eta,\cdot)\vert_k \d \eta.
\end{split}
\end{equation}
Hier bezeichnet $\widehat a(\eta,\xi)$ wiederum die Fouriertransformierte des Symbols bezüglich $x$. Weiter sei $\mathcal{C}_{k,\ell}$ der Abschluß der Symbolklasse in der $ |\cdot |_{k,\ell}$-Norm.
Im folgenden werden Symbole matrixwertig sein.

\medskip\noindent
{\em Zerlegung der Eins:} Sei im folgenden $\Theta\in\rmC_0^\infty(\R^n)$ mit $\supp\Theta\subset [-3/4,3/4]^n$ sowie $\Theta>0$ auf $[-1/2,1/2]^n$. Dann sind die Funktionen
\begin{equation}
\varphi_k(x) = \Theta(x-k) \bigg(\sum_{j\in\Z^n} \Theta(x-j)^2\bigg)^{-1/2}
\end{equation}
Elemente von $\rmC_0^\infty(\R^n)$ mit $\sum_{k} \varphi_k(x)^2=1$ und es existiert für jeden Multiindex $\alpha\in\N_0^n$ eine Konstante $C_\alpha$ mit
\begin{equation}
    \sum_{k\in\Z^n} | \D^\alpha \varphi_k(x)|^2 \le C_\alpha.
\end{equation}
Weiterhin impliziert $x,y\in\supp\varphi_k$ stets $|x-y|\le 3\sqrt n/2$. Ausgehend von dieser Partition der Eins definieren wir
\begin{equation}
    \psi_k(\xi)= \varphi_k(\xi / \sqrt{|\xi|}), 
\end{equation}
so daß
\begin{equation}\label{eq:6:6.35} 
    \sum_{k\in\Z^n} \psi_k(\xi)^2 = 1\qquad\text{und}\qquad |\xi|^\alpha \sum_{k\in\Z^n} |\partial_\xi^\alpha \psi_k(\xi)|^2 \le C_\alpha
\end{equation}
gilt. Weiter impliziert $\xi,\eta\in\supp\psi_k$ stets $|\sqrt{|\xi|}-\sqrt{|\eta|}|\le C$ und damit 
\begin{equation}\label{eq:6:6.36}
 |\xi-\eta|\le C \langle\xi\rangle^{1/2} \qquad\text{für alle $\xi,\eta\in\supp\psi_k$}.
\end{equation}
Weiterhin nützlich ist
\begin{equation}\label{eq:6:6.37}
  \sum_{k\in\Z^n} |\psi_k(\xi)-\psi_k(\eta)|^2 \le C \frac{|\xi-\eta|^2}{ \langle\xi\rangle^{1/2} \langle\eta\rangle^{1/2}},
\end{equation} 
was nur für $|\xi-\eta|\le \langle\xi\rangle^{1/2}$ nichttrivial ist und dann aus \eqref{eq:6:6.35} für $|\alpha|=1$ per Integration über die Strecke von $\xi$ zu $\eta$ folgt.

\medskip\noindent
{\em Positiv semidefinite Matrizen.}
Sei $A\in\C^{d\times d}$ selbstadjungiert und positiv semidefinit. Dann betrachten wir die zugeordneten quadratischen Formen\footnote{Zur Notation: Wir definieren $\cspro{u}{A}{v} = u^* A v=\overline u^\top A v$ für $A\in\C^{d\times d}$ und $u,v\in\C^{d}$.} $\cspro{v}A{v}$ auf $\C^d$ und  schreiben $A \preceq B$ für zwei solche Matrizen, falls 
\begin{equation}
   \forall_{v\in\C^d} \quad:\quad \cspro{v}A{v}\le \cspro{v}B{v}
\end{equation}
gilt. 

Eine erste Anwendung ist folgendes Lemma.

\begin{lem}[{\cite[Lemma 2.2]{Lax:1966}}]\label{Matrizenlemma}
	Seien $A,B\in\C^{d\times d}$ selbstadjungiert und es gelte $A\pm B \succeq 0$. Dann gilt
	\begin{equation}
		\forall_{v,w \in \mathbb{C}^d} \quad: \quad \vert \cspro{v}B{w} \vert \le \cspro{v}A{v}^{{1}/{2}}\cspro{w}A{w}^{{1}/{2}}. 
	\end{equation}
\end{lem}
\begin{proof}
	Nach Voraussetzung gilt
	\begin{equation}\begin{split}
		&\cspro{v+w}{A+B}{v+w} \ge 0,\\
		&\cspro{v-w}{A-B}{v-w} \ge 0.
		\end{split}
	\end{equation}
	Addieren der beiden Gleichungen liefert
	\begin{equation}
		\cspro{v}A{v}+2\Re \cspro{w}B{v} +\cspro{w}Aw \ge 0
	\end{equation}
	und damit nach Multiplikation von $w$ mit einer komplexen Zahl vom Betrag Eins (so dass $\Re \cspro{\lambda w}B{v} = - | \cspro{w}B{v}|$ gilt) die Behauptung. 
\end{proof}

\medskip\noindent
{\em Ein Hilfslemma.} Wir beginnen mit einer Hilfsaussage, welche auch für sich genommen von Interesse ist. 

\begin{lem}[{\cite[Lemma 3.1]{Lax:1966}}]\label{lem3.1:Lax}
Sei $p\in \mathcal{C}_{0,2}$ eine von $\xi$ unabhängige selbstadjungierte Matrix und sei weiter $\Phi: \mathbb{R}^n\rightarrow\mathbb{R}$ Lipschitz-stetig mit Lipschitzkonstante $K$ auf $\supp \widehat{u}$. Dann gilt
\begin{equation}
		| \Re \spro {\Phi(\D)pu}{\Phi(\D)u}-\Re\spro{p\Phi(\D)u}{\Phi(\D)u}| \le\frac{1}{2}| p|_{0,2}K^2|| u||^2.
\end{equation}	 
\end{lem}
\begin{proof}
	Die Rechenregeln der Fouriertransformation liefern
	\begin{equation}\label{lem3.1:Lax;1}
		\begin{split} J[u]&=  \Re \spro {\Phi(\D)pu}{\Phi(\D)u}-\Re\spro{p\Phi(\D)u}{\Phi(\D)u} \\
		&= \left(2\pi\right)^{-\frac{n}{2}}\Re\iint\cspro {\widehat{u}(\xi)}{\widehat{p}(\xi-\eta)}{\widehat{u}(\eta)}\left(\Phi(\xi)-\Phi(\eta)\right)\Phi(\xi)\d \xi\d\eta.
		\end{split}
	\end{equation}
	Weil $p$ selbstadjungiert ist, folgt $\widehat{p}(\xi) = {\widehat{p}}(-\xi)^*$. Durch Vertauschen der Integrationsvariablen erhalten wir
	\begin{equation}\label{lem3.1:Lax;2}
	\begin{split}	
	J[u]&= \left(2\pi\right)^{-\frac{n}{2}}\Re\iint \cspro{ \widehat{u}(\eta)}{\widehat{p}(\eta-\xi)}{\widehat{u}(\xi)}\left(\Phi(\eta)-\Phi(\xi)\right)\Phi(\eta)\d \xi\d\eta\\
		&= \left(2\pi\right)^{-\frac{n}{2}}\Re\iint\cspro{\widehat{u}(\xi)}{\widehat{p}(\xi-\eta)}{\widehat{u}(\eta)}\left(\Phi(\eta)-\Phi(\xi)\right)\Phi(\eta)\d \xi\d\eta.
	\end{split}	
	\end{equation}
	und Addition von $(\ref{lem3.1:Lax;1})$ und $(\ref{lem3.1:Lax;2})$ liefert
	\begin{equation}
		J[u]= \frac{1}{2}\left(2\pi\right)^{-\frac{n}{2}}\Re\iint \cspro{\widehat{u}(\xi)}{\widehat{p}(\xi-\eta)}{\widehat{u}(\eta)}\left(\Phi(\xi)-\Phi(\eta)\right)^2\d\xi\d\eta.
	\end{equation}
	Mit  $\left(\Phi(\xi)-\Phi(\eta)\right)^2\le K^2 |\xi-\eta|^2$ und $| \widehat{p}(\xi-\eta,\cdot)|_{0,2} = (2\pi)^{-\frac{n}{2}}\int \langle\xi-\eta\rangle^2 |\widehat{p}(\xi-\eta)|\d\eta$ sowie der Ungleichung von Cauchy--Schwarz folgt die Behauptung. 
\end{proof}

Damit kommen wir zum Hauptresultat dieses Abschnitts, der scharfen G\r{a}rdingungleichung für Systeme nach Lax und Nirenberg.

\begin{thm}[{\cite[Theorem 3.1]{Lax:1966}}]
Sei $a\in\mathcal{C}_{0,2}\cap\mathcal{C}_{2,0}$ selbstadjungiert und positiv-semidefinit. Dann erfüllt der zugehörige Operator $A$ die Ungleichung
\begin{equation}
	\Re \spro{Au}{u} \ge - K|| u||_{\left(-{1}/{2}\right)}^2.
\end{equation}
\end{thm}
\begin{proof}
Wir machen Gebrauch von der zu Beginn des Kapitels konstruierten Zerlegung der Eins. Sei dazu $	u_k:= \psi_k(\D)u$. Dann gilt $\widehat{u_k}(\xi) = \psi_k(\xi) \widehat{u}(\xi)$ und mit der Darstellung 
\begin{equation}
	\widehat{Au}(\xi) = \left(2\pi\right)^{-\frac{n}{2}}\int \widehat{a}(\xi-\eta,\eta)\widehat{u}(\eta)\d \eta
\end{equation}	 
folgt für die Differenz $J[u]$ der quadratischen Formen $\spro{Au}{u}$ und $\sum_k \spro{Au_k}{u_k}$
\begin{align}
 J[u]&=\spro{Au}{u}-\sum\limits_k \spro{Au_k}{u_k} \notag\\
& = \left(2\pi\right)^{-\frac{n}{2}}\iint \cspro{\widehat{u}(\xi)}{\widehat{a}(\xi-\eta,\eta)}{\widehat{u}(\eta)}\left(1-\sum\limits_k \psi_k(\eta)\psi_k(\xi)\right)\d \xi\d\eta\notag\\
	&= \frac{1}{2}\left(2\pi\right)^{-\frac{n}{2}} \iint \cspro{\widehat{u}(\xi)}{\widehat{a}(\xi-\eta,\eta)}{\widehat{u}(\eta)}\sum\limits_k\vert\psi_k(\eta)-\psi_k(\xi)\vert^2\d \xi\d\eta.
\end{align}
Damit folgt mit \eqref{eq:6:6.37} und erneutem Anwenden von Cauchy--Schwarz
\begin{equation}
\begin{split}	
| J[u] | &\le \frac{C}{2}\left(2\pi\right)^{-\frac{n}{2}}\iint| \cspro{\widehat{u}(\xi)}{\widehat{a}(\xi-\eta,\eta)}{\widehat{u}(\eta)}| \, |\xi-\eta|^2\langle \xi\rangle^{-\frac{1}{2}} \langle\eta\rangle^{-\frac{1}{2}}\d \xi \d \eta\\ 
& \le \frac{C}{2}| a|_{0,2}|| u||_{(-{1}/{2})}^2.
\end{split}
\end{equation}
Weiterhin gilt $\sum_k \|u_k\|^2_{(-1/2)} = \|u\|^2_{(-1/2)}$.
Damit ist es ausreichend, die gewünschte Ungleichung für die Funktionen $u_k$ einzeln (und mit von $k$ unabhängiger Konstanten) zu zeigen. Wir halten ein beliebiges $k$ fest und setzen im Folgenden
 $v= u_k$. Sei $\tau\in \supp \widehat{v}$ fixiert. Für $\mu, \eta \in \supp \widehat{v}$ gilt mit \eqref{eq:6:6.36}
\begin{equation}
	1+|\mu|^2 \le 1+\left(| \eta|+| \eta-\mu|\right)^2 \le 1+|\eta|^2+ c_1 \langle \mu \rangle,
\end{equation}
also
\begin{equation}
	\langle \mu\rangle \le c_2\langle \eta\rangle
\end{equation}
und somit
\begin{equation}\label{Abschaetzung 0-Norm}
	\langle \tau\rangle^{-1}|| v||^2 \lesssim|| v||_{\left(-{1}/{2}\right)}^2.
\end{equation}
Für $l=1,\ldots, n$ definieren wir $\widehat{a}^l(\xi-\eta,\tau)=\frac{\partial}{\partial \tau_l}\widehat{a}(\xi-\eta,\tau)$ und wenden den Mittelwertsatz an. Dies liefert für $\eta\in \supp \psi_k$ 
\begin{equation}
\begin{split}	&| \widehat{a}(\xi-\eta,\eta)-\widehat{a}(\xi-\eta,\tau) -\sum\limits_{l=1}^n (\eta_l-\tau_l)\widehat{a}^l\left(\xi-\eta,\tau\right)| \lesssim| \eta-\tau|^2| \widehat{a}(\xi-\eta,\cdot)|_2\langle\eta\rangle^{-2} \\&
\lesssim | \widehat{a}(\xi-\eta,\cdot)|_2 \langle\eta\rangle^{-1} \lesssim| \widehat{a}(\xi-\eta,\cdot)|_2\langle\eta\rangle^{-\frac{1}{2}}\langle\xi\rangle^{-\frac{1}{2}}.
\end{split}
\end{equation}
Setzen wir nun $p(x)=a(x,\tau)$, $p^l(x)= a^l(x,\tau) $ und $v^l:=(\D_l-\tau_l)v$ (also $\widehat{v}^l(\xi)=\left(\xi_l-\tau_l\right)\widehat{v}(\xi)$), dann folgt
\begin{equation}
	|\spro{Av}{v}-\spro{pv}{v}-\sum\spro{p^lv^l}{v}| \le c | a|_{2,0}|| u||_{\left(-{1}/{2}\right)}^2.
\end{equation}
Es verbleibt also, die Ungleichung 
\begin{equation}
	Q[v] \ge -C|| v||_{\left(-{1}/{2}\right)}^2
\end{equation}
für die quatratische Form
\begin{equation}
	Q[v]=\spro{pv}{v}+\Re\sum\spro{p^lv^l}{v} 
\end{equation}
zu zeigen. Diese entspricht der in $\xi$ linearen Approximation an das Symbol $a(x,\xi)$ in $\xi=\tau$.

Für jedes $l\in\{1,\dots, n\}$ gilt für $\xi = \pm c\langle\tau\rangle^{\frac{1}{2}}\e_l$ mit einem später geschickt gewählten $c$, dass
\begin{equation}
	a(x,\xi+\tau) = a(x,\tau)\pm\langle\tau\rangle^{\frac{1}{2}}a^l(x,\tau) + R(x,\xi,\tau). 
\end{equation}
Weil $a(x,\xi+\tau)$ positiv semi-definit ist, ist für eine geeignete Konstante $C$ auch
\begin{equation}
	a(x,\tau) \pm c\langle\tau\rangle^{\frac{1}{2}}a^l(x,\tau)+C\langle\tau\rangle \frac{| a(x,\cdot)|_{2}}{\langle\tau\rangle^2}\succeq 0
\end{equation}
positiv semi-definit. Da Addieren eines Terms der Form $c \langle\tau\rangle^{-1}$ zu $p(x)$ gemäß \eqref{Abschaetzung 0-Norm} einen Fehler der Ordnung $\mathcal O\left(\vert\vert v\vert\vert_{\left(-{1}/{2}\right)}^2\right)$ verursacht, können wir ohne Einschränkung annehmen, dass  
\begin{equation}
	p(x) \pm \langle\tau\rangle^{\frac{1}{2}}p^l(x) \succeq 0
\end{equation}
positiv semi-definit ist. Wir wenden darauf Lemma \ref{Matrizenlemma} an und erhalten
\begin{equation}
	c\langle\tau\rangle^{\frac{1}{2}} |\spro{p^lv}{v}|\le\spro{pv}{v}^{\frac{1}{2}}\spro{pv^l}{v^l}^{\frac{1}{2}}.
\end{equation}
Unter Ausnutzen der Ungleichung $\sqrt{ab}\le \frac{n}{2}a+\frac{1}{2n}b$ folgt
\begin{equation}
	|\spro{p^lv^l}{v}|\le \frac{1}{2n}\spro{pv}{v}+\frac{n}{2c^2}\frac{1}{\langle\tau\rangle}\spro{pv^l}{v^l},
\end{equation}
was
\begin{equation}
	Q[ v]=\spro{pv}{v}+\Re\sum\nolimits_l \spro{p^lv^l}{v} \ge \frac{1}{2}\spro{pv}{v} -\frac{n}{2c^2}\frac{1}{\langle\tau\rangle}\sum\nolimits_l \spro{pv^l}{v^l}
\end{equation}
impliziert. Lemma \ref{lem3.1:Lax}, angewandt auf $\Phi(\xi) = \xi_l-\tau_l$, liefert in Verbindung mit $(\ref{Abschaetzung 0-Norm})$
\begin{equation}
\sum\nolimits_l \spro{pv^l}{v^l}=\Re\sum\nolimits_l \spro{pv}{\left(\D_l-\tau_l\right)v}+\mathcal{O}\left(|| v||_{\left(-{1}/{2}\right)}^2\right) 
\end{equation}
und damit
\begin{equation}
	Q[v] \ge \frac{1}{2}\spro{pv}{v}-\frac{n}{2c^2}\frac{1}{\langle\tau\rangle}\Re\spro{pv}{| D-\tau|^2v)}+\mathcal{O}\left(|| v||_{\left(-{1}/{2}\right)}^2\right),
\end{equation}
was wir äquivalent ausdrücken durch
\begin{equation}
	2Q[v] \ge \Re\spro{pv}{\left(1-\frac{n}{c^2}\frac{| \D-\tau|^2}{\langle\tau\rangle}\right)v}+\mathcal{O}\left(|| v||_{\left(-{1}/{2}\right)}^2\right).
\end{equation}
Auf $\supp\widehat{v}$ gilt
\begin{equation}
	\frac{|\xi-\tau|^2}{\langle\tau\rangle}\le \tilde{C}^2 
\end{equation}
für eine Konstante $\tilde{C}>0$. Setzen wir nun $c:=\tilde{C}\sqrt{2n}$ und 
\begin{equation}
	\Phi(\xi):=\begin{cases}
		\left(1-\frac{n}{c^2}\frac{| \xi-\tau|^2}{\langle\tau\rangle}\right)^{\frac{1}{2}}, &| \xi-\tau| \le \tilde C(\langle\tau\rangle)^{\frac{1}{2}},\\
		\sqrt{\frac{1}{2}}, & \text{sonst},
	\end{cases}
\end{equation}
dann ist $\Phi(\xi)$ Lipschitz-stetig mit Lipschitz-Konstante $K=\mathcal O(\langle\tau\rangle)^{-\frac{1}{2}}$, erneute Anwendung von Lemma \ref{lem3.1:Lax} liefert
\begin{equation}
	2Q[v] \ge \Re\spro{pv}{\Phi(\D)^2v}+\mathcal{O}\left(|| v||_{\left(-{1}/{2}\right)}^2\right) = \Re\spro{p\Phi(\D)v}{\Phi(\D)v}+\mathcal{O}\left(|| v||_{\left(-{1}/{2}\right)}^2\right).
\end{equation}
Weil $a(x,\tau)$ positiv definit ist, folgt $\Re\spro{p\Phi(\D)v}{\Phi(\D)v}\ge 0$ und damit die gewünschte Ungleichung. 
\end{proof}

