% !TEX root = main.tex
\chapter{Die Ungleichung von G\r{a}rding}
\cite{Garding:1953}
\cite{Lax:1966}

\section{Dirichletformen}
Im folgenden sollen Dirichletformen 
\begin{equation}
   P(x,\D,\overline\D)[u,v] = \sum_{|\alpha|,|\beta|\le m} \int_\Omega p_{\alpha,\beta}(x) \big(\D^\alpha u(x)\big) \overline{\big(\D^\beta v(x)\big)} \d x,\qquad u,v\in\rmC_0^\infty(\Omega)
\end{equation}
zu eingem gegebenen Gebiet $\Omega\subset\R^n$ und für Koeffizienten $p_{\alpha,\beta}\in\rmC^\infty(\overline{\Omega})$ mit $p_{\beta,\alpha}(x)=\overline{p_{\alpha,\beta}(x)}$  betrachtet werden. Wir schreiben kurz $P(x,\D,\overline\D)[u]=P(x,\D,\overline\D)[u,u]$ und wenn klar ist, welche Form wir betrachten nur $P[u,v]$ beziehungsweise $P[u]$. Zugeordnet zu einer solchen Form betrachten wir das Symbol
\begin{equation}
   P(x,\zeta,\overline\zeta) = \sum_{\alpha,\beta} p_{\alpha,\beta}(x) \zeta^\alpha\overline\zeta^\beta,\qquad \zeta\in\C^n,
\end{equation}
und das zugeordnete Hauptsymbol
\begin{equation}
   p(x,\zeta,\overline\zeta) = \sum_{|\alpha|=|\beta|=m} p_{\alpha,\beta}(x) \zeta^\alpha\overline\zeta^\beta,\qquad \zeta\in\C^n.
\end{equation}
Auf Grund der Symmetriebedingung ist das Symbol $P(x,\xi,\xi)$ reellwertig f\"ur alle $\xi\in\R^n$, ebenso ist 
$P(x,\D,\overline \D)[u]$ reell. Spezialfälle solcher Dirichletformen sind die Innenprodukte des $\rmH^m_0(\Omega)$. Für diese schreiben wir kurz
\begin{equation}
  \spro{u}{v}_{(m)} = \sum_{|\alpha|\le m} \spro{\D^\alpha u}{\D^\alpha v} = \sum_{|\alpha|\le m} \int_\Omega   \big(D^\alpha u(x)\big) \overline{\big(\D^\alpha v(x)\big)} \d x.
\end{equation}
Es stellt sich die Frage, unter welchen Voraussetzungen eine Dirichletform äquivalent zu einem solchen Innenprodukt ist. Gilt $a_{\alpha,\beta}\in \rmB^\infty(\Omega)$, so ergibt sich stets eine obere Schranke der Form
\begin{equation}
    P[u] \le C \spro{u}{u}_{(m)}.
\end{equation}
Untere Schranken sind komplizierter. Wir nennen die Form (gleichmäßig) elliptisch, falls 
\begin{equation}
  \inf_{x\in\Omega}  \inf_{|\xi|=1} |p(x,\xi,\xi)| > 0.
\end{equation}
Unter dieser Voraussetzung zeigen wir, dass es Konstanten $a,b\in\R$ gibt, für welche
\begin{equation} 
   \spro{u}{u}_{(m)} \le a\, P[u] + b\, \spro{u}{u}
\end{equation}
gilt. Dies ist dazu äquivalent, dass der Quotient $P [u]  / \spro{u}{u}_{(m)}$ nach unten beschränkt ist.

\begin{thm}[\cite{Garding:1953}, Theorem 2.1]
Sei $P[u]$ eine gleichmäßig elliptische Dirichletform auf einem beschränkten Gebiet $\Omega\Subset\R^n$. Dann gilt
\begin{equation}
  \inf_{u\in\rmH_0^m(\Omega)} \frac{P[u]}{\spro{u}{u}_{(m)}} > -\infty.
\end{equation}
\end{thm}
\begin{proof}
{\sl Schritt 1.} Wir beginnen mit dem Spezialfall, dass $P(x,\xi,\xi) = P(\xi)$ unabhängig von $x$ und homogen vom Grad $2m$ in $\xi$ ist. Dann folgt mit dem Satz von Plancherel
\begin{equation}
 P[u] =  \sum_{|\alpha|=|\beta|=m} \int_\Omega  p_{\alpha,\beta} \big(\D^\alpha u(x)\big)\overline{\big(\D^\beta u(x)\big)} \d x
 =   \sum_{|\alpha|=|\beta|=m} \int  p_{\alpha,\beta} \xi^\alpha \xi^\beta  |\widehat u(\xi)|^2 \d\xi 
\end{equation}
mit $p_{\alpha,\beta}\in\C$ den Koeffizienten der Form und für beliebiges $u\in\rmC_0^\infty(\Omega)$. Da weiterhin die Elliptizitätsbedingung $p(\xi)\ge \tilde c |\xi|^{2m}$ 
vorausgesetzt ist und $\Omega$ beschränkt ist, gilt
\begin{equation}
  P[u] \ge \tilde c \int |\xi|^{2m} |\widehat u(\xi)|^2 \d\xi \ge  2c\; \spro{u}{u}_{(m)}
\end{equation}
mit einer geeigneten Konstanten $c>0$. $\bullet$\qquad{\sl Schritt 2.}  Wir führen den allgemeinen Fall auf den gerade gezeigten Spezialfall zurück. Dazu bezeichne im Folgenden
\begin{align}
	w(\rho) = \sup\left\{| p_{\alpha,\beta}(x)-p_{\alpha,\beta}(y)| : |\alpha|, |\beta| \le m, | x-y| \le \rho\right\}.
\end{align} 
Da nach Voraussetzung die Koeffizienten aus $\rmC^\infty(\overline\Omega)$ sind, sind sie insbesondere gleichmäßig stetig und somit gilt
$\lim_{\rho\to0} w(\rho)=0$.  Sei nun $x_0\in \Omega$ beliebig, $u\in \rmH_0^m(\Omega\cap \partial B_\rho(x_0))$ und bezeichne $p_0(x,\xi, \xi):=p(x_0,\xi,\xi)$ die Form mit in $x_0$ eingefrorenen Koeffizienten. Dann gilt
\begin{align}
\begin{split}
\big| P[u]-P_0[u] \big| &=\left| \sum\limits_{|\alpha|, |\beta| \le m} \int_\Omega \left(p_{\alpha,\beta}(x)-p_{\alpha,\beta}(x_0)\right){\D^\alpha(u(x)}\overline{\D^\beta u(x))}\mathrm{d}x\right|\\
	&\le w(\rho) \sum\limits_{|\alpha|,|\beta| \le m} \spro{\D^\alpha u}{\D^\beta u} \le  w(\rho) \sum\limits_{|\alpha|,|\beta| \le m}\|\D^\alpha u\|\, \|\D^\beta u\| \le
	 w(\rho) \|u\|_{(m)}^2
\end{split}
\end{align}
unter zweifacher Ausnutzung der Ungleichung von Cauchy--Schwarz. Also folgt unter Verwendung des oben betrachteten Spezialfalls
\begin{align}
	P[u] \ge P_0[u] - \big| P[u] - P_0[u] \big| \ge \big( 2c - w(\rho) \big) \|u\|_{(m)}^2 \ge  c \| u\|_{(m)}^2
\end{align} 
für obiges $c$ und alle $u\in \rmH_0^m(\Omega \cap \partial B_\rho(x_0))$,  $\rho$ hinreichend klein. Die Abschätzung ist gleichmäßig in $x_0\in\Omega$.
$\bullet$\qquad {\sl Schritt 3.} Da $\Omega$ beschränkt ist, finden wir zu jedem $\epsilon>0$ eine endliche Überdeckung $\Omega\subset\bigcup_{k=1}^N B_{{\epsilon}/{2}}(x_k)$ und eine dieser Überdeckung untergeordnete Zerlegung der Eins $\psi_k\in\rmC_0^\infty(B_{\epsilon/2}(x_k))$ mit
\begin{align}
 \sum\limits_{k=1}^N |\psi_k(x)|^2 = 1,\qquad \ x\in \Omega.
\end{align}
Sei nun $u\in\rmH_0^m(\Omega)$ beliebig. Dann gilt
\begin{equation}
\begin{split}
	P[u]&= \sum_{k=1}^N \sum\limits_{|\alpha|,|\beta|\le m} \int_\Omega |\psi_k(x)|^2p_{\alpha\beta}(x)\big(\D^\alpha u(x)\big)\overline{\big(\D^\beta u(x)\big)}\d x\\
	&=  \sum_{k=1}^N \bigg( \sum\limits_{|\alpha|,|\beta|\le m} \int_\Omega  p_{\alpha\beta}(x)\big(\D^\alpha \psi_k(x) u(x)\big)\overline{\big(\D^\beta \psi_k(x) u(x)\big)}\d x
	- R_k[u]\bigg),
\end{split}
\end{equation}
wobei der Restterm $R_k[u]$ eine Form der Ordnung $m$ mit verschwindendem Hauptsymbol\footnote{Zur Erinnerung: $r_k(x,\xi,\xi)=0$ heißt, daß im Integral nie ein Produkt von zwei Ableitungen der Ordnung $m$ steht.} ist. Es gilt also mit Cauchy--Schwarz $|R_k[u]|\le c_k \|u\|_{(m)} \|u\|_{(m-1)}$, während jeder Summand der äußeren Summe mit Schritt 2 abgeschätzt werden kann. Also folgt 
\begin{equation}\label{eq:6.16}
P[u] \ge \sum_{k=1}^N  \bigg( c \|\psi_k u\|_{(m)}^2 - c_k \|u\|_{(m)} \|u\|_{(m-1)} \bigg) .
\end{equation}
Weiter gilt 
\begin{equation}
   \|\psi_k u\|_{(m)}^2 =  \sum_{|\alpha|\le m} \int_\Omega |\D^\alpha (\psi_k(x) u(x)) |^2 \d x
   =   \sum_{|\alpha|\le m} \int_\Omega  |\psi_k(x)|^2  |\D^\alpha u(x) |^2 \d x + \tilde R_k[u]
\end{equation}
mit einer weiteren Form $\tilde R_k[u]$  der Ordnung $m$ und verschwindendem Hauptsymbol. Also gilt wiederum $|\tilde R_k[u]|\le b_k \|u\|_{(m)}\|u\|_{(m-1)}$ und  \eqref{eq:6.16} liefert mit $b= \sum_{k=1}^N (c_k + c b_k  )$
\begin{equation}
P[u] \ge c \|u\|_{(m)}^2 - b \|u\|_{(m)}\|u\|_{(m-1)} .
\end{equation}
Mit $\|u\|_{(m-1)}\le \|u\|_{(m)}$ folgt die Behauptung.
\end{proof}


\section{$\Psi$do}



\begin{lem}\label{lem3.1:Lax}
	Sei das Symbol $p(x)$ eine von $\xi$ unabhängige hermitesche Matrix, $p\in \mathcal{C}_{0,2}$. Ist $\Phi: \mathbb{R}^n\rightarrow\mathbb{R}, \ \xi \mapsto\Phi(\xi)$ Lipschitz-stetig mit Lipschitzkonstante $K$ auf $\supp \widehat{u}$, dann gilt
	\begin{align}
		| \Re \spro {\Phi(\D)pu}{\Phi(\D)u}-\Re\spro{p\Phi(D)u}{\Phi(d)u}| \le\frac{1}{2}| p|_{0,2}K^2|| u||_0^2.
\end{align}	 
\end{lem}
\begin{proof}
	Die Rechenregeln der Fouriertransformation liefern
	\begin{align}\label{lem3.1:Lax;1}
		\begin{split} J:&=  \Re \spro {\Phi(\D)pu}{\Phi(\D)u}-\Re\spro{p\Phi(D)u}{\Phi(d)u} \\
		&=\Re \left( \spro{\Phi(\cdot)\left(\widehat{u}\ast \widehat{p}\right)}{\Phi(\cdot)\widehat{u}}-\spro{\widehat{p}\ast\widehat{\Phi(\cdot)u}}{\Phi(\cdot)\widehat{u}}\right)\\
		&= \left(2\pi\right)^{-\frac{n}{2}}\Re\int\int\widehat{p}(\xi-\eta)\widehat{u}(\eta)\widehat{u}(\xi)\left(\Phi(\xi)-\Phi(\eta)\right)\Phi(\xi)\d \xi\d\eta.
		\end{split}
	\end{align}
	Weil $p$ hermitesch ist, folgt $\widehat{p}(\xi) = \overline{\widehat{p}}(-\xi)$. Durch Vertauschen der Integrationsvariablen erhalten wir also
	\begin{align}\label{lem3.1:Lax;2}
	\begin{split}	J&= \left(2\pi\right)^{-\frac{n}{2}}\Re\int\int\overline{\widehat{p}}(\eta-\xi)\widehat{u}(\eta)\widehat{u}(\xi)\left(\Phi(\eta)-\Phi(\xi)\right)\Phi(\eta)\d \xi\d\eta\\
		&= \left(2\pi\right)^{-\frac{n}{2}}\Re\int\int\widehat{p}(\xi-\eta)\widehat{u}(\eta)\widehat{u}(\xi)\left(\Phi(\eta)-\Phi(\xi)\right)\Phi(\eta)\d \xi\d\eta.
	\end{split}	
	\end{align}
	Addieren von $(\ref{lem3.1:Lax;1})$ und $(\ref{lem3.1:Lax;2})$ liefert
	\begin{align}
		J= \frac{1}{2}\left(2\pi\right)^{-\frac{n}{2}}\Re\int\int \widehat{p}(\xi-\eta)\widehat{u}(\eta)\widehat{u}(\xi)\left(\Phi(\xi)-\Phi(\eta)\right)^2\d\xi\d\eta.
	\end{align}
	Mit $| \widehat{p}(\xi-\eta,\cdot)|_0 = | \widehat{p}(\xi-\eta)|$ und der Ungleichung von Cauchy-Schwarz folgt die Behauptung. 
\end{proof}

\begin{thm}
Sei $a(x,\xi)\in\mathcal{C}_{0,2}\cap\mathcal{C}_{2,0}$ hermitesch und positiv-semidefinit. Dann erfüllt der Operator $A$ die Ungleichung
\begin{align}
	\Re \spro{Au}{u} \ge - K|| u||_{\left(-\frac{1}{2}\right)}^2.
\end{align}

\end{thm}
\begin{proof}
	Wir machen Gebrauch von der Zerlegung der Eins und lokalisieren wie im eindimensionalen Fall: Sei
	\begin{align}
		u_k:= \psi_k(\D)u, \text{ d.h. } \widehat{u_k}(\xi) = \psi_k(\xi) \widehat{u}(\xi).
\end{align}
Wegen 
\begin{align}
	\widehat{Au}(\xi) = \left(2\pi\right)^{-\frac{n}{2}}\int \widehat{a}(\xi-\eta,\eta)\widehat{u}(\eta)\d \eta
\end{align}	 
folgt
\begin{align}
	\begin{split} I:&=\spro{Au}{u}-\sum\limits_k \spro{Au_k}{u_k} = \left(2\pi\right)^{-\frac{n}{2}}\int\int \widehat{a}(\xi-\eta,\eta)\widehat{u}(\eta)\widehat{u}(\xi)\big(1-\sum\limits_k \psi_k(\eta)\psi_k(\xi)\big)\d \xi\d\eta\\
	&= \frac{1}{2}\left(2\pi\right)^{-\frac{n}{2}} \int\int \widehat{a}(\xi-\eta,\eta)\widehat{u}(\eta)\widehat{u}(\xi)\sum\limits_k\big(\psi_k(\eta)-\psi_k(\xi)\big)^2\d \xi\d\eta,
\end{split}
\end{align}
denn $\sum\limits_k\big(\psi_k(\eta)-\psi_k(\xi)\big)^2=\sum\limits_k\psi_k(\eta)^2-2\psi_k(\xi)\psi_k(\eta)+\psi_k(\xi)^2=2-2\sum\limits_k\psi_k(\xi)\psi_k(\eta).$
Damit folgt
\begin{align}
\begin{split}	&| I| \overset{\text{Zerlegung}}{\le}\frac{1}{2}\int\int|\widehat{a}(\xi-\eta,\eta)\widehat{u}(\eta)\widehat{u}(\xi)| C |\xi-\eta|^2\left(1+| \xi|\right)^{-\frac{1}{2}} \left(1+| \eta|\right)^{-\frac{1}{2}}\d \xi \d \eta\\ 
& \le \frac{1}{2}| a|_{0,2}|| u||_{-\frac{1}{2}}^2
\end{split}.
\end{align}
Damit ist es ausreichend, die gewünschte Ungleichung für die Funktionen $u_k$ zu zeigen. Wir halten ein beliebiges $k$ fest und setzen $v:= u_k.$ Sei $\tau\in \supp \widehat{v}.$ Für $\mu, \eta \in \supp \widehat{v}$ gilt (nach Konstruktion der Z.d.E.)
\begin{align}
	1+|\mu| \le 1+| \eta|+| \eta-\mu| \le 1+|\eta|+c_1(1+|  \mu |)^{\frac{1}{2}},
\end{align}
also
\begin{align}
	1+| \mu| \le c_2\left(1+|\eta|\right)
\end{align}
und somit
\begin{align}\label{Abschaetzung 0-Norm}
	\left(1+|\tau|\right)^{-1}|| v||_0^2 \le C_3|| v||_{\left(-\frac{1}{2}\right)}^2
\end{align}
Wir setzen $\widehat{a}^l(\xi-\eta,\tau):=\frac{\partial}{\partial \tau_l}\widehat{a}(\xi-\eta,\tau)$ und wenden darauf den Mittelwertsatz an. Für $\eta\in \supp \psi_k$ ergibt sich dann
\begin{align}
\begin{split}	&| \widehat{a}(\xi-\eta,\eta)-\widehat{a}(\xi-\eta,\tau) -\sum\limits_{l=1}^n (\eta_l-\tau_l)\widehat{a}^l\left(\xi-\eta,\tau\right)| \le C_4| \eta-\tau|^2| \widehat{a}(\xi-\eta,\cdot)|_2\left(1+|\eta|\right)^{-2} \\&
\le | \widehat{a}(\xi-\eta,\cdot)|_2 \left(1+|\eta|\right)^{-1} \le| \widehat{a}(\xi-\eta,\cdot)|_2\left(1+|\eta|\right)^{-\frac{1}{2}}\left(1+|\xi|\right)^{-\frac{1}{2}}.
\end{split}
\end{align}
Setzen wir $p(x):=a(x,\tau),\ p^l(x):= a^l(x,\tau) $ und $v^l:=(D_l-\tau_l)v$ (also $\widehat{v}^l=\left(\xi_l-\tau_l\right)\widehat{v}$, dann folgt weiter
\begin{align}
	|\spro{Av}{v}-\spro{p(x)v}{v}-\sum\spro{p^l(x)v^l}{v}| \le c | a|_{2,0}|| u||_{\left(-\frac{1}{2}\right)}^2
\end{align}
Damit verbleibt es, die Ungleichung \begin{align}
	Q \ge -C|| v||_{\left(-\frac{1}{2}\right)}^2
\end{align}
für die quatratische Form
\begin{align}
	Q=\spro{pv}{v}+\Re\sum\spro{p^lv^l}{v} 
\end{align}
zu zeigen.\\
Für jedes $l\in\{1,\cdots n\}$ gilt für ein $c$ (das wir später wählen) und für $\xi = \pm c(1+| \tau|)^{\frac{1}{2}}e_l$

\begin{align}
	a(x,\xi+\tau) = a(x,\tau)\pm(1+|\tau|)^{\frac{1}{2}}a^l(x,\tau) + R(x,\xi,\tau)
\end{align}
Weil $a(x,\xi+\tau)$ positiv definit ist, folgt dass für eine geeignete Konstante $C_1$ auch
\begin{align}
	a(x,\tau) \pm c(1+|\tau|)^{\frac{1}{2}}a^l(x,\tau)+C_1(1+| \tau|)\frac{| a(x,\cdot|_{2}}{\left(1+| \tau|\right)^2}\ge 0
\end{align}
positiv definit ist.\\
ToDO\\
Ohne Einschränkung sei 
\begin{align}
	p(x) \pm (1+| \tau|)^{\frac{1}{2}}p^l(x) \ge 0
\end{align}
positiv definit. Dann gilt nach (ToDo: Lemma 2.2)
\begin{align}
	c\left(1+|\tau|\right)^{\frac{1}{2}} |\spro{p^lv^}{v}|\le\spro{pv}{v}^{\frac{1}{2}}\spro{pv^l}{v^l}^{\frac{1}{2}}.
\end{align}
Wenden wir darauf die Ungleichung $\sqrt{ab}\le \frac{n}{2}a+\frac{1}{2n}b$ an, so folgt
\begin{align}
	|\spro{p^lv^l}{v}|\le \frac{1}{2n}\spro{pv}{v}+\frac{n}{2c^2}\frac{1}{1+|\tau|}\spro{pv^l}{v^l},
\end{align}
was 
\begin{align}
	Q=\spro{pv}{v}+\Re\sum\spro{p^lv^l}{v} \ge \frac{1}{2}\spro{pv}{v} -\frac{n}{2c^2}\frac{1}{1+| \tau|}\sum\spro{pv^l}{v^l}
\end{align}
impliziert. Lemma \ref{lem3.1:Lax}, angewandt auf $\Phi(\xi) = \xi_l-\tau_l$, liefert in Verbindung mit $(\ref{Abschaetzung 0-Norm})$
\begin{align}
\sum\spro{pv^l}{v^l}=\Re\sum \spro{pv}{\left(D_l-\tau_l\right)v}+\mathcal{O}\left(|| v||_{\left(-\frac{1}{2}\right)}^2\right) 
\end{align}
und damit
\begin{align}
	Q \ge \frac{1}{2}\spro{pv}{v}-\frac{n}{2c^2}\frac{1}{1+| \tau|}\Re\spro{pv}{| D-\tau|^2v)}+\mathcal{O}\left(|| v||_{\left(-\frac{1}{2}\right)}^2\right)
\end{align}
oder äquivalent
\begin{align}
	2Q \ge \Re\spro{pv}{\left(1-\frac{n}{2c^2}\frac{| D-\tau|^2}{1+|\tau|}\right)v}+\mathcal{O}\left(|| v||_{\left(-\frac{1}{2}\right)}^2\right).
\end{align}
Auf $\supp\widehat{v}$ gilt
\begin{align}
	\frac{|\xi-\tau|^2}{1+| \tau|}\le \tilde{C}^2 
\end{align}
für eine Konstante $\tilde{C}>0$. Setzen wir nun $c:=\tilde{C}\sqrt{n}$ und 
\begin{align}
	\Phi(\xi):=\begin{cases}
		\left(1-\frac{n}{2c^2}\frac{| \xi-\tau|^2}{1+|\tau|}\right)^{\frac{1}{2}}, &| \xi-\tau| \le C(1+|\tau|)^{\frac{1}{2}},\\
		\sqrt{\frac{1}{2}}, & \text{sonst},
	\end{cases}
\end{align}
Dann ist $\Phi(\xi)$ Lipschitz-stetig mit Lipschitz-Konstante $K=\mathcal O(1+| \tau|)^{-\frac{1}{2}}$, erneute Anwendung von Lemma \ref{lem3.1:Lax} liefert
\begin{align}
	2Q \ge \Re\spro{pv}{\Phi(D)^2v}++\mathcal{O}\left(|| v||_{\left(-\frac{1}{2}\right)}^2\right) = \Re\spro{p\Phi(D)v}{\Phi(D)v}+\mathcal{O}\left(|| v||_{\left(-\frac{1}{2}\right)}^2\right).
\end{align}
Weil $a(x,\tau)$ positiv definit ist, folgt $\Re\spro{p\Phi(D)v}{\Phi(D)v}\ge 0$ und damit die gewünschte Ungleichung. 
\end{proof}

