% !TEX root = main.tex
\chapter{Die Ungleichung von G\r{a}rding}
\cite{Garding:1953}
\cite{Lax:1966}

\section{Dirichletformen}
Im folgenden sollen Dirichletformen 
\begin{equation}
   P(x,\D,\overline\D)[u,v] = \sum_{|\alpha|,|\beta|\le m} \int_\Omega a_{\alpha,\beta}(x) \big(\D^\alpha u(x)\big) \overline{\big(\D^\beta v(x)\big)} \d x,\qquad u,v\in\rmC_0^\infty(\Omega)
\end{equation}
zu eingem gegebenen Gebiet $\Omega\subset\R^n$ und für Koeffizienten $a_{\alpha,\beta}\in\rmC^\infty(\Omega)$ mit $a_{\beta,\alpha}(x)=\overline{a_{\alpha,\beta}(x)}$  betrachtet werden. Wir schreiben kurz $P(x,\D,\overline\D)[u]=P(x,\D,\overline\D)[u,u]$ und wenn klar ist, welche Form wir betrachten nur $P[u]$. Zugeordnet zu einer solchen Form betrachten wir das Symbol
\begin{equation}
   P(x,\zeta,\overline\zeta) = \sum_{\alpha,\beta} p_{\alpha,\beta}(x) \zeta^\alpha\overline\zeta^\beta,\qquad \zeta\in\C^n,
\end{equation}
und das zugeordnete Hauptsymbol
\begin{equation}
   p(x,\zeta,\overline\zeta) = \sum_{|\alpha|=|\beta|=m} p_{\alpha,\beta}(x) \zeta^\alpha\overline\zeta^\beta,\qquad \zeta\in\C^n.
\end{equation}
Auf Grund der Symmetriebedingung ist das Symbol $P(x,\xi,\xi)$ reellwertig f\"ur alle $\xi\in\R^n$, ebenso ist 
$P(x,\D,\overline \D)[u]$ reell. Spezialfälle solcher Dirichletformen sind die Innenprodukte des $\rmH^m_0(\Omega)$. Für diese schreiben wir kurz
\begin{equation}
  \spro{u}{v}_{(m)} = \sum_{|\alpha|\le m} \spro{\D^\alpha u}{\D^\alpha v} = \sum_{|\alpha|\le m} \int_\Omega   \big(D^\alpha u(x)\big) \overline{\big(\D^\alpha v(x)\big)} \d x.
\end{equation}
Es stellt sich die Frage, unter welchen Voraussetzungen eine Dirichletform äquivalent zu einem solchen Innenprodukt ist. Gilt $a_{\alpha,\beta}\in \rmB^\infty(\Omega)$, so ergibt sich stets eine obere Schranke der Form
\begin{equation}
    P(x,\D,\overline\D)[u] \le C \spro{u}{u}_{(m)}.
\end{equation}
Untere Schranken sind komplizierter. Wir nennen die Form (gleichmäßig) elliptisch, falls 
\begin{equation}
  \inf_{x\in\Omega}  \inf_{|\xi|=1} |p(x,\xi,\xi)| > 0.
\end{equation}
Unter dieser Voraussetzung zeigen wir, dass es Konstanten $c,t\in\R$ gibt, für welche
\begin{equation} 
   \spro{u}{u}_{(m)} \le c P(x,\D,\overline \D) [u] + t \spro{u}{u}
\end{equation}
gilt. Auf beschränkten Gebieten ist dies äquivalent dazu, dass der Qutient $P(x,\D,\overline \D) [u]  / \spro{u}{u}_{(m)}$ nach unten beschränkt ist.

\begin{thm}[\cite{Garding:1953}, Theorem 2.1]
Sei $p(f,f)$ ein zu $p$ gehörendes Dirichletintegral (ToBeDefined). Dann gilt 
\begin{align}
	\inf\limits_{f \in H_0^m(\Omega)} \frac{p(f,f)}{(f,f)} > -\infty.
\end{align}
\end{thm}
\begin{proof}
	Wir beginnen mit dem Spezialfall, dass $p(x,\xi) = p(\xi)$ unabhängig von $x$ ist. Die Fourier-Transformierte von $\D^\alpha f$ ist dann von der Form
	\begin{align}
		\widehat{\D^\alpha f} = (-i)^m\xi^\alpha \hat{f}, 
	\end{align}
	d.h. mit Plancherel ergibt sich
	\begin{align}
		(f,f)_m = (2\pi)^{-n}\int \vert\hat{f}(x)\vert^2 \left(\sum\limits_{\vert \beta \vert = m} \xi_\beta^2\right)\mathrm{d}\xi.
	\end{align}
	Da $p$ homogen vom Grad $2m$ und positiv definit ist, existiert eine Konstante $c>0$ mit $\inf_{\vert \xi \vert =1} p(\xi)/(\sum  \xi_\beta^2) = c >0$. Für $p(f,f)$ folgt
	\begin{align}
		\begin{split}&p(f,f) = \int\limits_\Omega \sum\limits_{\vert\alpha\vert = \vert \beta\vert =m}p_{\alpha\beta}\D^\alpha f(x)\overline{\D^\beta f(x)}\mathrm{d}x =\left(2\pi\right)^{-n}\int\vert \mathcal{F}(\xi)\vert^2\sum\limits_{\vert\alpha\vert =\vert\beta\vert = m}p_{\alpha\beta}\xi^{\alpha}\xi^\beta\mathrm{d}\xi\\
& = \left(2\pi\right)^{-n} \int \vert \mathcal{F}(\xi)\vert^2 p(\xi)\mathrm{d}\xi \ge c \left(2\pi\right)^{-n}\int\vert\mathcal{F}(\xi)\vert^2  \left(\sum\limits_{\vert \beta\vert=m} \vert\xi^\beta\vert^2\right)		= c (f,f)_m.
\end{split}
\end{align}	
Den allgemeinen Fall führen wir auf diesen Spezialfall zurück. Im Folgenden bezeichne
\begin{align}
	w(\rho):= \sup\left\{\vert p_{\alpha\beta}(x)-p_{\alpha\beta}(y)\vert : \vert\alpha\vert, \vert\beta\vert \le m, \vert x-y\vert \le \rho\right\}.
\end{align} 
Sei $x_0\in \Omega$ beliebig, $f\in H_0^1\left(\Omega\cap \partial B_\rho(x_0)\right)$ und $p_0(x, \xi):=p(x_0,\xi)$ unabhängig von $x$. Dann gilt
\begin{align}
\begin{split}
	&\vert p(f,f)-p_0(f,f)\vert=\left\vert \int\limits_\Omega\sum\limits_{\vert\alpha\vert=\vert\beta\vert =m} \left(p_{\alpha\beta}(x)-p_{\alpha\beta}(x_0)\right){\D^\alpha(f(x)}\overline{D^\beta f(x))}\mathrm{d}x\right\vert\\
	&\le \frac{1}{2}w(\rho)\int\limits_\Omega \sum\limits_{\vert\alpha\vert=\vert\beta\vert =m}  \vert \D^\alpha f(x)\vert^2+\vert \D^\beta f<(x) \vert^2 \mathrm{d}x = w(\rho)d_m\vert f\vert_m^2,
	\end{split}
\end{align}
wobei $d_m$ die Anzahl der Ableitungen der Ordnung $m$ bezeichne. Damit folgt unter Verwendung des oben betrachteten Spezialfalls
\begin{align}
	p(f,f) \ge p_0(f,f)-\vert p(f,f)-p_0(f,f)\vert \ge (2c-w(\rho)d_m)\vert f\vert_m^2
\end{align} 
für ein $c>0$ und alle $f\in H_0^m\left(\Omega \cap \partial B_\rho(x_0)\right)$. Aus der gleichmäßigen Stetigkeit der $p_{\alpha\beta}$ folgt $w(\rho) \rightarrow 0, \ (\rho \rightarrow 0)$ und damit 
\begin{align}
	p(f,f) \ge c\vert f\vert_m^2
\end{align}
für $\rho$ klein genug.\\
Da $\Omega$ beschränkt ist, finden wir zu jedem $\epsilon>0$ eine endliche Überdeckung $\Omega=\bigcup_{k=1}^N B_{\frac{\epsilon}{2}}(x_k)$ und eine dieser Überdeckung untergeordnete Zerlegung der Eins $\left\{\psi_k\right\}$ mit
\begin{align}
 \psi_k \in H_0^m\left(B_\epsilon (x_k)\right),\  1=\sum\limits_{k=1}^N \vert h_k(x)\vert^2, \ x\in \Omega.
\end{align}
Jedes $f\in H_0^m(\Omega)$ lässt sich dann schreiben als
\begin{align}
	p(f,f) = \sum\limits_{k=1}^N\underbrace{\int\limits_\Omega \sum\limits_{\vert\alpha\vert=\vert\beta\vert=m}\vert h_k(x)\vert^2p_{\alpha\beta}(x)\D^\alpha f(x)\overline{D^\beta f(x)}\mathrm{d}x}_{=:A_k}.
\end{align}
Für $A_k$ gilt
\begin{align}
	A_k= \int\limits_\Omega \sum\limits_{\vert\alpha\vert=\vert\beta\vert=m}p_{\alpha\beta}(x)\D^\alpha(h_k(x)f(x))\overline{\D^\beta(h_k(x)f(x))}\d x -R_k=p(h_kf,h_kf)-R_k,
\end{align}
wobei $R_k$ eine Summe von Integralen über Produkte einer beschränkten Funktion mit zwei Ableitungen von $f$ der Ordnung $\le m$ (und nicht beide der Ordnung $m$) darstellt.
Wir schätzen $R_k$ nach oben ab und erhalten
\begin{align}
 A_k \ge p(h_kf,h_kf)-a_k\vert f\vert_m\vert f\vert_{m-1} \ge c \vert h_kf\vert_m^2-a_k\vert f\vert_m\vert f\vert_{m-1}
\end{align}
für geeignet gewählte Konstanten $a_k$. Weil außerdem
\begin{align}
\begin{split}
	&\vert h_kf\vert_m^2 = \int\limits_\Omega \sum\limits_{\vert\alpha\vert=\vert\beta\vert=m}p_{\alpha\beta}\D^\alpha(h_kf(x))\overline{D^\beta(h_kf(x))}\d x \\
	&\ge \int\limits_\Omega \vert h_k(x)\vert^2 \sum\limits_{\vert\alpha\vert=m}\vert \D^\alpha (f(x))\vert^2 \d x-b_k \vert f\vert_m\vert f\vert_{m-1}
	\end{split}
\end{align}
gilt, lässt sich $A_k$ weiter abschätzen zu
\begin{align}
	A_k \ge c\int\limits_\Omega \vert h_k(x)\vert^2 \sum\limits_{\vert\alpha\vert=m}\vert \D^\alpha (f(x))\vert^2 \d x-(cb_k+a_k)\vert f\vert_m\vert f\vert_{m-1}.
\end{align}
Aufsummieren der $A_k$ liefert
\begin{align}
	p(f,f) \ge c \sum\limits_{k=1}^N \int\limits_\Omega\vert h_k(x)\vert^2  \sum\limits_{\vert\alpha\vert=m}\vert \D^\alpha (f(x))\vert^2 \d x-a\vert f\vert_m\vert f\vert_{m-1} =c\vert f\vert_m^2 -a \vert f\vert_m\vert f\vert_{m-1},
\end{align}
wobei $a= \sum_{k=1}^N (a_k+cb_k)$. Für $t\in \R$ und $f\in H_0^m(\Omega)$ definieren wir $\vert f\vert^2:= c \vert f\vert_m^2+t\vert f\vert_0^2$ sowie
\begin{align}
\begin{split}
	&p_t(f,f) := p(f,f)+t(f,f) \ge c\vert f\vert_m^2-a\vert f\vert_m\vert f\vert_{m-1}+t\vert f\vert_0^2 \ge \vert f\vert^2\left(1-a\vert f\vert^2\vert f\vert_m\vert f\vert_{m-1}\right)\\
	& \ge \vert f\vert^2\big(1-\underbrace{a \sqrt{c}\vert f\vert_{m-1}\vert f\vert^{-1}}_{\rightarrow 0, \ t \rightarrow \infty, \text{ glm in } f}\big)
	\end{split}
\end{align}
Damit folgt $p_t(f,f) \ge \frac{1}{2}\left(c\vert f\vert_m^2+t\vert f\vert_0^2\right)$ für große $t$ oder äquivalent
\begin{align}
	p(f,f) \ge \frac{1}{2}c\vert f\vert_m^2
\end{align}
gleichmäßig in $f$. 
\end{proof}


\section{$\Psi$do}



\begin{lem}
	Sei das Symbol $p(x)$ eine von $\xi$ unabhängige hermitesche Matrix, $p\in \mathcal{C}_{0,2}$. Ist $\Phi: \mathbb{R}^n\rightarrow\mathbb{R}, \ \xi \mapsto\Phi(\xi)$ Lipschitz-stetig mit Lipschitzkonstante $K$ auf $\supp \widehat{u}$, dann gilt
	\begin{align}
		\vert \Re \spro {\Phi(\D)pu}{\Phi(\D)u}-\Re\spro{p\Phi(D)u}{\Phi(d)u}\vert \le\frac{1}{2}\vert p\vert_{0,2}K^2\vert\vert u\vert\vert_0^2.
\end{align}	 
\begin{proof}
	Die Rechenregeln der Fouriertransformation liefern
	\begin{align}\label{lem3.1:Lax;1}
		\begin{split} J:&=  \Re \spro {\Phi(\D)pu}{\Phi(\D)u}-\Re\spro{p\Phi(D)u}{\Phi(d)u} \\
		&=\Re \left( \spro{\Phi(\cdot)\left(\widehat{u}\ast \widehat{p}\right)}{\Phi(\cdot)\widehat{u}}-\spro{\widehat{p}\ast\widehat{\Phi(\cdot)u}}{\Phi(\cdot)\widehat{u}}\right)\\
		&= \left(2\pi\right)^{-\frac{n}{2}}\Re\int\int\widehat{p}(\xi-\eta)\widehat{u}(\eta)\widehat{u}(\xi)\left(\Phi(\xi)-\Phi(\eta)\right)\Phi(\xi)\d \xi\d\eta.
		\end{split}
	\end{align}
	Weil $p$ hermitesch ist, folgt $\widehat{p}(\xi) = \overline{\widehat{p}}(-\xi)$. Durch Vertauschen der Integrationsvariablen erhalten wir also
	\begin{align}\label{lem3.1:Lax;2}
	\begin{split}	J&= \left(2\pi\right)^{-\frac{n}{2}}\Re\int\int\overline{\widehat{p}}(\eta-\xi)\widehat{u}(\eta)\widehat{u}(\xi)\left(\Phi(\eta)-\Phi(\xi)\right)\Phi(\eta)\d \xi\d\eta\\
		&= \left(2\pi\right)^{-\frac{n}{2}}\Re\int\int\widehat{p}(\xi-\eta)\widehat{u}(\eta)\widehat{u}(\xi)\left(\Phi(\eta)-\Phi(\xi)\right)\Phi(\eta)\d \xi\d\eta.
	\end{split}	
	\end{align}
	Addieren von $(\ref{lem3.1:Lax;1})$ und $(\ref{lem3.1:Lax;2})$ liefert
	\begin{align}
		J= \frac{1}{2}\left(2\pi\right)^{-\frac{n}{2}}\Re\int\int \widehat{p}(\xi-\eta)\widehat{u}(\eta)\widehat{u}(\xi)\left(\Phi(\xi)-\Phi(\eta)\right)^2\d\xi\d\eta.
	\end{align}
	Mit $\vert \widehat{p}(\xi-\eta,\cdot)\vert_0 = \vert \widehat{p}(\xi-\eta)\vert$ und der Ungleichung von Cauchy-Schwarz folgt die Behauptung. 
\end{proof}
\end{lem}


