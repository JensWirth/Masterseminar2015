\documentclass[11pt,bibtotoc]{scrbook}
\usepackage{german}
\usepackage[utf8]{inputenc}
%
\usepackage{hyperref}
%
\usepackage{wasysym}\let\iint\relax\let\iiint\relax
\usepackage{amsfonts,amssymb,amsthm,amsmath,CJK,mathrsfs}
\usepackage[a4paper, scale=0.76]{geometry}
%
\usepackage{ifthen}
\usepackage{makeidx}
\makeindex
\newcommand{\eIndex}[2][]{%
\emph{#2}%
\ifthenelse{\equal{#1}{}}{\index{#2}}{\index{#1!#2}}%	
}%
\def\spro#1#2{\pmb(#1,#2\pmb)}
\def\abs#1{\lvert#1\rvert}
\def\Abs#1{\left\lvert#1\right\rvert}
\def\norm#1{\lVert#1\rVert}
\def\Norm#1{\left\lVert#1\right\rVert}
%
% Aufzaehlungen roemisch numeriert....
\renewcommand{\labelenumi}{{\rm\bf(\roman{enumi})}}
%
\newtheorem{thm}{Satz}[chapter]
\newtheorem{lem}[thm]{Lemma}
\newtheorem{cor}[thm]{Korollar}
\renewcommand{\proofname}{Beweis}
\theoremstyle{definition}
\newtheorem{df}[thm]{Definition}
\newtheorem{exa}{Beispiel}[chapter]
\newtheorem{rem}[thm]{Bemerkung}
%
%
% DEFINITIONEN
\def\N{\mathbb N}    % N
\def\R{\mathbb R}    % Koerper R
\def\C{\mathbb C}    % Koerper C
\def\e{\mathrm e}     % Eulersche Zahl e
\def\i{\mathrm i}       % imaginäre Einheit i
\def\D{\mathrm D}    % Differentialoperator D
\def\d{\,\mathrm d}   % Differential d für Integrale
\def\rmC{\mathrm C}% Raum C
\def\rmL{\mathrm L} % Raum L
\def\rmH{\mathrm H}% Raum H
\def\mcP{\mathcal P}   % curly P
\def\mcQ{\mathcal Q}  % curly Q  
\def\mcR{\mathcal R}  % curly R
\let\epsilon\varepsilon
%% von Tillmann
% NOT TO BE USED, WILL BE REMOVED LATER
%\def\pr{\prime}
%\def\tr{\top}%{\mathrm{T}}
%\def\al{\alpha}
%\def\be{\beta}
%\def\de{\delta}
%\def\ep{\epsilon}
%\def\ga{\gamma}
%\def\x{\xi}
%\def\y{\eta}
%\def\z{\zeta}
%\def\Om{\Omega}
%\def\pd{\partial}
%\def\infi{\infty}
\def\til#1{\widetilde{#1}}
%\def\mto{\mapsto}
%\def\ti{\times}
%\def\Lap{\Delta}
\def\cc#1{\overline{#1}}
\def\ul#1{\underline{#1}}
\def\ska#1#2{\langle#1,#2\rangle}
\def\Ska#1#2{\left\langle#1,#2\right\rangle}
%\def\subs{\subset}
%\def\Lin{\operatorname{span}}
%\def\Grad{\operatorname{deg}}
\def\gdw{\Leftrightarrow}
%%%%%%%%%%%%%%%%%%%%
%
\DeclareMathOperator{\supp}{supp}          % Traeger (supp)
\DeclareMathOperator{\diam}{diam}          % Durchmesser  (diam)
\DeclareMathOperator{\dist}{dist}               % Abstand (dist)
\let\div\relax
\DeclareMathOperator{\div}{div}                % Divergenz
\DeclareMathOperator{\spann}{span}        % Lineare Huelle (span)
\DeclareMathOperator{\Grad}{deg}           % Grad eines Polynoms
\let\Re\relax
\DeclareMathOperator{\Re}{Re}
\let\Im\relax
\DeclareMathOperator{\Im}{Im}
%
\begin{document}
\titlehead{Priv.-Doz. Dr. Jens Wirth\\Institut f\"ur Analysis, Dynamik und Modellierung\\Universit\"at Stuttgart}
\lowertitleback{\copyright 2015. Texte geschrieben von den Teilnehmern des Seminars.  Einzelne Beiträge sind nicht namentlich gekennzeichnet. Ausarbeitungen basieren auf Originalliteratur. }
\title{Mikrolokale Analysis}
\subtitle{Masterseminar}
\author{---}
\date{Sommersemester 2015}
\maketitle
\tableofcontents
%
\part{Analysis von Differentialoperatoren}
%
% !TEX root = main.tex
\chapter{Einleitung}
Die folgenden Abschnitte basieren im wesentlichen auf Hörmanders Arbeit \cite{Hormander:1955}. In dieser Einleitung sollen die verwendete Notation festgelegt und die wichtigsten operatortheoretischen Konzepte zusammengefaßt werden.

\section{Funktionalanalytische Grundlagen}

Seien $V$ und $W$ Banachräume und $\mathcal D_T\subseteq V$ ein linearer Teilraum. Ein (im allgemeinen unbeschr\"ankter) \eIndex{Operator} $T$ mit Definitionsbereich $\mathcal D_T$ ist eine lineare Abbildung $T:V\supset\mathcal D_T \to W$. Er heißt \eIndex[Operator]{beschr\"ankt}, falls es eine Konstante $C>0$ mit 
\begin{equation}
\forall v\in\mathcal D_T \quad:\quad \|Tv \|_W \le C \|v\|_V
\end{equation}
gibt. Weiter heißt
\begin{equation}
\mathcal R_T = \{ Tv : v\in \mathcal D_T \} \subseteq W
\end{equation}
der \eIndex[Operator]{Wertebereich} und 
\begin{equation}
\mathcal G_T = \{ (v,Tv) : v \in \mathcal D_T \} \subseteq V\times W 
\end{equation}
der \eIndex[Operator]{Graph} von $T$. Eine Teilmenge $\mathcal G\subseteq V\times W$ ist genau dann Graph eines Operators, wenn $\mathcal G$ linear ist und aus
$(0,w)\in\mathcal G$ stets $w=0$ folgt. Wir versehen $V\times W$ mit der Norm $\|(v,w)\|_{V\times W}^2 = \|v\|_V^2 + \|w\|_W^2$.


Der Operator $T$ wird als \eIndex[Operator]{abgeschlossen} bezeichnet, falls der Graph $\mathcal G_T$ ein abgeschlossener Teilraum des Produktraumes $V\times W$ ist. Weiter heißt $T$ \eIndex[Operator]{abschließbar}, falls der Abschluß von $\mathcal G_T$ in $V\times W$ Graph eines Operators ist. Dieser wird als Abschluß von $T$ bezeichnet.

\begin{thm}[{\cite[Theorem~1.1]{Hormander:1955}}]\label{thm:1:1.1}
Sei $T_1$ abgeschlossen und $T_2$ abschließbar und gelte $\mathcal D_{T_1}\subseteq \mathcal D_{T_2}$. Dann existiert eine Konstante $C$ mit  
\begin{equation}
   \| T_2 u\|_W^2 \le C ( \|T_1 u\|_W^2 + \|u\|_V^2 ). 
\end{equation}
\end{thm}
\begin{proof}
Wir betrachten die Abbildung $\mathcal G_{T_1} \ni (u,T_1u) \mapsto T_2 u \in W$. Diese ist linear und überall definiert. Wir zeigen, daß sie auch abgeschlossen ist. Angenommen, $u_n\to u$ und $T_1 u_n\to T_1 u$ und $T_2 u_n$ sei konvergent. Da $T_2$ abschließbar ist, konvergiert $T_2 u_n \to T_2 u$. Also ist die Abbildung abgeschlossen, nach dem Satz vom abgeschlossenen Graphen stetig und somit beschr\"ankt.
\end{proof}

Ist $T$ injektiv, so bestimmt $\mathcal G_{T^{-1}} = \{ (w,v) : (v,w)\in\mathcal G_T\}$ einen Graphen. Der zugehörige Operator wird als $T^{-1}$ bezeichnet. Dieser ist genau dann abgeschlossen, wenn $T$ abgeschlossen ist. Ist $T^{-1}$ beschränkt, gilt also
\begin{equation}\label{eq:1:1.6} 
   \forall u\in\mathcal D_T \quad:\quad \|u\|_V\le C \|Tu\|_W
\end{equation}
mit einer von $u$ unabhängigen Konstanten $C$, so sagen wir $T$ ist \eIndex[Operator]{beschränkt invertierbar}.

Im folgenden sei $V=W=H$ ein Hilbertraum. Zur Vereinfachung der Notation verwenden wir immer $\|\cdot\|$ f\"ur die Norm in $H$. Weiter sei das Innenprodukt in $H$ durch $\spro{u}{v}$ bezeichnet und $H\times H$ mit dem entsprechenden Innenprodukt $\spro{(u_0,u_1)}{(v_0,v_1)} = \spro{u_0}{v_0} + \spro{u_1}{v_1}$ versehen. Auf $H\times H$ sei der Operator $J$ mit $J(v,w) = (-w,v)$ definiert. Dann ist zu jedem dicht definierten Operator $T$ mit Graphen $\mathcal G_T$ durch $\mathcal G_{T^*}=(J\mathcal G_T)^\perp$ ein Graph gegeben. Der zugeh\"orige Operator $T^*$ wird als zu $T$ \eIndex[Operator]{adjungiert} bezeichnet. Er ist immer abgeschlossen und es gilt
\begin{equation}
   \spro{Tu}{v} = \spro{u}{T^*v}
\end{equation}
f\"ur $u\in\mathcal D_T$ und $v\in\mathcal D_{T^*}$.

\begin{thm}[{\cite[Lemma~1.1]{Hormander:1955}}]\label{thm:1:1.2}
Sei $T$ ein dicht definierter Operator. Dann gilt $\mathcal R_T=H$ genau dann, wenn 
$T^*$ beschränkt invertierbar ist.
\end{thm}
\begin{proof}
Sei $\mathcal R_T=H$. Dann existiert zu jedem $u\in H$ ein $w\in H$ mit $Tw=u$. Damit folgt
\begin{equation}
   \spro{u}{v} = \spro{Tw}{v} = \spro{w}{T^*v}, \qquad |\spro{u}{v}| \le C_u \|T^*v\|
\end{equation}
f\"ur alle $u\in H$ und $v\in\mathcal D_{T^*}$ und mit dem Satz über die gleichmäßige Beschränktheit die Behauptung.

Angenommen, $T^*$ erf\"ullt die Ungleichung \eqref{eq:1:1.6}. Dann ist der selbstadjungierte Operator $TT^*$ wegen
\begin{equation}
  \spro{TT^*v}{v} = \spro{T^*v}{T^*v} \ge C^{-2} \spro{v}{v}
\end{equation}
strikt positiv und stetig invertierbar. Dann ist aber $TT^* (TT^*)^{-1}  = I$ und somit $\mathcal R_T = H$.
\end{proof}

\begin{cor}[{\cite[Lemma~1.2]{Hormander:1955}}]\label{cor:1:1.3}
Ein dicht definierter Operator $T$ besitzt eine Rechtsinverse $S$ genau dann, wenn $T^*$ beschränkt invertierbar ist.
\end{cor}
\begin{proof}
Wenn $T$ eine Rechtsinverse $S$ besitzt, impliziert $RS=I$ schon $\mathcal R_T=H$ und mit obigem Satz folgt die beschränkte Invertierbarkeit von $T^*$. Gilt umgekehrt \eqref{eq:1:1.6}, so ist $S=T^*(TT^*)^{-1}$ das gesuchte. Da $T^*(TT^*)^{-1/2}$ eine Isometrie ist, ist $S$ stetig.
\end{proof}

\section{Differentialoperatoren} 
Wir nutzen im folgenden Multiindexschreibweise. Für Multiindices $\alpha,\beta\in\mathbb N_0^n$ sei
\begin{equation}
|\alpha|=\sum_{j=1}^n \alpha_j,\qquad \alpha! = \prod_{j=1}^n \alpha_j!,\qquad (\alpha+\beta)_j=\alpha_j+\beta_j.
\end{equation} 
Weiter sei zu $\zeta\in\C^n$ durch
\begin{equation}
   \zeta^\alpha = \prod_{j=1}^n \zeta_j^{\alpha_j}
\end{equation}
das Monom, sowie durch
\begin{equation}
   \D^\alpha = \prod_{j=1}^n \left(- \i \frac{\partial}{\partial x_j}\right)^{\alpha_j}
\end{equation}
ein formaler Differentialoperator definiert. Sei nun $\Omega\subseteq\R^n$ ein Gebiet. Ein Differentialoperator der Ordnung $m$ ist ein formaler Ausdruck der Form
\begin{equation}
    P(x,\D) = \sum_{|\alpha|\le m} a_\alpha(x) \D^\alpha
\end{equation}
mit Koeffizientenfunktionen $a_\alpha\in \rmC^\infty(\Omega)$. Er agiert in nat\"urlicher Weise auf $u\in\rmC^\infty_0(\Omega)$ (oder auf $u\in\mathscr D'(\Omega)$ in distributionellem Sinne).

Versieht man $\rmC_0^\infty(\Omega)$ durch
\begin{equation}
    \spro{u}{v} = \int_{\Omega} u(x) \overline{v(x)} \d x
\end{equation}
mit einem $\rmL^2$-Innenprodukt, so kann man zu $P(x,\D)$ den formal adjungierten Operator $P^*(x,\D))$ betrachten. Dieser erf\"ullt
\begin{equation}
   \forall u,v\in\rmC_0^\infty(\Omega)\quad:\quad \spro{ P(x,\D)u}{v}=\spro{u}{  P^*(x,\D)v}
\end{equation}
und ist durch
\begin{equation}
    P^*(x,\D) v(x) = \sum_{|\alpha|\le m}  \D^\alpha\big(\overline{a_\alpha(x)}v(x)\big)
\end{equation}
gegeben.

\begin{lem}[{\cite[Lemma~1.4]{Hormander:1955}}]
Der Differentialoperator  $P(x,\D)$ versehen mit Definitionsbereich
\begin{equation}
  \mathcal D= \{ u\in\rmC^\infty(\Omega) \cap \rmL^2(\Omega) :   P(x,\D)u\in\rmL^2(\Omega) \}
\end{equation}
ist $\rmL^2$-$\rmL^2$-abschließbar.
\end{lem}
\begin{proof}
Sei $u_n\in\mathcal D$ eine Folge mit $u_n\to 0$ in $\rmL^2(\Omega)$ und $P(x,\D) u_n \to w\in\rmL^2(\Omega)$. Dann gilt f\"ur jedes 
$v\in\rmC^\infty_0(\Omega)$
\begin{equation}
 \spro{w}{v}=\lim_{n\to\infty}   \spro{ P(x,\D) u_n}{v} = \lim_{n\to\infty} \spro{u_n}{P^*(x,\D)v} = 0
\end{equation}
somit $w=0$.
\end{proof}

Die Aussage gilt allgemeiner, es ergibt sich $\rmL^p$-$\rmL^q$-, $\rmC$-$\rmC$ sowie $\rmL^p$-$\rmC$-Abschließbarkeit bei entsprechend gewähltem $\mathcal D$. Insbesondere ist $P(x,\D)$ mit Definitionsbereich $\rmC^\infty_0(\Omega)$ abschließbar als Operator auf $\rmL^2(\Omega)$. 

\begin{df} 
Sei $P(x,\D)$ ein Differentialausdruck und $\Omega$ ein Gebiet. Dann wird der $\rmL^2$-$\rmL^2$-Abschluß des durch  $P(x,\D)$ auf $\rmC_0^\infty(\Omega)$ definierten Operators mit $P_0$ und als \eIndex[Differentialoperator]{minimaler Operator} zum Differentialausdruck $P(x,\D)$ und Gebiet $\Omega$ bezeichnet. 
Weiterhin heißt $P := (  (P^*)_0)^*$ \eIndex[Differentialoperator]{maximaler Operator} zu $P(x,\D)$ und $\Omega$.
\end{df}

Der so definierte maximale Operator besitzt den Definitionsbereich
\begin{equation}
   \mathcal D_P = \{ u\in\rmL^2(\Omega) :  P(x,\D)u \in\rmL^2(\Omega)\},
\end{equation}
wobei die Anwendung von $P(x,\D)$ im distributionellen Sinne zu verstehen ist. Wir betrachten ein einfaches Beispiel. Dazu sei $\Omega=(a,b)\subseteq\R$ ein Intervall und $P(x,\D) = \D^2 = - \partial^2$ die Zuordnung der zweiten Ableitung. Dann besitzt der minimale Operator den Definitionsbereich
\begin{equation}
    \{ u\in\rmH^2(\Omega) :  u(a)=\partial u(a)=u(b)=\partial u(b) = 0\} = \rmH^2_0(\Omega),
\end{equation}
also den Sobolevraum $\rmH^2_0(\Omega)$ der am Rand verschwindenden Funktionen, und der maximale Operator gerade den gesamten Sobolevraum $\rmH^2(\Omega)$ als Definitionsbereich. Im Falle h\"oherer Raumdimensionen wird der Definitionsbereich des maximalen Operators in der Regel echt größer als der Sobolevraum passender Ordnung sein.
Der Unterschied zwischen minimalen und maximalen Operatoren besteht in der Wahl von Randbedingungen. 

\begin{thm}[{\cite[Lemma~1.7]{Hormander:1955}}]\label{thm:1:loesbarkeit}
Die Gleichung $Pu=f$ hat genau dann für jedes $f\in\rmL^2(\Omega)$ eine L\"osung $u\in\mathcal D_{P}$, wenn $\overline  P_0$ beschränkt invertierbar ist, also wenn f\"ur den formal adjungierten Differentialausdruck
\begin{equation}
  \forall u\in\rmC^\infty_0(\Omega)\quad:\quad   \|u\| \le C \|   P^*(x,\D) u \|
\end{equation}
gilt.
\end{thm}
\begin{proof}
Folgt aus Satz~\ref{thm:1:1.2} in Verbindung mit der Definition des maximalen Operators $P$.
\end{proof}

Solche und allgemeinere Ungleichungen stehen im Zusammenhang mit Enthaltenseinsbeziehungen zwischen Definitionsbereichen minimaler Operatoren. Dazu definieren wir zuerst: 
\begin{df}\label{df:1:1.2}
Seien $P(x,\D)$ und $Q(x,\D)$ Differentialausdrücke und $\Omega$ ein Gebiet. Gilt dann $\mathcal D_{P_0}\subseteq \mathcal D_{Q_0}$, so heiße $P(x,\D)$ \eIndex[Differentialoperator]{stärker} als $Q(x,D)$ bzw. $Q(x,\D)$ \eIndex[Differentialoperator]{schwächer} als $P(x,\D)$. Gilt $\mathcal D_{P_0}=\mathcal D_{Q_0}$, so heißen beide \eIndex[Differentialoperator]{gleich stark}.
\end{df}
Die  Definition hängt im Falle konstanter Koeffizienten nicht von der Wahl des Gebietes ab. Im Falle variabler Koeffizienten (siehe später) sind die Gebiete hinreichend klein zu wählen um eine sinnvolle Definition erhalten.

\begin{lem}[{\cite[Lemmata~1.5 und~1.6]{Hormander:1955}}]\label{lm: Qu stetig}
Seien $P(x,\D)$ und $Q(x,\D)$ Differentialausdr\"ucke auf einem Gebiet $\Omega$. 
\begin{enumerate}
\item
$Q(x,\D)$ ist genau dann schwächer als $P(x,\D)$, wenn
\begin{equation}\label{eq:1:1.21}
    \forall u\in\rmC^\infty_0(\Omega)\quad:\quad \| Q(x,\D)u\|^2 \le C \big( \| P(x,\D)u\|^2 + \|u\|^2 \big)
\end{equation}
gilt.
\item
$Q_0 u$ ist für jedes $u\in\mathcal D_{P_0}$ genau dann beschränkt, wenn
\begin{equation}
    \forall u\in\rmC^\infty_0(\Omega)\quad:\quad  \sup_{x\in\Omega} | Q(x,\D) u(x) |^2 \le C \big( \| P(x,\D)u\|^2 + \|u\|^2 \big)
\end{equation}
gilt. In diesem Fall besitzt $Q_0u$ insbesondere einen stetigen auf $\partial\Omega$ verschwindenden Repräsentanten.
\end{enumerate}
\end{lem}
\begin{proof}
{\bf (i)}\quad Die Hinrichtung folgt direkt aus Satz~\ref{thm:1:1.1}. Für die Rückrichtung nehmen wir an, die Ungleichung gilt. Dann existiert zu $u\in\mathcal D_{P_0}$ eine Folge $u_n\in\rmC_0^\infty(\Omega)$ mit $u_n\to u$ und $P(x,\D) u_n\to P_0 u$ in $\rmL^2(\Omega)$. Dann ist aber $Q(x,\D)(u_n-u_m)$ Cauchy und da $Q_0$ abgeschlossen ist folgt $u\in\mathcal D_{Q_0}$. $\bullet$\qquad
{\bf (ii)}\quad Für die Hinrichtung betrachtet man $Q(x,\D)$ als Operator $\mathcal G_{P_0}\to\rmL^\infty(\Omega)$ und wendet Satz~\ref{thm:1:1.1} an. Für die Rückrichtung sei $u\in\mathcal D_{P_0}$ beliebig und $u_n\in\rmC_0^\infty(\Omega)$ eine Folge mit $u_n\to u$ und $P(x,\D)u_n\to P_0u$ in $\rmL^2(\Omega)$.
Dann konvergiert $ Q(x,\D)u_n$ gleichmäßig und Behauptung folgt. Insbesondere verschwindet $Q_0u = \lim_{n\to\infty}  Q(x,\D) u_n$ auf dem unendlich fernen Punkt der Einpunktkompaktifizierung von $\Omega$.
\end{proof}

\section{Randwertprobleme}
Sei im folgenden $P(x,\D)$ und $\Omega$ fixiert. Dann sind  $\mathcal G_P$ und $\mathcal G_{P_0}$ abgeschlossene Teilräume von $\rmL^2\times\rmL^2$. Da
$\mathcal G_{P_0}\subseteq \mathcal G_P$ gilt, kann man den Quotientenraum
\begin{equation}
   \mathcal C = \mathcal G_P / \mathcal G_{P_0}
\end{equation}
betrachten. Dieser wird als \eIndex[Differentialoperator]{Cauchyraum} des Differentialausdrucks $P(x,\D)$ auf dem Gebiet $\Omega$ bezeichnet. Für $u\in\mathcal D_P$ sei
\begin{equation}
     \Gamma u := [ (u,Pu) ]_{\mathcal G_{P_0}} \in \mathcal C
\end{equation}
das zugeordnete Cauchydatum. Man kann sich $\Gamma u$ als eine Charakterisierung der Randwerte von $u$ auf $\Omega$ vorstellen, unterscheiden sich zwei Funktionen $u$ und $v$ nur in einer kompakten Teilmenge $\Omega' \Subset \Omega$, so gilt $\Gamma u=\Gamma v$.
Sei nun $B\subseteq\mathcal C$ ein linearer Unterraum. Dann heißt 
\begin{equation}
    Pu = f,\qquad \Gamma u\in B,
\end{equation}
f\"ur gegebenes $f\in\rmL^2(\Omega)$ das zugeordnete \eIndex{Randwertproblem} mit homogener \eIndex[Randwertproblem]{Randbedingung} $\Gamma u\in B$. Die Randbedingung 
$B$ bestimmt damit eine Einschr\"ankung $\widehat P$ des Operators $P$ auf den Definitionsbereich
\begin{equation}
    \mathcal D_{\widehat P} = \{ u \in \mathcal D_P : \Gamma u\in B \}.
\end{equation}
Der so definierte Operator ist abgeschlossen genau dann, wenn $B$ abgeschlossen ist. Die Menge der abgeschlossenen Operatoren $\widehat P$ mit $P_0\subseteq\widehat P\subseteq P$ steht in Bijektion zu den abgeschlossenen Teilräumen des Cauchyraumes $\mathcal C$.
Das Randwertproblem heißt \eIndex[Randwertproblem]{korrekt gestellt}, falls $\widehat P$ stetig invertierbar ist.

\begin{thm}[Vi\v{s}ik, [{\cite[Theorem~1.2]{Hormander:1955}}]
Zu einem Differentialausdruck $P(x,\D)$ existieren auf einem Gebiet $\Omega$ genau dann korrekt gestellte Randwertprobleme, wenn $P_0$ und $(P^*)_0$
beschr\"ankt invertierbar sind.
\end{thm} 
\begin{proof} 
Angenommen, es existiert ein korrekt gestelltes homogenes Randwertproblem  für $P(x,\D)$. Da $P_0\subseteq \widehat P$ gilt, muß dann $P_0^{-1}$ beschränkt sein. Weiterhin ist $\widehat P^{-1}$ rechtsinvers zu $P$, damit muß $(P^*)_0^{-1}$ nach Korollar~\ref{cor:1:1.3} beschränkt sein.

Seien nun $P_0$ und $ (P^*)_0$ beschränkt invertierbar. Da $P_0^{-1}$ beschr\"ankt ist, ist $\mathcal R_{P_0}\subseteq \rmL^2(\Omega)$ abgeschlossen.  Bezeichne nun $\pi$ den Orthogonalprojektor auf $\mathcal R_{P_0}$. Ist nun $S$ eine Rechtsinverse zu $P$ (die nach  Korollar~\ref{cor:1:1.3} existiert), so gilt für
den durch  
\begin{equation}
  Tf = P_0^{-1} \pi f + S(I-\pi) f,\qquad f\in\rmL^2(\Omega) 
\end{equation}
definierten Operator $T$ nach Konstruktion $PT f = \pi f + (I-\pi) f = f$ und $T$ besitzt eine Inverse $\widehat P \subseteq P$. Weiter ist $T\supset P_0^{-1}$ und $\widehat P\supset P_0$ , so daß $P_0\subseteq \widehat P\subseteq P$. Da $\widehat P^{-1}$ beschränkt und auf ganz $\rmL^2(\Omega)$ definiert ist, ist der Satz bewiesen.
\end{proof}

Seien nun $P_0$ und  $(P^*)_0$ beschränkt invertierbar und bezeichne 
\begin{equation}
  U=\ker P = \{u\in\rmL^2(\Omega) : Pu =0\}
\end{equation}  
den Kern des maximalen Operators. Dann ist $\Gamma U\subseteq \mathcal C$ abgeschlossen und die Einschränkung $\gamma:=\Gamma|_U$ wegen
\begin{multline}
  \|u\|^2\ge \|\Gamma u\|^2 = \inf_{w\in \mathcal D_{P_0}} (\|u-w\|^2 + \|Pu - Pw\|^2) \\   \ge \inf_{w\in \rmL^2} (\|u-w\|^2 + C^{-2} \|w\|^2) = \frac{C^{-2}}{1+C^{-2}}\|u\|^2
\end{multline}
ein Isomorphismus. 
\begin{thm}[{\cite[Theorem~1.3]{Hormander:1955}}]
Seien $P_0$ und $\overline  P_0$ beschränkt invertierbar und sei $B\subseteq\mathcal C$ abgeschlossener Teilraum. Das zugehörige homogene Randwertproblem $\widehat P$ ist genau dann korrekt gestellt, wenn $\mathcal C = B \dotplus \Gamma U $ als direkte Summe\footnote{Direkt, nicht notwendig orthogonal.} gilt.
\end{thm}
\begin{proof}
Angenommen, $\mathcal C = B \dotplus \Gamma U$. Dann ist der Lösungsoperator über eine Rechtsinverse $S$ zu $P$ und den Projektor $\pi\in\mathcal L(\mathcal C)$ auf  $\Gamma U$ entlang $B$ durch
\begin{equation}
     Tf = Sf-\gamma^{-1}\pi\Gamma Sf
\end{equation}
ausdrückbar, da für alle $f\in\rmL^2(\Omega)$ nach Konstruktion $PTf=f-P \gamma^{-1}\pi\Gamma S f=f-0=f$ sowie $\Gamma T f=\Gamma Sf - \Gamma\gamma^{-1}\pi \Gamma Sf=(I-\pi)\Gamma Sf\in B$ gilt und somit $T=\widehat P^{-1}$ stetig ist.  

Ist umgekehrt $\widehat P^{-1}$ die stetige Inverse zu $\widehat P$, so induziert
$\mathcal G_P\ni(v,Pv) \mapsto \Gamma(v-\widehat P^{-1}Pv)\in\Gamma U$ (was offenbar auf $\mathcal G_{P_0}$ verschwindet und auf $\Gamma U$ die Identität ist) einen Projektor $\pi\in\mathcal L(\mathcal C)$. Da weiterhin $v\in\mathcal D_{\widehat P}$ genau dann gilt, wenn $v=\widehat P^{-1}Pv$ erfüllt ist, folgt $B=\ker\pi$.
\end{proof}

\section{Differentialoperatoren mit konstanten Koeffizienten}
Im folgenden sollen vorerst nur Operatoren mit konstanten Koeffizienten betrachtet werden. Diese gehören zu Differentialausdrücken der Form
\begin{equation}
     P(\D) = \sum_{|\alpha|\le m} a_\alpha \D^\alpha
\end{equation}
mit $a_\alpha\in\C$. Für diese gilt 
\begin{equation}
     P(\D) \e^{\i x\cdot\zeta}  =  P(\zeta) \e^{\i x\cdot\zeta}
\end{equation}
für alle $\zeta\in\C^n$ und mit $x\cdot\zeta = \sum_{j=1}^n x_j\zeta_j$. Das Polynom $ P(\zeta)$ heißt das \eIndex[Differentialoperator]{Symbol} des Differentialausdrucks $P(\D)$. Ist nun $\Omega\subseteq\R^n$ beschränkt und $u\in\rmC_0^\infty(\Omega)$, so ist die Fourier--Laplace-transformierte von $u$
\begin{equation}
    \widehat u (\zeta) = (2\pi)^{-n/2} \int_\Omega \e^{-\i x\cdot\zeta } u(x) \d x,\qquad \zeta\in\C^n
\end{equation}
eine ganze Funktion auf $\C^n$. Die Anwendung des Operators $ P(\D)$ entspricht im Fourierbild der Multiplikation mit $ P(\zeta)$. Der formal adjungierte Operator $  P^*(\D)$ entspricht dabei dem Operator zum Symbol $\overline{P}(\zeta)=\overline{P(\overline\zeta)}$.

Eine erste Anwendung ist folgender Satz. Ein Beweis ergibt sich später nochmals in allgemeinerer Form.

\begin{thm}[{\cite[Theorem~2.1]{Hormander:1955}}]\label{thm:1:1.8}
Der Operator $P_0$ ist beschränkt invertierbar, es existiert also eine Konstante $C$ mit
\begin{equation}
   \forall u\in\rmC_0^\infty(\Omega)\quad:\quad \|u\|\le C\| P(\D) u\|.
\end{equation}
\end{thm}

Der Beweis basiert auf folgendem Lemma der Funktionentheorie. 
\begin{lem}[Malgrange, {\cite[Lemma~2.1]{Hormander:1955}}]
Sei $g\in\mathfrak A(\overline{\mathbb D})$ analytisch in einer Umgebung der Kreisscheibe $|z|\le 1$  und $r$ ein Polynom mit höchstem Koeffizienten $A$. Dann gilt
\begin{equation}
   |Ag(0)|^2 \le (2\pi)^{-1} \int_0^{2\pi} |g(\e^{\i\theta}) r(\e^{\i\theta})|^2 \d\theta.
\end{equation}
\end{lem}
\begin{proof}
Seien $z_j$ die Nullstellen von $r$ innerhalb der Kreisscheibe $|z|<1$ und
\begin{equation}
    r(z) = q(z) \prod_{j} \frac{z_j-z}{\overline{z_j}z-1}.
\end{equation} 
Dann gilt auf der Kreislinie $|r(z)|=|q(z)|$ und $q(z)$ ist analytisch in $\mathbb D$. Also folgt
\begin{equation}
  (2\pi)^{-1} \int  |g(\e^{\i\theta}) r(\e^{\i\theta})|^2 \d\theta = (2\pi)^{-1} \int  |g(\e^{\i\theta}) q(\e^{\i\theta})|^2 \d\theta \ge |g(0)q(0)|^2.
\end{equation}
Nun ist aber $q(0)/A$ gerade das Produkt der Nullstellen von $r(z)$ außerhalb des Einheitskreises und damit $|q(0)|\ge |A|$ und das Lemma folgt.
\end{proof}

\begin{proof}[Beweis zu Satz~\ref{thm:1:1.8}]
Bezeichne $p(\zeta)$ den homogenen Hauptteil von $P(\zeta)$. Sei weiter $\xi_0\in\R^n$ mit $p(\xi_0)\ne0$. Wendet man nun obiges Lemma auf die Funktion
$\widehat u(\zeta+ t\xi_0)$ und das Polynom $P(\zeta+t\xi_0)$ (verstanden als Funktionen von $t$) an, so erhält man
\begin{equation}
   |\widehat u(\zeta) p(\xi_0)|^2 \le (2\pi)^{-1} \int |\widehat u(\zeta+\e^{\i\theta} \xi_0) P(\zeta+\e^{\i\theta}\xi_0)|^2 \d\theta.
\end{equation}
Speziell mit $\zeta=\xi\in\R^n$ und Integration bezüglich $\xi$ ergibt sich
\begin{align}
   |p(\xi_0)|^2 \int |\widehat u(\xi)|^2 \d\xi &\le (2\pi)^{-1} \iint |\widehat u(\xi+\e^{\i\theta}\xi_0) P(\xi+\e^{\i\theta}\xi_0)|^2 \d\xi\d\theta \notag\\
   &= (2\pi)^{-1} \iint |\widehat u(\xi+\i\xi_0\sin\theta) P(\xi+\i\xi_0\sin\theta)|^2 \d\xi\d\theta
\end{align} 
und unter Ausnutzung der Parseval-Identität
\begin{equation}
   |p(\xi_0)|^2 \int |u(x)|^2\d x \le (2\pi)^{-1} \iint |P(D) u(x)|^2 \e^{2x\cdot\xi_0 \sin\theta} \d x\d\theta
\end{equation}
und mit $C=\sup_{x\in\Omega} \e^{|x\cdot\xi_0|} /|p(\xi_0)|$ folgt die Behauptung.
\end{proof}

\begin{cor}\label{cor:1:1.10}
\begin{enumerate}
\item
Maximale Differentialoperatoren mit konstanten Koeffizienten sind auf jedem beschränkten Gebiet surjektiv; zu jedem $f\in\rmL^2(\Omega)$ existiert ein $u\in\mathcal D_P$ mit $Pu=f$.
\item
Für Differentialoperatoren mit konstanten Koeffizienten existieren
zu jedem beschränkten Gebiet korrekt gestellte Randwertprobleme.
\end{enumerate}
\end{cor}
 % Einleitung, Teil 1
% !TEX root = main.tex
\chapter{Minimale Operatoren}

Ein erstes Ziel dieses Kapitels ist es, den Vergleich der Stärke
zweier Differentialausdrücke mit konstanten Koeffizienten  (Definition \ref{df:1:1.2})  auf einen Vergleich ihrer Symbole zurückzuführen. 

\section{Die Stärke von Differentialausdrücken mit konstanten Koeffizienten}

Seien $\mcP=P(\D)$ und $\mcQ=Q(\D)$ zwei Differentialausdrücke mit konstanten Koeffizienten und zugehörigen Symbolen $P(\xi)$ und $Q(\xi)$. Wir nehmen zuerst an, es gibt eine Konstante $C>0$ mit $|Q(\xi)|\le C|P(\xi)|$ für alle $\xi\in\R^n$. Dann ist $\mcQ$ schwächer als $\mcP$, da mit der Formel von Plancherel 
\begin{equation}
    \|Q(\D) u\|^2 = \int |Q(\xi)\widehat u(\xi)|^2 \d\xi \le C^2 \int |P(\xi)\widehat u(\xi)|^2 \d\xi = C^2 \|P(\D)u \|^2 
\end{equation}
für alle $u\in\rmC_0^\infty(\R^n)$ gilt. Man beachte, daß in diesem Falle jede reelle Nullstelle von $P$ auch Nullstelle von $Q$ sein muß. Das soll im folgenden verallgemeinert und die genutzte Abschätzung $C=\sup_{\xi\in\R^n} |Q(\xi)|/|P(\xi)|$ für Operatoren auf beschränkten Gebieten  durch eine Bedingung der Form
\begin{equation}\label{eq:2:1}
\sup_{\xi\in\R^n}\frac{\widetilde{Q}(\xi)}{\widetilde{P}(\xi)}<\infty
\end{equation}
ersetzt werden, wobei $\widetilde{P}$ dem Polynom $P$ durch
\begin{equation}\label{eq:2:2}
\widetilde{P}(\zeta)^2=\sum_\alpha\abs{P^{(\alpha)}(\zeta)}^2,\qquad \zeta=\xi+\i\eta \in\C^n
\end{equation}
zugeordnet wird und $\widetilde Q$ analog definiert sei. Dabei bezeichne 
\begin{equation}
P^{(\alpha)}(\xi)=\partial^\alpha P(\xi)
\end{equation}
die $\alpha$-te Ableitung des Polynoms $P(\xi)$. Es wird sich zeigen, daß die Bedingung \eqref{eq:2:1} genau dann gilt, wenn $\mcQ$ schwächer als $\mcP$ ist.

In den folgenden Betrachtungen benötigt man häufig die leibnizsche Formel
\begin{equation}\label{eq:2:leib}
\forall u,v\in \rmC^\infty(\Omega) \quad:\quad P(\D)(uv)=\sum_{\alpha}(P^{(\alpha)}(\D)v)\left(\frac{\D^\alpha u}{\alpha!}\right).
\end{equation} 
Für $P(\D)=\D^\alpha$ ist dies die gewöhnliche Leibnizregel, allgemein folgt \eqref{eq:2:leib} unter Ausnutzung der Linearität. Wir treffen weiter folgende Konvention:
Im Falle griechischer Indizes
ist die Ableitung nach einem Multiindex $\alpha\in\N^n_0$ gemeint,
im Fall lateinischer Indizes oder Zahlen definieren wir
\begin{equation}
P^{(k_1,\dots,k_l)}(\xi)=\partial_{k_1}\cdots\partial_{k_l}P(\xi).
\end{equation}

\begin{thm}[{\cite[Theorem~2.1]{Hormander:1955}}]\label{thm:2:2.1}
Sei $\Omega$ ein beschränktes Gebiet.
Der Differentialausdruck $\mcQ$ ist genau dann schwächer als $\mcP$,
wenn
\begin{equation}\label{eq:thm:2:2.1}
\sup_{\xi\in\R^n}\frac{\widetilde{Q}(\xi)}{\widetilde{P}(\xi)}<\infty
\end{equation}
erfüllt ist.
\end{thm}

Wir zeigen zunächst, wie in beiden Richtungen abgeschätzt wird.
Für die Rückrichtung benötigen wir ein Lemma,
das wir später beweisen.

\begin{proof}
{\em Hinrichtung.}
Sei $\mcQ$ schwächer als $\mcP$.
Dann gilt für ein $C>0$ und alle $u\in\rmC_0^\infty(\Omega)$ die Ungleichung
\begin{equation}\label{eq:2:4}
\norm{\mcQ u}^2\leq C(\norm{\mcP u}^2+\norm{u}^2).
\end{equation}
Für jedes $\xi\in\R^n$ und ein festes $\psi\in \rmC^\infty_0(\Omega)$, $\psi\neq0$,
definieren wir $\psi_\xi(x)=\e^{\i x\cdot\xi}\psi(x)$.
Dies bringt durch die Formel von Leibniz die Ableitungen von $P(\xi)$ ins Spiel
\begin{equation}\label{eq:2.5}
P(\D)\psi_\xi(x)=\e^{\i x\cdot\xi}\sum_\alpha P^{(\alpha)}(\xi)\frac{\D^\alpha\psi(x)}{\alpha!},
\end{equation}
analoges gilt für $Q(\D)$.
Schreiben wir kurz
\begin{equation}\label{eq:2.6}
\Psi_{\alpha\beta}=\frac1{\alpha!\beta!}\spro{\D^\alpha\psi}{\D^\beta\psi},
\end{equation}
so liest sich Ungleichung \eqref{eq:2:4} mit $u=\psi_\xi$ als
\begin{equation}\label{eq:2.7}
\sum_{\alpha,\beta}Q^{(\alpha)}(\xi)\overline{{Q}^{(\beta)}(\xi)}\Psi_{\alpha\beta}
\leq C\bigg(\sum_{\alpha,\beta}P^{(\alpha)}(\xi)\overline{{P}^{(\beta)}(\xi)}\Psi_{\alpha\beta}+\Psi_{00}\bigg).
\end{equation}
Bei $\left(\Psi_{\alpha\beta}\right)_{\alpha,\beta}$ handelt es sich um die gramsche Matrix
eines Skalarproduktes auf einem $\C^M$, da für alle $t=(t_\alpha)_{|\alpha|\le m}\in\C^M\setminus\{0\}$
\begin{equation}
  \sum_{\alpha,\beta} t_\alpha \overline{t_\beta} \Psi_{\alpha\beta} = \int_\Omega \left|\sum_\alpha \frac{t_\alpha \D^\alpha\psi (x)}{\alpha! }\right|^2 \d x 
  = \int \left|\sum_\alpha \frac{t_\alpha \xi^\alpha}{\alpha!} \right|^2 |\widehat\psi(\xi)|^2 \d\xi \ne0
\end{equation}
gilt.
Dies ist gleichbedeutend damit,
dass die Vektoren $D^\alpha\psi\in L^2$, $\abs{\alpha}\leq m$, linear unabhängig sind.
Es gibt also Konstanten $\Psi_-,\Psi_+\in\R_+$, so dass
\begin{equation}\label{eq:2.8}
\abs{t}^2\Psi_-\leq\sum_{\alpha,\beta}t_\alpha \overline{t_\beta}\Psi_{\alpha\beta}\leq\abs{t}^2\Psi_+.
\end{equation}
Die Ungleichungen \eqref{eq:2.7} und \eqref{eq:2.8} kombinieren für $t_\alpha=Q^{(\alpha)}(\xi)$ und damit $\widetilde Q(\xi)^2=|t|^2$ 
und unter Ausnutzung von $\widetilde P(\xi)^2\ge c^{-1}>0$ zu
\begin{equation}\label{eq:2.9}
\widetilde{Q}(\xi)^2 \Psi_- \le C  \bigg(\sum_{\alpha,\beta}P^{(\alpha)}(\xi)\overline{{P}^{(\beta)}(\xi)}\Psi_{\alpha\beta}+\Psi_{00}\bigg) \leq  C(1+c) \widetilde{P}(\xi)^2 \Psi_+
\end{equation}
für alle $\xi\in\R^n$ und ergeben damit gerade die Ungleichung \eqref{eq:thm:2:2.1}.
%
$\bullet$\qquad {\em Rückrichtung.}
Sei die Ungleichung \eqref{eq:thm:2:2.1} erfüllt, das heißt gelte
\begin{equation}\label{eq:2.10}
\widetilde{Q}(\xi)\leq C\widetilde{P}(\xi),
\end{equation}
für alle $\xi\in\R^n$ und mit einer Konstanten $C>0$,
also auch $\abs{Q(\xi)}\leq C\widetilde{P}(\xi)$.
Für $u\in\rmC_0^\infty(\Omega)$ gilt
\begin{align}
\|Q(\D) u\|^2 &= \int |Q(\xi) \widehat{u}(\xi)|^2 \d\xi
\leq C \int |\widetilde P(\xi) \widehat{u}(\xi)|^2 \d\xi \notag\\&  = C\sum_\alpha\int | P^{(\alpha)}(\xi)\widehat{u}(\xi)|^2\d\xi = C\sum_\alpha\norm{P^{(\alpha)}(\D)u}^2.\label{eq:2.11}
\end{align}
Um den Beweis abzuschließen,
fehlt also nur noch eine Ungleichung der Form
\begin{equation}
\norm{P^{(\alpha)}(\D)u}\leq C_{\mcP,\Omega}\norm{P(\D)u}, \label{eq:2:wantedineq}
\end{equation}
mit $C_{\mcP,\Omega}>0$ unabhängig von $u\in\rmC_0^\infty(\Omega)$ und $\alpha\in\N_0^n$. Diese werden wir mit der Methode der Energieintegrale herleiten (Korollar \ref{cor:2:2.5}).
\end{proof}

\begin{rem}
Der Beweis zeigt insbesondere, daß in Formel \eqref{eq:thm:2:2.1} die Funktion $\widetilde{Q}(\xi)$ durch $|Q(\xi)|$ ersetzt werden kann.
Definieren wir weiterhin
\begin{equation}
W(\mcP)=\{\mcQ=Q(\D):\mcQ~\text{ist schwächer als}~\mcP\},
\end{equation}
so ist wegen $|Q_1(\xi)+Q_2(\xi)|^2\leq2(|Q_1(\xi)|^2+|Q_2(\xi)|^2)$
die Menge $W(\mcP)$ ein $\C$-Vektorraum.
Dies folgt auch im ganz allgemeinen Fall aus Formel \eqref{eq:1:1.21}.
\end{rem}

\section{Energieintegrale}

Im folgenden entwickeln wir eine Methode, um sesquilineare Differentialausdrücke abzuschätzen.
Wir betrachten dazu für $u,v\in\rmC_0^\infty(\Omega)$
\begin{equation}
F(\D,\overline{\D})u\bar{v}
=\sum_{\alpha,\beta}a_{\alpha\beta}(\D^\alpha u)(\cc{\D^\beta v}),
\end{equation}
mit Koeffizienten $a_{\alpha\beta}\in\C$. Uns interessieren insbesondere die Terme
\begin{equation}\label{eq:2:qf}
\Phi_F(u)=\int F(\D,\overline{\D})u(x)\bar{u}(x)\d x
=\int F(\xi,\xi)\abs{\hat{u}(\xi)}^2\d \xi,
\end{equation}
welche quadratische Formen $\Phi_F$ auf $\rmC^\infty_0(\Omega)\subseteq \rmL^2(\Omega)$
definieren.
Es besteht eine eins-zu-eins Korrespondenz
zwischen sesquilinearen Differentialausdrücken $F(\D,\cc{\D})$
und den Symbolen
\begin{equation}
F(\zeta,\cc{\zeta})=\sum_{\alpha,\beta}a_{\alpha\beta}\zeta^\alpha\cc{\zeta}^\beta, \zeta\in\C^n,
\end{equation}
analog zu den Symbolen von Differentialausdrücken.

Die quadratischen Formen $\Phi_F$
sind schon durch die Werte $F(\xi,\xi)$, $\xi\in\R^n$ bestimmt.
Verschiedene quadratische Differentialausdrücke bzw.~Symbole
können also die gleiche quadratische Form definieren.
Folgendes Lemma zeigt, dass es eine eins-zu-eins
Korrespondenz zwischen den auf reelle Werte eingeschränkten
Symbolen $F(\xi,\xi)$, $\xi\in\R^n$,
und den quadratischen Formen $\Phi_F$ gibt.

\begin{lem}[{\cite[Lemma~2.5]{Hormander:1955}}]\label{lem:2:qfsym}
Sei $\Omega$ ein beliebiges Gebiet.
Dann gilt 
\begin{equation}\label{eq:lem:2:qfsym}
\forall u\in\rmC_0^\infty(\Omega) \quad:\quad \int F(\D,\overline{\D})u\bar{u}\d x=0,
\end{equation}
genau dann, wenn $F(\xi,\xi)=0$ für alle $\xi\in\R^n$.
\end{lem}
\begin{proof}
Es ist nur die Hinrichtung zu beweisen. Gelte also \eqref{eq:lem:2:qfsym}.
Sei $u_\eta(x)=u(x)\e^{\i {x}\cdot{\eta}}$ für ein festes $u\in \rmC^\infty_0(\Omega)$, $u\neq0$.
Dann ist
\begin{align}
\int F(\D,\overline{\D})u_\eta(x)\overline{u_\eta(x)}\d x
&=\int F(\xi,\xi)\abs{\widehat{u}(\xi-\eta)}^2\d \eta\notag\\&
=\int F(\xi+\eta,\xi+\eta)\abs{\widehat{u}(\xi)}^2\d \xi=q(\eta)
\end{align}
für alle $\eta\in\R^n$.
Auf der rechten Seite steht wieder ein Polynom $q(\eta)$ in $\eta\in\R^n$
und nach Voraussetzung müssen dessen Koeffizienten verschwinden.
Wegen
\begin{equation}
(\xi+\eta)^\alpha=\sum_{\beta+\gamma=\alpha}\begin{pmatrix}\alpha\\\beta\end{pmatrix}\xi^\beta\eta^\gamma=\eta^\alpha+\dots
\end{equation}
gilt für die homogenen Hauptteile $\pi_q(\xi)$ bzw.~$\pi_F(\xi)$
von $p(\xi)$ bzw.~$F(\xi,\xi)$ die Gleichung $\pi_q(\xi)=\norm{u}^2\pi_F(\xi)$.
Also würde $F(\xi,\xi)\neq0$ zu einem Widerspruch führen.
\end{proof}

Unser Ziel ist es, Ableitungen von Differentialausdrücken zu kontrollieren.
Dazu betrachten wir zugeordnete Ableitungen sesquilinearer Differentialausdrücke. Wir beginnen mit
\begin{equation}
\frac{\partial}{\partial x_k}
\left[F(\D,\overline{\D})u\bar{u}\right]
=\i\left[(\D_k-\overline{\D}_k)F(\D,\overline{\D})\right]u\bar{u}.
\end{equation}
Für einen Vektor $\ul{G}=(G_k)_{k=1,\dots,n}$ sesquilinearer Differentialausdrücke $G_k$ ergibt sich damit formal als Divergenz die Formel
\begin{subequations}
\begin{equation}\label{eq:2.15a}
\div(\ul{G}(\D,\overline{\D})u\bar{u})=\sum_{k=1}^n\frac{\partial}{\partial x_k}\left[G_k(\D,\overline{\D})u\bar{u}\right]=F(\D,\overline{\D})u\bar{u},
\end{equation}
wobei
\begin{equation}\label{eq:2.15b}
F(\zeta,\bar{\zeta})=\i\sum_{k=1}^n(\zeta_k-\bar{\zeta}_k)G_k(\zeta,\bar{\zeta})=-2\sum_{k=1}^n\eta_kG_k(\zeta,\bar{\zeta})
\end{equation}
\end{subequations}
für $\zeta=\xi+\i\eta$ mit $\xi,\eta\in\R^n$.
\begin{lem}[{\cite[Lemma~2.2]{Hormander:1955}}]\label{lem:2:2.2}
Ein Polynom $F(\zeta,\overline{\zeta})$ in $\zeta=\xi+\i\eta$ und $\overline{\zeta}=\xi-\i\eta$
kann genau dann in der Form \eqref{eq:2.15b} dargestellt werden,
wenn $F(\xi,\xi)=0$ für alle $\xi\in\R^n$ gilt.

In diesem Fall gilt
\begin{equation}\label{eq:2:2.16}
G_k(\xi,\xi)=-\frac12\left.\frac{\partial F(\xi+\i\eta,\xi-\i\eta)}{\partial \eta_k}\right\rvert_{\eta=0}
\end{equation}
für alle $\xi\in\R^n$.
\end{lem}
\begin{proof}
Die Hinrichtung ist offensichtlich.
Die Rückrichtung und Formel \eqref{eq:2:2.16}
folgen aus der (reellen) Taylorformel von $F(\xi+\i\eta,\xi-\i\eta)$
mit den Entwicklungspunkten $(\xi,0)\in\R^{2n}$.
\end{proof}

Die Polynome $G_k(\zeta,\overline{\zeta})$ sind nicht eindeutig durch $F(\zeta,\overline\zeta)$ bestimmt.
Mit Hilfe einer speziellen quadratischen Differentialform und obiger Formel
gewinnen wir folgende Ungleichung,
mit der wir die Ableitungen der Differentialausdrücke kontrollieren können.

\begin{lem}
Sei $\Omega$ ein Gebiet, so dass für ein $k\in\{1,\ldots,n\}$ der $k$-Durchmesser $B_k=\sup_{x,y\in\Omega}\abs{x_k-y_k}$ endlich ist. Dann gilt
für Differentialausdrücke $P(\D)$ und $Q(\D)$, sowie alle Funktionen $u\in\rmC_0^\infty(\Omega)$ die Ungleichung
\begin{equation}\label{eq:lem:2:2.4}
\abs{\spro{P^{(k)}(\D)u}{\overline{Q}(\D)u}}\leq\norm{P(\D)u}\left(\norm{\overline{Q}^{(k)}(\D)u}+B_k\norm{\overline{Q}(\D)u}\right).
\end{equation}
\end{lem}

\begin{proof}
Wir betrachten den quadratischen Differentialausdruck mit dem Symbol
\begin{equation}\label{eq:2:spsym}
F(\zeta,\bar{\zeta})=P(\zeta)Q(\bar{\zeta})-Q(\zeta)P(\bar{\zeta}).
\end{equation}
Dieser erfüllt offensichtlich $F(\xi,\xi)=0$ für alle $\xi\in\R^n$.
Nach Lemma \ref{lem:2:2.2} erhalten wir,
nachdem wir mit $-\i x_k$ multipliziert haben
\begin{equation}\label{eq:2:2.22}
-\i x_kF(\D,\overline{\D})u\bar{u}=-\i x_k\sum_{j=1}^n\frac{\partial}{\partial x_j}\left[G_j(\D,\overline{\D})u\bar{u}\right],
\end{equation}
wobei nach Formel \eqref{eq:2:2.16} das Symbol
\begin{equation}
G_k(\zeta,\zeta)=-\i\left(P^{(k)}(\zeta)Q(\cc{\zeta})-Q^{(k)}(\zeta)P(\cc{\zeta})\right)\label{eq:2:Gk},
\quad\zeta\in\C^n
\end{equation}
gewählt werden kann.
Man beachte bei der Berechnung von \eqref{eq:2:Gk},
dass $P(\zeta),Q(\zeta)$ holomorph
und $P(\overline{\zeta}),Q(\overline{\zeta})$ anti-holomorph sind.
Integrieren wir Gleichung \eqref{eq:2:2.22},
so liefert eine partielle Integration
in der Variablen $x_k$ auf der rechten Seite
\begin{equation}\label{eq:2:ints}
\int-\i x_kF(\D,\cc{\D})u(x)\cc{u(x)}\d x=\i\int G_k(\D,\cc{\D})u(x)\cc{u(x)}\d x.
\end{equation}
Wir setzen nun \eqref{eq:2:spsym} und \eqref{eq:2:2.22} ein,
schreiben die rechte Seite von \eqref{eq:2:ints}
in Skalarprodukte um und bringen einen Term auf die andere Seite.
Es resultiert die Gleichung
\begin{equation}
\spro{P^{(k)}(\D)u}{\overline{Q}(\D)u}
=\spro{P(\D)u}{\overline{Q}^{(k)}(\D)u}-\i\int x_k\left(P(\D)u\cc{\overline{Q}(\D)u}-Q(\D)u\cc{\overline{P}(\D)u}\right)\d x.
\end{equation}
Ohne Beschränkung der Allgemeinheit können wir das Gebiet $\Omega$ so legen,
dass $\abs{x_k}\leq B_k/2$ für alle $x\in\Omega$ gilt.
Also erhalten wir mit Cauchy-Schwarz
und der Dreiecksungleichung die Ungleichung \eqref{eq:lem:2:2.4}.
\end{proof}

\begin{cor}[{\cite[Lemma~2.7]{Hormander:1955}}]\label{cor:2:2.5}
Ist $P(\xi)$ vom Grad $m$ in $\xi_k$, so gilt
\begin{equation}\label{eq:cor:2:2.5}
\norm{P^{(k)}(\D)u}\leq mB_k\norm{P(\D)u}
\end{equation}
für alle $u\in\rmC_0^\infty(\Omega)$.
\end{cor}
\begin{proof}
Setzen wir $\overline{Q}(\xi)=P^{(k)}(\xi)$, so erhalten wir aus \eqref{eq:lem:2:2.4} die Ungleichung
\begin{equation}\label{eq:2.28}
\norm{P^{(k)}(\D)u}^2\leq\norm{P(\D)u}\left(\norm{P^{(kk)}(\D)u}+B_k\norm{P^{(k)}(\D)u}\right)
\end{equation}
für alle $u\in\rmC_0^\infty(\Omega)$. Wir gehen nun Induktiv in $m$ vor.
Für $m=1$ ist $P^{(kk)}=0$, $P^{(k)}\neq0$ und somit entspricht die Gleichung \eqref{eq:cor:2:2.5}
gerade der Gleichung \eqref{eq:lem:2:2.4} nach Kürzen eines Faktors.
Angenommen das Korollar ist für $m-1$ erfüllt,
dann erhalten wir durch Kombination von \eqref{eq:2.28} und \eqref{eq:cor:2:2.5}
\begin{equation}
\norm{P^{(k)}(\D)u}^2\leq\norm{P(\D)u}\left((m-1)B_k\norm{P^{(k)}(\D)u}+B_k\norm{P^{(k)}(\D)u}\right)
\end{equation}
und der Induktionsschritt  ist gezeigt.
\end{proof}

\begin{cor}[{\cite[Lemma~2.8]{Hormander:1955}}]\label{cor:2:2.6}
Für festes beschränktes Gebiet $\Omega$ und beliebiges $\alpha\in\N^n_0$ gilt
\begin{equation}
\forall u\in\rmC_0^\infty(\Omega)\quad:\quad \norm{P^{(\alpha)}(\D)u}\leq C_{m,\alpha,\Omega}\norm{P(\D)u},
\end{equation}
mit $C_{m,\alpha,\Omega}=m(m-1)\cdots(m-\abs{\alpha})B^\alpha$, wobei $B^\alpha=\prod_{l=1}^n B^{\alpha_l}_l$.
Die Konstante $C_{m,\alpha,\Omega}$ hängt also nur vom Grad $m$ des Polynoms $P(\xi)$,
der Ableitungsordnung $\abs{\alpha}$ und der Ausdehnung des Gebiets $\Omega$ ab.
\end{cor}

\begin{proof}
Mehrfache Anwendung von Korollar \ref{cor:2:2.5}.
\end{proof}

Das letzte Korollar ist gerade der im Beweis
von Satz \ref{thm:2:2.1} gebrauchte Baustein.
Dabei können wir $C_{\mcP,\Omega}=\max_{|\alpha|\le m} C_{m,\alpha,\Omega}$
für die Konstante in \eqref{eq:2:wantedineq} wählen.

\section{Beispiele und spezielle Symbole}

Wir betrachten als instruktive Beispiele
einige klassische Differentialausdrücke zweiter Ordnung.
Zur Vereinfachung der Notation schreiben wir
\begin{align}
f(\xi)\asymp g(\xi)\quad&\gdw&
\exists_{C_-,C_+>0}\forall_{\xi\in\R^n}\quad&:\quad C_-g(\xi)\leq f(\xi)\leq C_+g(\xi),\\
f(\xi)\apprle g(\xi)\quad&\gdw&
\exists_{C>0}\forall_{\xi\in\R^n}~\quad&:\quad f(\xi)\leq Cg(\xi).
\end{align}

\begin{exa}\label{exa:2:lap}
Wir betrachten das zum \eIndex[Differentialoperator]{Laplaceoperator} $\Delta$
korrespondierende Symbol $P(\xi)=\xi_1^2+\dots+\xi_n^2=\abs{\xi}^2$.
Das regularisierte Symbol ist gegeben durch
\begin{equation}
\widetilde{P}(\xi)^2=\abs{\xi}^4+4\abs{\xi}^2+4n.
\end{equation}
Beachten wir, dass stets
\begin{equation}\label{eq:2:sqrest}
a^2+b^2\leq(a+b)^2\leq2(a^2+b^2),\quad\text{für}~a,b\geq0,
\end{equation}
so sehen wir, dass
\begin{equation}
\widetilde{P}(\xi)^2\asymp\abs{\xi}^4+1.
\end{equation}
Also ist der Laplaceoperator stärker als alle Differentialausdrücke von gleichem oder kleinerem Grad.
Diese sind alle von der Form
\begin{equation}
Q(\xi)=\xi^\top A\xi+\ska{b}{\xi}+c,
\end{equation}
mit einer Matrix $A\in\C^{n\times n}$, $b\in\C^n$ und $c\in\C$,
wobei wir $A^\top=A$ annehmen können.
Wir untersuchen, welche davon gleich stark wie der Laplaceoperator sind.
Es ist
\begin{equation}\label{eq:2:rpQLap1}
\widetilde{Q}(\xi)=\abs{\xi^\top A\xi+\ska{b}{\xi}+c}^2+\sum_{k=1}^n\abs{2e^\top_kA\xi+b_k}^2
+4\sum_{1\leq k\leq l\leq n}\abs{a_{kl}}^2.
\end{equation}
Da die Ausdrücke $\widetilde{Q}(\xi)$ und $1+\abs{\xi^\top A\xi}^2$
auf kompakten Mengen positiv und beschränkt sind (falls $\mcQ\neq0$) genügt es,
diese für $\abs{\xi}\geq R$ mit festem $R>0$ groß genug zu betrachten.
Ist $\abs{\xi^\top A\xi}\neq0$ falls $\xi\neq0$,
so gilt $\abs{\xi^\top A\xi}\asymp\abs{\xi}^2$.
In diesem Fall kann man, für $R$ groß genug,
die linearen Terme in \eqref{eq:2:rpQLap1} vernachlässigen
und erhält insgesamt
\begin{equation}
\widetilde{Q}(\xi)\asymp\abs{\xi^\top A\xi}^2+1\asymp\abs{\xi}^4+1\asymp\widetilde{P}(\xi).
\end{equation}
Im anderen Fall betrachtet man \eqref{eq:2:rpQLap1}
für alle $\xi\in\R^n$ mit $\xi^\top A\xi=0$,
und stellt fest, dass $\widetilde{Q}(\xi)$ für diese $\xi$ nur linear wächst,
$Q(\D)$ also echt schwächer als $P(\D)$ ist.

Im Fall einer reellen Matrix gilt $\xi^\top A\xi\neq0$ genau dann für alle $\xi\neq0$,
wenn entweder alle Eigenwerte positiv oder alle negativ sind.
%Im Fall einer komplexen Matrix ist
%\begin{equation}
%\abs{\xi^\top A\xi}^2=\abs{\xi^\top A_\Re\xi}^2+\abs{\xi^\top A_\Im\xi}^2,
%\end{equation}
%mit $A=A_\Re+\i A_\Im$, $A_\Re,A_\Im\in\R^{n\times n}$.
%Die weitere Untersuchung gestaltet sich allerdings recht aufwendig
%und soll nicht weiter vertieft werden.
\end{exa}

\begin{exa}\label{exa:2:schroe}
Der Schrödingergleichung für ein freies Teilchen\index{Differentialoperator!Schrödinger-}
\begin{equation}
\i\frac{\partial}{\partial t}\psi(x,t)=-\Delta\psi(x,t),\end{equation}
lässt sich das Symbol $P(\xi)=\xi^2_1+\dots+\xi^2_{n-1}-\xi_n$ zuordnen.

Wir bestimmen alle Operatoren gleicher Stärke.
Das Symbol $Q(\xi)$ ist genau dann schwächer wie $P(\xi)$,
wenn
\begin{equation}\label{eq:2:subschroe}
Q(\xi)^2\apprle(\xi^2_1+\dots+\xi^2_{n-1}-\xi_n)^2+\xi^2_1+\dots+\xi^2_{n-1}+1.
\end{equation}
Also hat $Q(\xi)$ maximalen Grad 2 in $\xi_1,\dots,\xi_{n-1}$,
Grad 1 in $\xi_n$ und ist somit von der Form
\begin{equation}
Q(\xi)=a_0+\sum^n_{k=1}a_k\xi_k+\sum^n_{k,l=1}a_{kl}\xi_k\xi_l,
\end{equation}
mit $a_{nn}=0$.
Wir betrachten die Gleichung \eqref{eq:2:subschroe}
für spezielle Vektoren:
\begin{equation}\label{eq:2:subschroespec}
Q(\xi)^2\apprle(\xi^2_1+\dots+\xi^2_{n-1}+1),
\quad\text{für alle}~\xi\in\R^n~\text{mit}~\xi_n=\xi^2_1+\dots+\xi^2_{n-1}.
\end{equation}
Da wir $\xi_1,\dots,\xi_{n-1}$ in \eqref{eq:2:subschroespec} frei wählen können folgt, dass
\begin{equation}
Q(\xi)=a_0+\sum_{k=1}^{n-1}a_k\xi_k+a_n(\xi_n-\xi^2_1-\dots-\xi^2_{n-1}),
\end{equation}
mit beliebigen $a_0,a_1,\dots,a_n\in\C$.
Es ist $Q(\xi)$ genau dann gleich Stark wie $P(\xi)$,
wenn $a_n\neq0$.
\end{exa}

\begin{exa}\label{exa:2:heat}
Der Wärmeleitungsgleichung\index{Differentialoperator!parabolisch}
\begin{equation}
\frac{\partial}{\partial t}T(x,t)=\Delta T(x,t),
\end{equation}
entspricht das Symbol $P(\xi)=\xi^2_1+\dots+\xi^2_{n-1}+\i\xi_n$.

Ein Symbol $Q(\xi)$ ist genau dann schwächer als $P(\xi)$, wenn
\begin{equation}
Q(\xi)^2\apprle(\xi^2_1+\dots+\xi^2_{n-1})^2+\xi^2_1+\dots+\xi^2_{n-1}+\xi^2_n+1.
\end{equation}
Diese Ungleichung ist genau dann erfüllt, wenn
\begin{equation}
Q(\xi)=a_0+\sum_{k=1}^na_k\xi_k+\sum_{k,l=1}^{n-1}a_{kl}\xi_k\xi_l,
\end{equation}
mit beliebigen $a_0,a_k,a_{kl}\in\C$.
Das Symbol $Q(\xi)$ ist genau dann gleich stark wie $P(\xi)$,
wenn $a_n\neq0$ und $\sum_{k,l=1}^{n-1}a_{kl}\xi_k\xi_l\neq0$ falls $\xi\notin e_n\R$
mit dem $n$-ten Standartbasisvektor $e_n$.
\end{exa}

\begin{exa}\label{exa:2:hyper}
Als letztes Beispiel betrachten wir die ultrahyperbolische Gleichung\index{Differentialoperator!(ultra)hyperbolisch}
\begin{equation}
\Delta_1u=\Delta_2u,
\end{equation}
mit $\Delta_1=\partial_1^2+\dots+\partial_m^2$, $\Delta_2=\partial^2_{m+1}+\dots+\partial^2_n$.
Das korrespondierende Symbol ist $P(\xi)=\xi^2_1+\dots+\xi^2_m-(\xi^2_{m+1}+\dots+\xi^2_n)$.

Ein Symbol $Q(\xi)$ ist genau dann gleich stark wie $P(\xi)$,
wenn
\begin{equation}
Q(\xi)^2\apprle(\xi^2_1+\dots+\xi^2_m-(\xi^2_{m+1}+\dots+\xi^2_n))^2+\xi^2_1+\dots+\xi^2_n+1.
\end{equation}
Setzen wir wie in Beispiel \ref{exa:2:schroe} $\xi^2_1+\dots+\xi^2_m=\xi^2_{m+1}+\dots+\xi^2_n$,
so erhalten wir für diese $\xi$:
\begin{equation}\label{eq:2:rpQhyp}
Q(\xi)^2\apprle\xi^2_1+\dots+\xi^2_n+1.
\end{equation}
Da wir in \eqref{eq:2:rpQhyp} noch genug Wahlfreiheit haben folgt,
dass $Q(\xi)$ von der Form
\begin{equation}
Q(\xi)=a_0+\sum_{k=1}^na_k\xi^k+b(\xi_1^2+\dots+\xi^2_m-(\xi^2_{m+1}+\dots+\xi^2_n)),
\end{equation}
mit beliebigen $a_0,a_1,\dots,a_n,b\in\C$ sein muss
und $Q(\xi)$ ist genau dann gleich stark wie $P(\xi)$,
wenn $b\neq0$.
\end{exa}

Wir wollen einige Unterschiede der betrachteten Beispiele festhalten.
Vergleichen wir die Beispiele \ref{exa:2:lap}, \ref{exa:2:hyper}
mit den Beispielen \ref{exa:2:schroe}, \ref{exa:2:heat}.
In den ersten beiden Fällen hängt der Vergleich der Stärke nur von den Hauptteilen,
d.h.~den Potenzen größter Ordnung ab.
In den zweiten zwei Fällen hängt die Vergleich auch von niederen Termen ab.
Dies hängt damit zusammen, dass sich die Symbole $P(\xi)$ in
\ref{exa:2:schroe} und \ref{exa:2:heat} nicht auf allen (nicht-trivialen)
Unterräumen des $\R^n$ wie die höchste im Symbol vorkommende Potenz verhalten.

Auch die Paare \ref{exa:2:lap}, \ref{exa:2:heat}
und \ref{exa:2:schroe}, \ref{exa:2:hyper}
weisen einen große Unterschied auf.
Bei den ersten beiden Operatoren hat man viel Wahlfreiheit
für gleich starke Operatoren.
Bei den zweiten zwei Operatoren hat man im Hauptteil stets
nur einen frei wählbaren Parameter.
Dies hängt damit zusammen, dass in den Symbolen $P(\xi)$
in \ref{exa:2:schroe} und \ref{exa:2:hyper} Differenzterme auftreten.

\begin{df}
\begin{enumerate}
\item
Ein Differentialausdruck $P(\D)$ heißt \eIndex[Differentialoperator]{vom Haupttyp},
falls er gleich stark ist wie jeder Differentialausdruck mit gleichem Hauptteil.
\item
Ein Differentialausdruck $P(\D)$ heißt \eIndex[Differentialoperator]{elliptisch},
falls er stärker ist als jeder Differentialausdruck von kleiner oder gleichem Grad.
\end{enumerate}
\end{df}

Wie eben diskutiert sind also der Laplaceoperator
und der Operator aus Beispiel \ref{exa:2:hyper} vom Haupttyp,
die anderen beiden nicht. Der Laplaceoperator ist als einziger der angegebenen Operatoren elliptisch.
Wir geben nun Klassifikationen für diese Operatortypen.

\begin{thm}[{\cite[Theorem~2.3]{Hormander:1955}}]\label{thm:2:2.9}
Ein Symbol $P(\xi)$ ist genau dann vom Haupttyp,
wenn die partiellen Ableitungen $p^{(k)}(\xi)=\partial_k p(\xi)$
des Hauptteils $p(\xi)$ für kein $\xi\in\R^n\setminus\{0\}$
alle gleichzeitig Null sind.
\end{thm}

\begin{proof}
{\it Hinrichtung.} Ist $P(\xi)$ vom Haupttyp und vom Grad $m>0$,
dann gilt das gleiche für $p(\xi)$.
Dann ist $p(\xi)$ auch stärker als $p(\xi)+\xi^\alpha$
mit $\abs{\alpha}=m-1$.
Nach Theorem \ref{thm:2:2.1} und der Linearität von $W(\mcP)$
folgt dann
\begin{equation}\label{eq:2:homquot}
\sup_{\xi\in\R^n}\frac{(\xi^2_1+\dots+\xi^2_n)^{m-1}}{\sum\abs{p^{(\alpha)}(\xi)}^2}<\infty.
\end{equation}
Angenommen es gibt ein $\xi_0\in\R^n$,
so dass $p^{(i)}(\xi_0)=0$ für alle $i=1,\dots,n$.
Dann gilt auch $p(\xi_0)=0$, da nach Eulers Formel für homogene Polynome folgt
\begin{equation}
mp(\xi)=\sum_{k=1}^n\xi_kp^{(k)}(\xi)
\end{equation}
Wir setzen $\xi=t\xi_0$, $t\in\R$, in Gleichung \eqref{eq:2:homquot},
und erhalten einen Quotienten aus Polynomen in der Variablen $t$.
Aufgrund der Homogenität gilt
\begin{equation}
p(t\xi_0)=t^mp(\xi_0)=0,~p^{(1)}(t\xi_0)=t^{m-1}p^{(1)}(\xi_0)=0,
\dots,~p^{(n)}(t\xi_0)=t^{m-1}p^{(n)}(\xi_0)=0,
\end{equation}
und somit hat das Polynom im Zähler echt kleineren Grad als $2(m-1)$.
Dies liefert einen Widerspruch für $t\to\infty$. $\bullet$\qquad {\it Rückrichtung.}
Nehmen wir nun umgekehrt an, $P(\xi)$ erfüllt die Bedingung aus Theorem \ref{thm:2:2.9}
und $Q(\xi)$ hat den gleichen Hauptteil $p(\xi)$ wie $P(\xi)$.
Durch Weglassen positiver Term erhalten wir 
\begin{equation}
\widetilde{P}(\xi)^2>\sum_{k=1}^n\abs{P^{(k)}(\xi)}^2=\pi(\xi)+r(\xi).
\end{equation}
Dabei fassen wir die höchsten Terme des Ausdrucks
$\sum_{k=1}^n\abs{P^{(k)}(\xi)}^2$ zu $\pi(\xi)$ zusammen.
Diese sind dann gegeben durch
\begin{equation}
\pi(\xi)=\sum_{k=1}^n\abs{p^{(k)}(\xi)}^2,
\end{equation}
was direkt aus der binomischen Formel
und der Additivität des Grads von Polynomen folgt.
Also hat $r(\xi)$ Grad echt kleiner als $2(m-1)$.
Nach Annahme ist $\pi(\xi)\neq0$ falls $\xi\neq0$
and somit $\abs{r(\xi)}/\pi(\xi)\leq\tfrac12$,
für $\abs{\xi}$ groß genug.
Für diese $\xi$ ist dann auch $\widetilde{P}(\xi)^2>\pi(\xi)/2$.
Mit der Dreiecksungleichung erhalten wir nun
\begin{equation}
\frac{\abs{Q(\xi)^2}}{\widetilde{P}(\xi)^2}
\leq\frac{\abs{P(\xi)}^2}{\widetilde{P}(\xi)^2}+\frac{\abs{Q(\xi)-P(\xi)}^2}{\widetilde{P}(\xi)^2}
\leq1+2\frac{\abs{Q(\xi)-P(\xi)}^2}{\pi(\xi)},
\end{equation}
für $\abs{\xi}$ groß genug.
Das Polynom im Zähler des zweiten Terms hat maximal den Grad $2(m-1)$.
Insgesamt ist damit $\sup_{\xi\in\R^n}\abs{Q(\xi)}/\widetilde{P}(\xi)<\infty$,
also $Q(\D)$ schwächer als $P(\D)$.
\end{proof}

Der Satz und sein Beweis bestätigen also die Anmerkungen nach den Beispielen.
Elliptische Operatoren lassen ähnlich sich klassifizieren.

\begin{thm}
Ein Differentialausdruck $P(\D)$ ist genau dann elliptisch,
wenn der homogene Hauptteil $p(\xi)$ von $P(\xi)$ für alle $\xi\in\R^n\setminus\{0\}$ die Bedingung $p(\xi)\ne0$ erfüllt.
\end{thm}


Zum Abschluss betrachten wir Produktoperatoren auf Produkträumen.
Dies sind Differentialoperatoren mit Symbolen von der Form
\begin{equation}\label{eq:2:prodpol}
P(\xi)=P_a(\xi_a)P_b(\xi_b),
\end{equation}
mit $\xi_a=(\xi_1,\dots,\xi_m)$ und $\xi_b=(\xi_{m+1},\dots,\xi_n)$,
wobei $0<m<n$ fest gewählt ist
und $P_a(\xi)$ ein Polynom in $\R^m$
und $P_b(\xi)$ ein Polynom in $\R^{n-m}$ ist.
Wir erhalten:

\begin{thm}
Für Produktpolynome wie in \eqref{eq:2:prodpol}
gilt die Gleichung
\begin{equation}
W(\mcP)=\spann\left(W(\mcP_a) W(\mcP_b)\right),
\end{equation}
$W(\mcP)$ ist also das Tensorprodukt von $W(\mcP_a)$ und $W(\mcP_b)$.
\end{thm}
\begin{proof}
Wir betrachten zunächst einen Multiindex $\alpha\in\mathbb N_0^n$ und setzen $\alpha_a=(\alpha_1,\dots,\alpha_m,0,\dots,0)$
und $\alpha_b=(0,\dots,0,\alpha_{m+1},\dots,\alpha_n)$. Dann gilt
\begin{equation}
P^{(\alpha)}(\xi)
=\partial^\alpha\left(P_a(\xi_a)P_b(\xi_b)\right)
=P^{(\alpha_a)}_a(\xi_a)P^{(\alpha_b)}_b(\xi_b).
\end{equation} 
Also folgt, dass
\begin{equation}\label{eq:2:prodeq}
\widetilde{P}(\xi)=\sum_\alpha\abs{P^{(\alpha)}(\xi)}^2
=\sum_{\alpha_a,\alpha_b}\abs{P^{(\alpha_a)}_a(\xi_a)}^2\abs{P^{(\alpha_b)}_b(\xi_b)}^2
=\widetilde{P}_a(\xi_a)\widetilde{P}_b(\xi_b).
\end{equation}
Dies zeigt $\spann\left(W(\mcP_a) W(\mcP_b)\right)\subseteq W(\mcP)$. 

Nehmen wir nun an, dass $Q(\xi)\in W(\mcP)$.
Betrachten wir $Q(\xi_a,\xi_{b})$ mit variablem $\xi_a$ und festen $\xi_{b}$,
so folgt aus Gleichung \eqref{eq:2:prodeq},
dass $Q(\xi_a,\xi_{b})\in W(\mcP_a)$.
Analoges gilt bei vertauschten Rollen von $a$ und $b$.
Wir wählen eine Vektoraumbasis $p_1(\xi_a),\dots,p_l(\xi_a)$ von $W(\mcP_a)$.
Es gilt
\begin{equation}
\sum_{k=1}^lp_k(\xi_l)a_k(\xi_b),
\end{equation}
mit Polynomen $a_k(\xi_b)$.
Wir zeigen $a_k(\xi_b)\in W(\mcP_b)$ für alle $k=1,\dots,K$.
Da die $p_1(\xi_a),\dots,p_K(\xi_a)$ linear unabhängig sind,
gibt es $\xi_{a,1},\dots,\xi_{a,K}$,
so dass $\left(p_k(a,\xi_l)\right)_{k,l=1,\dots,K}$ invertierbar ist.
Das Gleichungssystem
\begin{equation}
Q(\xi_{a,l},\xi_b)=\sum_{k=1}^lp_k(\xi_{a,l})a_k(\xi_b)\quad\text{für}~k=1,\dots,K,
\end{equation}
kann also nach den $a_k(\xi_b)$ aufgelöst werden kann.
Dies impliziert nach obigem $a_k(\xi_b)\in W(\mcP_b)$.
\end{proof}

\section{Ein weiteres Vergleichsresultat}
Wir wollen ein weiteres Vergleichsresultat angeben, welches ebenfalls aus dem Quotienten der regularisierten Symbole $\widetilde{P}$ und $ \widetilde{Q}$ hervorgeht. Es handelt sich hierbei um die Stetigkeit von $Q(\D) u$ für $u$ aus dem Definitionsbereich des minimalen Operators $P_0$. Das Resultat verallgemeinert den sobolevschen Einbettungssatz.

\begin{thm}[{\cite[Theorem~2.6]{Hormander:1955}}]
Sei $Q(D)$ schwächer als $P(D)$. Angenommen, $Q(\D)u$ ist für jedes $u\in\mathcal D_{P_0}$ beschränkt. Dann gilt
\begin{equation}\label{satz: bedingung stetigkeit}
\int \frac{|Q(\xi)|^2}{\widetilde{P}(\xi)^2} \d \xi < \infty.
\end{equation}
Ist andererseits \eqref{satz: bedingung stetigkeit} erfüllt, dann ist $Q(D)u$ stetig und es existiert für jedes $\varepsilon >0$ eine kompakte Menge $K \subseteq \Omega$ mit
\begin{align*}
 \forall x \in \Omega \backslash K\quad:\quad |Q(\D)u(x)| < \varepsilon,
\end{align*}
d.h. $Q(\D)u$ verschwindet auf  $\partial\Omega$.
\end{thm}
\begin{proof}{\it Hinrichtung.}
Angenommen für jedes $u \in \mathcal{D}_{P_0}$ gilt $Q(\D)u\in\rmL^\infty(\Omega)$. 
Damit folgt aus Lemma \ref{lm: Qu stetig}  für alle $u \in \rmC_0^{\infty}(\Omega)$
\begin{equation}
\sup_{x \in \Omega} | Q(\D)u(x)|^2 \leq C ( \Vert  P(\D)u\Vert^2 + \Vert u\Vert^2 ),
\end{equation}
also insbesondere
\begin{equation}\label{eq:abschneide null}
|Q(\D)u(0)|^2 \leq C( \Vert  P(\D)u\Vert^2 + \Vert u\Vert^2 ) 
\end{equation}
mit einer von $u\in\rmC_0^\infty(\Omega)$ unabhängigen Konstanten $C>0$. Für eine schnell fallende Funktion $ \varphi \in \mathscr S(\R^n)$ aus dem Schwartzraum definieren wir
\begin{equation}
v(x)= (2 \pi)^{-\frac{n}{2}} \int_{\Omega}\e^{\i x \cdot \xi} \frac{\varphi (\xi)}{\widetilde{P}(\xi)} \d \xi,
\end{equation}
so daß mit dem Satz von Plancherel
\begin{equation}\label{eq:abschneide phi}
\Vert P^{(\alpha)}(\D)v \Vert^2 = \int |\widehat{v}(\xi)|^2 |P^{(\alpha)}(\xi)|^2 \d \xi = \int \frac{|\varphi(\xi)|^2}{\widetilde{P}(\xi)^2} | P^{(\alpha)} (\xi)|^2 \d \xi \leq \int |\varphi(\xi)|^2 \d \xi= \Vert \varphi \Vert^2
\end{equation}
folgt.
Wir wählen eine geeignete Abschneidefunktion $\psi \in \rmC_0^{\infty}(\Omega)$ mit $\psi(x) \leq 1$ auf $\Omega$ und $\psi(x)=1$ in einer Umgebung von $x=0$. Dann gilt $\tilde{u} = \psi v \in \rmC_0^{\infty}(\Omega)$ und mit der leibnizschen Produktformel folgt
\begin{equation}
P(\D)\tilde{u} = P(\D)(\psi v) = \sum_{\alpha} \frac{\D^{\alpha} \psi}{\alpha !} P^{(\alpha)}(\D) v.
\end{equation}
Mithilfe von \eqref{eq:abschneide phi} ergibt sich also
\begin{equation}\label{eq:abschatz phi}
\Vert P(\D) \tilde{u} \Vert \leq C \Vert \varphi \Vert
\end{equation}
mit einer geeigneten von $\varphi$ unabhängigen Konstanten $C>0$.
Wegen
\begin{equation}
v = (2 \pi)^{-\frac{n}{2}} \int \e^{\i x \cdot \xi} \widehat{v}(\xi) \d \xi, \qquad Q(\D)\e^{\i x \cdot \xi} = \e^{\i x \cdot \xi} Q(\xi), \qquad Q(\D)\tilde{u}(0) = Q(\D)v(0),
\end{equation}
und \eqref{eq:abschneide null}, \eqref{eq:abschatz phi} folgt schließlich
\begin{equation}
|Q(\D)\tilde{u}(0)|^2 = |Q(\D)v(0)|^2 = \left|(2 \pi)^{-\frac{n}{2}} \int \frac{Q(\xi)}{\widetilde{P}(\xi)} \varphi(\xi) \d \xi \right|^2 \leq {C} \Vert \varphi \Vert^2
\end{equation}
und da $\varphi\in\mathscr S(\R^n)$ beliebig war folgt mit dem Darstellungssatz von Fréchet-Riesz
\begin{equation}\label{eq:behauptung L2}
\int \frac{|Q(\xi)|^2}{\widetilde{P}(\xi)^2} \d \xi < \infty
\end{equation}
und somit die Behauptung. $\bullet$\qquad {\it Rückrichtung.}
Gelte nun \eqref{eq:behauptung L2}. Dann folgt für $u \in \rmC_0^{\infty}(\Omega)$ mit der Ungleichung von Cauchy--Schwarz
\begin{align}
|Q(\D)u(x)|^2 &=  \left|(2 \pi)^{-\frac{n}{2}} \int \e^{\i x \cdot \xi} Q(\xi) \widehat{u}(\xi)  \d \xi \right|^2 \notag
\\ & \leq (2 \pi)^{-\frac{n}{2}} \int \frac{|Q(\xi)|^2}{\widetilde{P}(\xi)^2} \d \xi \int \widetilde{P}(\xi)^2 | \widehat{u}(\xi)|^2 \d \xi = C \sum_{\alpha} \Vert P^{(\alpha)}(\D)u\Vert^2.
\end{align}
Mit Korollar \ref{cor:2:2.6} folgt also für alle $u \in \rmC_0^{\infty}(\Omega)$
\begin{equation}
|Q(\D)u(x)|^2 \leq C \Vert P(\D)u\Vert^2
\end{equation}
gleichmäßig in $x\in\Omega$ und damit mit Lemma \ref{lm: Qu stetig} die Behauptung.
\end{proof}

\section{Kompaktheitskriterien}
Satz \ref{thm:2:2.1} gibt ein Kriterium dafür, daß der Operator
\begin{equation}\label{eq:2:map}
\mcR_{P_0}\ni P(\D)u\mapsto Q(\D)u\in  \mcR_{Q_0}
\end{equation}
$\rmL^2$--$\rmL^2$-beschränkt ist. Im folgenden fragen wir nach der Kompaktheit dieses Operators. 

\begin{df}
Seien $V$ und $W$ zwei Banachräume und $T: V \supset \mathcal{D}_T \rightarrow W$ ein linearer Operator. $T$ heißt \eIndex[Operator]{kompakt}, wenn für jede Folge $(u_n)_{n \in \mathbb{N}}$ in $\mathcal{D}_T$ mit $\Vert u_n \Vert \leq 1$ für alle $n \in \mathbb{N}$ eine Teilfolge $(u_{n_k})_{k \in \mathbb{N}}$ existiert, so daß deren Bildfolge $(Tu_{n_k})_{k\in \mathbb{N}}$ in $W$ konvergiert.
\end{df}

Wir sagen im folgenden der Differentialausdruck $P(\D)$ \eIndex[Differentialoperator]{dominiert} den Differentialausdruck $Q(\D)$, wenn die Abbildung aus \eqref{eq:2:map} kompakt ist. Die Eigenschaft dominierend zu sein ist wiederum unabhängig vom beschränkten Gebiet $\Omega$.

\begin{thm}[{\cite[Theorem~2.15]{Hormander:1955}}]\label{thm:2:Abbildung kompakt}
Der Differentialausdruck $P(\D)$ dominiert genau dann den Differentialausdruck $Q(\D)$,
wenn
\begin{equation}\label{eq:2:tozero}
\lim_{|\xi|\to\infty} \frac{\widetilde{Q}(\xi)}{\widetilde{P}(\xi)} = 0
\end{equation}
gilt.
\end{thm}
\begin{proof} {\it Rückrichtung.}
Wir nehmen zunächst an, dass \eqref{eq:2:tozero} erfüllt ist und betrachten eine Folge ${(u_n)}_{n \in \mathbb{N}} $ in $\rmC^\infty_0(\Omega)$ mit
\begin{equation}\label{eq:2:PDb}
\norm{P(\D)u_n}\leq 1.
\end{equation}
Wir zeigen, dass dann eine Teilfolge ${(u_{n_k})}_{k \in \mathbb{N}}$ in $\rmC^\infty_0(\Omega)$ existiert,
so dass  $\left(Q(\D)u_{n_k}\right)_{k \in \mathbb{N}}$ konvergent ist. Nach Voraussetzung gilt 
\begin{equation}
\frac{\widetilde{Q}(\xi)^2}{\widetilde{P}(\xi)^2} = \frac{\sum_{\alpha} | Q^{(\alpha)} (\xi)|^2  }{ \sum_{\alpha} | P^{(\alpha)} (\xi)|^2} \rightarrow 0\qquad  \text{für} \ \ \xi \rightarrow \infty;
\end{equation}
insbesondere ist $\widetilde Q(\xi)/\widetilde P(\xi)$ beschränkt. Also existiert eine Konstante $C>0$, so dass
\begin{equation}
|Q(\xi)|^2 \leq C \sum_{\alpha} | P^{(\alpha)} (\xi)|^2 \label{eq:1.5}
\end{equation}
gilt. Damit folgt mit dem Satz von Plancherel für beliebiges $u\in\rmC_0^\infty(\Omega)$
\begin{multline}
\int |Q(\D)u(x)|^2 \d x = \int |Q(\xi) \widehat{u}(\xi)|^2 \d \xi \\
\leq  \sum_{\alpha}C\int | P^{(\alpha)} (\xi) \widehat u(\xi)|^2 \d \xi =\sum_{\alpha} C\int |P^{(\alpha)}(\D)u(x)|^2 \d x.
\end{multline}
Weiter liefert uns  Satz \ref{thm:2:2.1} eine Konstante $\widetilde{C}>0$ mit
\begin{equation}
\Vert P^{(\alpha)}(\D)u \Vert \leq \widetilde{C} \Vert P(\D)u \Vert, \label{eq:ableitung leq PD}
\end{equation}
woraus sich dann mit \eqref{eq:2:PDb} sofort
\begin{equation}
\norm{Q(\D)u_n}\leq C' \norm{P(\D)u_n} \leq C' 
\end{equation}
für die gegebene Folge $(u_n)_{n\in\N}$ ergibt. Da $\Omega$ beschränkt ist, ist auch die $\rmL^1$-Norm $\|Q(\D)u_n\|_1$ gleichmäßig beschränkt und die 
Fouriertransformierten $Q(\xi) \widehat{u}_n(\xi)$ sind gleichmäßig in $\xi$ und $n$ beschränkt. Wir zeigen, daß sie auch gleichgradig stetig sind. Sei dazu $\varepsilon>0$ und $\delta>0$ so gewählt, dass für $|\xi_1 - \xi_2| < \delta$ stets $|\e^{\i x \cdot \xi_1} - \e^{\i x \cdot \xi_2}|\le\epsilon$ gilt. Dann folgt mit Ungleichung von Cauchy--Schwarz 
\begin{align}
| Q(\xi_1)\widehat{u}_n(\xi_1) - Q(\xi_2)\widehat{u}_n(\xi_2)| & \leq (2\pi)^{-n/2} \int_{\Omega} \left| \e^{\i x \cdot \xi_1} - \e^{\i x \cdot \xi_2} \right| | Q(\D) u_n(x)| \d x \notag
\\ & \leq \varepsilon |\Omega|^{\frac{1}{2}} \Vert Q(\D)u_n \Vert \leq \varepsilon \widetilde{C}.
\end{align}
Also können wir nach dem Satz von Arzel\`a-Ascoli eine lokal gleichmäßig konvergente Teilfolge $(Q(\xi)\widehat{u}_{n_k})_{k\in\N}$ auswählen, d.h. für jedes Kompaktum $K\Subset\R^n$ konvergiert $Q(\xi)\widehat{u}_{n_k}(\xi)$ gleichmäßig.
Nach Voraussetzung \eqref{eq:2:tozero} lässt sich für jedes $\varepsilon > 0$ eine kompakte Menge $K \Subset \R^n$ finden, so dass 
\begin{equation}
\forall \xi\in\R^n\setminus K \quad:\quad \frac{\abs{Q(\xi)}}{\widetilde{P}(\xi)} <\varepsilon
\end{equation}
gilt. Damit folgt nun
\begin{align}
\int_{\R^n\setminus K}\abs{Q(\xi)}^2\abs{\widehat{u}_{n_k}(\xi)-\widehat{u}_{n_l}(\xi)}^2\d \xi
&\leq\varepsilon^2\int\widetilde{P}(\xi)^2\abs{\widehat{u}_{n_k}(\xi)-\widehat{u}_{n_l}(\xi)}^2\d \xi\notag\\
&=\varepsilon^2\sum_\alpha \int  \abs{ P^{(\alpha)}(\xi) (\widehat{u}_{n_k}(\xi)-\widehat{u}_{n_l}(\xi) ) }^2\d \xi\notag\\
&=\varepsilon^2\sum_\alpha \norm{P^{(\alpha)}(\D)(u_{n_k}-u_{n_l})}^2\label{eq:2:Kcsum}
\end{align}
und diese Summe ist nach \eqref{eq:ableitung leq PD} und \eqref{eq:2:PDb} beschränkt.
Da $Q(\xi)\widehat u_{n_k}(\xi)$ auf $K$ gleichmäßig konvergiert, gilt außerdem
\begin{equation}
\int_K\abs{Q(\xi)}^2\abs{\widehat{u}_{n_k}(\xi)-\widehat{u}_{n_l}(\xi)}^2\d \xi\longrightarrow0
\quad\text{für}~ k,l \rightarrow \infty,
\end{equation}
und damit die Konvergenz der Teilfolge $\left(Q(\D)u_{n_k} \right)_{k \in \mathbb{N}}$ in $\rmL^2(\Omega)$. 
Da die Funktionen $P(\D)u$ für $u\in \rmC^\infty_0(\Omega)$ dicht in $\mcR_{P_0}$ liegen, dominiert $P(\D)$ den Operator $Q(\D)$.
$\bullet$\qquad{\it Hinrichtung.}
Wir nehmen nun an, dass $P(\D)$ den Operator $Q(\D)$ dominiert und zeigen, dass \eqref{eq:2:tozero} erfüllt ist.
 Sei dazu $(\xi_n)_{n \in \mathbb{N}}$ eine beliebige Folge in $\mathbb{R}^n$ mit $\xi_n \rightarrow \infty$. Wir zeigen, dass $\widetilde{Q}(\xi_{n})/\widetilde{P}(\xi_{n})\to0$ für $n \rightarrow \infty$ gilt. Da nach Satz~\ref{thm:2:2.1} der Quotient $\widetilde Q(\xi)/\widetilde P(\xi)$ beschränkt ist, reicht es zu zeigen, daß für jede Folge $(\xi_n)_{n \in \mathbb{N}}$, für welche der Quotient $\widetilde Q(\xi_n)/\widetilde P(\xi_n)$ konvergiert, dessen Grenzwert Null sein muß. Durch Wahl einer Teilfolge kann man weiter annehmen, daß $\xi_n-\xi_m\to\infty$ für $n,m\to\infty$, $n\ne m$ gilt. Für eine solche Folge konstruieren wir nun eine Folge von Testfunktionen.
Sei dazu  $u\in \rmC^\infty_0(\Omega)$, $u\neq0$, und für $n\in\N$ definieren wir
\begin{equation}
u_n(x)=u(x)\frac{\e^{\i x \cdot \xi_n}}{\widetilde{P}(\xi_n)}.
\end{equation}
Dann liefert die leibnizsche Produktregel
\begin{equation}
P(\D)u_n(x)=\e^{\i x \cdot \xi_n}\sum_\alpha\frac{P^{(\alpha)}(\xi_n)}{\widetilde{P}(\xi_n)}\frac{\D^\alpha u(x)}{\alpha!}
\end{equation}
und wir finden Konstante $C>0$ mit
\begin{equation}\label{eq:2:PDunb}
\norm{P(\D)u_n}\leq C.
\end{equation}
Ausmultiplizieren liefert nun
\begin{equation}\label{eq:2:kz}
\norm{Q(\D)u_n-Q(\D)u_m}^2=\norm{Q(\D)u_n}^2+\norm{Q(\D)u_m}^2-2\Re\spro{Q(\D)u_n}{Q(\D)u_m}, 
\end{equation}
wobei das Innenprodukt durch
\begin{equation}
\spro{Q(\D)u_n}{Q(\D)u_m}=
\sum_{\alpha,\beta}\frac{Q^{(\alpha)}(\xi_n)}{\widetilde{P}(\xi_n)}\frac{\cc{Q^{(\beta)}(\xi_n)}}{\widetilde{P}(\xi_n)}
\frac1{\alpha!\beta!}\int_\Omega \e^{\i x \cdot(\xi_n-\xi_m)} \D^\alpha u(x) \cc{\D^\beta u(x) }\d  x
\end{equation}
gegeben ist. Da die Faktoren $Q^{(\alpha)}(\xi_n)/\widetilde{P}(\xi_n)$ nach Satz \ref{thm:2:2.1} beschränkt und $\D^\alpha u\cc{\D^\beta u}$ integrierbar sind, folgt mit dem Riemann--Lebesue-Lemma $\spro{Q(\D)u_n}{Q(\D)u_m}\to0$ für $n,m\to\infty$, $n\neq m$. Aufgrund der Kompaktheit finden wir eine Teilfolge $(u_{n_k})_{k\in\N}$ für die $(Q(\D)u_{n_k})_{k \in \mathbb{N}}$ konvergiert. Damit muß nach \eqref{eq:2:kz}  aber $\|Q(\D)u_{n_k}\|\to0$, $k\to\infty$ gelten, d.h.
\begin{equation}
\norm{Q(\D)u_{n_k}}^2=\sum_{\alpha,\beta}\frac{Q^{(\alpha)}(\xi_{n_k}) \overline{Q^{(\beta)}(\xi_{n_k})}}{\widetilde{P}(\xi_{n_k})^2} \underbrace{\frac{1}{\alpha! \beta!}\int \D^{\alpha} u \overline{\D^{\beta} u} \d x}_{=\Psi_{\alpha \beta}} \longrightarrow 0
\end{equation}
und damit wegen der Äquivalenz der Innenprodukte in \eqref{eq:2.8} auch 
\begin{equation}
\frac{\widetilde{Q}(\xi_{n_k})}{\widetilde{P}(\xi_{n_k})} \longrightarrow 0.
\end{equation}
\end{proof}
\begin{exa}
Wir betrachten einen Differentialausdruck $P(\D)$, welcher durch
\begin{equation}
P(\D) = -\i\partial_1 + \partial_2^2
\end{equation}
gegeben ist. Für das dazugehörige Symbol $P(\xi)$ ergibt sich dann
\begin{equation}
P(\xi) = \xi_1 - \xi_2^2.
\end{equation}
Für jedes $\xi \not = 0$ gilt somit
\begin{align}
|P(\xi t)| \rightarrow \infty \qquad \text{für} \ t \rightarrow \infty,
\end{align}
aber nicht zwingend
\begin{align}
|P(\xi)| \rightarrow \infty \qquad \text{für} \ \xi \rightarrow \infty,
\end{align}
da die Nullstellenmenge
\begin{equation}
N= \{ (\xi_1,\xi_2)\in \mathbb{R}^2 : \ \xi_1 = \xi_2^2 \}
\end{equation}
parabelförmig ins Unendliche läuft. Betrachtet man jedoch das regularisierte Symbol
\begin{equation}
\widetilde{P}(\xi)^2 = \sum_{|\alpha| \leq 2} |P^{\alpha}(\xi)|^2 = |\xi_1 - \xi_2^2|^2 + |2\xi_2|^2 + 2,
\end{equation}
dann gilt $\widetilde{P}(\xi) \rightarrow \infty$ für $\xi \rightarrow \infty$. Speziell mit $Q(\D)=I$ impliziert obiger Satz somit aus
$1/\widetilde P(\xi)\to0$ die Kompaktheit von $P_0^{-1}$.
\end{exa}
Wir werden im Folgenden zeigen, dass der Operator $P_0^{-1}$ genau dann kompakt ist, wenn das dazugehörige Symbol $P(\xi)$ von allen Variablen abhängt. Dazu vorbereitend geben wir zunächst folgende Definition.
\begin{df}
Sei $P(\xi)$ das Symbol eines Differentialausdrucks $P(\D)$. Dann bezeichnen wir mit
\begin{equation}
\Lambda(P)=\{ \nu \in \mathbb{R}^n| \ \ \forall \xi \in \mathbb{R}^n, \forall t \in \mathbb{R} \;:\; P(\xi + t\nu) = P(\xi)  \}
\end{equation}
den \eIndex[Polynom]{Linienraum}\footnote{Dieser wurde von G\r{a}rding eingeführt und beschreibt den größten Unterraum des $\R^n$, so daß das Polynom $P$ auf den Quotienten $\R^n/\Lambda(P)$ projiziert werden kann.} von $P$. Wir nennen $P(\xi)$ \eIndex[Differentialoperator]{vollständig}, wenn $\Lambda(P)=\{0\}$ gilt.
\end{df}
\begin{lem}[{\cite[Lemma~2.13]{Hormander:1955}}]\label{Lema:homogene Polynome} 
Sei $Q(\xi)$ ein homogenes Polynom vom Grad $m \in \mathbb{N}$ und  $\nu \in \mathbb{R}^n$ mit
\begin{equation}
Q^{(\alpha)}(\nu)=0
\end{equation}
für alle $\alpha\in\N_0^n$ mit $|\alpha|=m-1$. Dann gilt $\nu \in \Lambda(Q)$.
\end{lem}
\begin{proof}
Für $m=1$ gilt offensichtlich $\alpha = 0$ und es ist nichts zu zeigen. Sei also $m>1$ und wir nehmen an, dass die Behauptung schon für Polynome vom Grad kleiner $m$ gezeigt ist. Die partielle Ableitung $\partial Q/\partial_{\xi_j}$ verschwindet entweder identisch oder ist homogen vom Grad $m-1$. Da sie ebenso die Voraussetzungen des Lemmas erfüllt gilt also nach Induktionsvoraussetzung
\begin{equation}
\partial Q(\xi + t \nu)/ \partial \xi_j = \partial Q(\xi)/\partial \xi_j.
\end{equation}
Da dies für alle $j\in\{1,\ldots,n\}$ gilt, ist der Ausdruck $Q(\xi + t \nu) - Q(\xi)$ unabhängig von $\xi$ und wir erhalten
\begin{equation}
Q(\xi + t \nu) - Q(\xi) = Q(t \nu).
\end{equation}
Speziell mit $t=1$ und $\xi=\nu$ folgt
\begin{equation}
Q(\xi + t\nu) - Q(\xi) = Q(2 \nu) - Q(\nu)= Q(\nu), 
\end{equation}
also $2^mQ(\nu)=Q(2\nu) = 2 Q(\nu)$ und damit $Q(\nu)=0$, also insbesondere
\begin{equation}
Q(\xi + t\nu) = Q(\xi)
\end{equation}
und damit $\nu \in \Lambda(Q)$.
\end{proof}
Mit diesem Lemma und Satz \ref{thm:2:Abbildung kompakt} können wir nun das versprochene Kompaktheitskriterium für $P_0^{-1}$ angeben.
\begin{thm}[{\cite[Theorem~2.17]{Hormander:1955}}]
Der Operator $P_0^{-1}$ ist genau dann kompakt, wenn das zu $P(\D)$ gehörende Symbol $P(\xi)$ ein vollständiges Polynom ist.
\end{thm}
\begin{proof}{\em Hinrichtung.}
Angenommen $P(\xi)$ ist nicht vollständig, es existiert also ein von Null verschiedener Vektor $0 \not = \nu \in \mathbb{R}^n$ mit
\begin{equation}
P(\xi + t \nu)  = P(\xi)
\end{equation}
für alle $\xi\in\R^n$ und $t\in\R$. Differenzieren dieser Identität liefert
\begin{equation}
\widetilde{P}(\xi + t \nu) = \widetilde{P}(\xi)
\end{equation}
für alle $\xi\in\R^n$ und $t\in\R$ und $\widetilde{P}(\xi) \not \rightarrow \infty$ für $\xi \rightarrow \infty$. Also impliziert Satz \ref{thm:2:Abbildung kompakt} mit $Q(\xi)=1$, dass $P_0^{-1}$ nicht kompakt sein kann. $\bullet$\qquad {\em Rückrichtung.} Sei das Symbol $P(\xi)$ vollständig, d.h. es gilt $\Lambda (P)= \{0\}$. Wir zerlegen das Symbol $P(\xi)$ in seine homogenen Komponenten $P_k(\xi)$, 
\begin{equation}\label{eq:zerlegung des polynoms}
P(\xi) = \sum_{k=0}^m P_k(\xi),\qquad \deg P_k=k.
\end{equation}
Gilt nun $\nu \in \Lambda (P)$, dann folgt für beliebiges $\xi\in\R^n$ und $t\in\R$
\begin{equation}
\sum_{k=0}^m P_k (\xi + t \nu) = \sum_{k=0}^m P_k (\xi)
\end{equation}
und damit unter Ausnutzung der Homogenität auch
\begin{equation}
P_k (\xi + t \nu) = P_k (\xi) 
\end{equation}
für jeden einzelnen Summanden. Also gilt $\Lambda(P) \subseteq \bigcap_{k \leq m} \Lambda (P_k)$. Die umgekehrte Inklusion ergibt sich sofort aus \eqref{eq:zerlegung des polynoms}, woraus wir dann schließlich
\begin{equation}\label{eq:schnitt der homogenen teile}
\Lambda(P) = \bigcap_{k \leq m} \Lambda (P_k) = \{0\}
\end{equation}
folgern. Wir müssen also zeigen, dass $\widetilde{P}(\xi) \rightarrow \infty$ für $\xi \rightarrow \infty$ gilt um Satz \ref{thm:2:Abbildung kompakt} mit $Q(\xi) =1$ erneut anwenden zu können. Dafür zeigen wir, dass die Menge
\begin{equation}
M_{C,P} := \{\xi \in \R^n | \ \widetilde{P}(\xi) \leq C\}
\end{equation}
für jedes $C$ beschränkt ist. {\em Dazu zeigen wir, daß für beliebige Polynome $P$ und Konstanten $C$ die zugeordnete Menge $M_{C,P}$ modulo $\Lambda(P)$ (also auf dem Quotientenraum $\R^n/\Lambda(P)$) beschränkt ist.} Der Beweis erfolgt per Induktion über $m$.

\noindent{\em Induktionsanfang:} Wir betrachten Polynome vom Grad $1$. Diese sind von der Form 
$P(\xi) =  a \cdot \xi + b$ mit $ a  \in \mathbb{C}^n$, $b \in \mathbb{C}$. Für diese gilt $\Lambda(P)=\{\Re a\}^{\perp}\cap\{\Im a\}^\perp$ und $M_{C,P}$ ist wegen $\widetilde P(\xi) \ge (\Re P(\xi))^2+(\Im P(\xi))^2$ ein zu 
$(\spann\{\Re a,\Im a\})^\perp$ paralleler Streifen. Letzterer ist als Teilmenge des Quotientenraumes $\mathbb{R}^n/(\spann\{\Re a,\Im a\})^{\perp}\simeq \spann\{\Re a,\Im a\}$ beschränkt. $\bullet$

\noindent{\em Induktionsschritt:} Angenommen, die Aussage wurde schon für alle Polynome vom Grad kleiner $m$ gezeigt. Sei weiter $P$ vom Grad $m$.
Nach Definition gilt $\widetilde{P}(\xi) \leq C$ für alle $\xi \in M_{C,P}$, insbesondere also auch $|P^{(\alpha)}(\xi)|\leq C$ für jeden Multiindex $\alpha$. Für $|\alpha|=m-1$ unterscheiden sich $P^{(\alpha)}(\xi)$ und $P_m^{(\alpha)}(\xi)$ nur um eine Konstante, d.h. es gilt $P^{(\alpha)}-P_m^{(\alpha)}=p_\alpha \in\mathbb{C}$ und damit $|P_m^{(\alpha)}(\xi)| \leq C'$ für eine neue Konstante $C'>0$ und alle $\xi \in M_{C,P}$. Gilt nun  $P_m^{(\alpha)}(\xi)=0$ für $\xi\in\R^n$ und alle $|\alpha|=m-1$, 
so folgt mit Lemma \ref{Lema:homogene Polynome}, daß $\xi \in \Lambda(P_m)$. 
Damit ist der Nullraum der linearen Abbildung $\xi \mapsto ( P^{(\alpha)}_m(\xi))_{|\alpha|=m-1} \in\C^M$, $M=\#\{\alpha : |\alpha|=m-1\}$, enthalten in $\Lambda(P_m)$
und somit gilt
\begin{equation}
    \widetilde {P}(\xi)^2 \ge \sum_{|\alpha|=m-1} |P^{(\alpha)}(\xi)|^2 =\sum_{|\alpha|=m-1} |P^{(\alpha)}_m(\xi)+p_\alpha |^2  \ge C'' \dist( \xi,\Lambda(P_m) )^2 - C'''
\end{equation}
mit geeigneten Konstanten $C''>0$ und $C'''$. Damit folgt die Beschränktheit von $M_{C,P}$ modulo $\Lambda(P_m)$ in $\mathbb{R}^n/\Lambda(P_m)$. 

Da $P_m^{(\alpha)}(\xi+\eta) = P_m^{(\alpha)}(\xi)$ für $\eta\in\Lambda(P_m)$ und alle $\alpha$ gilt, ist $\widetilde{P_m}(\xi)$ definiert auf $\R^n/\Lambda(P_m)$ und beschränkt auf Kompakta in diesem Raum. Insbesondere ist $\widetilde {P_m}(\xi)\le C''''$ auf $M_{C,P}$. 
Betrachten wir nun das Polynom
\begin{equation}
R(\xi)=P(\xi) - P_m(\xi) = \sum_{k=0}^{m-1} P_k(\xi),
\end{equation}
so folgt mit der Dreiecksungleichung
\begin{equation}
\widetilde{R}(\xi) \leq \widetilde{P}(\xi) + \widetilde{P}_m(\xi).
\end{equation}
Da das Polynom $R$ vom Grad $m-1$ ist, impliziert die Induktionsvoraussetzung, dass $M_{C,R}$ modulo $\Lambda(R)=\bigcap_{k=1}^{m-1} \Lambda (P_k)$ beschränkt ist. Aus $\widetilde{P}_m(\xi) \leq C''''$ in $M_{C,P}$ ergibt sich somit für alle $\xi\in M_{C,P}$
\begin{equation}
\widetilde{R}(\xi) \leq \widetilde{P}(\xi) + \widetilde{P}_m(\xi) \leq C + C'''',
\end{equation}
also $M_{C,P} \subseteq M_{C+C'''',R}$, womit die Beschränktheit von $M_{C,P}$ modulo $\Lambda(R)$ folgt. Zusammen mit der Beschränktheit modulo $\Lambda(P_m)$ folgt die Behauptung. $\bullet$

Da nach Voraussetzung $P$ vollständig und somit $\Lambda(P)$ trivial ist, ist die Menge $M_{C,P}\subset\R^n$ für jedes $C$ beschränkt und $\widetilde P(\xi)\to \infty$ für $\xi\to\infty$.
\end{proof}



 % Minimale Operatoren :: Tilmann Kleiner, Andreas Bitter
% !TEX root = main.tex
\chapter{Maximale Operatoren}

% Voraussichtliche Struktur

Sei $P(\D)$ ein Differentialausdruck und $\Omega\subset\R^n$ ein Gebiet. Wir erinnern an die Definition des maximalen Operators $P$, dieser versteht sich als größtmögliche distributionelle Fortsetzung von $P(\D)$ und ist auf dem Definitionsbereich
\begin{equation}
   \mathcal D_P = \{ u\in\rmL^2(\Omega)\;:\; P(\D)u\in\rmL^2(\Omega) \}
\end{equation}
gegeben. Maximale Operatoren sind im wesentlichen durch ihren Definitionsbereich bestimmt. Es gilt
\begin{thm}
Angenommen, für zwei Differentialausdrücke $P(\D)$ und $Q(\D)$ und die zugeordneten maximalen Operatoren auf einem beschränkten Gebiet $\Omega$ gilt $\mathcal D_P\subseteq\mathcal D_Q$. Dann gilt
entweder $Q(\xi)=a P(\xi) + b$ mit Konstanten $a,b\in\C$ oder es gibt einen Vektor $\nu\in\R^n$ und Polynome $p$ und $q$ in einer Variablen mit
$P(\xi)=p(\xi\cdot\nu)$ und $Q(\xi)=q(\xi\cdot\nu)$ sowie $\deg q\le\deg p$.
\end{thm}
\begin{proof}[Beweisidee] Vgl. \cite{Hormander:1955}, der Beweis beruht auf strukturellen Argumenten zu Polynomringen über $\C$. Wesentlich ist, daß $\mathcal D_P\subseteq \mathcal D_Q$ Abschätzungen der Form
\begin{equation}
   |Q(\zeta)|^2 \le C \big( |P(\zeta)|^2 + 1\big)
\end{equation}
für alle {\em komplexen} $\zeta\in\C^n$ impliziert. Wir zeigen dies kurz. Aus Satz \ref{thm:1:1.1} folgt
\begin{equation}
\norm{Qu}^2\leq C(\norm{Pu}^2+\norm{u}^2)
\end{equation}
für alle $u\in\mathcal{D}_P$. Da wir das Gebiert $\Omega$ als beschränkt vorausgesetzt haben, gilt $\rmC^\infty(\overline{\Omega})\subset\mathcal{D}_P$. Also folgt mit $u(x)=\e^{\i x\cdot \zeta}$ für $\zeta\in\C^n$
\begin{equation}
\abs{Q(\zeta)}^2\int_\Omega \big|\e^{\i x\cdot \zeta}\big|^2\d x \leq C \left( \abs{P(\zeta)}^2\int_\Omega \big|\e^{\i x\cdot \zeta}\big|^2\d x+\int_\Omega \big|\e^{\i x\cdot \zeta}\big|^2\d x  \right)
\end{equation}
und damit nach Division beider Seiten durch $ \int_\Omega \big|\e^{\i x\cdot \zeta}\big|^2\d x$ die gewünschte Ungleichung.
\end{proof}

Nichttriviale Vergleichsresultate, wie wir sie in Kapitel~\ref{chap2} für minimale Operatoren erhalten haben, sind damit für maximale Operatoren von vornherein ausgeschlossen. 



\section{Differentialoperatoren von lokalem Typ} % bearbeitet durch Thomas Hamm
\begin{df}
Der Differentialausdruck $P(\D)$ heißt \eIndex[Differentialoperator]{von lokalem Typ}\footnote{Die Bezeichnung hat sich so nicht durchgesetzt; vollständige Operatoren von lokalem Typ entsprechen den heute als \eIndex[Differentialoperator]{(lokal) hypoelliptisch} bezeichneten Operatoren.} (auf einem beschränkten Gebiet $\Omega$), falls $u\varphi\in\mathcal D_{P_0}$ für alle $u\in\mathcal D_{P}$ und alle $\varphi\in \rmC_0^\infty(\Omega)$ gilt.
\end{df}
Es wird sich zeigen, daß die Definition unabhängig vom Gebiet $\Omega$ ist. Gilt die definierende Bedingung für ein beschränktes Gebiet, so gilt sie für jedes. Im folgenden sei $\Omega$ fest gewählt. Identifiziert man man den Graphen der Operatoren $P$ und $P_0$ mit den Definitionsbereichen versehen mit der Graphennorm, dann charakterisiert der \eIndex[Differentialoperator]{Cauchyraum} $\mathcal C = \mathcal D_{P}/\mathcal D_{P_0}$ eines Operators von lokalem Typ das Randverhalten der Funktionen aus $\mathcal D_{P}$ in der Umgebung von $\partial\Omega$.

\begin{lem}[{\cite[Lemma~3.4]{Hormander:1955}}]\label{lem : loktyp}
Der Differentialausdruck $P(\D)$ ist genau dann von lokalem Typ, wenn für jedes $u\in\mathcal D_P$ und alle Multiindices $\alpha\in\N_0^n$ stets $P^{(\alpha)}(\D)u \in \rmL^2_{\rm loc}(\Omega)$ gilt. In diesem Fall existiert für jedes Teilebiet $\omega\Subset \Omega$ mit kompaktem Abschluss in $\Omega$ ein $C>0$ mit
\begin{equation}
\int_{\omega} \abs{P^{(\alpha)}(\D)u(x)}^2\d x\leq C\left( \int_{\Omega} \abs{P(\D)u(x)}^2\d x + \int_{\Omega} \abs{u(x)}^2 \d x \right) \label{eq:lm3.4}
\end{equation}
für alle $u\in\mathcal{D}_P$.
\end{lem}
\begin{proof}
Sei zuerst $P(\D)$ von lokalem Typ, $u\in\mathcal{D}_P$ und $\omega\Subset\Omega$ relativ kompakt. Sei weiter $\varphi\in\rmC_0^\infty(\Omega)$ so gewählt, dass $\varphi(x)=1$ auf $\omega$ gilt. Dann gilt $v=u\varphi\in\mathcal D_{P_0}$ und $u=v$ auf $\omega$. Daraus folgt $P^{(\alpha)}(\D)u|_\omega=P^{(\alpha)}(\D)v|_\omega$ und es ist $P^{(\alpha)}(\D)v\in\rmL^2(\Omega)$ nach Korollar \ref{cor:2:2.6} und damit $P^{(\alpha)}(\D)u|_\omega \in\rmL^2(\omega)$. 
Also ist $P^{(\alpha)}(\D)u \in \rmL^2_{\rm loc}(\Omega)$. $\bullet$\qquad
Sei umgekehrt $P^{(\alpha)}(\D)u \in \rmL^2_{\rm loc}(\Omega)$ für alle Multiindices $\alpha\in\N_0^n$. Sei weiter $\varphi\in C_0^\infty(\Omega)$. Dann ist wegen Leibniz
\begin{equation}
P(\D)v=\sum_\alpha \dfrac{\D^\alpha \varphi}{{\alpha}!}P^{(\alpha)}(\D)u
\end{equation}
und damit nach Voraussetzung $P(\D)v\in \rmL^2(\Omega)$ und somit $v\in\mathcal{D}_P$. Da außerdem $\supp v\Subset\Omega$ kompakt ist gilt $\dist(\supp v,\partial\Omega)=\delta>0$. Wir wählen eine Funktion $\psi\in\rmC_0^\infty(B_{\delta/2})$ mit $\psi\ge0$ und $\int\psi(x)\d x=1$ und setzen $\psi_k(x)=k^{n} \psi(k x)$. Dann ist
$v_k=v*\psi_k\in\rmC_0^\infty(\Omega)$ und es gilt $\partial^\alpha v_k\to \partial^\alpha v$, $k\to\infty$, in $\rmL^2(\Omega)$ 
und somit $P(\D)v_k\to P(\D)v$. Damit ist $v\in \mathcal{D}_{P_0}$ und $P(\D)$ von lokalem Typ. $\bullet$ \qquad 
Die Ungleichung \eqref{eq:lm3.4} folgt aus Satz \ref{thm:1:1.1} angewandt auf die Abbildung
\begin{equation}
\mathcal{D}_P \ni u \mapsto P^{(\alpha)}u\in \rmL^2(\omega).
\end{equation}
Insbesondere gilt \eqref{eq:lm3.4} genau dann für alle $\omega\Subset\Omega$, wenn $P(\D)$ von lokalem Typ ist.
\end{proof}

%
%
%

\begin{thm}[{\cite[Theorem~3.12]{Hormander:1955}}]
   Angenommen, $P(\D)$ ist von lokalem Typ. Dann ist $P$ der Abschluß seiner Einschränkung auf $\mathcal D_P\cap \rmC^\infty(\Omega)$.
\end{thm}
\begin{proof}
Sei $\{\Omega_i\}_{i\in \N}$ eine abzählbare offene Überdeckung von $\Omega$ mit $\Omega_i$ beschränkt  und $\{\varphi_i\}_{i\in \N}$ mit $\varphi_i\in \rmC_0^\infty (\Omega_i)$ eine untergeordnete lokal endliche Zerlegung der Eins. Zu einem beliebigen $u\in\mathcal{D}_P$ setze nun $u_i=u\varphi_i$. Dann ist $\sum_{i\geq 1} u_i=u$, $\sum_{i\geq 1} P(\D)u_i=P(\D)u$ und da $P$ von lokalem Typ ist, ist $u_i\in\mathcal{D}_{P_0}$. Da $P_0$ gerade der $\rmL^2$-$\rmL^2$-Abschluß auf $\rmC_0^\infty(\Omega)$ ist, können wir für gegebenes $\epsilon>0$ Funktionen $v_i\in \rmC_0^\infty(\Omega)$ wählen mit $\norm{u_i-v_i}<\epsilon 2^{-i}$ und $\norm{P(\D)u_i-P(\D)v_i}<\epsilon 2^{-i}$. Weiterhin können wir ohne Einschränkung davon ausgehen, dass $\supp v_i \subset \Omega_i$. Damit folgt
\begin{equation}
\norm{v-u}\leq \sum_{i=1}^\infty \norm{v_i-u_i} < \epsilon
\end{equation}
und
\begin{equation}
\norm{P(\D)v-P(\D)u}\leq \sum_{i=1}^\infty \norm{P(\D)v_i-P(\D)u_i} < \epsilon
\end{equation}
und $v,P(\D)v\in \mathcal D_P\cap \rmC^\infty(\Omega)$ und damit die Behauptung.
\end{proof}

%%
%
%

Im folgenden sollen Kriterien dafür gefunden werden, daß ein gegebener Differentialausdruck $P(\D)$ von lokalem Typ ist. Ein erstes notwendiges Kriterium liefert folgendes Lemma.

\begin{lem}[{\cite[Lemma~3.5]{Hormander:1955}}]
Sei $\Omega$ ein beschränktes Gebiet und $P(\D)$ von lokalem Typ. Dann existiert für jedes $A>0$ ein $C>0$, sodass
\begin{equation}
\widetilde{P}(\zeta)^2=\sum_\alpha \abs{P^{(\alpha)}(\zeta)}^2\leq C(1+\abs{P(\zeta)}^2)
\end{equation}
für alle $\zeta\in\C^n$ mit $\abs{\Im \zeta} < A$.
\end{lem}
\begin{proof}
Setze $u(x)=\e^{\i x\cdot\zeta}$. Dann ist $P^{(\alpha)}(\D)u(x)=P^{(\alpha)}(\zeta)\e^{\i x\cdot\zeta}$ und mit Lemma \ref{lem : loktyp} folgt
\begin{equation}
\abs{P^{(\alpha)}(\zeta)}^2 \int_{\omega} \e^{-2x\cdot \eta}\d x \leq C(1+\abs{P(\zeta)}^2)\int_\Omega \e^{-2x\cdot \eta}\d x
\end{equation}
wobei $\eta = \Im \zeta$ und $\omega\Subset\Omega$. Setzen wir weiter $\delta = \sup \{2\abs{x}:x\in\Omega\}$, so folgt
\begin{equation}
\int_\omega \e^{-2x\cdot \eta}\d x \geq \int_\omega \e^{-\delta \abs{\eta}}=\abs{\omega}\e^{-\delta \abs{\eta}}
\end{equation}
und
\begin{equation}
\int_\Omega \e^{-2x\cdot \eta}\d x \leq \int_\Omega \e^{\delta \abs{\eta}}=\abs{\Omega}\e^{\delta \abs{\eta}}.
\end{equation}
Insgesamt also $\abs{P^{(\alpha)}(\zeta)}^2\leq C\e^{2\delta \abs{\eta}}(1+\abs{P(\zeta)}^2)\leq \widetilde{C}(1+\abs{P(\zeta)}^2) $, wenn $\abs{\eta}=\abs{\Im \zeta}$ beschränkt. Summation über alle $\alpha$ liefert die Behauptung.
\end{proof}


%
% Satz 3.3.
%

Es zeigt sich, dass das gerade gefundene Kriterium auch hinreichend ist. Wir formulieren den Satz, der Beweis wird sich über mehrere Abschnitte hinziehen und nachfolgend in einzelnen Teilschritten gezeigt werden.

\begin{thm}[{\cite[Theorem~3.3]{Hormander:1955}}]\label{thm:3:3.3}
Sei $P(\D)$ ein Differentialausdruck mit Symbol $P(\xi)$ und gelte $\Lambda(P)=\{0\}$. Dann sind \"aquivalent:
\begin{enumerate}
\item Für jedes $A>0$ existiert ein $R>0$, so daß $P(\xi+\i\eta)\ne0$ für alle $|\eta|<A$ und $|\xi|>R$.
\item Für jedes $\vartheta\in\R^n$ gilt $\lim_{\xi\to\infty} P(\xi+\vartheta)/P(\xi)=1$.
\item Für jeden Multiindex $\alpha\in\N_0^n$, $\alpha\ne0$, gilt $\lim_{\xi\to\infty} P^{(\alpha)}(\xi)/P(\xi)=0$.
\item Für jedes $A>0$ existiert ein $C>0$, so daß $\widetilde P(\xi+\i\eta)^2\le C\big(|P(\xi+\i\eta)|^2+1\big)$ für $|\eta|\le A$.
\item Für jedes $A>0$ gilt $\lim_{\xi\to\infty} P(\xi+\i\eta)=\infty$ gleichmäßig in $|\eta|\le A$.
\item $P(\D)$ ist von lokalem Typ.
\end{enumerate}
\end{thm} 

 Der Schritt {\bf (vi)} $\Rightarrow$ {\bf(iv)} wurde gerade gezeigt. Die Voraussetzung $\Lambda(P)=\{0\}$ kann entfallen, wenn die Aussagen {\bf (i)}, {\bf (ii)}, {\bf (iii)}
 und {\bf (v)} jeweils modulo $\Lambda(P)$ gefordert werden.


\begin{lem}[{\bf(i) $\Rightarrow$ (ii)};{\cite[Lemma~3.7]{Hormander:1955}}]
Angenommen für jedes $A>0$ existiert ein $B>0$, sodass $P(\xi+\i\eta)\neq 0$ für alle $\xi,\eta\in\R^n$ mit $\abs{\eta}<A$ und $\abs{\xi}>B$. Dann gilt
\begin{equation}
\lim_{\xi\to\infty}\dfrac{P(\xi+\vartheta)}{P(\xi)}=1 \label{eq:lm3.7}
\end{equation}
für alle $ \vartheta\in\R^n$.
\end{lem}
\begin{proof}
Durch Verschieben des Koordinatensystems zeigen wir \ref{eq:lm3.7} ohne Einschränkung für den Vektor $\vartheta=(1,0,\ldots,0)$. Sei $\epsilon > 0$ beliebig. Dann existiert nach Voraussetzung ein $B>0$, sodass $P(\xi+\i\eta)\neq 0$ für alle $\abs{\eta} < \epsilon^{-1}$ und $\abs{\xi}>B$. Sei nun $\xi=(\xi_1,\ldots,\xi_n)\in\R^n$ mit $\abs{\xi}>B+\epsilon^{-1}$ fest und $t_1,\ldots,t_m\in \C$ die Nullstellen des Polynoms $P(\cdot,\xi_2,\ldots,\xi_n)$. Für $\zeta=(t_k,\xi_2,\ldots,\xi_n)$ gilt dann wegen $P(\zeta)=0$, dass $\abs{\Re \zeta} \leq B$ oder $\abs{\Im \zeta} \geq \epsilon^{-1}$. Für $\abs{\xi-\zeta}= (\abs{\xi-\Re \zeta}^2+\abs{\Im \zeta})^{1/2}$ folgt dann in beiden Fällen $\abs{\xi_1-t_k}=\abs{\xi-\zeta}\geq \epsilon^{-1}$. Weiterhin haben wir für $P$ die Darstellung
\begin{equation}
P(\xi)=c\prod_{k=1}^m (\xi_1-t_k).
\end{equation}
und damit
\begin{equation}
\dfrac{P(\xi+\vartheta)}{P(\xi)}=\prod_{k=1}^m \dfrac{\xi_1+1-t_k}{\xi_1-t_k}=\prod_{k=1}^m\left( 1+\dfrac{1}{\xi_1-t_k} \right).
\end{equation}
Zusammen mit der Abschätzung für die Linearfaktoren folgt
\begin{equation}
\left|\dfrac{P(\xi+\vartheta)}{P(\xi)}-1\right|\leq \epsilon\sum_{k=1}^m \binom{m}{k}\epsilon^{k-1} \leq m\epsilon  \sum_{k=0}^{m-1}\binom{m-1}{k}\epsilon^k = m\epsilon(1+\epsilon)^{m-1}
\end{equation}
für alle $\xi \in\R^n$ mit $\abs{\xi}>B+\epsilon^{-1}$.
\end{proof}
\begin{lem}[{\cite[Lemma~2.10]{Hormander:1955}}]
Sei $P$ ein Polynom. Dann gilt 
\begin{equation}
\spann\{P^{(\alpha)}:\abs{\alpha}\leq \deg P\}=\spann\{P(\cdot+\vartheta):\vartheta\in\R^n\}.
\end{equation}
\end{lem}
\begin{proof}
\relax[$\supseteq$] Da $P$ ein Polynom ist, folgt aus der Taylorformel
\begin{equation}
P(\xi+\vartheta)=\sum_\alpha \dfrac{P^{(\alpha)}(\xi)}{\alpha!}\vartheta^{\alpha}
\end{equation}
für jedes feste $\vartheta\in\R^n$. $\bullet$\qquad [$\subseteq$] Betrachten wir nun den Unterraum, der von allen Translaten von $P$ aufgespannt wird. Dieser enthält alle Linearkombinationen der Form
\begin{equation}
\sum_{i=1}^m t_i P(\xi+\vartheta_i)=\sum_{i=1}^mt_i\sum_\alpha \dfrac{P^{(\alpha)}(\xi)}{\alpha!}\vartheta_i^\alpha =\sum_\alpha \dfrac{P^{(\alpha)}(\xi)}{\alpha!} \sum_{i=1}^mt_i\vartheta_i^\alpha=\sum_\alpha c_\alpha \dfrac{P^{(\alpha)}(\xi)}{\alpha!}
\end{equation}
mit Koeffizienten $t_i\in\C$ und Vektoren $\vartheta_i\in\R^n$. Für geeignet gewählte $ m,t_i,\vartheta_i $ nehmen dabei die Koeffizienten $c_\alpha$ beliebige Werte an. Also enthält dieser Unterraum auch alle Linearkombinationen der $P^{(\alpha)}$.
\end{proof}
\begin{lem}[{\bf(ii) $\Rightarrow$ (iii),(iv)};{\cite[Lemma~3.8]{Hormander:1955}}]
Gilt
\begin{equation}
\lim_{\xi\to\infty}\dfrac{P(\xi+\vartheta)}{P(\xi)}=1
\end{equation}
für alle reellen $\vartheta$, so auch für alle komplexen und die Konvergenz ist gleichmäßig in $\vartheta$, wenn $\abs{\vartheta}$ beschränkt. Außerdem gilt dann für alle Multiindices $\alpha\neq0$
\begin{equation}
\lim_{\xi\to\infty}\dfrac{P^{(\alpha)}(\xi+\vartheta)}{P(\xi+\vartheta)}=0
\end{equation}
ebenfalls gleichmäßig in $\vartheta$, wenn $\abs{\vartheta}$ beschränkt.
\end{lem}
\begin{proof}
Nach obigem Lemma existiert für $P^{(\alpha)}$ eine Darstellung der Form
\begin{equation}
P^{(\alpha)}(\xi)=\sum_{i=1}^m t_i P(\xi+\vartheta_i).
\end{equation}
Für $\alpha\neq0$ enthält der linke Term keine Monome der Ordnung $\deg P$ und durch Koeffizientenvergleich folgt $\sum t_i=0$. Damit folgt
\begin{equation}
\dfrac{P^{(\alpha)}(\xi)}{P(\xi)}=\sum_{i=1}^m t_i\dfrac{P(\xi+\vartheta_i)}{P(\xi)} \rightarrow \sum_{i=1}^m t_i=0
\end{equation}
nach Voraussetzung für $\xi\to\infty$. Weiterhin folgt mit der Taylorformel
\begin{equation}
\dfrac{P(\xi+\vartheta)}{P(\xi)}=1+\sum_{\alpha\neq 0} \dfrac{P^{(\alpha)}(\xi)}{P(\xi)}\dfrac{\vartheta^\alpha}{\alpha!}
\end{equation}
und damit die Konvergenz für komplexe $\vartheta$, sowie die gleichmäßige Konvergenz. Ebenso folgt damit für
\begin{equation}
\dfrac{P^{(\alpha)}(\xi+\vartheta)}{P(\xi+\vartheta)}=\sum_{i=1}^mt_i\dfrac{P(\xi+\vartheta+\vartheta_i)}{P(\xi)}\dfrac{P(\xi)}{P(\xi+\vartheta)} \rightarrow 0
\end{equation}
die gleichmäßige Konvergenz für $\xi\to\infty$, $\alpha\neq0$, wenn $\abs{\vartheta}$ beschränkt.
\end{proof}
Der Beweis der Implikation {\bf (iv) $\Rightarrow$ (v)} verläuft mit Mitteln aus dem Beweis von Satz \ref{thm:2.17}. Analog zum dortigen Induktionsbeweis zeigt man $\widetilde P(\zeta)^2\to\infty$ auf $\C^n / \Lambda_*(P)$,  wobei 
\begin{equation}
\Lambda_*(P)=\{ \vartheta\in\C^n : \forall_{\zeta\in\C^n}\; P(\zeta+\vartheta)=P(\zeta)  \}
\end{equation} 
den entsprechenden komplexe Linienraum bezeichnet. Damit impliziert die untere Schranke aus {\bf (iv)} auf dem Streifen $|\Im\zeta|\le A$ die Divergenz von $|P(\zeta)|$. Da die Implikation {\bf (v) $\Rightarrow$ (i)} offensichtlich ist, ist damit die Äquivalenz der Bedingungen {\bf (i)}--{\bf(v)} gezeigt. Also ist jede dieser Bedingungen notwendig dafür, dass $P(\D)$ von lokalem Typ ist. Um zu beweisen, dass diese auch hinreichend sind, benötigen wir eine Fundamentallösung, welche im folgenden Abschnitt konstruiert wird.

\begin{exa}
Die äquivalenten Bedingungen  {\bf (i)}--{\bf(v)} liefern einfache Kriterien dafür, wann Operatoren von lokalem Typ sind. So sind alle elliptischen Operatoren von lokalem Typ, da für diese der homogene Hauptteil $p(\xi)$ des Symbols $P(\xi)$ keine (reellen) Nullstellen besitzt und damit $1/P(\xi) = \mathcal O(|\xi|^{-m})$, $\xi\to\infty$, gilt. Daraus folgt sofort {\bf (iii)}. 

Umgekehrt impliziert {\bf (i)}, dass das Symbol nur auf einem Kompaktum reelle Nullstellen besitzen darf. Damit sind die im letzten Kapitel schon erwähnten 
Schrödingeroperatoren $\i\partial_1-\Delta_2$ auf $\Omega\subset \R\times\R^n$ und (ultra-) hyperbolischen Operatoren $\Delta_1-\Delta_2$ auf $\Omega\subset \R^k\times\R^l$, $k,l\ne0$, beide nicht von lokalem Typ. Andererseits liefert eine kurze Rechnung, daß der Wärmeleitungsoperator $\partial_1-\Delta_2$ auf $\Omega\subset\R\times\R^n$ von lokalem Typ ist. Sein Symbol ist durch $P(\xi) = \i \xi_1 - \xi'\cdot\xi'$ in $\xi=(\xi_1,\xi')\in\R^n$ gegeben und erfüllt $|P(\xi)|^2 = |\xi_1|^2+|\xi'|^4$ und damit $1/P(\xi)\to0$, $\xi' / P(\xi)\to0$ und folglich {\bf (iii)}.
\end{exa}

\section{Konstruktion von Fundamentallösungen eines vollständigen Operators von lokalen Typ} % bearbeitet von Matthias Hofmann

Im folgenden konstruieren wir eine Fundamentallösung, also eine Distribution $E\in\mathscr D'(\R^n)$ mit
\begin{equation}
E*(P(\D) u) = u, \quad u \in \rmC_0^\infty(\R^n),
\end{equation}
welche zusätzliche Regularitätseigenschaften besitzt.
Um genau zu sein,  wollen wir zeigen:
\begin{thm}\label{fundamental_exist}
Sei $P$ vollständig und von lokalem Typ, dann besitzt $P(\D)$ eine Fundamentallösung $E$, welche 
\begin{enumerate}
\item im Gebiet $\R^n\setminus\{0\}$  durch eine unendlich differenzierbare Funktion $E(x)$ dargestellt wird und
\item für die $P^{(\alpha)}(\D) E*u$ für jedes $u\in \rmL^2(\R^n)$ mit kompakten Träger $\supp u\Subset\R^n$ lokal quadrat-integrierbar ist. 
\end{enumerate}
\end{thm}
\begin{rem}
Ist $P(\D)$ vollständig und von lokalem Typ so besitzt jede Fundamentallösung diese Eigenschaften, da die Differenz zweier Fundamentallösungen unendlich oft differenzierbar ist. Dies werden wir später sehen.
\end{rem}

Zunächst einige Worte zur Beweisidee.  Falls $P(\xi)\neq 0$ für alle $\xi\in \mathbb R$, so ließe sich $E$ bestimmen aus
\begin{equation}\label{distr1}
E*u(x) = (2\pi)^{-n/2}\int \e^{\i x\cdot \xi} \frac{\widehat u(\xi)}{P(\xi)}\, \mathrm d\xi, \quad u \in \rmC_0^\infty,
\end{equation}
oder äquivalent
\begin{equation}\label{distr2}
\langle E,\check u\rangle = (2\pi)^{-n/2} \int \frac{\widehat u(\xi)}{P(\xi)}\, \mathrm d\xi, \quad u \in\rmC_0^\infty,
\end{equation}
wobei $\check u$ durch $\check u(x)=u(-x)$ gegeben ist. Wir überlegen uns also eine Verallgemeinerung von \eqref{distr2}. 

\begin{proof}[Beweis von Satz \ref{fundamental_exist}]
Wir werden diesen Beweis in einigen Schritten führen.
\vspace{1mm}\\
\textbf{Schritt 1:} \emph{Konstruktion einer Fundamentallösung.}
Nach Satz~\ref{thm:3:3.3} {\bf (v)} gilt $|P(\xi)|\ge 1$ falls $\xi\in \mathbb R$ und $|\xi|\ge C$ mit geeigneter Konstante $C$.  Damit folgt $|P(\xi)\ge 1|$ falls $\xi_2^2+ \ldots + \xi_n^2 \ge C^2$.  

Wir können annehmen, dass der Koeffizient vor der höchsten Potenz von $\xi_1$ von $P(\xi)$ konstant ist (vgl. Lemma ??). 
%Da die Nullstellen des Polynoms stetig von den Parametern abhängen, die nicht im führenden Koeffizienten stehen.  
So finden wir außerdem leicht eine zweite Konstante $C'$ sodass $|P(\xi_1, \xi_2, \cdots, \xi_nu)|\ge 1$ für $\xi_2^2+\ldots + \xi_n^2< C$ and $|\xi_1|\ge C'$.  

Jetzt setzen wir für $u\in\rmC_0^\infty(\mathbb R^n)$
\begin{equation}\label{distr3}
\langle E, \check u\rangle  = (2\pi)^{-n/2} \int \, \mathrm d\xi_2 \cdots \, \mathrm d\xi_n \oint \frac{\widehat u(\xi)}{P(\xi)} \, \mathrm d\xi_1= (2\pi)^{-n/2} \oint \frac{\widehat u(\xi)}{P(\xi)} \, \mathrm d\xi.
\end{equation}
wobei das Integral über $\xi_1$ über die reelle Achse gehe, falls $\xi_2^2+ \cdots + \xi_n^2 \ge C^2$, und über die reelle Achse mit dem Intervall $(-C', C')$ ersetzt durch einen Halbkreis in der unteren Halbebene, falls $\xi_2^2+ \cdots+ \xi_n^2 < C^2$. Damit gilt $|P(\xi)|\ge 1$  in dem Integral.  Damit ergibt sich analog zu \eqref{distr1}
\begin{equation}
E*u(x) = (2\pi)^{-n/2}\oint e^{\i x\cdot \xi } \frac{\widehat u(\xi)}{P(\xi)}\, \mathrm d\xi, \quad u \in \rmC_0^\infty.
\end{equation} 
Damit folgt für $u\in\rmC_0^\infty(\mathbb R^n)$
\begin{equation}
E*(P(\D)u) = (2\pi)^{-n/2} \oint \e^{\i x\cdot\xi} P(\xi)  \frac{\widehat{u} (\xi)}{P(\xi)}\, \mathrm d\xi=(2\pi)^{-n/2} \oint \e^{\i x\cdot \xi} \widehat u(\xi) \, \mathrm d\xi
\end{equation}
und da der Integrand holomorph in $\xi_1$ ist, können wir den Integrationsweg als reale Achse wählen. Damit folgt
\begin{equation}
E*(P(\D) u)(x) = (2\pi)^{-n/2} \int \e^{\i x\cdot \xi} \widehat u(\xi) \, \mathrm d\xi = u(x),
\end{equation}
und $E$ ist somit eine Fundamentallösung.

\textbf{Schritt 2:} \emph{Beweis der Quadratintegrierbarkeit von $P^{(\alpha)}(\D) E * u$.}
Wir teilen das Integral \eqref{distr3} in zwei Teile auf. Falls $R=\sqrt{C^2 + C'^2}$, haben wir $|\xi|<R$ und der Teil des Integrals von \eqref{distr3}, wo $\xi$ nicht real ist. Schreibe also 
\begin{equation}\label{distr4}
E=E_1 + E_2,
\end{equation}
\begin{equation}\label{distr5}
\langle E_1,\check u\rangle = (2\pi)^{-n/2} \int_{|\xi|\ge R} \frac{\widehat u(\xi)}{P(\xi)}\, \mathrm d\xi, \quad E_2(\check u)= (2\pi)^{-n/2} \oint_{|\xi|\le R} \frac{\widehat u(\xi)}{P(\xi)}\, \mathrm d\xi
\end{equation}
Die Variable $\xi$ in der Definition von $E_1$ kann nur reelle Werte im Integral annehmen. Die Distribution $E_2$ ist eine ganze Funktion, da für $u\in\rmC_0^\infty$ erhält man mit der Definition von $\widehat u$
\begin{equation}
\langle E_2, \check u\rangle = (2\pi)^{-\nu} \oint_{|\xi |\le R} \frac{\mathrm d\xi}{P(\xi)} \int u(x) \e^{-\i x\cdot\xi}\, \mathrm dx= (2\pi)^{-n} \int \check u(x) \, \oint_{|\xi|\le R} \frac{e^{\i x\cdot \xi}}{P(\xi)}\, \mathrm d\xi \, \mathrm dx
\end{equation}
erhält. Damit ist $E_2$ regulär und
\begin{equation}
E_2(x) = (2\pi)^{-n} \oint_{|\xi|\le R} \frac{\e^{\i x\cdot \xi}}{P(\xi)}\, \mathrm d\xi,
\end{equation}
die analytisch ist, da das Integral gleichmäßig konvergent auf Kompakta ist. 

Sei $u\in L^2$ mit kompaktem Träger. Die Faltung $P^{\alpha}(\D)E_2 *u$ ist dann eine analytische Funktion.  Die Aussage ({\bf ii}) folgt dann, wenn wir zeigen, dass $P^{(\alpha)}(\D) E_1 * u$ quadratintegrierbar ist.  Sei $\phi \in\rmC_0^\infty$. Dann ist die Funktion $U*\phi$ auch in $C_0^\infty$ und im Sinne von \eqref{distr5} folgt
\begin{equation}
\langle P^{(\alpha)} (\D) E_1 * u, \check \phi\rangle  = P^{(\alpha)}(\D) E_1 * u * \phi(0) = \int_{|\xi| \ge R} \frac{P^{(\alpha)}(\xi)}{P(\xi)} \widehat u(\xi) \widehat \phi(\xi) \, \mathrm d\xi,
\end{equation}     
so dass die Fouriertransformierte von $P^{(\alpha)}(\D) E_1 * u$ eine Funktion ist die für $|\xi|<R$ verschwindet und gleich $\widehat u(\xi)P^{(\alpha)}(\xi)/P(\xi)$ falls $|\xi|\ge R$. Damit ist $P^{(\alpha)}(\D) E_1 * u$ auch quadratintegrierbar und ({\bf ii}) folgt.

\textbf{Schritt 3:} \emph{Beweis der Differenzierbarkeit in $\mathbb R^n\setminus\{0\}$.}
Im Folgenden wenden wir uns dem Beweis von ({\bf i}) zu. Da $E_2$ eine ganze Funktion ist, genügt es zu zeigen, dass $E_1$ unendlich oft differenzierbar für $x\neq 0$ ist.  Wir beginnen mit folgenden Lemma:
\begin{lem}\label{3.2lem1}
Sei $0\neq y \in \mathbb R^n$ fest und sei zu $\tau\in\R$
\begin{equation}\label{3.2inf}
M(\tau) = \inf_{\zeta, \xi} |\zeta-\xi|,
\end{equation}
wobei $\zeta\in \mathbb C^n$ mit $P(\zeta)=0$, und $\xi\in \mathbb R^n$ mit $y\cdot \xi= \tau$. Dann existieren positive Zahlen $a$ und $b$ mit
\begin{equation}\label{3.2to}
M(\tau) \tau^{-b} \to a\quad \text{für } \tau \to \infty.
\end{equation}
\end{lem}
\begin{proof}
%Nach Theorem ?? ({\bf i}) ist der Restriktionsbereich für $\zeta$ kompakt   NEIN IST ER NICHT

 und das Infimum in \eqref{3.2inf} wird angenommen. Weiter folgt leicht, dass $M(\tau)$ stetig von $\tau$ abhängt.
\end{proof}
  

%so dass die Fouriertransformierte von $P^{(\alpha)}(\D) E_1 * u$ eine Funktion ist die für $|\xi|<R$ verschwindet und gleich $\hat u(\xi) P^{(\alpha)}(\xi)/P(\xi)$ wenn $|\xi|\ge R$. Da $P^{(\alpha)}(\xi)/P(\xi)$ beschränkt is für $|\xi|\ge R$ nach Theorem ?? ({\bf iii})  und dass $\hat u(\xi)$ quadratintegrierbar ist, folgern wir, dass die Fouriertransformierte von $P^{(\alpha)}(D) E_1 * u$ quadratintegrierbar ist. Damit ist auch $P^{(\alpha)}(D) E_1 * u$ quadratintegrierbar ist, was den Beweis von ({\bf ii}) abschließt.  

%Im Folgenden wollen wir (\bf{i})

% fehlt was
\begin{lem}\label{3.2lem2}
Es gibt positive Konstanten $c$ und $d$ so dass für hinreichend große $|\xi|$ wir 
\begin{equation}
|\zeta- \xi|\ge c |\xi|^{d}
\end{equation} 
für alle reelle $\xi$ und alle $\zeta$ mit $P(\zeta)=0$ haben.
\end{lem} 
\begin{proof}
Sei der Vektor $y$ von Lemma ?? als $(0,\ldots, 0, 1, 0, \ldots, 0)$, erhalten wir für große $|\xi_i|$
\begin{equation}
|\zeta-\xi|\ge a_i |\xi_i|^{b_i}
\end{equation}
wobei $a_i$ und $b_i$ positive Zahlen sind. Damit ergibt sich mit $c'=\min a_i$ und $d=\min b_i$ 
\begin{equation}
|\zeta - \xi|\ge c'(\max |\xi_i|)^d \ge c |\xi|^d.
\end{equation}
\end{proof}
\begin{lem}\label{3.2lem3}
Sei $y\in \mathbb R^n$ und $\eta \in \mathbb R^n$ zwei feste Vektoren. Dann gibt es eine Konstante $C$ so dass
\begin{equation}
\left |(\eta\cdot\nabla)^{k+j} \left ( \frac{1}{P(\xi)} \right )  \right | \le \frac{(k+j)! C^{k+j}}{|\xi|^{kd} ( 1+ | y\cdot\xi | )^{bj}},\qquad  |\xi|\ge R, \quad j, k=1,2,\ldots,
\end{equation}
wobei $b$ and $d$ aus dem vorigen Lemma seien. Dabei ist
\begin{equation}
(\eta\cdot\nabla)^{k+j} \left (\frac{1}{P(\xi)} \right ) =  \frac{\mathrm d^{k+j}}{\mathrm d t^{k+j}}  \frac{1}{P(\xi+ t\eta)} \bigg|_{t=0}.
\end{equation}
\end{lem}
\begin{proof}
Wir schreiben $P(\xi+t\eta) = A \prod_{1}^m (t-t_r) $. Da $\zeta = \xi+ t_r \eta$ eine Nullstelle von $P$ ist und $\zeta - \xi = t_r \eta$ können die Zahlen $t_r$ abgeschätzt werden durch Lemma \ref{3.2lem1} und \ref{3.2lem2} zu
\begin{equation}\label{3.2tr}
|t_r| \ge a' (1+|\langle y, \xi\rangle|)^b,\quad |t_r|\ge c' |\xi|^d, (|\xi|\ge R).
\end{equation}
Dann ist die $(k+j)$-te  Ableitung von $1/P(\xi+t\eta)$ für $t=0$ eine Summe von Terme die alle von der Form $A^{-1}$ geteilt durch das Produkt von $k+j+m$ der Nullstellen $t_r$. Die Zahl der Terme kann abgeschätzt werden durch
\begin{equation}
m(m+1) \cdots (m+k+j-1) =(k+j)! \binom{m+k+j-1}{k+j}<(k+j)! 2^{m+k+j-1}.
\end{equation}
Weiter ist $A$ unabhängig von $\xi$. Dies folgt leicht mit Theorem \ref{thm:3:3.3} ({\bf iii}). Sonst würde folgen 
\begin{equation}
\frac{P^{(\alpha)}(\xi + t\eta)}{P(\xi + t\eta)}\to \frac{A^{(\alpha)}(\xi)}{A(\xi)}\neq 0\quad(t\to \infty).
\end{equation}
für ein $\xi\in \mathbb R^n$ und einem Multiindex $\alpha\neq 0$.

% umsetzen
Dann folgt das Lemma, indem wir $j$ der Terme mit Nullstellen durch die erste Ungleichung aus \eqref{3.2tr}, $k$ der Terme mit der zweiten Ungleichung und die weiteren $m$ durch eine Konstante.  
\end{proof}
Seien $y$ und $\eta$ zwei feste Vektoren und sei $b$ und $d$ wie im vorigen Lemma.  Wir zeigen zunächst, dass
\begin{equation}
F= (x\cdot \eta)^l (y \cdot \nabla)^k E_1
\end{equation}
stetig ist, falls
\begin{equation}\label{3.2l}
l\ge \frac{k}{b}+r, % l hängt von k ab
\end{equation}
wobei $r=\lfloor n/d +1\rfloor$.  Dies schließt tatsächlich den Beweis des Theorems ab, da dann $E_1$ beliebig oft stetig differenzierbar ist außerhalb von $0$.  %Zusatz wie im Original?

Nach Definition von $F$ bzw. $E_1$ folgt 
\begin{equation}\label{3.2F}
\langle F, \check u\rangle = (2\pi)^{-n/2} \int_{|\xi|\ge R} \frac{(y\cdot \xi)^k}{P(\xi)} \left ( (\eta\cdot \D)^l \hat u(\xi)\right )\, \mathrm d\xi.
\end{equation}
Mit partieller Integration folgt $F(\check u)=G(\check u) + I(\check u)$ mit
\begin{equation}\label{3.2G}
\langle G,\check u\rangle = (2\pi)^{-n/2} \int_{|\xi|\ge R} \hat u(\xi) \left ( (-\eta \cdot \nabla)^l \left ( \frac{(y\cdot \xi)^k}{P(\xi)} \right ) \right )\, \mathrm d\xi
\end{equation}
\begin{equation}\label{3.2I}
\langle I,\check u\rangle = i^{-1} (2\pi)^{-n/2} \sum_{j=0}^{l-1} \int_{|\xi|=R} \left ( (-\eta \cdot \nabla)^j \left ( \frac{(y\cdot \xi)^k}{P(\xi)}\right ) \right ) \left ( (\eta \cdot \nabla)^{l-1-j} \hat u(\xi) \right ) (\eta \cdot \mathrm n) \, \mathrm d\sigma.
\end{equation}
Nach Lemma \ref{3.2lem3} können wir den Integranden von \eqref{3.2G} abschätzen, den wir mit $g(\xi)$ bezeichnen.
\begin{equation}
\begin{split}
|g(\xi)| &= \left | \sum_{j=0}^k \binom{l}{j} (y\cdot i\eta)^j (y \cdot \xi)^{k-j} \left ( (-D\cdot \eta)^{l-j} \frac{1}{P(\xi)} \right ) \right |\\
&\le \sum_{j=0}^k \binom{l}{j} \frac{k!}{(k-j)!} |y\cdot \eta|^j |y\cdot \xi|^{k-j}  \frac{(l-j)! C^{l-j}}{|\xi|^{rd} (1+|y\cdot \xi|^{b(l-j-r)})} 
\end{split}
\end{equation}
wo im ersten Schritt die Leibnizformel benutzt worden ist. Mit \eqref{3.2l} folgt
\begin{equation}
b(l-j-r) -(k-j)= b\left ( l-\frac{k}{b} -r \right )  + j(1-b) >0.
\end{equation}
und erhalten somit
\begin{equation}
|g(\xi)| \le |\xi|^{-rd} l! \sum_{j=0}^k \binom{k}{j} |y\cdot \eta|^j C^{l-j} = | \xi|^{-rd} l! C^l \left ( \frac{|y\cdot\eta|}{C+1} \right )^k.
\end{equation}
Die Funktion $|\xi|^{-rd}$ ist integrierbar über $|\xi|>R$, da $rd >n$.  Damit folgt, dass die Distribution $G$ gegeben ist durch die stetige Funktion
\begin{equation}
G(x) = (2\pi)^{-n} \int_{|\xi|\ge R} g(\xi) e^{i x\cdot \xi}\, \mathrm d\xi.
\end{equation}
Weiter folgt ähnlich wie zuvor mit der Definition von $\hat u$ eingesetzt in \eqref{3.2I}, dass $I$ durch die analytische Funktion 
\begin{equation}
I(x) = i^{-1}(2\pi)^{-n} \sum_{j=0}^{l-1} \int_{|\xi|=R} (x\cdot \eta)^{l-1-j} e^{ix\cdot \xi} \left ( (-\eta \cdot \nabla)^{j} \left ( \frac{(y\cdot \xi)^k}{P(\xi)} \right ) \right )\, \mathrm d\sigma
\end{equation} 
gegeben ist. Damit ist $F=G+I$ stetig.
\end{proof}

%
%
%
%

\begin{proof}[Beweis von Satz~\ref{thm:3:3.3} {\bf (i)--(v) $\Longrightarrow$ (vi)}]
Sei $P$ ein vollständiger Operator, der die Voraussetzungen I-V von Theorem \ref{thm:3:3.3} erfüllt und sei $\Omega\subset \mathbb R^n$ eine beliebige Umgebung um die $0$ und $\Omega'\subset \subset \Omega$ kompakt enthalten.  Sei $\rho(x)\in C_=^\infty$, sodass die für $|x|\ge \varepsilon$ verschwindet und $1$ in einer Umgebung um den Ursprung ist. Wir definieren die Distribution
\begin{equation}
F= \rho E.
\end{equation}
Der Träger von $F$ ist damit in der Kugel $\overline {B_{\varepsilon}(0)}$ enthalten und es ist
\begin{equation}
P(D) F= P(D) E - P(D)((1-\rho)E) =\delta_0 + \omega(x)
\end{equation}  
wobei $P(D)E=\delta_0$ und $\omega(x):= -P(D)((1-\rho)E)$ eine unendlich oft differenzierbare Funktion ist nach vorigen Theorem, die für $|x|\ge \varepsilon$ und in einer Umgebung um die $0$ verschwindet.  

Sei $u\in \mathcal D_{P}$, also insbesondere $u\in L^2(\Omega)$ und $P(\D)u\in L^2(\Omega)$.  Dann folgt in $\Omega'$
\begin{equation}
u= u*\delta_0 = u * (P(\D) F-\omega) = F*(P(\D)u)- \omega *u.
\end{equation}
Und damit
\begin{equation}\label{3.2wichtig}
P^{(\alpha)}(\D) u = (P^{(\alpha)}(D)F)*(Pu)-(P^{(\alpha)}(\D) \omega)*u.
\end{equation}
Im folgenden wollen wir zeigen, dass $P^{(\alpha)}(\D)\in L^2_{\text{loc}}(\Omega)$ gilt für alle Multiindices $\alpha\in \mathbb N_0^n$. Dann folgt mit Lemma \ref{lem : loktyp}, dass $P$ vom lokalen Typ ist.  $(P^{(\alpha)}(D)\omega)*u\in C^\infty$ ist offenbar quadrat-integrierbar in $\Omega'$. 
Bezeichne $\phi$ die Funktion die gleich $Pu$ auf $\Omega'_\varepsilon:=\{x\in \Omega| \dist(\Omega', x)<\varepsilon\}$ und $0$ sonst ist. Dann folgt in $\Omega'$
\begin{equation}
(P^{(\alpha)}(\D)F)* (Pu) = (P^{(\alpha)}(\D)E)*\phi+ (P^{(\alpha)}(\D)((\rho-1)E))*\phi.
\end{equation}
$(P^{(\alpha)}(\D)E)*\phi$ ist nach Satz \ref{fundamental_exist} ({\bf ii}) quadrat-integrierbar. Offenbar besitzt $\phi$ kompakten Träger und leicht folgt auch, dass $P^{(\alpha)}(\D)((\rho-1)E)*\phi$ quadrat-integrierbar in $\Omega'$  ist. Daraus folgt schließlich mit \eqref{3.2wichtig}, dass $P^{(\alpha)}u\in L^2_{\text{loc}}$.
\end{proof}


%
%
%
%

\begin{thm}
Sei $P(\D)$ ein Differentialausdruck und $\Omega\subset\R^n$ ein beschränktes Gebiet. Dann sind die folgenden Aussagen äquivalent:
\begin{enumerate}
\item $P(\D)$ ist vollständig und von lokalem Typ.
\item Jede (distributionelle) Lösung $u\in\rmL^2(\Omega)$ der Gleichung $P(\D)u=0$ ist glatt,
\begin{equation}
    \ker P(\D) = \{ u\in\rmL^2(\Omega) \;:\; P(\D)u=0\} \subseteq\rmC^\infty(\Omega).
\end{equation}
\end{enumerate}
\end{thm}
\begin{proof}
\end{proof} % Maximale Operatoren :: Thomas Hamm, Matthias Hofmann
% !TEX root = main.tex
\chapter{Operatoren mit variablen Koeffizienten}

Zur Notation: Sei $\Omega\subset\R^n$ ein Gebiet und bezeichne weiterhin 
\begin{equation}
    \|u\|_{(m)} = \sum_{|\alpha|\le m} \|\D^\alpha u\|
\end{equation}
die $m$-te Sobolevnorm einer Funktion $u\in\rmC_0^\infty(\Omega)$. Im folgenden betrachten wir mininmale Operatoren zu Differentialausdrücken
\begin{equation}
   P(x,\D) = \sum_{|\alpha|\le m} a_\alpha(x) \D^\alpha
\end{equation}
mit Koeffizienten $a_\alpha\in\rmC^\infty(\Omega)$. Offensichtlich gilt $\|P(x,\D)u\|\le C\|u\|_{(m)}$ für alle $u\in\rmC_0^\infty(\Omega)$ mit einer nur von der Größe der Koeffizienten abhängenden Konstanten $C$.
Wir definieren \eIndex[Differentialoperator]{Symbol} und \eIndex[Differentialoperator]{Hauptsymbol}
\begin{equation}
   P(x,\zeta) = \sum_{|\alpha|\le m} a_\alpha(x)\zeta^\alpha,\qquad p(x,\zeta)=\sum_{|\alpha|=m} a_\alpha(x)\zeta^\alpha
\end{equation}
als Polynome auf $\C^n$ mit Koeffizienten aus $\rmC^\infty(\Omega)$. Analog zu den schon für den Fall konstanter Koeffizienten eingeführten Bezeichnungen nennen wir den Differentialausdruck $P(x,\D)$ im Punkt $x\in\Omega$
\begin{itemize}
\item \eIndex[Differentialoperator]{elliptisch}, falls für alle (reellen) $\xi\in\R^n\setminus\{0\}$ das Hauptsymbol $p(x,\xi)\ne0$ erfüllt;
\item \eIndex[Differentialoperator]{vom Haupttyp}, falls die Nullstellenmenge des Hauptsymbols 
\begin{equation}
    p(x,\xi) = 0 \quad \Longrightarrow \quad \nabla_\xi p(x,\xi)\ne0
\end{equation}
für alle $\xi\in\R^n\setminus\{0\}$ erfüllt.
\end{itemize}
Während die für Operatoren mit konstanten Koeffizienten gezeigten Aussagen und Abschätzungen im wesentlichen unabhängig vom Gebiet waren, 
treten bei Operatoren mit variablen Koeffizienten neue Effekte auf. Betrachtet man jedoch hinreichend {\em kleine} Gebiete, so kann man Abschätzungen auf den Fall konstanter Koeffizienten zurückführen. Das soll am Beispiel der Elliptizitätsabschätzung
\begin{equation}\label{eq:4:locEll}
\forall u\in\rmC_0^\infty(\omega)\quad:\quad    \| u\|_{(m)} \le C \| P(x,\D) u \|
\end{equation}
und der Haupttypabschätzung 
\begin{equation}\label{eq:4:locHT}
\forall u\in\rmC_0^\infty(\omega)\quad:\quad   \| u\|_{(m-1)} \le C \| P(x,\D) u \|
\end{equation}
diskutiert werden. Hierbei sei $x_0\in\Omega$ fest gewählt und $\omega\Subset\Omega$ eine hinreichend kleine Umgebung von $x_0$. Die Größe von $\omega$ hängt vom Verhalten der Koeffizienten ab. Die Gültigkeit einer solchen lokalen Abschätzung für jedes $x_0\in\Omega$ impliziert offenbar das entsprechende Resultat auf jeder relativ kompakten Teilmenge $\Omega'\Subset\Omega$. Konstanten hängen von der Wahl von $\omega$ beziehungsweise $\Omega'$ ab.



\section{Hinreichende Bedingungen für untere Schranken}


\begin{thm}\label{thm:4:4.1}
Angenommen, $P(x,\D)$ ist elliptisch. Dann existiert zu jedem $x_0\in\Omega$ eine Umgebung $\omega\Subset\Omega$, so dass
\eqref{eq:4:locEll} mit einer von $\omega$ abhängigen Konstanten $C$ gilt. 
\end{thm}
\begin{proof}
Ohne Beschränkung der Allgemeinheit nehmen wir an der Ursprung liege im Gebiet und es gelte $x_0=0$. Sei weiter $P(\D)=P(0,\D)=\sum_\alpha a_\alpha(0)\D^\alpha$ der Operator der entsteht, wenn man in die Koeffizienten den Wert $0$ einsetzt. Dann gilt für alle $u\in\rmC_0^\infty(\Omega)$
\begin{equation}
 \|u\|_{(m)} \le C  \| P(\D) u\| 
\end{equation}
da $P(0,\D)$ elliptischer Operator mit konstanten Koeffizienten ist. Weiter existiert zu jedem $\epsilon>0$ ein $\delta>0$, so dass
$|a_\alpha(x)-a_\alpha(0)|\le \epsilon$ für $|x|\le\delta$. Damit impliziert die Dreiecksungleichung
\begin{equation}
  \| P(\D) u - P(x,\D) u\| \le \epsilon \sum_{|\alpha|\le m} \|\D^\alpha u\| = \epsilon \|u\|_{(m)}
\end{equation}
für alle $u\in\rmC_0^\infty(B_\delta)$. Für $\epsilon$ klein genug folgt damit die Behauptung.
\end{proof}

Ein Operator $P(x,\D)$ wird als  \eIndex[Differentialoperator]{von reellem Haupttyp} bezeichnet, falls sein Hauptsymbol $p(x,\xi)$ reellwertig ist. Unter der Voraussetzung stimmen $P(x,\D)$ und sein formal adjungierter ${}^tP(x,\D)$ bis auf einen Operator der Ordnung $m-1$ überein. Allgemeiner heißt $P(x,\D)$ \eIndex[Differentialoperator]{wesentlich normal}, falls die  \eIndex{Poissonklammer} des Hauptsymbols $p$ mit $\overline p$ definiert durch $\overline p(x, \overline\zeta) = \overline{p(x,\zeta)}$
\begin{equation}
    \{ p,\overline p\} (x,\xi) = \sum_{j=1}^n \bigg(\frac{\partial p(x,\xi)}{\partial \xi_j} \frac{\partial \overline p (x,\xi)}{\partial x_j} - \frac{\partial \overline p(x,\xi)}{\partial \xi_j}\frac{\partial  p(x,\xi)}{\partial x_j} \bigg)    = 0 
\end{equation}
für alle $\xi\in\R^n$ verschwindet. In diesem Fall ist der Kommutator von $P(x,\D)$ und ${}^tP(x,\D)$ von Ordnung $2m-2$ (statt nur $2m-1$).

Das folgende Resultat ergibt sich aus \cite[Theorem~4.1]{Hormander:1955} zusammen mit der Vorbemerkung zu diesem Kapitel. Die Umkehrung dieses Satzes gilt nicht.

\begin{thm}[{\cite[Theorem~4.1]{Hormander:1955}}]\label{thm:4:4.2}
Angenommen, $P(x,\D)$ ist vom Haupttyp und wesentlich normal. Dann existiert zu jedem $x_0\in\Omega$ eine Umgebung $\omega\Subset\Omega$, so dass
\eqref{eq:4:locHT} mit einer von $\omega$ abhängigen Konstanten $C$ gilt.
\end{thm}
\begin{proof}[Beweisskizze]
Für den Originalbeweis konstruiert man wiederum Energieintegrale und schätzt diese mittels partieller Integration ab.
\end{proof}

\begin{cor}
Angenommen, $P(x,\D)$ erfüllt die Voraussetzungen von Satz~\ref{thm:4:4.1} oder Satz~\ref{thm:4:4.2}. Dann existiert zu jedem $\Omega'\Subset\Omega$ und jedem
$f\in\rmL^2(\Omega')$ ein $u\in\mathscr D'(\Omega')$ mit $P(x,\D)u=f$ in $\Omega'$.
\end{cor}
\begin{proof}
Mit $P(x,\D)$ erfüllt auch ${}^tP(x,\D)$ die Voraussetzungen des Satzes. Insbesondere ist damit der zu ${}^tP(x,\D)$ auf $\Omega'$ assoziierte minimale Operator beschränkt invertierbar und damit nach Satz~\ref{thm:1:loesbarkeit} der zu $P(x,\D)$ assoziierte maximale Operator surjektiv.
\end{proof}

\section{Notwendige Bedingungen für untere Schranken}
Im folgenden sei $\Omega \subset \R^n$ eine offene Umgebung der Null und $P(x,\D)$ ein Differentialausdruck auf $\Omega$ mit Hauptsymbol $p(x,\xi)$. 
\begin{lem}\label{lem1}
Angenommen, es existiert eine Funktion $u \in \rmC^\infty (\Omega)$ mit 
\begin{equation}
\label{grad}
p(x, \nabla u) = \mathbf{o} \big(|x|^2\big), \qquad  x \rightarrow 0,
\end{equation}
welche eine Taylorentwicklung 
\begin{equation}\label{taylor}
u(x) = \i \xi\cdot x + x\cdot A x + \mathcal O\big(|x|^3\big),\qquad x\to0
\end{equation}
mit $\xi\in\R^n$, einer symmetrischen Matrix $A=(\boldsymbol\alpha_{i,j})_{i,j}\in\C^{n\times n}$ und $B = (\Re \boldsymbol\alpha_{i,j})_{i,j} $ negativ definit  besitzt. 
Gilt weiterhin  
\begin{equation}
\label{normpart}
\sum_{j=1}^{n}\bigg|\frac{\partial p(0,\xi)}{\partial \xi_j}\bigg|^2 \neq 0,
\end{equation}
so folgt
\begin{equation}
\label{lm 1 eq}
\sup_{v\in\rmC_0^\infty(\Omega)} \frac{ \lVert v \rVert_{(m-1)} }{ \lVert P(x,\D)v \rVert} = \infty.
\end{equation}
\end{lem}
\begin{proof}
Da die Matrix $B$ negativ definit ist, gilt
\begin{equation}
\Re u(x) = x\cdot Bx + \mathcal O\big(|x|^3\big) \le - 2 a |x|^2 + \mathcal O\big(|x|^3\big),\qquad x\to0
\end{equation}
mit geeignet gewähltem $a>0$. Damit folgt
\begin{equation}
\label{realteil}
\Re u(x) \le - a |x|^2 
\end{equation}
für hinreichend kleine $|x|$. Nach Verkleinerung von $\Omega$ können wir also ohne Einschränkung annehmen, dass (\ref{realteil}) überall in $\Omega$ gilt. 
%
Sei nun $\phi \in \rmC_0^\infty(\Omega)$ und $t>0$. Dann erfüllt $v_t := \phi \e^{tu} \in \rmC_0^\infty(\Omega)$ die Abschätzung $|v_t(x)| \le |\phi(x)| \e^{-at|x|^2}$. Wir zeigen nun
\begin{equation}
\frac{\lVert v_t \rVert_{(m-1)}}{\lVert P(x,\D)v_t \rVert} \longrightarrow \infty, \qquad t \rightarrow \infty,
\end{equation}
woraus die Behauptung (\ref{lm 1 eq}) folgt. Dazu nutzen wir
\begin{equation}
P(x,\D)v_t(x) = \e^{tu(x)} \sum_{j=0}^{m} b_j(x) t^j
\end{equation}
mit Koeffizienten $b_j\in\rmC^\infty(\Omega)$. Es reicht, $b_m$ und $b_{m-1}$ zu berechnen. Mit der Leibnizformel erhalten wir
\begin{equation}
P(x,\D)(\phi \e^{tu}) = \sum_{\alpha} \dfrac{\D^\alpha \phi}{\alpha!} P^{(\alpha)}(x,\D)\e^{tu}.
\end{equation}
Wir zerlegen nun
\begin{equation}
P(x,\D) = p(x,\D) + q(x,\D) + r(x,\D),
\end{equation}
wobei $p$ und $q$ homogen der Ordnung $m$ bzw. $m-1$ seien sowie $r$ Ordnung $<m-1$ habe. Dann sind die Koeffizienten von $t^m \e^{tu}$ und $t^{m-1} \e^{tu}$ in
\begin{equation}
\phi p(x,\D)\e^{tu} + \bigg( \phi q(x,\D) + \sum_{j=1}^{n} (\D_j \phi) p^{(j)}(x,\D) \bigg) \e^{tu}
\end{equation}
ebenfalls $b_m$ und $b_{m-1}$, denn alle anderen Summanden in der Leibnizformel haben Grad kleiner als $m-1$ in $t$. Nach Berechnung des ersten Summanden erhalten wir $b_m = (-\i)^m \phi p(x,\nabla u)$ und insbesondere wegen (\ref{grad})
\begin{equation}
b_m(x) = \mathbf{o} (|x|^2), \quad x \rightarrow 0.
\end{equation}
Weiter erhalten wir
\begin{equation}
	\label{4.17}
b_{m-1} = (-\i)^{m-1} \bigg( \phi (\i^{m-1}g + q(x,\nabla u)) + \sum_{j=1}^{n} (\D_j \phi) p^{(j)} (x,\nabla u) \bigg),
\end{equation}
wobei $g$ eine glatte Funktion ist, die von $u$ und den Koeffizienten von $p$ abhängt. Wir wählen nun $\phi$ so, dass $\phi(0) =1$ gilt und sodass (\ref{4.17}) für $x=0$ verschwindet, was wegen (\ref{normpart}) möglich ist. Wegen Stetigkeit von $b_{m-1}$ ist dann $b_{m-1}(x) = \mathbf{o}(1)$, wenn $x$ gegen Null strebt.

Nach diesen Überlegungen können wir für $\varepsilon >0$ eine Umgebung $U$ der Null wählen, sodass
\begin{equation}
|b_m(x)| < \varepsilon |x|^2, \hspace{0.5cm} |b_{m-1}(x)| < \varepsilon, \hspace{0.5cm} \forall \, x \in U.
\end{equation}
Mit Gleichung (\ref{realteil}) erhalten wir für alle $x\in U$ die Abschätzung
\begin{equation}
|P(x,\D)v_t| \le t^{m-1} \e^{-at|x|^2}(\varepsilon t |x|^2 + \varepsilon + C/t)
\end{equation}
und mit der Transformation $x \mapsto x/\sqrt{t}$ im Integral folgt
\begin{equation}
\lVert P(x,\D) v_t \rVert_{L^2(U)} ^2 \le t^{2m-2-n/2} \varepsilon^2 \int_U (|x|^2 + 1 +C/\varepsilon t)^2 \e^{-2a|x|^2} \d x,
\end{equation}
wobei das letzte Integral für $t \rightarrow \infty$ gegen
\begin{equation}
B^2 = \int_U (|x|^2 + 1)^2 \e^{-2a|x|^2} \d x
\end{equation}
konvergiert. Für große $t$ liefert dies die Abschätzung
\begin{equation}
\lVert P(x,\D) v_t \rVert_{L^2(U)} ^2 \le 2 t^{2m-2-n/2} \varepsilon^2  B^2.
\end{equation}
Da aus (\ref{realteil}) ebenso
\begin{equation}
\lVert P(x,\D) v_t \rVert_{L^2(U^c)} ^2 = \mathcal{O}(t^{2m}e^{-2ct})
\end{equation}
für eine Konstante $c>0$ folgt, erhalten wir
\begin{equation}
	\label{4.18}
\lVert P(x,\D) v_t \rVert^2 \le 4 B^2 t^{2m-2-n/2} \varepsilon^2.
\end{equation}
Als schätzen wir $\lVert v_t \rVert_{(m-1)}$ nach unten ab. Wir nehmen ohne Einschräung an, dass $\xi_1\neq 0$ gilt. Wir setzen $\alpha' = (m-1,0,\ldots,0)$ und erhalten
\begin{equation}
\D^{\alpha'} v_t = (\phi (\D_1 u)^{m-1}t^{m-1}+ \ldots)\e^{tu}.
\end{equation}
Wegen $\phi(0) (\D_1u(0))^{m-1} = \xi_1^{m-1} \neq 0$ gilt $|\phi| |\D_1u|^{m-1} \ge 2c > 0$ für eine Konstante $c > 0$ in einer Umgebung der Null und wegen $\Re u(x) = \mathcal{O} (|x|^2)$ existiert eine Konstante $a' > 0$ sodass $\Re u(x) \ge -a' |x|^2$. Für große $t$ erhalten wir
\begin{equation}
|\phi| |\D_1u|^{m-1} \ge ct^{m-1}\e^{-ta'|x|^2}.
\end{equation}
Weiter folgt für große $t$    
\begin{equation}
	\label{4.19}
\lVert v_t \rVert_{(m-1)}^2 \ge \lVert \D^{\alpha'} v_t\rVert^2 \ge \int\limits_{t|x|^2 < 1}c^2t^{2m-2}\e^{-2ta'|x|^2} \d x = C^2 t^{2m-2-n/2}
\end{equation}
für eine Konstante $C > 0$ und eine Umgebung $V$ der Null. Kombiniert man (\ref{4.18}) und (\ref{4.19}), so erhält man für große $t$
\begin{equation}
\dfrac{\lVert v_t \rVert_{(m-1)}}{\lVert P(x,\D) v_t \rVert} \ge \dfrac{C}{2B\varepsilon}
\end{equation}
und da $\varepsilon>0$ beliebig war folgt die Behauptung des Lemmas.
\end{proof}

\begin{lem}\label{lem2}
Wir nehmen an, es gelte (\ref{normpart}). Dann gibt es eine Funktion $u \in \rmC^\infty(\Omega)$ mit (\ref{grad}) und (\ref{taylor}) genau dann, wenn
\begin{align}
	\label{4.15}
p(0, \xi) &= 0, \\ 	\label{4.16}
\sum_{k=1}^{n} \boldsymbol\alpha_{j,k}\frac{\partial p(0,\xi)}{\partial \xi_k} &= -\i\,  p_j(0,\xi), \hspace{0.5cm} j=1, \ldots, n, 
\end{align}
wobei $ p_j(x,\xi) = \partial_{x_j}  p(x,\xi)$ die partielle Ableitung des Hauptsymbols bezeichne.
\end{lem}

\begin{proof} [Beweisskizze]
Betrachten wir die Taylorentwicklung, so stellen wir fest, dass (\ref{grad}) genau dann erfüllt ist, wenn $p(x,\nabla u)$ und alle Ableitungen der Ordnung $\le 2$ im Nullpunkt verschwinden. Das Verschwinden von $p(x,\nabla u)$ wird gerade durch (\ref{4.15}) beschrieben. Wir berechnen weiter
\begin{equation}
\dfrac{\partial}{\partial x_j} p(0,\nabla u) = 
p_j(0,\i\xi) + \sum_{k=1}^{n} \boldsymbol\alpha_{j,k} \dfrac{\partial p(0,\i\xi)}{\partial \xi_k}
\end{equation}
und bemerken unter Verwendung der Homogenität von $p$, dass die linke Seite genau dann für alle $j$ verschwindet, wenn (\ref{4.16}) gilt. Es bleibt zu zeigen, dass für geeignet gewähltes $u$ unter den gegebenen Bedingungen auch die zweiten Ableitungen verschwinden. Hierzu leitet man die obige Gleichung ein weiteres Mal ab und stellt unter Verwendung von (\ref{normpart}) fest, dass das resultierende System partieller Differentialgleichungen die Voraussetzungen des Satzes von Cauchy-Kovalevsky erfüllt, woraus die Existenz eines solchen $u$ folgt.
\end{proof}

\begin{lem}\label{lem3}
Seien $\zeta, f \in \C^n$ mit $\zeta \neq 0$. Dann gibt es genau dann eine symmetrische Matrix $A = (\boldsymbol\alpha_{i,j})_{i,j}$ mit negativ definitem Realteil
$B=(\Re\boldsymbol \alpha_{i,j})_{i,j}$ und
\begin{equation}\label{laag1}
   A \zeta = f,
\end{equation}
wenn die Vektoren $\zeta$ und $f$
\begin{equation} \label{laag2}
\Re f\cdot \overline \zeta < 0
\end{equation}
erfüllen.
\end{lem}
\begin{proof}
Es gelte (\ref{laag1}). Multiplikation beider Seiten mit $\overline \zeta_k$ und Aufaddieren ergibt unter Verwendung der Symmetrie von $A$
und mit $\zeta=\xi+\i\eta$
\begin{equation}
 f\cdot \overline \zeta = \sum_{k=1}^{n}f_k \overline{\zeta_k} = \sum_{j,k=1}^{n} \boldsymbol\alpha_{kj} \zeta_j \overline \zeta_k
=  \sum_{j,k=1}^{n} \boldsymbol\alpha_{kj} \xi_j \xi_k +  \sum_{j,k=1}^{n} \boldsymbol\alpha_{kj} \eta_j \eta_k,
\end{equation}
Da für mindestens ein $j$ ein $\xi_j$ oder $\eta_j$ ungleich Null ist, folgt (\ref{laag2}) wegen der Definitheit von $B$. $\bullet$\qquad 
Es gelte nun umgekehrt (\ref{laag2}). Wir nehmen zunächst an, $\zeta$ ist proportional zu einem reellen Vektor. Nach Multiplikation von $f$ und $\zeta$ mit derselben komplexen Zahl können wir dann annehmen, dass $\zeta=\xi$ reell ist. Mit $A = B + \i C$, $f= g+\i h$ wird aus (\ref{laag1}) das Gleichungssystem bestehend aus $B \xi = g$ und $C \xi = h$. Dass eine reelle, symmetrische Matrix $C$ mit $C \xi = h$ existiert, ist offensichtlich. Schreiben wir weiter $g = g' + \xi (g\cdot \xi)/ 2 |\xi|^2$, so gilt $g'\cdot\xi = g\cdot\xi /2 < 0$. Man rechnet dann leicht nach, dass die Matrix $B$, gegeben durch
\begin{equation}
B x = \dfrac{g\cdot\xi }{2|\xi|^2} x + \dfrac{x\cdot g'}{\xi\cdot g'} g',\qquad x\in\R^n,
\end{equation}
negativ definit und symmetrisch ist und $A=B+\i C$ erfüllt das gesuchte. Sei nun $\zeta$ nicht proportional zu einem reellen Vektor. Wir zeigen, dass 
\begin{equation}
A = \dfrac{\Re( f\cdot\overline\zeta )}{|\zeta|^2} \mathrm I + \i C
\end{equation}
für ein reelles $C\in\R^{n\times n}$ die gewünschten Eigenschaften erfüllt. Die Bedingung an $C$ schreibt sich dann als
\begin{equation}
	\i C \zeta = f',
\end{equation}
wobei $f' = f - \zeta \frac{\Re( f\cdot\overline\zeta)}{|\zeta|^2}$ die Eigenschaft
\begin{equation}
	\label{Ebene}
\Re (f'\cdot\overline\zeta) = 0
\end{equation}
hat. Um die Existenz eines solchen $C$ zu zeigen, stellen wir zunächst fest, dass die Menge der Vektoren in $\C^n$, die als $\i C \zeta$, $C \in \R^{n \times n}$ symmetrisch, geschrieben werden können, einen reellen Vektorraum bilden. Dieser ist für ein $g \in \C^n$ in der Hyperebene $\{ z \in \C^n : \Re g\cdot \overline z = 0 \}$ enthalten. Für jedes $\xi \in \R^n$ ist die Matrix $C$, gegeben durch $C x =  (x\cdot\xi)\xi$, symmetrisch und es folgt
\begin{equation}
\Re \i (\xi\cdot\overline g)(\zeta\cdot\xi) = 0.
\end{equation}
Insbesondere ist $(\xi\cdot\overline g)(\zeta\cdot\xi) $ immer reell. Unter Verwendung, dass $\zeta$ nicht proportional zu einem reellen Vektor ist, berechnet man schnell, dass $g$ ein reelles Vielfaches von $\zeta$ sein muss. Insbesondere ist $\Re(z\cdot\overline\zeta) = 0 \Leftrightarrow \Re(z\cdot \overline g) = 0$ und $f'$ lässt sich wegen (\ref{Ebene}) als $\i C \zeta$ schreiben für eine reelle, symmetrische Matrix $C$.  Es folgt die Behauptung.
\end{proof}

\begin{thm}
\label{Umkehrung}
Es gelte die Ungleichung (\ref{eq:4:locHT}), d.h.
\begin{equation}
  \| u\|_{(m-1)} \le C \| P(x,\D) u \| \quad \forall u\in\rmC_0^\infty(\Omega).
\end{equation}
Dann folgt 
\begin{equation}
	\label{Beh}
\{p,\overline p\} (x,\xi) = 0,
\end{equation}
falls $p(x,\xi) = 0$, $x \in \Omega$ und $ \xi \in \R^n$.
\end{thm}
\begin{proof}
Wir nehmen ohne Einschränkung $x=0$ an. Weiterhin gelte
\begin{equation}
\sum_{j=1}^{n}|\partial p(0,\xi)/\partial \xi_j|^2 \neq 0,
\end{equation}
denn sonst ist (\ref{Beh}) trivialerweise erfüllt. Die Lemmata \ref{lem1} und \ref{lem2} zeigen dann, dass die Gleichung (\ref{4.16}) für keine symmetrische Matrix mit negativ definitem Realteil erfüllt sein kann. Mit Lemma \ref{lem3} folgt deshalb $\{p,\overline p\}(x,\xi) \ge 0$. Das analoge Argument mit $-\xi$ anstatt $\xi$ liefert $\{p,\overline p\}(x,-\xi) \ge 0$. Weil $\{p,\overline p\}(x,\xi)$ eine ungerade Funktion in $\xi$ ist, folgt die Behauptung.
\end{proof}
\begin{rem}
In Satz \ref{Umkehrung} folgt nur das Verschwinden der Poissonklammer auf der Nullstellenmenge von $p(x,\xi)$ und insbesondere nicht die wesentliche Normalität von $P(x,D)$. Folglich handelt es sich nicht um die Umkehrung von Satz \ref{thm:4:4.2}, wo wesentliche Normalität vorausgesetzt wurde.
\end{rem} % Operatoren von reellem Haupttyp :: Julian Mauersberger
% !TEX root = main.tex
%\chapter{Ein unlösbarer Operator}
%\cite{Lewy:1957}
%\cite{Hormander:1960b}


\section{Das Beispiel von Lewy}


Bis Mitte des 20. Jahrhunderts ging man davon aus, dass die lokale Existenz glatter Lösungen für lineare Differentialgleichungen stets gegeben sei. Das Beispiel von Hans Lewy zeigt eindrucksvoll, dass dies nicht zwingend der Fall sein muss.  In diesem Abschnitt soll eine lineare, partielle Differentialgleichung erster Ordnung in drei Variablen mit komplexwertigen $\rmC^\infty$-Koeffizienten vorgestellt werden, welche in \emph{keiner} offenen Menge eine glatte/distributionelle Lösung besitzt. Hierfür betrachtet man den durch
\begin{equation}\label{lewy:differentialausdruck}
\mathscr L := -\frac{\partial}{\partial x_1} -\i\frac{\partial}{\partial x_2} +2\i(x_1+\i x_2)\frac{\partial}{\partial y_1}
\end{equation}
definierten Differentialausdruck auf $\R^3$. Das erste überraschende Resultat von Lewy ist in folgendem Lemma enthalten:
\begin{lem}[{\cite{Lewy:1957}}]\label{thm:1_lewy}
Zu einer reellwertigen Funktion $\psi\in \rmC^1(\mathbb{R})$ besitze das Problem
\begin{equation}\label{eq:1_lewy:gleichung}
\mathscr Lu=\psi'(y_1)
\end{equation}
in einer Umgebung $\Omega\subset\R^3$ von $(0,0,y_1^0)$ eine $\rmC^1$-Lösung $u$. Dann ist $\psi$ analytisch in $y_1=y_1^0$.
\end{lem}

\begin{proof}
Wir integrieren $(\partial_1+\i\partial_2) u$ für eine Lösung $u$ von \eqref{eq:1_lewy:gleichung} über einen Kreis in der $x_1$-$x_2$ Ebene um den Punkt $(0,0,y_1^0)$. Der Radius wird dabei so klein gewählt, dass der Kreis in $\Omega$ liegt. Sei dazu
\begin{equation}
x_1^2+x_2^2=y_2=\mathrm{const},\qquad y_1=\mathrm{const},
\end{equation}
$t=\log\sqrt{y_2}=\log\sqrt{x_1^2+x_2^2}$ und $\theta$ derjenige Winkel gegeben durch
\begin{equation}
x_1+\i x_2=\sqrt{y_2} \, \e^{\i\theta} =\e^{t+\i\theta}.
\end{equation}
Dann erhält man durch einfaches Nachrechnen $\overline x \overline\partial_x = \overline\partial_t$, also
\begin{equation}\label{thm:1_lewy:proof1}
\frac{\partial}{\partial x_1} +\i\frac{\partial}{\partial x_2}=
\e^{-t+\i\theta} \left(\frac{\partial}{\partial t}
+\i\frac{\partial}{\partial \theta}\right).
\end{equation}
Diese Identität zusammen mit partieller Integration liefert
\begin{align}\label{thm:1_lewy:abl_gleich}
\begin{split}
\int_0^{2\pi} \left(\frac{\partial}{\partial x_1} +\i\frac{\partial}{\partial x_2}\right)u\d\theta 
&= \int_0^{2\pi} \e^{-t+\i\theta} \left(\frac{\partial}{\partial \,t}+\i\frac{\partial}{\partial \theta}\right)u \d\theta \\
&= \int_0^{2\pi} \e^{-t+\i\theta} \left(\frac{\partial u}{\partial \,t} +\i\frac{\partial u}{\partial \theta}\right)\d\theta \\
&= \int_0^{2\pi} \e^{-t+\i\theta} \left(\frac{\partial u}{\partial \,t} +u\right)\d\theta .
\end{split}
\end{align}
Weiter impliziert $\sqrt{y_2}=\e^t$ für jede differenzierbare Funktion $w$
\begin{align}\label{thm:1_lewy:int_gleichheit1}
\begin{split}
\frac{\partial w}{\partial\, t}+ w  = 2\sqrt{y_2} \frac{\partial}{\partial y_2} \left( \sqrt{y_2} w(\log\sqrt{y_2})\right).
\end{split}
\end{align}
Eingesetzt in das letzte Integral aus \eqref{thm:1_lewy:abl_gleich} ergibt sich
\begin{align}\label{thm:1_lewy:abl_gleich_final}
\begin{split}
\int_0^{2\pi} \left(\frac{\partial}{\partial x_1} +\i\frac{\partial}{\partial x_2}\right)u\d\theta 
= 2\left(\frac{\partial}{\partial y_2}\right)\int_0^{2\pi} \e^{\i\theta}\sqrt{y_2}u\d\theta.
\end{split}
\end{align}
Setzen wir nun 
\begin{equation}
I(y_1,y_2)=\i\int_0^{2\pi} \e^{\i\theta} \sqrt{y_2} u\d\theta,
\end{equation}
so liefert \eqref{eq:1_lewy:gleichung} und danach \eqref{thm:1_lewy:abl_gleich_final}
\begin{align}
\begin{split}
\frac{\partial I}{\partial y_1} +\i\frac{\partial I}{\partial y_2} 
&= \i\int_{0}^{2\pi} \frac{\partial}{\partial y_1}\left(\e^{\i\theta}\sqrt{y_2}u\right)+\i\frac{\partial}{\partial y_2}\left(\e^{\i\theta}\sqrt{y_2}u\right)\d\theta \\
&= \i\int_{0}^{2\pi} \e^{\i\theta}\sqrt{y_2}\left(\frac{\partial}{\partial y_1}u\right) +\i\e^{\i\theta}\frac{\partial}{\partial y_2}(\sqrt{y_2}u)\d\theta\\
&= \i\int_{0}^{2\pi} \e^{\i\theta}\sqrt{y_2} \left(
	\frac{\psi'(y_1)}{2\i(x_1+\i x_2)}
	+ \frac{\left(\frac{\partial}{\partial x_1} + \i\frac{\partial}{\partial x_2}\right)u}{2\i(x_1+\i x_2)} 
	+ \frac{\i}{\sqrt{y_2}} \frac{\partial}{\partial y_2}(\sqrt{y_2} u)
\right)\d\theta \\
&= \i\int_{0}^{2\pi} \e^{\i\theta}\sqrt{y_2} \left( 
	\frac{\psi'(y_1)}{2\i\sqrt{y_2}\e^{\i\theta}} 
	+ \frac{\left(\frac{\partial}{\partial x_1} + \i\frac{\partial}{\partial x_2}\right)u}{2\i\sqrt{y_2}\e^{\i\theta}}
	+ \frac{\i}{\sqrt{y_2}} \frac{\partial}{\partial y_2}(\sqrt{y_2} u)
\right)\d\theta	 \\
&= \int_0^{2\pi} \frac{\psi'(y_1)}{2}\d\theta 
	+ \frac{1}{2}\int_0^{2\pi} \left(\frac{\partial}{\partial x_1} +\i \frac{\partial}{\partial x_2}\right)u\d\theta
	- \int_0^{2\pi} \e^{\i\theta}\frac{\partial}{\partial y_2}(\sqrt{y_2} u) \d\theta \\
&= 2\pi\frac{\psi'(y_1)}{2}
	+  \left(\frac{\partial}{\partial y_2}\right)\int_0^{2\pi} \e^{\i\theta}\sqrt{y_2}u\d\theta
	- \left(\frac{\partial}{\partial y_2}\right) \int_0^{2\pi} \e^{\i\theta}\sqrt{y_2}u\d\theta\\
&= \pi\psi'(y_1).
\end{split}
\end{align}
Ferner ist
\begin{equation}
J(y):=J(y_1,y_2):=I(y_1,y_2)-\pi\psi(y_1),
\end{equation}
eine $\rmC^1$-Funktion. Nach Konstruktion erfüllt diese die Cauchy-Riemannschen Differentialgleichungen
\begin{equation}
\frac{\partial J}{\partial y_1} +\i\frac{\partial J}{\partial y_2} = 0
\end{equation}
 und ist somit analytisch in $y=y_1+\i y_2$. Ihr Definitionsbereich enthält alle $y=(y_1,y_2)$ mit $y_2>0$ hinreichend klein und $y_1$ nahe $y_1^0$.
 Da weiter für $y_2=0$ nach Konstruktion $I(y_1,0)=0$ gilt, folgt
\begin{align*}
 J(y_1,0)=-\pi\psi'(y_1)
\end{align*}
mit nach Voraussetzung reellem $\psi$ und $J$ kann mithilfe des Spiegelungsprinzips in die untere komplexe Halbebene analytisch fortgesetzt werden. Somit ist $\psi'(y_1)$ 
und damit auch $\psi(y_1)$ analytisch in einer Umgebung des Punktes $y_1=y_1^0$ und die Behauptung ist gezeigt.
\end{proof}

Betrachtet man nun \eqref{eq:1_lewy:gleichung} mit reellwertigem $\psi\in \rmC^\infty(\R)$, welches \textit{nicht} analytisch in $y_1=y_1^0$ ist, so liefert die Kontraposition des obigen Lemmas, dass \eqref{eq:1_lewy:gleichung} in keiner Umgebung $\Omega$ von $(0,0,y_1^0)$ eine Lösung $u\in \rmC^1(\Omega)$ besitzen kann. Lewy nutzte obiges Lemma um eine Funktion $f\in\rmC^\infty(\R)$ zu konstruieren, so dass
\begin{equation} 
   \mathscr L u = f(y_1)
\end{equation}
in {\em keinem} Gebiet $\Omega\subset\R^3$ eine Lösung in einem Hölderraum $\rmC^{1,\alpha}(\Omega)$ besitzen kann.




\section{Hörmanders Unlösbarkeitskriterium}
Hörmander zeigte in \cite{Hormander:1960a}, dass für jedes Gebiet $\Omega\subset\R^n$ eine glatte Funktion $f\in\rmC^\infty(\R^n)$ existiert, für welche keine Distribution $u\in\mathscr{D}'(\Omega)$ mit $\mathscr Lu=f$ existiert. Dazu zeigte er eine notwendige Bedingung für die Lösbarkeit eines Differentialausdrucks erster Ordnung. In \cite{Hormander:1960b} verallgemeinerte er diese noch auf Operatoren höherer Ordnung.

Zuerst zeigen wir ein Analogon zu Satz~\ref{thm:4:Umkehrung}, welches es erlaubt von Lösbarkeitsaussagen auf das Verschwinden der Poissonklammer 
von $p$ und $\overline p$ zu schließen. 

\begin{thm}[{\cite[Theorem 1]{Hormander:1960b}}]\label{thm:3.1_hoer}
Sei $\Omega\subset\mathbb{R}^n$ ein Gebiet und $P(x,\D)$ ein Differentialausdruck der Ordnung $m$ mit Hauptsymbol $p(x,\xi)$.
Angenommen, für jedes $f\in \rmC_0^\infty(\Omega)$  existiert eine distributionelle Lösung $u\in\mathscr D'(\Omega)$ zu
\begin{equation}\label{eq:3.1_hoer}
P(x,\D)u=f.
\end{equation}
Dann gilt für alle $x\in\Omega$ und $\xi\in\mathbb{R}^n$ mit $p(x,\xi)=0$
\begin{equation}\label{eq:3.1_hoer_aussage}
 \{p,\overline{p}\}(x,\xi)=0.
\end{equation}
\end{thm}
\begin{proof}
Wir zeigen dies indirekt, beginnen mit einer Umformulierung der Voraussetzung als Ungleichung und konstruieren danach geeignete Testfunktionen um einen Widerspruch herzuleiten. 
\noindent {\sl Schritt 1.}  
 Bezeichne $\rmB^\infty_0(\omega)=\{f\in\rmC^\infty(\Omega) \mid \supp f\subseteq\overline\omega\}$ und 
\begin{equation}\label{eq:M_N}
   M_N = \{ f\in \rmB^\infty_0(\omega)\mid \exists_{u\in\mathscr D'(\omega)} \; \forall_{\psi\in\rmC_0^\infty(\omega)}\; |\langle u,\psi\rangle | \le N \|\psi\|_{(N)} \quad\text{und}\quad P(x,\D)u=f \}.
\end{equation}
Für jedes $u\in\mathscr D'(\Omega)$ und $\omega\Subset\Omega$ gilt $ |\langle u,\psi\rangle | \le N \|\psi\|_{(N)}$ für alle $\psi\in\rmC_0^\infty(\omega)$ und ein hinreichend großes $N$. Die Voraussetzung impliziert also
\begin{equation}
   \bigcup_{N=1}^\infty M_N = \rmB^\infty_0(\omega).
\end{equation} 
Die Mengen $M_N$ sind abgeschlossen (siehe nachfolgendes Lemma \ref{lm:closed}),  konvex und symmetrisch. Weiter ist $\rmB^\infty_0(\omega)$ metrisierbar, der Bairesche Kategoriensatz impliziert also, dass mindestens eine der Mengen $M_N$ einen inneren Punkt (wegen Symmetrie den Ursprung) besitzt. Es gibt also ein $k$ und $\epsilon>0$, so dass
\begin{equation}\label{eq:fepsInMN}
    \{ f\in \rmB^\infty_0(\omega)\mid \|f\|_{(k)}<\epsilon\} \subset M_N
\end{equation}
für ein $N$ gilt.
%
$\bullet$\qquad {\sl Schritt 2.} 
Angenommen, die Behauptung gilt nicht. Sei also ohne Beschränkung der Allgemeinheit $0\in\Omega$ und gelte für ein $\xi\in\R^n$ sowohl $p(0,\xi)=0$ als auch\footnote{Da $\{p,\overline p\}$ reellwertig und ungerade in $\xi$ ist, erfüllt für $\{p,\overline p\}(0,\xi)\ne0$ entweder $\xi$ oder $-\xi$ die Bedingung $\{p,\overline p\}(0,\xi)<0$.}   $\{p,\overline p\}(0,\xi)<0$. Analog zu Lemma~\ref{thm:4:lem2} folgt die Existenz einer Funktion $u\in\rmC^\infty(\Omega)$ mit 
\begin{equation}
    p(x,\nabla u(x)) = \mathcal O(|x|^q),\qquad x\to0
\end{equation}
sowie
\begin{equation}
   u(x) = \i \xi\cdot x + x\cdot Ax + \mathcal O(|x|^3),\qquad x\to0 
\end{equation}
zu vorgegebener Ordnung $q$, obigem Vektor $\xi$ und geeignet (mit Lemma~\ref{thm:4:lem3}) gewählter symmetrischer Matrix $A$ mit negativ definitem Realteil. 
%
$\bullet$\qquad {\sl Schritt~3.} Wir definieren die Hilfsfunktionen
\begin{equation}
   f_{\tau, k}(x) = \tau^{-k} \chi(\tau x),\qquad \text{wobei $\chi\in\rmC_0^\infty(\R)$ mit}\quad \widehat \chi(-\xi) = (2\pi)^{-n/2}  \int \e^{\i x\cdot\xi} \chi(x) \d x \ne 0,
\end{equation}
sowie für $q=2(r+1)$, $r=n+k+m+N$ und dem oben konstruierten $u(x)$ die Hilfsfunktionen
\begin{equation}
   v_{\tau,k,N} (x) = \tau^{n+1+k} \e^{\tau u(x)} \sum_{j=0}^{r-1} \tau^{-j} \varphi_j(x) 
\end{equation}
mit noch zu bestimmenden Koeffizienten $\varphi_j\in\rmC_0^\infty(\Omega)$. Nach Konstruktion gilt für die Sobolevnormen
\begin{equation}\label{eq:cond1}
    \limsup_{\tau\to\infty} \| f_{k,\tau}\|_{(k)} < \infty,
\end{equation}
und für $\varphi_0(0)=1$ folgt
\begin{equation}\label{eq:cond3}
   \frac1\tau \int f_{\tau,k}(x) v_{\tau,k,N}(x)\d x = \int \e^{\tau u(\frac x\tau)} \chi(x) \sum_{j=0}^{r-1} \varphi_j(\frac x\tau) \tau^{-j} \d x \longrightarrow \int \e^{\i x\cdot\xi}\chi(x)\d x\ne0
\end{equation}
für $\tau\to\infty$. Die verbleibenden $\varphi_j$ werden so gewählt, dass 
\begin{equation}\label{eq:cond2}
   \limsup_{\tau\to\infty} \|{}^tP(x,\D) v_{\tau,k,N}\|_{(N)} <\infty 
\end{equation}
gilt. Dazu nutzt man, dass 
\begin{equation}
{}^tP(x,\D) v_{\tau,k,N}(x)=\tau^{n+1+k+m}  \e^{\tau u(x)} \sum_{j=0}^m \tau^{-j} a_j(x) 
\end{equation}
mit Koeffizienten $a_j\in\rmC_0^\infty(\Omega)$ gilt und wählt $\varphi_j$ so, dass $a_j(x)=\mathcal O(|x|^{q-2j})$ (was auf eine erneute Anwendung des Satzes von Cauchy--Kowalewskaja hinausläuft). Danach zeigt eine einfache Rechnung, dass für jedes $\psi\in\rmC_0^\infty(\omega)$ mit $\psi(x)=\mathcal O(|x|^{2s})$, $x\to0$, $s\geq 0$
und $\omega\Subset\Omega$ hinreichend klein stets die Normabschätzung $\limsup_{\tau\to\infty} \|\tau^{s-N}\psi \e^{\tau u}\|_{(N)}<\infty$ gilt und \eqref{eq:cond2} folgt.~$\bullet$ \qquad {\sl Schritt~4.} Wir zeigen, dass \eqref{eq:fepsInMN} im Widerspruch zur Existenz der gerade konstruierten Hilfsfunktionen steht. 
Zum Einen impliziert \eqref{eq:fepsInMN} die Existenz einer Distribution $u$ mit $P(x,\D)u=f$ für $f\in \rmB^\infty_0(\omega)$ mit $\|f\|_{(k)}<\epsilon$. Damit gilt für jede Testfunktion $v\in\rmC_0^\infty(\omega)$ die Abschätzung
\begin{equation}\label{eq:4.74}
   \bigg| \int f(x) v(x) \d x\bigg| = |\langle u, {}^t P(x,\D) v \rangle| \le N \| {}^t P(x,\D) v\|_{(N)}.
\end{equation}
Speziell mit $f=c f_{k,\tau}$ und $v=v_{k,N,\tau}$ folgt für $\tau\to\infty$ ein Widerspruch: Für $c>0$ klein genug impliziert \eqref{eq:fepsInMN} zusammen mit \eqref{eq:cond1}, dass $f\in M_N$. Also gilt \eqref{eq:4.74}. Nun ist die rechte Seite aber wegen \eqref{eq:cond2} gleichmäßig in $\tau$ beschränkt, die linke strebt mit \eqref{eq:cond3} für $\tau\to\infty$  gegen Unendlich. Widerspruch.
\end{proof}


\begin{rem}
Die im obigen Beweis verwendete Menge $\rmB^\infty_0(\omega)$ ist gerade der Abschluss von $\rmC_0^\infty(\omega)$ in $\rmC^\infty(\Omega)$
(also auch in $\rmC^\infty(\R^n)$). Damit kann man für jedes Gebiet $\Omega\subset\R^n$ den entsprechenden Raum auch als
\begin{equation}
\rmB_0^\infty(\Omega) =\{f\in \rmC^\infty(\Omega) \mid \forall_{\alpha\in\N_0^n}\;\forall_{\epsilon >0} \;\exists_{K_{\alpha,\epsilon}\Subset\Omega} \;:\; \sup\nolimits_{x\in \Omega\setminus K_{\alpha,\epsilon}}|\D^\alpha f(x)|<\epsilon \}
\end{equation}
charakterisieren. Weiter sei 
\begin{equation}
\rmB^\infty(\Omega) = \{ f\in\rmC^\infty(\Omega) \mid  \forall_{\alpha\in\N_0^n} \;:\; \sup\nolimits_{x\in\Omega} |\D^\alpha f(x)|<\infty \}.
\end{equation}
Nach Konstruktion ist $\rmB^\infty_0(\Omega)$ abgeschlossener Teilraum von $\rmB^\infty(\Omega)$ und für $\omega\Subset\Omega$ ist die Einschränkung auf $\omega$ surjektiv von $\rmB^\infty_0(\Omega)$ auf $\rmB^\infty(\omega)$. Das nachfolgende Lemma liefert damit insbesondere die im Beweis verwendete Abgeschlossenheit der Mengen $M_N$.
\end{rem}


\begin{lem}\label{lm:closed}
Sei $\omega\Subset\Omega$ und
\begin{equation}
M_N:= \{f\in \rmB^\infty(\omega)\mid \exists_{u\in\mathscr D'(\omega)}\; \forall_{\psi\in\rmC_0^\infty(\omega)}\; |\langle u,\psi\rangle | \le N \|\psi\|_{(N)} \quad\text{und}\quad P(x,\mathrm{D})u=f\quad \text{in $\omega$} \}.
\end{equation}
Dann ist $M_N\subset \rmB^\infty(\omega)$ abgeschlossen.
\end{lem}

\begin{proof}
Die Menge der Distributionen $u\in\mathscr D'(\omega)$, welche die  Bedingung 
\begin{equation}\label{lewy:uglSchwartz}
|\langle u,\psi\rangle|\leq N\|\psi\|_{(N)} 
\end{equation}
für alle $\psi\in\rmC_0^\infty(\omega)$ erfüllen, ist folgenkompakt. Wir zeigen dies mit einem Diagonalfolgenargument.
Sei dazu $(u_n)_{n\in\N}$ eine Folge in $\mathscr D'(\omega)$ mit dieser Schranke. Sei weiter $(\psi_j)_{j\in\N}$ eine Folge
von $\rmC_0^\infty(\omega)$ Funktionen, die in $\rmH^N_0(\omega)$ (also dem Abschluss von $\rmC_0^\infty(\omega)$ in der $\|\cdot\|_{(N)}$-Norm) dicht ist.
Diese Existiert wegen der Separabilität von $\rmH_0^N(\omega)$.

Da die Folge $\langle u_n,\psi_1\rangle$ in $\C$ beschränkt ist, existiert eine Teilfolge $(u_{n_k}^{(1)})_{k\in\mathbb{N}}$ derart, dass
$\langle u_{n_k}^{(1)},\psi_1\rangle$ konvergiert. Weiter finden wir auch eine Teilfolge $(u_{n_k}^{(2)})_{k\in\mathbb{N}}$ von $u_{n_k}^{(1)}$, so dass $\langle u_{n_k}^{(2)},\psi_2\rangle$ konvergiert, etc. Wählen wir nun die Diagonalfolge $v_k = u_{n_k}^{(k)}$, so konvergiert $\langle v_k,\psi_j\rangle$ nach Konstruktion für alle $j$. 
Also konvergiert wegen der vorausgesetzten Schranke \eqref{lewy:uglSchwartz} die Folge   $\langle v_k,\psi\rangle$  für alle $\psi\in\rmH^N_0(\omega)$ und da $\rmC_0^\infty(\omega)\subset \rmH_0^N(\omega)$ gilt in $\mathscr D'(\omega)$. Der Grenzwert erfüllt offenbar ebenfalls \eqref{lewy:uglSchwartz}.

Sei also $(f_n)_{n\in\mathbb{N}}$ eine gegen $f\in \rmB^\infty(\omega)$ konvergente Folge aus $M_N$. Dann gilt  $f_n=P(x,\D)u_n$ für gewisse $u_n\in\mathscr{D}'(\omega)$ mit  $|\langle u,\psi\rangle | \le N \|\psi\|_{(N)}$ für alle $\psi\in\rmC_0^\infty(\omega)$. Auf Grund der gerade gezeigten Folgenkompaktheit existiert eine in $\mathscr D'(\omega)$  konvergente Teilfolge $(u_{n_k})_{k\in\N}$. Sei nun $u=\lim_{k\rightarrow\infty}u_{n_k}$. Dann erfüllt $u$ ebenfalls  
 \eqref{lewy:uglSchwartz} und da $P(x,\D)u_{n_k}\to P(x,\D)u$ in $\mathscr D'(\omega)$ konvergiert, folgt $P(x,\D)u=f$.
\end{proof}
\begin{thm}[{\cite[Theorem 2]{Hormander:1960b}}]\label{thm:2_hoer}
Sei $P(x,\D)$ ein Differentialausdruck der Ordnung $m$ mit  Hauptsymbol $p(x,\xi)$ und existiere zu jedem $\omega\Subset\Omega$ 
ein $x\in\omega$ und ein $\xi\in\R^n$ mit
\begin{equation}
 \{p,\overline p\}(x,\xi)\ne 0.
\end{equation}
Dann existieren Funktionen $f\in \rmB_0^\infty(\Omega)$, so dass
\begin{equation}\label{lewy:pxDu=f}
P(x,\mathrm D)u=f
\end{equation}
in keiner der Mengen $\omega\subseteq\Omega$ eine Lösung $u\in\mathscr D'(\omega)$ besitzt. Die Menge dieser Funktionen $f$ ist von zweiter Kategorie\footnote{Eine Teilmenge $A$ eines topologischen Raumes $B$ heißt von erster Kategorie, falls eine abzählbare Menge nirgends dichter Teilmengen aus $B$ existiert, deren Vereinigung $A$ ergibt. Ist dies nicht der Fall, so heißt $A$ von zweiter Kategorie.}.
\end{thm}

\begin{proof} Der Beweis folgt {\cite[Theorem 3.2]{Hormander:1960a}}.
{\sl Schritt 1.}
Sei zunächst $\omega\subseteq\Omega$ eine feste, nichtleere Menge und $M$ definiert durch
\begin{equation}
M=\{f\in \rmB_0^\infty(\Omega)\mid \exists\,u\in\mathscr{D}'(\omega) : P(x,\mathrm{D})u=f\quad \text{auf $\omega$}\}.
\end{equation}
Wir zeigen, dass $M$ von erster Kategorie ist.  Sei hierzu $\omega_1\Subset\omega$ offen und nichtleer, d.h. insbesondere ist $\overline{\omega_1}\subset\omega$ kompakt. Dann existiert für jede Distribution $u\in\mathscr{D}'(\omega)$ ein $N\in\mathbb{N}$, so dass $u$ die Ungleichung \eqref{lewy:uglSchwartz}
für alle $\psi\in \rmC_0^\infty(\omega_1)$ erfüllt ist. Seien nun wieder Mengen $M_N$ durch
\begin{equation}
M_N:= \{f\in \rmB^\infty(\omega_1)\mid \exists\,u\in\mathscr D'(\omega_1): P(x,\mathrm{D})u=f\;\mathrm{in}\;\omega_1\;\wedge\; u\mathrm{\;erf"ullt\;}\eqref{lewy:uglSchwartz}\},
\end{equation}
gegeben. Diese sind nach Lemma~\ref{lm:closed} abgeschlossen, offenbar konvex und auch symmetrisch. 
Keines der $M_N$ besitzt einen inneren Punkt. 
Anderenfalls gäbe es wiederum ein $N$, ein $k$ und ein $\epsilon>0$, so dass alle $f\in \rmB^\infty(\omega_1)$ mit
$\|f\|_{(k)}<\epsilon$ in $M_N$ liegen. Also gäbe es zu jedem $f\in\rmC_0^\infty(\omega_1)$ eine Konstante $\delta\ne0$, so dass $P(x,\D)u=\delta f$ in $\mathscr D'(\omega_1)$ eine L\"osung besitzt. Da aber dann $\delta^{-1}u$ schon $P(x,\D)u=f$ löst und $f$ beliebig war, widerspricht dies Satz~\ref{thm:3.1_hoer}. 

Gleiches trifft auch auf ihre Urbilder $\widetilde M_N$ unter der stetigen Einschränkung $\rmB^\infty_0(\Omega)\to \rmB^\infty(\omega_1)$ zu. Insbesondere sind diese Mengen abgeschlossen und besitzen keine inneren Punkte. Also ist $\widetilde M_N$ von erster Kategorie und ebenso die abzählbare Vereinigung dieser Mengen. Da aber $M\subset \bigcup_N \widetilde M_N$ gilt, folgt die  Behauptung.
$\bullet$\qquad {\sl Schritt 2.}
Wir zeigen nun, dass \eqref{lewy:pxDu=f} tatsächlich keine Lösung in $\Omega$ besitzt. Sei dazu $(\omega_j)_{j\in\mathbb{N}}$ eine abzählbare Basis der Topologie von $\Omega$ und bezeichne
\begin{equation}
M^{(j)}:=\{f\in \rmB_0^\infty(\Omega)\mid \exists u\in\mathscr{D}'(\omega_j): P(x,\mathrm{D})u=f\;\mathrm{auf\;}\omega_j\}.
\end{equation}
Dann folgt aus Schritt 1, dass alle $M^{(j)}$ und folglich auch $\bigcup M^{(j)}$ von erster Kategorie sind. Nach Definition der $M^{(j)}$ kann aber $P(x,\mathrm{D})u=f$ für $f\not\in  \bigcup M^{(j)}$ auf keinem $\omega_j$ gelöst werden. Da für jede beliebige, offene, nichtleere Menge $\omega\subseteq\Omega$ ein Index $j_0$ existiert, so dass $\omega_{j_0}\subset\omega$ gilt, besitzt $P(x,\mathrm{D})u=f$ auf keiner solchen Menge $\omega$ eine Lösung. 
\end{proof}



\begin{exa}
Für Lewys Beispiel 
\begin{equation}
p(x,\xi)=-i\xi_1+\xi_2-2(x_1+ix_2)\xi_3
\end{equation}
in $n=3$ Dimensionen aus dem ersten Teil dieses Kapitels erhalten wir $\{p,\overline{p}\}(x,\xi)=-8\xi_3$. Wegen
\begin{equation}
\xi_3 =1,\quad \xi_1=-2x_2,\quad \xi_2=2x_1\qquad\Rightarrow\qquad p(x,\xi)=0
\end{equation}
ist $\{p,\overline{p}\}(x,\xi)=-8\neq 0$ und \eqref{eq:3.1_hoer_aussage} gilt nicht für alle $x\in\mathbb{R}^n$ womit folglich die Voraussetzungen von Satz \ref{thm:2_hoer} erfüllt sind.
\end{exa} % Ein unloesbarer Operator :: Robin Lang
%
%
\part{Untere Schranken an Pseudodifferentialoperatoren}
%
% !TEX root = main.tex
\chapter{Einleitung}
Hier sollen die Grundlagen dafür gelegt werden, auch Operatoren mit variablen Koeffizienten richtig behandeln zu können. Die Darstellung basiert auf \cite{Hormander:1965}, \cite{Hormander:1966}, sowie \cite[Kapitel 18]{Hormander:1985}. Zuerst verallgemeinern wir den Begriff des Differentialoperators soweit, daß wir auch Inverse und allgemeinere Funktionen von solchen Operatoren behandeln können. 

\section{Operatoren und Symbole}
Sei $\Omega$ eine Mannigfaltigkeit. Differentialoperatoren auf $\Omega$ können dann in lokalen Karten definiert werden oder global durch ihre Eigenschaften charakterisiert werden. Die bekannteste davon ist, daß eine stetige lineare Abbildung $P:\rmC_0^\infty(\Omega)\to\rmC^\infty(\Omega)$ genau dann Differentialoperator ist, wenn $P$ lokal ist, also wenn
\begin{equation}
  \forall f\in\rmC_0^\infty(\Omega)\quad:\quad \supp Pf \subseteq \supp f
\end{equation}
gilt. Für Verallgemeinerungen brauchbarer ist folgende Charakterisierung:

\begin{lem}
Eine  lineare Abbildung   $P:\rmC_0^\infty(\Omega)\to\rmC^\infty(\Omega)$ ist genau dann ein Differentialoperator der Ordnung $m$, wenn für alle $f\in\rmC_0^\infty(\Omega)$ und alle $g\in\rmC^\infty(\Omega)$ die Funktion 
\begin{equation}\label{eq:6:6.2}
\e^{-\i\lambda g} P(f\e^{\i\lambda g}) = \sum_{j=0}^m P_j(f,g) \lambda^j
\end{equation}
ein Polynom vom Grad $m$ in $\lambda$ ist.
\end{lem}
\begin{proof}
Die Hinrichtung ist klar. Zum Beweis der Rückrichtung sei $x'\in\Omega$ ein Punkt und $f\in\rmC_0^\infty(\Omega)$ mit $f=1$  in einer (in einem Kartengebiet liegenden) Umgebung $\omega$ von $x'$. Sei weiter $\xi\in\R^n$ und $g(x) = x\cdot\xi$. Dann folgt aus \eqref{eq:6:6.2}
\begin{equation}
    \e^{-\i \lambda x\cdot\xi} P ( f\e^{\i \lambda x\cdot\xi}) = \sum_{j=0}^m p_j(f;x,\xi) \lambda^j
\end{equation}
mit $\rmC^\infty$-Funktionen auf $\omega\times\R^n$ als Koeffizienten $p_j(f;x,\xi)$. Da auch
$p_j(f;x,\lambda\xi)=p_j(f;x,\xi)\lambda^j$ gilt, ist $p_j(f;x,\xi)$ homogen vom Grad $j$. Die einzigen auf ganz $\R^n$ glatten homogenen Funktionen sind Polynome,
$p_j(f;x,\xi) = \sum_{|\alpha|=j} a_\alpha(f;x) \xi^\alpha$. Sei nun $u\in\rmC_0^\infty(\omega)$. Dann gilt mit der Fourierschen Inversionsformel
\begin{equation}
    u(x) = (2\pi)^{-n/2} \int \e^{\i x\cdot\xi} \widehat u(\xi)\d\xi
\end{equation}
und da $u=fu$ folgt insbesondere 
\begin{align}
    Pu(x) &= P(fu)(x) = (2\pi)^{-n/2} \int P(f \e^{\i x\cdot\xi}) \widehat u(\xi)\d\xi \notag \\&= \sum_{j=0}^m (2\pi)^{-n/2} \int \e^{\i x\cdot\xi} p_j(f;x,\xi) \widehat u(\xi)\d\xi
    = \sum_{|\alpha|\le m} a_\alpha(f;x)\D^\alpha u.
\end{align}
Mit Linearität folgt die Behauptung.
\end{proof}

Sei nun $\Omega$ ein Gebiet.
Wir betrachten im folgenden allgemeiner Operatoren $P : \rmC_0^\infty(\Omega) \to \rmC^\infty(\Omega)$ und deren Beschreibung durch ein \eIndex[Pseudodifferentialoperator]{Symbol}
\begin{equation}\label{eq:6:6.6}
    \sigma_P(f; x,\xi) = \e^{-\i x\cdot\xi} P( f(x) \e^{\i x\cdot\xi}).
\end{equation}  
Dabei sei $f\in\rmC_0^\infty(\Omega)$ kompakt getragen mit $f(x)=1$ auf einer kompakten Teilmenge $\omega\Subset\Omega$. 
Dann gilt für jedes $u\in\rmC_0^\infty(\omega)$ 
\begin{equation}
   P u (x) = (2\pi)^{-n/2} \int \e^{\i x\cdot\xi} \sigma_P(f; x,\xi) \widehat u(\xi) \d\xi.
\end{equation}
Der Operator $P$ wird als \eIndex{Pseudodifferentialoperator} der \eIndex[Pseudodifferentialoperator]{Ordnung} $m$ bezeichnet, falls jedes so entstehende Symbol 
$\sigma_P(f;\cdot,\cdot)$ zur nachfolgend definierten \eIndex{Symbolklasse} $S^m_{\rm loc}(\Omega\times\R^n)$ gehört.

\begin{df}
Eine Funktion $\sigma\in\rmC^\infty(\Omega\times\R^n)$ gehört zur Symbolklasse $S^m_{\rm loc}(\Omega\times\R^n)$ falls
\begin{equation} 
  \sup_{x\in K}  |  \partial_x^\beta\partial_\xi^\alpha \sigma(x,\xi) | \le C_{K,\alpha,\beta} \langle\xi\rangle^{m-|\alpha|}
\end{equation}
für alle Kompakta $K\Subset\Omega$ und alle Multiindices $\alpha,\beta\in\N^n_0$ gilt. Dabei bezeichnet $\langle\xi\rangle = \sqrt{1+|\xi|^2}$.
\end{df}

Ist nun $\sigma\in S^m_{\rm loc}(\Omega\times\R^n)$, so konvergiert für jedes $u\in\rmC_0^\infty(\Omega)$ das Integral
\begin{equation}\label{eq:6:6.9}
   P u (x) = (2\pi)^{-n/2} \int \e^{\i x\cdot\xi} \sigma(x,\xi) \widehat u(\xi) \d\xi
\end{equation}
und definiert eine glatte Funktion $Pu\in\rmC^\infty(\Omega)$. Es gilt sogar mehr

\begin{lem}
Sei $\sigma\in S^m_{\rm loc}(\Omega\times\R^n)$ und $Pu$ für $u\in\rmC_0^\infty(\Omega)$ durch \eqref{eq:6:6.9} definiert. Dann gilt für jedes $f\in\rmC_0^\infty(\Omega)$
und das durch \eqref{eq:6:6.6} zugeordnete Symbol
\begin{equation}
    \sigma_P(f;\cdot,\cdot)\in S^m_{\rm loc}(\Omega\times\R^n). 
\end{equation}
Weiterhin gilt
\begin{equation}
   \sigma_P(f;x,\xi) \sim \sum_{\alpha}  \frac{1}{\alpha!} \big( \partial_\xi^\alpha \sigma(x,\xi) \big) \big(\D_x^\alpha f(x)\big)
\end{equation}
und, falls $f(x)=1$ auf $\omega\Subset\Omega$ gilt, auch $\sigma_P(f;x,\xi)=\sigma(x,\xi) \mod\mathscr S(\overline{\omega}\times\R^n)$.
\end{lem}
\begin{proof}
Nach Definition gilt
\begin{align}
   \sigma_P(f;x,\xi) &=(2\pi)^{-n}  \e^{-\i x\cdot\xi} \int \e^{\i x\cdot\eta} \sigma(x,\eta) \int_\Omega \e^{-\i y\cdot(\eta-\xi)} f(y) \d y \d\eta \notag\\
   & = (2\pi)^{-n/2}  \int \e^{\i x\cdot(\eta-\xi)} \sigma(x,\eta) \widehat f(\eta-\xi)\d\eta \notag\\
   & = (2\pi)^{-n/2}  \int \e^{\i x\cdot\eta} \sigma(x,\eta+\xi) \widehat f(\eta)\d\eta \notag\\
   & = (2\pi)^{-n/2}  \int \e^{\i x\cdot\eta} \left( \sum_{|\alpha|<N} \frac{\eta^\alpha}{\alpha!} \partial_\xi^\alpha \sigma(x,\xi) + R_N(x,\xi,\eta) \right) \widehat f(\eta)\d\eta \notag\\  
   & =  \sum_{|\alpha|<N} \frac{1}{\alpha!} \big( \partial_\xi^\alpha \sigma(x,\xi) \big) \big(\D_x^\alpha f(x)\big)  +
   \rho_N(x,\xi)
\end{align}
unter Ausnutzung des taylorschen Satzes mit Entwicklungspunkt $\eta=0$. Weiter gilt unter Nutzung der Integraldarstellung des Restgliedes $R_N(x,\xi,\eta)$ 
\begin{equation}
   R_N(x,\xi,\eta) = N \int_0^1 \sum_{|\alpha|=N} \frac{(1-t)^N \eta^\alpha}{\alpha!} \partial_\xi^\alpha \sigma(x,\xi+t\eta) \d t 
\end{equation}
und somit für beliebige Kompakta $K\Subset\Omega$ und $N\ge m$
\begin{align}
   |\langle\xi\rangle^{N-m} \rho_N(x,\xi)|& \lesssim 
   \left|\langle\xi\rangle^{N-m}  \int \e^{\i x\cdot\eta} \int_0^1 \sum_{|\alpha|=N} \frac{(1-t)^N \eta^\alpha}{\alpha!} \partial_\xi^\alpha \sigma(x,\xi+t\eta) \d t  \widehat f(\eta) \d\eta   \right| \notag\\
&\lesssim  \int \int_0^1 \langle\eta\rangle^N \langle\xi+t\eta\rangle^{m-N} \langle\xi\rangle^{N-m} \d t | \widehat f(\eta) |\d\eta   
\lesssim \int \langle\eta\rangle^{2N} |\widehat f(\eta)|\d\eta
\end{align}
gleichmäßig in $x\in K$ (und unter Ausnutzung der Ungleichung von Peetre $\langle \xi\rangle^s \lesssim \langle\eta\rangle^s \langle\xi-\eta\rangle^s$ für $s\ge0$).
Weiter gilt für Multiindices $\beta,\gamma\in\N_0^n$
\begin{multline}
   \partial_x^\gamma \partial_\xi^\beta \rho_N(x,\xi) \\
   =  (2\pi)^{-n/2} N \sum_{\delta\le\gamma} \binom{\gamma}{\delta} \i^{|\delta|} \int \e^{\i x\cdot\eta} \int_0^1 \sum_{|\alpha|=N} \frac{(1-t)^N \eta^{\alpha+\delta}}{\alpha!} \partial_x^{\gamma-\delta} \partial_\xi^{\alpha+\beta} \sigma(x,\xi+t\eta) \d t  \widehat f(\eta) \d\eta 
\end{multline}
und obige Abschätzung liefert $\rho_N\in S^{m-N}_{\rm loc}(\Omega\times\R^n)$. Das impliziert die Behauptung.
\end{proof}

Pseudodifferentialoperatoren auf Gebieten kann man nicht notwendigerweise verketten. Wir bezeichnen einen Pseudodifferentialoperator $P$ als \eIndex[Pseudodifferentialoperator]{eigentlich getragen}, falls er $\rmC_0^\infty(\Omega)$ nach $\rmC_0^\infty(\Omega)$ abbildet. Dies gilt zum Beispiel dann, wenn das Symbol in $x$ kompakt getragen ist. Allgemeiner gibt es für eigentlich getragene Operatoren zu jedem Kompaktum $K\Subset\Omega$ ein Kompaktum $L\Subset\Omega$, so daß für $u\in\rmC_0^\infty(K)$ stets $\supp Pu \subset L$ gilt. Sei nun $f\in\rmC_0^\infty(K)$. Dann gilt für einen zweiten Pseudodifferentialoperator $Q$ definiert durch das Symbol $\tau\in S^{m_2}_{\rm loc}(\Omega\times\R^n)$
\begin{equation}
    \sigma_{Q\circ P} (f;x,\xi) = (2\pi)^{-n} \e^{-\i x\cdot\xi} \int \e^{\i x\cdot\eta}  \tau(x,\eta) 
      \int_{\Omega} \e^{-\i y\cdot(\eta-\xi)} \sigma_P(f;y,\xi) \d y \d\eta 
\end{equation}
und damit nach der Argumentation des letzten Beweises
\begin{equation}
    \sigma_{P\circ Q}(f;x,\xi) \sim \sum_\alpha \frac1{\alpha!} \big(\partial_\xi^\alpha \tau(x,\xi)\big) \big( \D_x^\alpha \sigma_P(f;x,\xi)\big).
\end{equation}
Das ist eine erste Form der Kompositionsformel für Pseudodifferentialoperatoren. Ist $f(x)=1$ auf einer kleineren Menge $\omega\Subset K$, so ergibt sich das Symbol der Komposition $P\circ Q$ für $x\in\omega$ und modulo $\mathscr S(\overline\omega\times\R^n)$ als asymptotische Summe
\begin{equation}
    \sum_{\alpha}  \frac1{\alpha!} \big(\partial_\xi^\alpha \tau(x,\xi)\big) \big( \D_x^\alpha \sigma(x,\xi)\big).
\end{equation}
Analog kann man den formal Adjungierten eines eigentlich getragenen Pseudodifferentialoperators bestimmen. So gilt (vorerst formal, also distributionell)
\begin{equation}
   P^* u (y) =\int \e^{\i x\cdot\xi}  \int_\Omega \e^{-\i y\cdot\xi} \overline{\sigma(y,\xi)} u(y)\d y\d\xi
\end{equation}
und damit wiederum
\begin{equation}
   \sigma_{P^*}(f;x,\xi) \sim \sum_\alpha \frac1{\alpha!} \partial_\xi^\alpha\D_x^\alpha \big(\overline{\sigma(x,\xi)} f(x)\big).
\end{equation}
Auf einer Menge $\omega\Subset\Omega$ auf welcher $f(x)=1$ gilt, vereinfacht sich dies modulo $\mathscr S(\overline\omega\times\R^n)$  zu
\begin{equation}
   \sum_\alpha \frac1{\alpha!} \partial_\xi^\alpha\D_x^\alpha \overline{\sigma(x,\xi)}.
\end{equation}
Pseudodifferentialoperatoren mit Symbolen aus $S^0_{\rm loc}(\Omega\times\R^n)$ sind $\rmL^2_{\rm comp}$-$\rmL^2_{\rm loc}$-beschränkt. Das liefert nachfolgendes Lemma. Zusammen mit obigen asymptotischen Entwicklungen erhält man die $\rmH^s_{\rm comp}$-$\rmH^{s-m}_{\rm loc}$-Beschränktheit für Operatoren mit Symbolen aus $S^m_{\rm loc}(\Omega\times\R^n)$ für alle $s$.
\begin{lem}
Sei $\sigma\in S^0_{\rm loc}(\Omega\times\R^n)$ und $u,v\in\rmC_0^\infty(\Omega)$. Dann gilt für den durch \eqref{eq:6:6.6} definierten Operator $P$
und jede Funktion $f\in\rmC_0^\infty(\Omega)$
\begin{equation}
   | \spro{Pu}{f v} | \le C \|u\|\|v\|.
\end{equation}
Die Konstante $C$ hängt dabei nur von endlich vielen Seminormen des Symbols und der Wahl von $f$ ab.
\end{lem}
\begin{proof}
Nach Definition von $P$ gilt
\begin{equation}
   \spro{Pu}{fv} = (2\pi)^{-n/2} \iint \e^{\i x\cdot \xi}\overline{ v(x) f(x)} \sigma(x,\xi) \widehat u(\xi)\d\xi\d x,
\end{equation}
mit der Hilfsfunktion
\begin{equation}\label{eq:6:h-def}
  h(\eta,\xi) =  (2\pi)^{-n/2} \int \e^{-\i x\cdot\eta} \overline{f(x)} \sigma(x,\xi) \d x
\end{equation}
und der Plancherel-Identität also
\begin{equation}
   \spro{Pu}{fv} = (2\pi)^{-n/2} \iint \overline{ \widehat v(\eta) }h(\eta-\xi,\xi) \widehat u(\xi)\d\xi\d\eta.
\end{equation}
Es bleibt die Funktion $h$ abzuschätzen. Partielle Integration in \eqref{eq:6:h-def} liefert
\begin{equation}
   |h(\eta,\xi)|  \le C_N \langle\eta\rangle^{-N}
\end{equation}
für alle $N$ und mit dem Schur-Test die Behauptung.
\end{proof}

\section{Hörmanders globale Pseudodifferentialoperatoren}

Im folgenden betrachten wir globale Symbole $\sigma(x,\xi)$, welche die Bedingung
\begin{equation}
    \sup_{x\in\R^n} |\partial_x^\beta\partial_\xi^\alpha \sigma(x,\xi)| \le C_{\alpha,\beta} \langle\xi\rangle^{m-|\alpha|}
\end{equation}
für alle Multiindices $\alpha,\beta\in\N_0^n$ erfüllen. Die Symbolabschätzungen sind also global gleichmäßig in $x$.
Wir bezeichnen die Menge dieser Symbole mit $S^m_{\rm unif}(\R^n\times\R^n)$. Gilt $\sigma\in S^m_{\rm unif}(\R^n\times\R^n)$
und $u\in\mathscr S(\R^n)$ eine Schwartzfunktion, so definiert
\begin{equation}\label{eq:6:6.17}
   Pu (x) = (2\pi)^{-n/2} \int \e^{\i x\cdot\xi} \sigma(x,\xi) \widehat u(\xi)\d\xi
\end{equation}
wiederum eine Schwartzfunktion. Das folgt durch geeignetes partielles integrieren. Interessanter für uns ist folgender Satz; Operatoren der Ordnung $0$ sind $\rmL^2$--$\rmL^2$-beschränkt.
 
\begin{thm}
Sei $\sigma\in S^0_{\rm unif}(\R^n\times\R^n)$ und $P$ durch \eqref{eq:6:6.17} definiert. Dann gibt es eine Konstante $C>0$, so daß für alle $u\in\mathscr S(\R^n)$
\begin{equation}
   \| Pu\| \le C \|u\|
\end{equation}
gilt.
\end{thm}
%\begin{proof}
%%Wir nehmen in einem ersten Schritt an, daß das Symbol $\sigma(x,\xi)$ kompakt in $x$ getragen ist und bezeichnen seine Fouriertransformierte 
%%als $\widehat \sigma(\zeta,\xi)$. Nach Voraussetzung ist diese gleichmäßig in $\xi$ schnell fallend in $\zeta$, insbesondere gilt also
%%\begin{equation}
%%   |\widehat \sigma(\zeta,\xi)| \le C_N \langle\zeta\rangle^{-N}, \qquad N>n. 
%%\end{equation}
%%Damit gilt mit der Fourierschen Inversionsformel und dem Satz von Fubini
%%\begin{align}
%%    P u(x) &= (2\pi)^{-n} \int \e^{\i x\cdot\xi} \left(\int \e^{\i x\cdot\zeta} \widehat \sigma(\zeta,\xi) \d\zeta\right) \widehat u(\xi)\d\xi \notag\\
%%    &= (2\pi)^{-n} \int \e^{\i x\cdot\zeta} \left( \int \e^{\i x\cdot\xi}\widehat \sigma(\zeta,\xi) \widehat u(\xi) \d\xi \right) \d\zeta
%%\end{align}
%%die Normabschätzung
%%\begin{equation}
%%  \| Pu \| \le (2\pi)^{-n/2} \int  \sup_{\xi\in\R^n} |\widehat \sigma(\zeta,\xi)| \d\zeta\, \|u\| \le C \|u\|. 
%%\end{equation}
%%Ersetzt man nun allgemein das Symbol $\sigma(x,\xi)$ durch $\chi(x)\sigma(x,\xi)$ für eine Abschneidefunktion $\chi\in\rmC_0^\infty(\R^n)$ 
%%mit $|\chi(x)|\le 1$, so hängt die Konstante $C$ in der Normabschätzung $\|\chi Pu\|\le C\|u\|$ nur von der Größe des Trägers von $\chi$ ab.
%
%%Der folgende Beweis kann analog für die $\rmL^p$--$\rmL^p$-Beschränktheit für $1<p<\infty$ genutzt werden, wenn man das im folgenden Abschnitt gegebene Littlewood--Paley-Lemma entsprechend nutzt. Wir wählen eine Funktion $\varphi\in\rmC_0^\infty(\R_+)$ derart, daß $\supp\varphi\subset[1,2]$
%%\begin{equation}
%% \varphi(s)\ge0\qquad\text{sowie}\qquad   \sum_{j\in\mathbb Z} \varphi(2^{-j}s) = 1,\qquad s>0,
%%\end{equation}
%%gilt. Dazu kann man zum Beispiel eine Abschneidefunktion $\chi\in\rmC^\infty(\R_+)$ mit $\chi(s)=1$ auf $s<1$, $\chi(s)=0$ auf $s>2$ und 
%%$\chi'(s)\le0$ wählen und $\varphi(s)=\chi(2s)-\chi(s)$ setzen. Sei weiter $\varphi_j(s)=\varphi(2^{-j}s)$ und
%%\begin{equation}
%%    P_j = P\circ \varphi_j(\D) .
%%\end{equation}
%%Dann ist $P_j$ ein Operator zum Symbol $\sigma(x,\xi)\varphi_j(\xi)$.
%\end{proof}


\begin{thm}
Seien $A$ und $B$ globale Pseudodifferentialoperatoren mit Symbolen $\sigma_A\in S^{m_1}_{\rm unif}(\R^n\times\R^n)$ und $\sigma_B\in S^{m_2}_{\rm unif}(\R^n\times\R^n)$. Dann gilt
\begin{enumerate}
\item die Verkettung $A\circ B$ ist ein globaler Pseudodifferentialoperator der Ordnung $m_1+m_2$ und für das zugehörige Symbol gilt
\begin{equation}
   \sigma_{A\circ B}(x,\xi) \sim \sum_\alpha \frac1{\alpha!} \big(\partial_\xi^\alpha \sigma_A(x,\xi) \big) \big(\D_x^\alpha \sigma_B(x,\xi)\big);
\end{equation}
\item der formal adjungierte Operator $A^*$ ist ein Pseudodifferentialoperator der Ordnung $m_1$ und für das zugehörige Symbol gilt
\begin{equation}
   \sigma_{A^*}(x,\xi) = \sum_{\alpha} \frac1{\alpha!} \partial_\xi^\alpha \D_x^\alpha \overline{\sigma_A(x,\xi)}.
\end{equation}
\end{enumerate}
\end{thm}
%\begin{proof}
%\end{proof}

\section{Lokalisierungen und Sobolevabschätzungen}
Sei im folgenden $\Theta\in\rmC_0^\infty(\R^n)$ mit $\supp\Theta\subset [-3/4,3/4]^n$ sowie $\Theta>0$ auf $[-1/2,1/2]^n$. Dann sind die Funktionen
\begin{equation}
\varphi_k(x) = \Theta(x-k) \bigg(\sum_{j\in\Z^n} \Theta(x-j)^2\bigg)^{-1/2}
\end{equation}
Elemente von $\rmC_0^\infty(\R^n)$ mit $\sum_{k} \varphi_k(x)^2=1$ und es existiert für jeden Multiindex $\alpha\in\N_0^n$ eine Konstante $C_\alpha$ mit
\begin{equation}
    \sum_{k\in\Z^n} | \D^\alpha \varphi_k(x)|^2 \le C_\alpha.
\end{equation}
Weiterhin impliziert $x,y\in\supp\varphi_k$ stets $|x-y|\le 3\sqrt n/2$. Ausgehend von dieser Partition der Eins definieren wir
\begin{equation}
    \psi_k(\xi)= \varphi_k(\xi / \sqrt{|\xi|}), 
\end{equation}
so daß
\begin{equation}\label{eq:6:6.35} 
    \sum_{k\in\Z^n} \psi_k(\xi)^2 = 1\qquad\text{und}\qquad |\xi|^\alpha \sum_{k\in\Z^n} |\partial_\xi^\alpha \psi_k(\xi)|^2 \le C_\alpha
\end{equation}
gilt. Weiter impliziert $\xi,\eta\in\supp\psi_k$ stets $|\sqrt{|\xi|}-\sqrt{|\eta|}|\le C$ und damit 
\begin{equation}
 |\xi-\eta|\le C \langle\xi\rangle^{1/2} \qquad\text{für alle $\xi,\eta\in\supp\psi_k$}.
\end{equation}
Weiterhin nützlich ist
\begin{equation}
  \sum_{k\in\Z^n} |\psi_k(\xi)-\psi_k(\eta)|^2 \le C \frac{|\xi-\eta|^2}{ \langle\xi\rangle^{1/2} \langle\eta\rangle^{1/2}},
\end{equation} 
was nur für $|\xi-\eta|\le \langle\xi\rangle^{1/2}$ nichttrivial ist und dann aus \eqref{eq:6:6.35} für $|\alpha|=1$ per Integration über die Strecke von $\xi$ zu $\eta$ folgt.

\begin{df}[Sobolevnormen]
Für $u\in\rmC_0^\infty(\Omega)$, $\Omega \subset\R^n$ ein Gebiet, definieren wir die Sobolevnorm der Ordnung $s$,
\begin{equation}
     \| u \|_{(s)} = \bigg(\int \langle\xi\rangle^{2s} |\widehat u(\xi)|^2\d\xi\bigg)^{1/2}. 
\end{equation}
Weiter bezeichne $\rmH^s_0(\Omega)$ den Abschluß von $\rmC_0^\infty(\Omega)$ bezüglich dieser Norm.
\end{df}


Als eine Anwendung wollen wir zu einem Operator $P$ zum Symbol $\sigma\in S^m_{\rm comp} (\Omega\times\R^n)$ die sesquilineare Form
\begin{equation}
   \spro{Pu}{v} = (2\pi)^{-n/2} \iint \e^{\i x\cdot\xi} \sigma(x,\xi) \widehat u(\xi) \d\xi \overline{v(x)}\d x 
   = (2\pi)^{-n/2} \iint \widehat u(\xi) \overline{\widehat v(\eta)} \widehat\sigma(\eta-\xi,\xi) \d\xi\d\eta
\end{equation} 
betrachten. Wir setzen $u_k(x) = \psi_k(\D) u(x)$, also $\widehat u_k(\xi) = \psi_k(\xi) \widehat u(\xi)$ und untersuchen die Differenz 
$\spro{Pu}{v}-\sum_k \spro{Pu_k}{v_k}$. Für diese gilt
\begin{align}
       \spro{Pu}{v} -\sum_{k\in\Z^n} \spro{Pu_k}{v_k}  &= (2\pi)^{-n/2}  \iint \widehat\sigma (\eta-\xi,\xi) \widehat u(\xi) \overline{\widehat v(\eta)} \bigg( 1 - \sum_{k\in\mathbb Z^n} \psi_k(\xi)\psi_k(\eta)\bigg) \d\xi\d\eta  \notag\\
       &= \frac12 (2\pi)^{-n/2} \iint \widehat\sigma (\eta-\xi,\xi) \widehat u(\xi) \overline{\widehat v(\eta)}  \sum_{k\in\mathbb Z^n} | \psi_k(\xi)-\psi_k(\eta)|^2 \d\xi\d\eta 
\end{align}
und damit
\begin{equation}
   \bigg|\spro{Pu}{v} - \sum_{k\in\Z^n}\spro{Pu_k}{v_k}\bigg| \le C \iint | \widehat \sigma(\eta-\xi,\xi) \widehat u(\xi) \overline{\widehat u(\eta)}| |\eta-\xi|^2 \langle\xi\rangle^{-1/2} \langle\eta\rangle^{-1/2} \d\xi\d\eta, 
\end{equation}
zusammen mit den Symbolabschätzungen also
\begin{equation}
    \bigg|\spro{Pu}{v} - \sum_{k\in\Z^n}\spro{Pu_k}{v_k}\bigg| \le C \|u\|_{(m-1/2)} \|v\|_{(-1/2)}.
\end{equation}
 Andererseits gilt für jeden einzelnen Summanden
 \begin{align}
     |\spro{Pu_k}{v_k} | &= \bigg|\iint \widehat u(\xi) \overline{\widehat v(\eta)} \psi_k(\xi)\psi_k(\eta) \widehat\sigma(\eta-\xi,\xi) \d\xi\d\eta\bigg|
     \notag\\
     &\le \|u\|_{(m)}  \|v\|_{(0)} \iint | \langle\xi\rangle^{-m}\psi_k(\xi)\psi_k(\eta) \widehat\sigma(\eta-\xi,\xi)|^2 \d\xi\d\eta.
 \end{align}





 % Einleitung, Teil 2
% !TEX root = main.tex
\chapter{Die Ungleichung von G\r{a}rding}

\section{Dirichletformen}
Im folgenden sollen Dirichletformen 
\begin{equation}
   P(x,\D,\overline\D)[u,v] = \sum_{|\alpha|,|\beta|\le m} \int_\Omega p_{\alpha,\beta}(x) \big(\D^\alpha u(x)\big) \overline{\big(\D^\beta v(x)\big)} \d x,\qquad u,v\in\rmC_0^\infty(\Omega)
\end{equation}
zu einem gegebenen Gebiet $\Omega\subset\R^n$ und für Koeffizienten $p_{\alpha,\beta}\in\rmC^\infty(\overline{\Omega})$ mit $p_{\beta,\alpha}(x)=\overline{p_{\alpha,\beta}(x)}$  betrachtet werden. Wir schreiben kurz $P(x,\D,\overline\D)[u]=P(x,\D,\overline\D)[u,u]$ und wenn klar ist, welche Form wir betrachten, nur $P[u,v]$ beziehungsweise $P[u]$. Zugeordnet zu einer solchen Form betrachten wir das Symbol
\begin{equation}
   P(x,\zeta,\overline\zeta) = \sum_{\alpha,\beta} p_{\alpha,\beta}(x) \zeta^\alpha\overline\zeta^\beta,\qquad \zeta\in\C^n,
\end{equation}
und das zugeordnete Hauptsymbol
\begin{equation}
   p(x,\zeta,\overline\zeta) = \sum_{|\alpha|=|\beta|=m} p_{\alpha,\beta}(x) \zeta^\alpha\overline\zeta^\beta,\qquad \zeta\in\C^n.
\end{equation}
Auf Grund der Symmetriebedingung ist das Symbol $P(x,\xi,\xi)$ reellwertig f\"ur alle $\xi\in\R^n$, ebenso ist 
$P(x,\D,\overline \D)[u]$ reell. Spezialfälle solcher Dirichletformen sind die Innenprodukte des $\rmH^m_0(\Omega)$. Für diese schreiben wir kurz
\begin{equation}
  \spro{u}{v}_{(m)} = \sum_{|\alpha|\le m} \spro{\D^\alpha u}{\D^\alpha v} = \sum_{|\alpha|\le m} \int_\Omega   \big(D^\alpha u(x)\big) \overline{\big(\D^\alpha v(x)\big)} \d x.
\end{equation}
Es stellt sich die Frage, unter welchen Voraussetzungen eine Dirichletform äquivalent zu einem solchen Innenprodukt ist. Gilt $a_{\alpha,\beta}\in \rmB^\infty(\Omega)$, so ergibt sich stets eine obere Schranke der Form
\begin{equation}
    P[u] \le C \spro{u}{u}_{(m)}.
\end{equation}
Untere Schranken sind komplizierter. Wir nennen die Form (gleichmäßig) elliptisch, falls 
\begin{equation}
  \inf_{x\in\Omega}  \inf_{|\xi|=1} |p(x,\xi,\xi)| > 0.
\end{equation}
Unter dieser Voraussetzung zeigen wir, dass es Konstanten $a,b\in\R$ gibt, für welche
\begin{equation} 
   \spro{u}{u}_{(m)} \le a\, P[u] + b\, \spro{u}{u}
\end{equation}
gilt. Dies ist dazu äquivalent, dass der Quotient $P [u]  / \spro{u}{u}_{(m)}$ nach unten beschränkt ist.

\begin{thm}[{\cite[Theorem 2.1]{Garding:1953}}]
Sei $P[u]$ eine gleichmäßig elliptische Dirichletform der Ordnung $m$ auf einem beschränkten Gebiet $\Omega\Subset\R^n$. Dann gilt
\begin{equation}
  \inf_{u\in\rmH_0^m(\Omega)} \frac{P[u]}{\spro{u}{u}_{(m)}} > -\infty.
\end{equation}
\end{thm}
\begin{proof}
{\sl Schritt 1.} Wir beginnen mit dem Spezialfall, dass $P(x,\xi,\xi) = P(\xi)$ unabhängig von $x$ ist. Dann folgt mit dem Satz von Plancherel
\begin{equation}
 P[u] =  \sum_{|\alpha|,|\beta|\le m} \int_\Omega  p_{\alpha,\beta} \big(\D^\alpha u(x)\big)\overline{\big(\D^\beta u(x)\big)} \d x
 =   \sum_{|\alpha|,|\beta|\le m} \int  p_{\alpha,\beta} \xi^\alpha \xi^\beta  |\widehat u(\xi)|^2 \d\xi 
\end{equation}
mit $p_{\alpha,\beta}\in\C$ den Koeffizienten der Form und für beliebiges $u\in\rmC_0^\infty(\Omega)$. Da weiterhin die Elliptizitätsbedingung $p(\xi)\ge \tilde c |\xi|^{2m}$ 
vorausgesetzt ist und $\Omega$ beschränkt ist, gilt
\begin{equation}
  P[u] \ge \tilde c \int |\xi|^{2m} |\widehat u(\xi)|^2 \d\xi  -  \tilde c' \int \langle \xi\rangle^{2m-1} |\widehat u(\xi)|^2 \d\xi  \ge  2c\; \spro{u}{u}_{(m)}
\end{equation}
mit einer geeigneten Konstanten $c$. $\bullet$\qquad{\sl Schritt 2.}  Wir führen den allgemeinen Fall auf den gerade gezeigten Spezialfall zurück. Dazu bezeichne im Folgenden
\begin{equation}
	w(\rho) = \sup\left\{| p_{\alpha,\beta}(x)-p_{\alpha,\beta}(y)| : |\alpha|, |\beta| \le m, | x-y| \le \rho\right\}.
\end{equation} 
Da nach Voraussetzung die Koeffizienten aus $\rmC^\infty(\overline\Omega)$ sind, sind sie insbesondere gleichmäßig stetig und somit gilt
$\lim_{\rho\to0} w(\rho)=0$.  Sei nun $x_0\in \Omega$ beliebig, $u\in \rmH_0^m(\Omega\cap \partial B_\rho(x_0))$ und bezeichne $P_0(x,\xi, \xi):=P(x_0,\xi,\xi)$ die Form mit in $x_0$ eingefrorenen Koeffizienten. Dann gilt
\begin{equation}
\begin{split}
\big| P[u]-P_0[u] \big| &=\left| \sum\limits_{|\alpha|, |\beta| \le m} \int_\Omega \left(p_{\alpha,\beta}(x)-p_{\alpha,\beta}(x_0)\right){\D^\alpha(u(x)}\overline{\D^\beta u(x))}\mathrm{d}x\right|\\
	&\le w(\rho) \sum\limits_{|\alpha|,|\beta| \le m} \spro{\D^\alpha u}{\D^\beta u} \le  w(\rho) \sum\limits_{|\alpha|,|\beta| \le m}\|\D^\alpha u\|\, \|\D^\beta u\| \le
	 w(\rho) \|u\|_{(m)}^2
\end{split}
\end{equation}
unter zweifacher Ausnutzung der Ungleichung von Cauchy--Schwarz. Also folgt unter Verwendung des oben betrachteten Spezialfalls
\begin{equation}
	P[u] \ge P_0[u] - \big| P[u] - P_0[u] \big| \ge \big( 2c - w(\rho) \big) \|u\|_{(m)}^2 \ge  c \| u\|_{(m)}^2
\end{equation} 
für obiges $c$ und alle $u\in \rmH_0^m(\Omega \cap \partial B_\rho(x_0))$,  $\rho$ hinreichend klein. Die Abschätzung ist gleichmäßig in $x_0\in\Omega$.
$\bullet$\qquad {\sl Schritt 3.} Da $\Omega$ beschränkt ist, finden wir zu jedem $\epsilon>0$ eine endliche Überdeckung $\Omega\subset\bigcup_{k=1}^N B_{{\epsilon}/{2}}(x_k)$ und eine dieser Überdeckung untergeordnete Zerlegung der Eins $\psi_k\in\rmC_0^\infty(B_{\epsilon/2}(x_k))$ mit
\begin{equation}
 \sum\limits_{k=1}^N |\psi_k(x)|^2 = 1,\qquad \ x\in \Omega.
\end{equation}
Sei nun $u\in\rmH_0^m(\Omega)$ beliebig. Dann gilt
\begin{equation}
\begin{split}
	P[u]&= \sum_{k=1}^N \sum\limits_{|\alpha|,|\beta|\le m} \int_\Omega |\psi_k(x)|^2p_{\alpha\beta}(x)\big(\D^\alpha u(x)\big)\overline{\big(\D^\beta u(x)\big)}\d x\\
	&=  \sum_{k=1}^N \bigg( \sum\limits_{|\alpha|,|\beta|\le m} \int_\Omega  p_{\alpha\beta}(x)\big(\D^\alpha \psi_k(x) u(x)\big)\overline{\big(\D^\beta \psi_k(x) u(x)\big)}\d x + R_k[u]\bigg),
\end{split}
\end{equation}
wobei der Restterm $R_k[u]$ eine Form der Ordnung $m$ mit verschwindendem Hauptsymbol\footnote{Zur Erinnerung: $r_k(x,\xi,\xi)=0$ heißt, daß im Integral nie ein Produkt von zwei Ableitungen der Ordnung $m$ steht.} ist. Es gilt also mit Cauchy--Schwarz $|R_k[u]|\le c_k \|u\|_{(m)} \|u\|_{(m-1)}$, während jeder Summand der äußeren Summe mit Schritt 2 abgeschätzt werden kann. Also folgt 
\begin{equation}\label{eq:6.16}
P[u] \ge \sum_{k=1}^N  \bigg( c \|\psi_k u\|_{(m)}^2 - c_k \|u\|_{(m)} \|u\|_{(m-1)} \bigg) .
\end{equation}
Weiter gilt 
\begin{equation}
   \|\psi_k u\|_{(m)}^2 =  \sum_{|\alpha|\le m} \int_\Omega |\D^\alpha (\psi_k(x) u(x)) |^2 \d x
   =   \sum_{|\alpha|\le m} \int_\Omega  |\psi_k(x)|^2  |\D^\alpha u(x) |^2 \d x + \tilde R_k[u]
\end{equation}
mit einer weiteren Form $\tilde R_k[u]$  der Ordnung $m$ und verschwindendem Hauptsymbol. Also gilt wiederum $|\tilde R_k[u]|\le b_k \|u\|_{(m)}\|u\|_{(m-1)}$ und  \eqref{eq:6.16} liefert mit $b= \sum_{k=1}^N (c_k + c b_k  )$
\begin{equation}\label{eq:6.18}
P[u] \ge c \|u\|_{(m)}^2 - b \|u\|_{(m)}\|u\|_{(m-1)} .
\end{equation}
Mit $\|u\|_{(m-1)}\le \|u\|_{(m)}$ folgt die Behauptung.
\end{proof}

\begin{rem} 
Durch Interpolation folgt aus \eqref{eq:6.18}
\begin{equation}
P[u] \ge c \|u\|_{(m)}^2 - \tilde b \|u\|_{(m-1/2)}^2
\end{equation}
mit den in Definiton~\ref{df:sob-norm} definierten Sobolevnormen. Da $\|u\|_{(m-1/2)} \lesssim \|u\|_{(m)}$ gilt, folgt insbesondere für jedes $\epsilon>0$
die Existenz einer Konstanten $K(\epsilon)>0$ mit
\begin{equation}
P[u] \ge -\epsilon \|u\|_{(m)}^2 - K(\epsilon) \|u\|_{(m-1/2)}^2.
\end{equation}
Es stellt sich die Frage, was mit $K(\epsilon)$ für $\epsilon\to0$ passiert. 
\end{rem}

\section{Die scharfe G\r{a}rdingungleichung}

Nun soll die gerade aufgeworfene Frage beantwortet werden. Wir betrachten dazu Pseudodifferentialoperatoren $A$ zu Symbolen $\sigma_A(x,\xi)$.

\begin{thm}[Hörmander, {\cite[Theorem 1.3.3]{Hormander:1966}}]
Sei $A\in {\rm OP}S^{2m}_{\rm loc}(\Omega\times\R^n)$ Pseudodifferentialoperator mit Hauptsymbol\footnote{Es gilt also $\sigma_A-a\in S^{2m-1}_{\rm loc}(\Omega\times\R^n)$.} $a(x,\xi)\ge0$. Dann gibt es  für alle $\omega\Subset \Omega$ ein $K>0$ mit
 \begin{equation}
    \Re \spro{Au}{u} \ge - K \|u\|_{(m-1/2)}^2 
 \end{equation}
 für alle $u\in\rmC_0^\infty(\omega)$.
\end{thm}

Es genügt, die Aussage für Operatoren der Ordnung Null zu beweisen. Wir folgen Lax und Nirenberg \cite{Lax:1966} und zeigen eine entsprechende Aussage für (matrixwertige) Operatoren. Zuerst benötigen wir einige Definitionen und Hilfsmittel:

\medskip\noindent
{\em Symbolklassen:} Für $ a\in S^0_{\rm comp}(\R^n\times\R^n)$ definieren wir die Normen
\begin{equation}
\begin{split}&	\vert \widehat{a}(\eta,\cdot)\vert_k := \sum\limits_{\vert \beta\vert \le k} \sup\limits_{\xi} \langle \xi\rangle^{\vert \beta\vert}\vert\D_\xi^\beta  \widehat{a}(\eta,\xi)\vert,\\
&   | a|_{k,\ell} := (2\pi)^{-n/2} \int \langle\eta\rangle^\ell   \vert \widehat{a}(\eta,\cdot)\vert_k \d \eta.
\end{split}
\end{equation}
Hier bezeichnet $\widehat a(\eta,\xi)$ wiederum die Fouriertransformierte des Symbols bezüglich $x$. Weiter sei $\mathcal{C}_{k,\ell}$ der Abschluß der Symbolklasse in der $ |\cdot |_{k,\ell}$-Norm.
Im folgenden werden Symbole matrixwertig sein.

\medskip\noindent
{\em Zerlegung der Eins:} Sei im folgenden $\Theta\in\rmC_0^\infty(\R^n)$ mit $\supp\Theta\subset [-3/4,3/4]^n$ sowie $\Theta>0$ auf $[-1/2,1/2]^n$. Dann sind die Funktionen
\begin{equation}
\varphi_k(x) = \Theta(x-k) \bigg(\sum_{j\in\Z^n} \Theta(x-j)^2\bigg)^{-1/2}
\end{equation}
Elemente von $\rmC_0^\infty(\R^n)$ mit $\sum_{k} \varphi_k(x)^2=1$ und es existiert für jeden Multiindex $\alpha\in\N_0^n$ eine Konstante $C_\alpha$ mit
\begin{equation}
    \sum_{k\in\Z^n} | \D^\alpha \varphi_k(x)|^2 \le C_\alpha.
\end{equation}
Weiterhin impliziert $x,y\in\supp\varphi_k$ stets $|x-y|\le 3\sqrt n/2$. Ausgehend von dieser Partition der Eins definieren wir
\begin{equation}
    \psi_k(\xi)= \varphi_k(\xi / \sqrt{|\xi|}), 
\end{equation}
so daß
\begin{equation}\label{eq:6:6.35} 
    \sum_{k\in\Z^n} \psi_k(\xi)^2 = 1\qquad\text{und}\qquad |\xi|^\alpha \sum_{k\in\Z^n} |\partial_\xi^\alpha \psi_k(\xi)|^2 \le C_\alpha
\end{equation}
gilt. Weiter impliziert $\xi,\eta\in\supp\psi_k$ stets $|\sqrt{|\xi|}-\sqrt{|\eta|}|\le C$ und damit 
\begin{equation}\label{eq:6:6.36}
 |\xi-\eta|\le C \langle\xi\rangle^{1/2} \qquad\text{für alle $\xi,\eta\in\supp\psi_k$}.
\end{equation}
Weiterhin nützlich ist
\begin{equation}\label{eq:6:6.37}
  \sum_{k\in\Z^n} |\psi_k(\xi)-\psi_k(\eta)|^2 \le C \frac{|\xi-\eta|^2}{ \langle\xi\rangle^{1/2} \langle\eta\rangle^{1/2}},
\end{equation} 
was nur für $|\xi-\eta|\le \langle\xi\rangle^{1/2}$ nichttrivial ist und dann aus \eqref{eq:6:6.35} für $|\alpha|=1$ per Integration über die Strecke von $\xi$ zu $\eta$ folgt.

\medskip\noindent
{\em Positiv semidefinite Matrizen etc.}
Sei $A\in\C^{d\times d}$ selbstadjungiert und positiv semidefinit. Dann betrachten wir die zugeordneten quadratischen Formen\footnote{D.h. $\cspro{u}{A}{v} = u^* A v$} $\cspro{v}A{v}$ auf $\C^d$ und  schreiben $A \preceq B$ für zwei solche Matrizen, falls 
\begin{equation}
   \forall_{v\in\C^d} \quad:\quad \cspro{v}A{v}\le \cspro{v}B{v}
\end{equation}
gilt. Eine erste Anwendung ist folgendes Lemma.

\begin{lem}[{\cite[Lemma 2.2]{Lax:1966}}]\label{Matrizenlemma}
	Seien $A,B\in\C^{d\times d}$ selbstadjungiert und es gelte $A\pm B \succeq 0$. Dann gilt
	\begin{equation}
		\forall_{v,w \in \mathbb{C}^d} \quad: \quad \vert \cspro{v}B{w} \vert \le \cspro{v}A{v}^{{1}/{2}}\cspro{w}A{w}^{{1}/{2}}. 
	\end{equation}
\end{lem}
\begin{proof}
	Nach Voraussetzung gilt
	\begin{equation}\begin{split}
		&\cspro{v+w}{A+B}{v+w} \ge 0,\\
		&\cspro{v-w}{A-B}{v-w} \ge 0.
		\end{split}
	\end{equation}
	Addieren der beiden Gleichungen liefert
	\begin{equation}
		\cspro{v}A{v}+2\Re \cspro{w}B{v} +\cspro{w}Aw \ge 0
	\end{equation}
	und damit nach Multiplikation von $w$ mit einer komplexen Zahl vom Betrag Eins (so dass $\Re \cspro{\lambda w}B{v} = - | \cspro{w}B{v}|$) die Behauptung. 
\end{proof}

\medskip\noindent
{\em Hilfslemma} Wir beginnen mit einer Hilfsaussauge, welche auch für sich genommen von Interesse sind. 

\begin{lem}[{\cite[Lemma 3.1]{Lax:1966}}]\label{lem3.1:Lax}
Sei $p\in \mathcal{C}_{0,2}$ eine von $\xi$ unabhängige selbstadjungierte Matrix und sei weiter $\Phi: \mathbb{R}^n\rightarrow\mathbb{R}$ Lipschitz-stetig mit Lipschitzkonstante $K$ auf $\supp \widehat{u}$. Dann gilt
\begin{equation}
		| \Re \spro {\Phi(\D)pu}{\Phi(\D)u}-\Re\spro{p\Phi(\D)u}{\Phi(\D)u}| \le\frac{1}{2}| p|_{0,2}K^2|| u||^2.
\end{equation}	 
\end{lem}
\begin{proof}
	Die Rechenregeln der Fouriertransformation liefern
	\begin{equation}\label{lem3.1:Lax;1}
		\begin{split} J:&=  \Re \spro {\Phi(\D)pu}{\Phi(\D)u}-\Re\spro{p\Phi(\D)u}{\Phi(\D)u} \\
		&= \left(2\pi\right)^{-\frac{n}{2}}\Re\iint\cspro {\widehat{u}(\xi)}{\widehat{p}(\xi-\eta)}{\widehat{u}(\eta)}\left(\Phi(\xi)-\Phi(\eta)\right)\Phi(\xi)\d \xi\d\eta.
		\end{split}
	\end{equation}
	Weil $p$ selbstadjungiert ist, folgt $\widehat{p}(\xi) = {\widehat{p}}(-\xi)^*$. Durch Vertauschen der Integrationsvariablen erhalten wir
	\begin{equation}\label{lem3.1:Lax;2}
	\begin{split}	
	J&= \left(2\pi\right)^{-\frac{n}{2}}\Re\iint \cspro{ \widehat{u}(\eta)}{\widehat{p}(\eta-\xi)}{\widehat{u}(\xi)}\left(\Phi(\eta)-\Phi(\xi)\right)\Phi(\eta)\d \xi\d\eta\\
		&= \left(2\pi\right)^{-\frac{n}{2}}\Re\iint\cspro{\widehat{u}(\xi)}{\widehat{p}(\xi-\eta)}{\widehat{u}(\eta)}\left(\Phi(\eta)-\Phi(\xi)\right)\Phi(\eta)\d \xi\d\eta.
	\end{split}	
	\end{equation}
	und Addition von $(\ref{lem3.1:Lax;1})$ und $(\ref{lem3.1:Lax;2})$ liefert
	\begin{equation}
		J= \frac{1}{2}\left(2\pi\right)^{-\frac{n}{2}}\Re\iint \cspro{\widehat{u}(\xi)}{\widehat{p}(\xi-\eta)}{\widehat{u}(\eta)}\left(\Phi(\xi)-\Phi(\eta)\right)^2\d\xi\d\eta.
	\end{equation}
	Mit  $\left(\Phi(\xi)-\Phi(\eta)\right)^2\le K^2 |\xi-\eta|^2$ und $| \widehat{p}(\xi-\eta,\cdot)|_{0,2} = (2\pi)^{-\frac{n}{2}}\int \langle\xi-\eta\rangle^2 |\widehat{p}(\xi-\eta)|\d\eta$ sowie der Ungleichung von Cauchy--Schwarz folgt die Behauptung. 
\end{proof}

Damit kommen wir zum Hauptresultat dieses Abschnitts, der scharfen G\r{a}rdingungleichung für Systeme nach Lax und Nirenberg.

\begin{thm}[{\cite[Theorem 3.1]{Lax:1966}}]
Sei $a\in\mathcal{C}_{0,2}\cap\mathcal{C}_{2,0}$ selbstadjungiert und positiv-semidefinit. Dann erfüllt der zugehörige Operator $A$ die Ungleichung
\begin{equation}
	\Re \spro{Au}{u} \ge - K|| u||_{\left(-{1}/{2}\right)}^2.
\end{equation}
\end{thm}
\begin{proof}
	Wir machen Gebrauch von der zu Beginn des Kapitels konstruierten Zerlegung der Eins: Sei $	u_k:= \psi_k(\D)u, \text{ d.h. } \widehat{u_k}(\xi) = \psi_k(\xi) \widehat{u}(\xi).$ Wegen 
\begin{equation}
	\widehat{Au}(\xi) = \left(2\pi\right)^{-\frac{n}{2}}\int \widehat{a}(\xi-\eta,\eta)\widehat{u}(\eta)\d \eta
\end{equation}	 
folgt
\begin{equation}
\begin{split} I:&=\spro{Au}{u}-\sum\limits_k \spro{Au_k}{u_k} = \left(2\pi\right)^{-\frac{n}{2}}\iint \cspro{\widehat{u}(\xi)}{\widehat{a}(\xi-\eta,\eta)}{\widehat{u}(\eta)}\left(1-\sum\limits_k \psi_k(\eta)\psi_k(\xi)\right)\d \xi\d\eta\notag\\
	&= \frac{1}{2}\left(2\pi\right)^{-\frac{n}{2}} \iint \cspro{\widehat{u}(\xi)}{\widehat{a}(\xi-\eta,\eta)}{\widehat{u}(\eta)}\sum\limits_k\vert\psi_k(\eta)-\psi_k(\xi)\vert^2\d \xi\d\eta.
\end{split}
\end{equation}
Damit folgt mit \eqref{eq:6:6.37} und erneutem Anwenden von Cauchy--Schwarz
\begin{equation}
\begin{split}	
| I| &\le \frac{C}{2}\left(2\pi\right)^{-\frac{n}{2}}\iint| \cspro{\widehat{u}(\xi)}{\widehat{a}(\xi-\eta,\eta)}{\widehat{u}(\eta)}| \, |\xi-\eta|^2\langle \xi\rangle^{-\frac{1}{2}} \langle\eta\rangle^{-\frac{1}{2}}\d \xi \d \eta\\ 
& \le \frac{C}{2}| a|_{0,2}|| u||_{(-{1}/{2})}^2.
\end{split}
\end{equation}
Damit ist es ausreichend, die gewünschte Ungleichung für die Funktionen $u_k$ zu zeigen. Wir halten ein beliebiges $k$ fest und setzen $v:= u_k.$ Sei $\tau\in \supp \widehat{v}$ fixiert. Für $\mu, \eta \in \supp \widehat{v}$ gilt mit \eqref{eq:6:6.36}
\begin{equation}
	1+|\mu|^2 \le 1+\left(| \eta|+| \eta-\mu|\right)^2 \le 1+|\eta|^2+ c_1 \langle \mu \rangle,
\end{equation}
also
\begin{equation}
	\langle \mu\rangle \le c_2\langle \eta\rangle
\end{equation}
und somit
\begin{equation}\label{Abschaetzung 0-Norm}
	\langle \tau\rangle^{-1}|| v||^2 \lesssim|| v||_{\left(-{1}/{2}\right)}^2
\end{equation}
Wir setzen $\widehat{a}^l(\xi-\eta,\tau):=\frac{\partial}{\partial \tau_l}\widehat{a}(\xi-\eta,\tau)$ und wenden darauf den Mittelwertsatz an. Für $\eta\in \supp \psi_k$ ergibt sich dann
\begin{equation}
\begin{split}	&| \widehat{a}(\xi-\eta,\eta)-\widehat{a}(\xi-\eta,\tau) -\sum\limits_{l=1}^n (\eta_l-\tau_l)\widehat{a}^l\left(\xi-\eta,\tau\right)| \lesssim| \eta-\tau|^2| \widehat{a}(\xi-\eta,\cdot)|_2\langle\eta\rangle^{-2} \\&
\lesssim | \widehat{a}(\xi-\eta,\cdot)|_2 \langle\eta\rangle^{-1} \lesssim| \widehat{a}(\xi-\eta,\cdot)|_2\langle\eta\rangle^{-\frac{1}{2}}\langle\xi\rangle^{-\frac{1}{2}}.
\end{split}
\end{equation}
Setzen wir $p(x):=a(x,\tau),\ p^l(x):= a^l(x,\tau) $ und $v^l:=(\D_l-\tau_l)v$ (also $\widehat{v}^l=\left(\xi_l-\tau_l\right)\widehat{v}$), dann folgt
\begin{equation}
	|\spro{Av}{v}-\spro{pv}{v}-\sum\spro{p^lv^l}{v}| \le c | a|_{2,0}|| u||_{\left(-{1}/{2}\right)}^2
\end{equation}
Damit verbleibt es, die Ungleichung \begin{equation}
	Q[v] \ge -C|| v||_{\left(-{1}/{2}\right)}^2
\end{equation}
für die quatratische Form
\begin{equation}
	Q[v]=\spro{pv}{v}+\Re\sum\spro{p^lv^l}{v} 
\end{equation}
zu zeigen. Für jedes $l\in\{1,\cdots n\}$ gilt für $\xi = \pm c\langle\tau\rangle^{\frac{1}{2}}\e_l$ mit einem später geschickt gewählten $c$, dass
\begin{equation}
	a(x,\xi+\tau) = a(x,\tau)\pm\langle\tau\rangle^{\frac{1}{2}}a^l(x,\tau) + R(x,\xi,\tau). 
\end{equation}
Weil $a(x,\xi+\tau)$ positiv definit ist, ist für eine geeignete Konstante $C$ auch
\begin{equation}
	a(x,\tau) \pm c\langle\tau\rangle^{\frac{1}{2}}a^l(x,\tau)+C\langle\tau\rangle \frac{| a(x,\cdot)|_{2}}{\langle\tau\rangle^2}\succeq 0
\end{equation}
positiv definit. Weil Addieren eines Terms der Form $c \langle\tau\rangle^{-1} I_n$ zu $p(x)$ gemäß \eqref{Abschaetzung 0-Norm} einen Fehler der Ordnung $\mathcal O\left(\vert\vert v\vert\vert_{\left(-\frac{1}{2}\right)}^2\right)$ verursacht, können wir ohne Einschränkung annehmen, dass  
\begin{equation}
	p(x) \pm \langle\tau\rangle^{\frac{1}{2}}p^l(x) \succeq 0
\end{equation}
positiv definit ist. Wir wenden darauf Lemma \ref{Matrizenlemma} an und erhalten
\begin{equation}
	c\langle\tau\rangle^{\frac{1}{2}} |\spro{p^lv}{v}|\le\spro{pv}{v}^{\frac{1}{2}}\spro{pv^l}{v^l}^{\frac{1}{2}}.
\end{equation}
Unter Ausnutzen der Ungleichung $\sqrt{ab}\le \frac{n}{2}a+\frac{1}{2n}b$ folgt
\begin{equation}
	|\spro{p^lv^l}{v}|\le \frac{1}{2n}\spro{pv}{v}+\frac{n}{2c^2}\frac{1}{\langle\tau\rangle}\spro{pv^l}{v^l},
\end{equation}
was
\begin{equation}
	Q[ v]=\spro{pv}{v}+\Re\sum\spro{p^lv^l}{v} \ge \frac{1}{2}\spro{pv}{v} -\frac{n}{2c^2}\frac{1}{\langle\tau\rangle}\sum\spro{pv^l}{v^l}
\end{equation}
impliziert. Lemma \ref{lem3.1:Lax}, angewandt auf $\Phi(\xi) = \xi_l-\tau_l$, liefert in Verbindung mit $(\ref{Abschaetzung 0-Norm})$
\begin{equation}
\sum\spro{pv^l}{v^l}=\Re\sum \spro{pv}{\left(\D_l-\tau_l\right)v}+\mathcal{O}\left(|| v||_{\left(-{1}/{2}\right)}^2\right) 
\end{equation}
und damit
\begin{equation}
	Q[v] \ge \frac{1}{2}\spro{pv}{v}-\frac{n}{2c^2}\frac{1}{\langle\tau\rangle}\Re\spro{pv}{| D-\tau|^2v)}+\mathcal{O}\left(|| v||_{\left(-{1}/{2}\right)}^2\right),
\end{equation}
was wir äquivalent ausdrücken durch
\begin{equation}
	2Q[v] \ge \Re\spro{pv}{\left(1-\frac{n}{c^2}\frac{| \D-\tau|^2}{\langle\tau\rangle}\right)v}+\mathcal{O}\left(|| v||_{\left(-{1}/{2}\right)}^2\right).
\end{equation}
Auf $\supp\widehat{v}$ gilt
\begin{equation}
	\frac{|\xi-\tau|^2}{\langle\tau\rangle}\le \tilde{C}^2 
\end{equation}
für eine Konstante $\tilde{C}>0$. Setzen wir nun $c:=\tilde{C}\sqrt{2n}$ und 
\begin{equation}
	\Phi(\xi):=\begin{cases}
		\left(1-\frac{n}{c^2}\frac{| \xi-\tau|^2}{\langle\tau\rangle}\right)^{\frac{1}{2}}, &| \xi-\tau| \le \tilde C(\langle\tau\rangle)^{\frac{1}{2}},\\
		\sqrt{\frac{1}{2}}, & \text{sonst},
	\end{cases}
\end{equation}
dann ist $\Phi(\xi)$ Lipschitz-stetig mit Lipschitz-Konstante $K=\mathcal O(\langle\tau\rangle)^{-\frac{1}{2}}$, erneute Anwendung von Lemma \ref{lem3.1:Lax} liefert
\begin{equation}
	2Q \ge \Re\spro{pv}{\Phi(\D)^2v}+\mathcal{O}\left(|| v||_{\left(-{1}/{2}\right)}^2\right) = \Re\spro{p\Phi(\D)v}{\Phi(\D)v}+\mathcal{O}\left(|| v||_{\left(-{1}/{2}\right)}^2\right).
\end{equation}
Weil $a(x,\tau)$ positiv definit ist, folgt $\Re\spro{p\Phi(\D)v}{\Phi(\D)v}\ge 0$ und damit die gewünschte Ungleichung. 
\end{proof}

  % Gardingsche Ungleichung :: Simon Barth
% !TEX root = main.tex
\chapter{Die Ungleichung von Melin}
%\cite{Melin:1971}

Die Ungleichung von Melin verallgemeinert die G\r{a}rdingungleichung. Während letztere für Operatoren $A\in\mathrm{OP}S^{2m}_{\rm unif}(\R^n\times\R^n)$ mit nichtnegativem Hauptsymbol untere Schranken der Form 
\begin{equation}
\forall_{v\in\mathscr S(\R^n)}\quad:\quad  \Re\spro{Av}{v} + \epsilon \|v\|^2_{(m)} \ge C(\epsilon) \|u\|_{(m-1/2)}^2
\end{equation}
für jedes $\epsilon>0$ und zugehörige $C(\epsilon)\in\R$ liefert, fragen wir nun nach Bedingungen an $A$, so dass diese Abschätzungen eine halbe Ordnung besser, also als
\begin{equation}
  \forall_{v\in\mathscr S(\R^n)}\quad:\quad   \Re\spro{Av}{v} + \epsilon \|v\|^2_{(m-1/2)} \ge C(\epsilon) \|u\|_{(m-1)}^2
\end{equation}
gelten. Die zu nutzende Grundidee ist ähnlich zum Beweis der scharfen G\r{a}rdingungleichung durch Hörmander \cite{Hormander:1966} beziehungsweise durch Lax und Nirenberg
\cite{Lax:1966}.

\section{Eine spezielle Klasse quadratischer Dirichletformen}
Wir betrachten zuerst Dirichletformen der speziellen Struktur
\begin{equation}
   P[v] = \Re  \int \overline{v(x)}p(x,\D)v(x)\d x 
\end{equation}
für ein Polynom zweiten Grades
\begin{equation}
    p(x,\xi) = \sum_{|\alpha+\beta|\le 2} \frac{ a_{\alpha,\beta}}{\alpha!\beta!} x^\beta \xi^\alpha,\qquad a_{\alpha,\beta}\in\C
\end{equation}
und Funktionen $v\in\mathscr S(\R^n)$. Diese speziellen Formen werden später als Taylorapproximanten auftreten und sollen zuerst untersucht werden. Verschiedene Wahlen der Koeffizienten $a_{\alpha,\beta}$ können dabei dieselbe Form erzeugen, wir wählen die Koeffizienten im folgenden reell.

Zugeordnet zu $P$ betrachten wir die (reelle) quadratische Form
\begin{equation}
  h(x,\xi) = \Re  \sum_{|\alpha+\beta|= 2} \frac{a_{\alpha,\beta}}{\alpha!\beta!} x^\beta\xi^\alpha
\end{equation}
auf $\R^n\times\R^n$ und die symmetrische Grammatrix $H$ mit $\cspro{X}{H}{X}=h(X)$, $X=(x,\xi)\in\R^n\times\R^n$.  Das folgende Lemma legt nahe, die Dirichletform $P$ mit Hilfe der kanonischen symplektischen Struktur\footnote{Wir versehen den $\R^{2n}$ mit der symplektischen Form
$\varsigma(X,Y) = x\cdot \eta - \xi\cdot y$, wobei $X=(x,\xi)$ und $Y=(y,\eta)$. Eine Matrix aus $\R^{2n\times 2n}$ heißt symplektisch, falls sie die symplektische Form erhält. Invertierbare symplektische Matrizen bilden die symplektische Gruppe $\mathrm{Sp}_{2n}(\R)$.} auf $\R^n\times\R^n$ zu untersuchen.
\begin{lem}
Es sei $p$ ein Polynom zweiten Grades auf $\R^n\times\R^n$ und $\chi\in\mathrm{Sp}_{2n}(\R)$ ein Element der symplektischen Gruppe. Dann gilt
\begin{align}
\forall_{v\in\mathscr S(\R^n)} \quad:\quad \Re \int \overline{v(x)}p(x,\D)v(x)\d x\ge 0
\end{align}
genau dann für das Polynom $p$, wenn es für das Polynom $p\circ \chi$ gilt.
\end{lem}
%\begin{proof}[Beweisidee]
%Es genügt, dies für spezielle Familien von Matrizen, welche die symplektische Gruppe erzeugen, nachzurechnen.
%\end{proof}
Im folgenden bezeichne
\begin{align}
J=\begin{pmatrix}
0 & I\\
-I& 0 
\end{pmatrix},\qquad\text{ $ I \in\R^{n\times n}$  Einheitsmatrix }
\end{align}
die symplektische Matrix. Dann ist die symplektische Form gerade durch $\cspro{X}{J}{Y}$ gegeben. Für eine reelle symmetrische Matrix $M$ hat  die Matrix $JM$ ein symmetrisch um den Ursprung angeordnetes, rein imaginäres Spektrum. Die positive Spur von $H$ werde durch
\begin{equation}
   {\mathrm{Tr}}^+ H = \sum_j\mu_j 
\end{equation}
bezeichnet, wobei die Folge $(\lambda_j)$ eine ihrer Vielfachheit entsprechenden Aufzählung der positiven Eigenwerte von $\i JM$ ist. Wir stellen fest, dass die positive Spur
der Matrix $H$ unter symplektischen Transformationen der quadratischen Form $h$ invariant ist. Also gilt für alle $\chi\in\mathrm{Sp}_{2n}(\R)$
\begin{align}
\mathrm{Tr}^+(\chi^\mathrm{T}H\chi) = \mathrm{Tr}^+ H\,.
\end{align}

Der nachfolgende Satz ist von entscheidender Bedeutung um notwendige und hinreichende Bedingungen für die Ungleichung von Melin zu formulieren.

\begin{thm}[{\cite[Theorem 2.4.]{Melin:1971}}]\label{satz:thm2.4}
Es gilt $P[u]\ge0$ genau dann für alle $u\in\mathscr S(\R^n)$, wenn die Bedingungen
\begin{enumerate}
\item  $h(x,\xi)\ge0$ auf $\R^n\times\R^n$;
\item $F \cdot X = \sum_{|\alpha+\beta|=1} \Re a_{\alpha,\beta} x^\beta\xi^\alpha = 0$ für $X=(x,\xi)\in\R^{2n}$ mit $h(x,\xi)=0$;
\item $\Re a_{0,0} - \frac12 \sum_{|\beta|=1} \Im a_{\beta,\beta} -\frac14  \cspro{F}{H^{-1}}{F} + {\mathrm{Tr}}^+ H \ge 0$
\end{enumerate}
erfüllt sind.
\end{thm}
\begin{proof}[Beweisidee]
Die Grundidee des Beweises besteht in einer geeigneten symplektische Transformation auf Normalform. Wir nehmen vorerst an, dass $JH$ invertierbar ist. Dann existiert eine symplektische Basis des $\R^n\times\R^n$, so dass $(JH)^{-1}\e_j = - \lambda_j f_j$ und $(JH)^{-1} f_j = \lambda_j e_j$ und damit ein $\chi\in\mathrm{Sp}_{2n}(\R)$ mit
\begin{equation}
    a_0^0 + F\cdot \chi X + h(\chi X) = a_0^0 + \sum_{j=1}^n \lambda_j^{-1} (t_j x_j  + \tau_j\xi_j + x_j^2 + \xi_j^2 ),\qquad X=(x,\xi)\in\R^n\times\R^n
\end{equation}
und für $H^{-1} F = (t,\tau)\in\R^n\times\R^n$. Es bleibt also die Ungleichung
\begin{equation}
  \Re \int \overline{v(x)} \bigg( a_0^0 + \sum_{j=1}^n \lambda_j^{-1} \big(  t_j x_j +\tau_j \D_j +x_j^2 + \D_j^2\big)\bigg) v(x) \d x \ge 0
\end{equation}
zu zeigen. Ersetzt man $v$ durch $\e^{-\i t \cdot x /2 } v(x)$ und analog im Fourierbild, so entspricht dies sich unter Ausnutzung von $\sum_j \lambda_j^{-1} (t_j^1 + \tau_j^2)= \cspro{F}{H^{-1}}{F}$ gerade
\begin{equation}
   (a_0^0 - \frac14 \cspro{F}{H^{-1}}{F}) \int |v(x)|^2 \d x + \sum_{j=1}^n  \int \overline{v(x)} \lambda_j^{-1} (x_j^2 + \D_j^2) v(x) \d x \ge 0.
\end{equation} 
Um dies zu zeigen, multiplizieren wir die offensichtliche Ungleichung
\begin{equation}
  0 \le \int |(\D - \i  y)u(y)|^2 \d y
\end{equation}
für $u\in\rmC_0^\infty(\R)$ aus und erhalten
\begin{equation}
    \int \overline{u(y)} (\D^2 + y^2) u(y) \d y \ge  \int |u(y)|^2 \d y,
\end{equation}
also nach Addition von $n$ dieser Ungleichungen
\begin{equation}
 \sum_{j=1}^n  \int \overline{v(x)} \lambda_j^{-1} (x_j^2 + \D_j^2) v(x) \d x
 \ge (\sum \lambda_j^{-1}) \int |v(x)|^2 \d x
 \end{equation}
 und damit die Behauptung. Für die Rückrichtung nutzt man $v(x) = \e^{-|x|^2/2}$. Für den allgemeinen Fall ersetzt man $H$ durch $H+\epsilon I$ und lässt $\epsilon\to0$ streben.
\end{proof}


\section{Die Ungleichung von Melin}
Im Folgenden bezeichnen $a_k\in S^{k}_\mathrm{loc}(\Omega\times\R^n\setminus\{0\})$ die homogenen Entwicklungsterme der Entwicklung $a\sim\sum_{j=0}^\infty a_{m-j}$ des Symbols $a$ von $A\in\mathrm{OP}S^{m}_{\rm loc}(\Omega\times\R^n)$. Wir formulieren das Hauptresultat dieses Abschnittes. 

\begin{thm}[{\cite[Theorem 3.1]{Melin:1971}}]
Sei $A\in\mathrm{OP}S^{m}_{\rm loc}(\Omega\times\R^n)$ mit homogenem Hauptsymbol $a_m(x,\xi)\ge0$ und $\mu=(m-1)/2$. Dann sind \"aquivalent:
\begin{enumerate}
\item F\"ur jedes $\epsilon>0$, jedes $K\Subset\Omega$ und jedes $s\in\R$ existiert ein $C$, so dass
\begin{equation}
    \Re \spro{Au}{u} + \epsilon \|u\|^2_{(\mu)}  \ge -C\|u\|_{(s)}^2\label{ungleichungvonmelin}
\end{equation}
für alle $u\in\rmC_0^\infty(K)$.
\item
Für alle $(x,\xi)\in\Omega\times\R^n$ mit $a_m(x,\xi)=0$ gilt
\begin{equation}\label{bedfuerungleichung}
   \Re a_{m-1} +  \frac 1 2 {\mathrm{Tr}}^+ H_{a_m} \ge 0,
\end{equation}
wobei $H_{a_m}$ die Hessematrix zu $a_m$ bezeichne.
\end{enumerate}
\end{thm}

Wir können $A$ durch einen eigentlich getragenen Pseudodifferentialoperator mit gleichen homogenen Symbolkomponenten ersetzen.  Weiter genügt es, die Ungleichung für Operatoren der Ordnung $m=1$ zu zeigen. Sei dazu $\Lambda_\rho\in\mathrm{OP}S^\rho_{\rm loc}(\Omega\times\R^n)$ ein eigentlich getragener Pseudodifferentialoperator auf $\Omega$ mit Symbol $\braket\xi^\rho$.
Der Operator $\Lambda_\rho$ besitzt auf $\rmC_0^\infty(\Omega)$ die Parametrix\footnote{Inverse modulo $\mathscr E'(\Omega)\to\rmC^\infty(\Omega)$}  $\Lambda_{-\rho}$, ist $\rmL^2$-symmetrisch und erfüllt für jedes $s\in\R$ mit geeigneten Konstanten
\begin{align}
   C_1 \|u\|_{(m-\rho)} - C'_1 \|u\|_{(s)} \le    \|\Lambda_\rho u\|_{(m)} \le   C_2 \|u\|_{(m-\rho)}.
\end{align}
Es sei nun $A_\rho:=\Lambda_{-\rho} \circ A\circ\Lambda_{-\rho}$. Wenn man in \eqref{ungleichungvonmelin} $u$ durch $\Lambda_{-\rho} u$ ersetzt, so wird ersichtlich, dass die Ungleichung für $A$ und $\mu=(m-1)/2$ genau dann erfüllt ist, wenn sie für $A_\rho$ mit $\mu=(m-\rho-1)/2$ und $s$ ersetzt durch $s-\rho$ erfüllt ist. Desweiteren gilt $A_\rho\in \mathrm{OP}S^{m-2\rho}(\Omega\times\R^n)$. 
Da auch \eqref{bedfuerungleichung} genau dann für $A$ gilt, wenn die Bedingung für $A_\rho$ gilt, können wir annehmen, dass $A\in\mathrm{OP}S^1_{\rm loc}(\Omega\times\R^n)$.


\section{Beweis der Hinrichtung}

Es sei $(x_0,\xi_0)\in\Omega\times\R^n$ mit $a_1(x_0,\xi_0)=0$ und $K\Subset \Omega$ so gewählt, dass  $(x_0,\xi_0)\in \overset{\circ} {K}\times\R^n$. Wir können ohne Beschränkung der Allgemeinheit annehmen, dass $x_0=0$. 
Nun sei 
\begin{align}
 v\in\mathrm C^\infty_0(\R^n) \quad \text{und} \quad v_\lambda(x):=\lambda^{n/2}v(\lambda x)\, \e^{\i\lambda^2x\cdot \xi_0}\, ,\quad\lambda>0
\end{align}
Es gilt $\|v\|=\|v_\lambda\|$ und $\supp v_\lambda \subset K$ für genügend große $\lambda$. Außerdem folgt mit entsprechenden Koordinatentransformationen
\begin{align}
\spro {A v_\lambda} {v_\lambda} = (2\pi)^{-n/2}\iint \e^{\i x\cdot \xi} a(x/\lambda,\lambda^2\xi_0 + \lambda\xi)\widehat v(\xi)\cc{v(x)}\d\xi\d x\,.
\end{align}
Da $a_1(0,\xi)=0$ ein Minimum  ist, verschwindet auch der Gradient von $a_1$ an dieser Stelle. Es gilt
\begin{align}\label{oentwrueck}
\braket{\lambda^2\xi_0 + \lambda\xi}^{-1} = \mathcal O(1)\frac{\braket{\xi}}{\lambda}\,.
\end{align}
Dabei stellt $\mathcal O(1)$ eine in $\xi$ und $\lambda\in[1,\infty)$ beschränkte Funktion dar. Nun soll eine Taylorentwicklung von $a$ um den Entwicklungspunkt $(0,\xi_0)$ vorgenommen werden. Diese ist (mit $a^\alpha_\beta(x,\xi) = \partial_\xi^\alpha \partial_x^\beta a(x,\xi)$)
\begin{align}
a(x/\lambda, \lambda^2\xi_0+\lambda\xi) = a_0(0,\xi_0) + \sum_{\abs{\alpha+\beta}=2} {a_1}^\alpha_\beta(0,\xi_0)\frac{x^\beta\xi^\alpha}{\alpha!\beta!} + \mathcal O(1) \lambda^{-1}\braket{\xi}^4\,.
\end{align}
Wird dies nun in die Ungleichung von Melin mit $u=v_\lambda$ und $s=-1$ eingesetzt, so folgt
\begin{align}
\Re\int \cc{v(x)}\sum_{\abs{\alpha+\beta}=2}\frac{{a_1}^\alpha_\beta(0,\xi_0)}{\alpha!\beta!}x^\beta\D^\alpha v(x)\d x + (\epsilon +\Re a_0(0,\xi_0))\|v\|_{(0)}^2 + \mathcal O(1)\lambda^{-1}\ge C\|v_\lambda\|_{(-1)}^2
\end{align}
Wegen $\widehat{v}_\lambda(\xi)=\widehat{v}(\xi/\lambda-\xi_0\lambda)$ und \eqref{oentwrueck} folgt außerdem
\begin{align}
\|v_\lambda\|_{(-1)} = \int \braket{\lambda^2\xi_0+\lambda \xi}^{-2}|\widehat{v}(\xi)|\d\xi = \mathcal{O}(1)\lambda^{-1} 
\end{align}
Mit $\lambda\to\infty$ und anschließend $\epsilon \to 0$, bekommen wir somit
\begin{align}
\Re\int \cc{v(x)}\sum_{\abs{\alpha+\beta}=2}\frac{{a_1}^\alpha_\beta(0,\xi_0)}{\alpha!\beta!}x^\beta\D^\alpha v(x)\d x + \Re a_0(0,\xi_0)\int|v(x)|^2\d x\ge 0
\end{align}
Die symmetrische Matrix $H$, die die obige quadratische Form erzeugt, hat die Koeffizienten $1/2\,\Re {a_1}^\alpha_\beta(0,\xi_0)$. Mit $H=1/2\, H_{a_1}(0,\xi_0)$ und Satz \ref{satz:thm2.4} folgen nun die Eigenschaften \eqref{bedfuerungleichung}.

\section{Beweis der Rückrichtung}

Um den Beweis durchzuführen werden wir die im letzten Abschnitt eingeführte Zerlegung der Eins $(\phi_k)$ und $(\psi_k)$ benutzen. Zusätzlich wird die Folge $(\xi^j)\subset \R^n$ so gewählt, dass $0\neq \xi^j\in\supp \psi_j$.

Wir approximieren $A$ durch eine Taylorentwicklung des Symbols und die zugehörigen Differentialoperatoren
\begin{align}
 A_j^\delta v (x):=\sum_{\abs \alpha \le 2} \frac{a^\alpha(x,\xi^j/\delta)}{\alpha!}\left( \mathrm D - \xi^j/\delta\right)^\alpha v(x)
\end{align}
für $v\in\mathrm {C}^\infty(\R^n)$ und $\delta\in(0,1]$. Hierfür zunächst einige Vorüberlegungen. Die Ungleichung \eqref{ungleichungvonmelin} verändert sich nicht, wenn das Symbol außerhalb einer kompakten Menge $K$ verändert wird. Sei nun $\psi\in\mathrm{C}^\infty_0(\R^n)$ mit $\psi(x)\in[0,1]$ für alle $x\in\R^n$ und $\psi(x)=1$ für alle $x\in k$. Wir können jetzt $a(x,\xi)$ durch $(1-\psi(x))\braket{\xi} + \psi(x)a(x,\xi)$ ersetzen ohne \eqref{ungleichungvonmelin} zu ändern. Deshalb sei nun $a$ außerhalb einer Kompakten Menge durch $\braket{\xi}$ gegeben. Somit ist $a\in S_\mathrm{unif}^1(\R^n\times\R^n)$ und $\spro{Au}{u}$ stetig bezüglich $\norm\cdot _{(s)}$ für $s\ge 1/2$. Die quadratische Form der Terme negativer Ordnung von $A$ sind aus dem gleichen Grund bezüglich $\norm \cdot _{(-1/2)}$ beschränkt. Es reicht also $s\in[-1/2, 1/2]$ zu betrachten und dabei $a=a_1+a_0$ anzunehmen.

\begin{lem}
Sei $P$ Pseudodifferentialoperator mit Symbol $\sigma\in S^1_{\rm comp}(\R^n\times\R^n)$. Dann gilt
\begin{align}\label{hilfsungl1}
\abs {
\spro {P u}  u - \sum_{j\in\Z^n}\spro{ P_j^\delta\psi_j(\delta\mathrm D )u} { \psi(\delta\mathrm D )u }
}
\le C\delta \Norm u ^2_{(0)} + C_\delta \Norm u ^2_{(s)}\ ,\quad u\in\mathrm {C}^\infty_0(\R^n)\ .
\end{align}
\end{lem}

\begin{proof}
Es sei $u\in\mathrm C^\infty_0(\R^n)$. Da das Symbol $\sigma$ in $x$ kompakt getragen ist, existiert seine Fouriertransformierte $\widehat\sigma (\eta,\xi)$ bezüglich $x$. Da außerdem $ P u\in\mathrm C^\infty_0(\R^n)$ gilt, folgt
\begin{align}
\begin{split}
\widehat{ P u}(\eta) &= (2\pi)^{-n}\int \e^{-\i\eta\cdot x} \int e^{\i\xi\cdot x}\sigma(x,\xi) \widehat u (\xi) \d\xi\d x = (2\pi)^{-n/2}\int \widehat\sigma(\eta - \xi,\xi)\widehat u(\xi)\d\xi \\
&= (2\pi)^ {-n/2}\int \sum_{j\in\Z^n} \frac 1 2\left[(\psi_j(\delta \xi) - \psi_j(\delta \eta))^2 + 2 \psi_j(\delta\xi)\psi(\delta \eta)\right] \widehat\sigma(\eta - \xi,\xi)\widehat u(\xi)\d\xi\,.
\end{split}
\end{align}
Desweiteren gilt $\widehat{\partial^\alpha_\xi\sigma(\cdot,\xi)} = \partial_\xi^\alpha\widehat \sigma(\cdot,\xi)$ und somit
\begin{align}
\mathcal{F}\left[\sigma^\alpha(\cdot,\xi^j/\delta)\left( \mathrm D - \xi^j/\delta\right)^\alpha\psi_j(\delta\mathrm D)u\right](\eta) = \int \widehat \sigma^\alpha(\eta-\xi,\xi^j/\delta)\left( \xi - \xi^j/\delta\right)^\alpha\psi_j(\delta\xi)\widehat u(\xi)\d\xi\,.
\end{align}
Unter Ausnutzung des Satzes von Plancherel folgt nun
\begin{align}
\spro {Pu} u  - \sum_{j\in\Z^n}\spro{ P_j^\delta\psi_j(\delta\mathrm D )u} { \psi(\delta\mathrm D )u } = (2\pi)^{-n/2} \iint H_\delta (\xi,\eta) \widehat u(\xi)\widehat u(\eta)\d\xi\d\eta\,,
\end{align}
dabei ist die Funktion $H_\delta$ durch
\begin{align}\label{eq:hdeltadef}
\begin{split}
H_\delta(\xi,\eta) := &\frac 1 2 \sum_{j\in \N}(\psi_j(\delta \xi) - \psi_j(\delta \eta))^2\widehat \sigma(\eta-\xi,\xi)\\
&+ \sum_{j\in\Z^n} \psi_j(\delta\xi)\psi_j(\delta \eta)\left[
\widehat \sigma(\eta-\xi,\xi) - \sum_{\abs \alpha \le 2} \frac{\widehat\sigma^\alpha(\eta-\xi,\xi^j/\delta)}{\alpha!}\left( \xi - \xi^j/\delta\right)^\alpha
\right]
\end{split}
\end{align}
definiert. Nun soll die Funktion $H_\delta$ nach oben abgeschätzt werden. Da $\sigma\in S^1_{\rm comp}(\R^n\times\R^n)$ folgt
\begin{align}
\abs{\braket{\eta}^{2N}\widehat\sigma^\alpha(\eta,\xi)}\le (2\pi)^{-n/2}\int_K |\braket{\mathrm D}^{2N}\sigma^\alpha(x,\xi)|\d x \le C_{\alpha,2N} \braket{\xi}^{1 - \abs \alpha}
\end{align}
und somit existiert für jedes $N\in\N_0$ und jeden Multiindex $\alpha$ eine Konstante $C_{\alpha,N}> 0 $, so dass
\begin{align}\label{eq:fourierordnungsabsch}
\abs {\widehat \sigma^\alpha(\eta,\xi)} \le C_{\alpha,N}\braket{\eta}^{-N}\braket{\xi}^{1-\abs\alpha}
\end{align}
Mit Hilfe des Taylorrestglieds
\begin{align}
R_N[\widehat \sigma(\eta,\cdot)](\xi,a):=N \int \sum_{\abs\alpha=N}\frac{(1-t)^{N-1}(\xi-a)^\alpha}{\alpha!}\widehat \sigma^\alpha(\eta,\xi+t(a-\xi)) \d t
\end{align}
und einer Taylorentwicklung im zweiten Argument von $\widehat \sigma$ um den Entwicklungspunkt $\xi^j/\delta$ folgt mit der Abschätzung \eqref{eq:fourierordnungsabsch}
\begin{align}
\Abs{\widehat \sigma(\eta-\xi,\xi) - \sum_{\abs \alpha \le 2} \frac{\widehat\sigma^\alpha(\eta-\xi,\xi^j/\delta)}{\alpha!}\left( \xi - \xi^j/\delta\right)^\alpha} = \Abs{R_3[\widehat \sigma(\eta-\xi,\cdot)](\xi,\xi^j/\delta)}\nonumber\\ \le
C_N\braket{\eta-\xi}^{-N}|\xi-\xi^j/\delta|^3\sup_{t\in[0,1]}(1+|\xi+t(\xi^j/\delta -\xi)|)^{-2}\,.\label{innerHdelta}
\end{align}
Aus den Eigenschaften der Partition der Eins $(\psi_j)_{j\in\Z^n}$ folgt nun $|\xi\delta-\eta\delta|\le C\abs{\eta\delta}^{1/2}$ für $j\psi_j(\delta\eta)\psi_j(\delta\xi)\neq 0$. Für alle $\delta\xi,\delta\eta\in\supp\psi_j$ mit $j\neq 0$ folgt nun
\begin{align}
|\xi -\eta| \le C \delta^{-1/2}|\eta|^{1/2}\le \frac{\abs\eta}{2}
\end{align}
für $|\eta|\ge 4 C^2\delta^{-1}$. Für $j\psi_j(\delta\eta)\psi_j(\delta\xi)\neq 0$ und $\abs\xi + \abs\eta>12C^2\delta^{-1}$ ist somit $\abs\xi$ und $\abs\eta$ größer als $4C^2\delta^{-1}$. Natürlich gilt das gleiche auch, nachdem $\xi$ durch $\xi^j/\delta$ und $\eta$ durch $\xi$ ersetzt wurde. Wird dies in \eqref{innerHdelta} eingesetzt, so folgt für $\delta(\abs\eta + \abs\xi)$ groß genug und einer Anwendung der Hölderungleichung für die Summe über $j$, dass
\begin{align}
\begin{split}
&\Abs{\sum_{j\in\Z^n\setminus\{0\}} \psi_j(\delta\xi)\psi_j(\delta \eta)\left[
\widehat \sigma(\eta-\xi,\xi) - \sum_{\abs \alpha \le 2} \frac{\partial ^\alpha\widehat\sigma(\eta-\xi,\xi^j/\delta)}{\alpha!}\left( \xi - \xi^j/\delta\right)^\alpha
\right]}\\&\le C_N\delta^{-3/2}\braket{\xi-\eta}^{-N}\abs\xi^{-1/2}
\end{split}
\end{align}
Der erste Term von $H_\delta$ kann wegen
\begin{align}
\sum|\psi(\delta\xi)-\psi(\delta\eta)|\le C \frac{\braket{\xi-\eta}^2}{(\braket\xi\braket\eta)^{{1}/{2}}}
\end{align}
für  $\delta(\abs\eta + \abs\xi)$ groß genug durch
\begin{align}
C_N\delta\braket{\eta-\xi}^{-N}
\end{align}
abgeschätzt werden. Alle verbliebenen Terme von $H_\delta$ können nun durch $\phi_{\delta,N}(\eta)\phi_{\delta,N}(\xi)$ mit $\phi_{\delta,N}\in\mathrm C^\infty_\delta(\R^n)$ abgeschätzt werden. Es gilt außerdem
\begin{align}
\begin{split}
\iint\braket{\xi-\eta}^{-(n+1)}|\widehat u(\xi)\widehat u(\eta)|\d\eta\d\xi &= \iint \braket{\eta}^{-(n+1)}|\widehat{u}(\xi)||\widehat u(\eta-\xi)|\d\eta\d\xi\\
&\le \Norm{\abs{\widehat u} * \abs{\widehat u}}_\infty \norm{\braket{\cdot}^{-(n+1)}}_1\le C_{n+1} \Norm{\widehat u}_2^2
\end{split}
\end{align}
Mit $|H_\delta(\xi,\eta)|\le C_N\delta\braket{\xi-\eta}^{-N} +\phi_{\delta,N}(\xi)\phi_{\delta,N}(\eta)$ und $N$ groß genug folgt nun
\begin{align}
\iint|H_\delta(\xi,\eta)\widehat u(\xi)\widehat u(\eta)|\d\xi\d\eta &\le \delta C_N \Norm{u}^2_{(0)} + \Norm{\phi_{\delta,N} \widehat u}_1^2 \le \delta C_N \Norm{u}^2_{(0)} + \|\widehat\phi_{\delta,N}\|_{(-s)}^2 \Norm{u}_{(s)}^2 \notag \\
&= \delta C_N \Norm{u}^2_{(0)} +  C_{\delta,N} \Norm{ u}_{(s)}^2\,.
\end{align}
\end{proof}

Natürlich gilt die Ungleichung \eqref{hilfsungl1} auch, wenn wir das nicht kompakt getragene Symbol $\braket{\xi}$ betrachten. Dies lässt sich ebenfalls durch einer Taylorentwicklung und nach einer Anwenung des Satzes von Plancherel zeigen.
Die obige Aussage zusammen mit dem Lemma zeigt, dass der Fehler, der durch die Partitionierung von $u$ im Fourierraum und die Taylorentwicklung des Symbols entsteht, durch in $u$ und $\delta$ beschränkten Funktionen und den Konstanten und Normen in \eqref{hilfsungl1} beschrieben werden kann. Solche Funktionen werden in Zukunft mit dem Symbol $\mathcal O(1)$ bezeichnet. Wir schreiben also
\begin{align}
\spro {Au}{u} = \sum_{j\in\Z^n} \spro {A_j^\delta \psi_j(\delta\D)u}{\psi_j(\delta\D)u} + C\delta \mathcal{O}(1) \Norm{u}_{(0)} + C_\delta \mathcal O(1)\Norm{u}_{(s)}\,.
\end{align}

Nun soll eine weitere Lokalisation der Funktion $u$ vorgenommen werden. Dafür sei $\varphi_{j,k}(x):=\varphi_k(x|\xi^j|^{1/2})$. Für alle $v\in\rmC^\infty(\R^n)$ folgt nach Ausmultiplizieren und der Regel von Leibniz
\begin{align}
\sum_{k\in\Z^n}\spro {A^\delta_j\varphi_{j,k}v} {\varphi_{j,k}v} = \sum_{k\in\Z^n}\sum_{|\alpha|\le 2}\sum_{\gamma\le\beta\le\alpha} \left({\frac{a^\alpha(\cdot,\xi^j/\delta)(-\xi^j/\delta)^{(\alpha-\beta)}}{(\alpha-\beta)!(\beta-\gamma)!\gamma!}(\D^\gamma\varphi_{j,k})(\D^{\beta-\gamma}v)}\ , {\varphi_{j,k}v}\right)
\end{align}
Da außerdem $0=\D^\gamma\sum_k\varphi_{j,k}^2 = 2 \sum_k\varphi_{j,k}\D^\gamma\varphi_{j,k}$ für $|\gamma|=1$ fallen in der oberen Summe alle Terme mit $|\gamma|=1$ weg und es folgt
\begin{align}\label{eq:vorueberlegung1}
\begin{split}
\sum_{k\in\Z^n}\spro {A^\delta_j\varphi_{j,k}v} {\varphi_{j,k}v} = &\sum_{k\in\Z^n}\spro {\varphi_{j,k}A^\delta_jv} {\varphi_{j,k}v} \\&+ \sum_{k\in\Z^n}\sum_{|\alpha|=2}\int \frac{\partial^\alpha\sigma(x,\xi^j/\delta)}{\alpha!} |\xi^j|(\varphi_k\D^\alpha\varphi_k)(x|\xi^j|^{1/2})|v(x)|^2\d x
\end{split}
\end{align}
Da $a\in S^1_\text{unif}(\R^n\times\R^n)$ und somit $|a^\alpha(x,\xi^j/\delta)|\le\braket{\xi^j/\delta}^{-1}\le \delta|\xi^j|^{-1}$ für $|\alpha|=2$ und außerdem
\begin{align}
\sum_{k\in\Z^n} |\varphi_k(x)\D^\alpha\varphi_k(x)|\le \left(\sum_{k\in\Z^n}|\phi_k(x)|^2\right)^{1/2}\left(\sum_{k\in\Z^n}|\D^\alpha\phi_k(x)|^2\right)^{1/2}\le C_\alpha^{1/2}
\end{align}
kann die zweite Summe aus \eqref{eq:vorueberlegung1} durch $\|v\|^2$ abgeschätzt werden. Wird nun $\psi_j(\delta\D)u$ für $v$ eingesetzt und über $j$ summiert, so wird ersichtlich, dass die zweite Summe wieder von der Form $C\delta \mathcal{O}(1) \Norm{u}_{(0)} + C_\delta \mathcal O(1)\Norm{u}_{(s)}$ ist, da
\begin{align}
\begin{split}
\left|\sum_{j,k\in\Z^n}\sum_{|\alpha|=2}\int \frac{a^\alpha(x,\xi^j/\delta)}{\alpha!} |\xi^j|(\varphi_k\D^\alpha\varphi_k)(x|\xi^j|^{1/2})|v(x)|^2\d x\right|\\ \le \delta C \sum_{j\in\Z^n} \norm {\psi_j(\delta\D)u}^2 = \delta C \norm {\sum_{j\in\Z^n} \psi_j(\delta \cdot)\widehat u}^2 = C \delta \Norm u ^2\,.
\end{split}
\end{align}
Mit der Bezeichnung $u_\delta^{j,k} = \varphi_{j,k} \psi_j(\delta\D)u$ bedeutet dies
\begin{align}
\spro {Au} u = \sum_{j,k\in\Z^n} \spro {A^\delta_j u_\delta^{j,k}}{u_\delta^{j,k}} + C\delta\mathcal O(1)\norm u _{(0)} + C_\delta \Norm u _{(s)}
\end{align}
Um den verbleibenden Term abzuschätzen wird nun eine Koordinatentransformation durchgeführt. Hierfür sei $x^k = \xi^k |\xi^k|^{-1/2}\in\supp \varphi_k$, $x^{j,k}=x^k |\xi^j|^{-1/2}$ und weiterhin sei $v^{j,k}_\delta$ durch die Relation
\begin{align}
u^{j,k}_\delta(x) = v^{j,k}_\delta ((x-x^{j,k})|\xi^j|^{1/2}) e^{ix\xi^j/\delta}
\end{align}
definiert. Es folgt
\begin{align}
|v^{j,k}_\delta(y)| = \abs{e^{-i(y+x^k)x^j}u^{j,k}_\delta((y+x^k)|\xi^j|^{(-1/2)})} = |\varphi_k(x^k+y)\psi_j(\delta\D)u((y+x^k)|\xi^j|^{(-1/2)})|
\end{align}
Da aus den Eigenschaften der Zerlegung der Eins $\supp \varphi_k \subset B(x^k,C)$ folgt, gilt auch
\begin{align}
\supp v^{j,k}_\delta \subset B(0,C)
\end{align}
Außerdem impliziert die Wahl der Transformation für $y=(x-x^{j,k})|\xi^j|^{1/2}$
\begin{align}
(\D - \xi^j/\delta)^\alpha u^{j,k}_\delta(x)=  e^{i(y+x^k)x^j}|\xi^j|^{\abs\alpha/2}\D^\alpha v^{j,k}_\delta(y)
\end{align}
Somit folgt mit der Koordinatentransformation und einer Taylorentwicklung der Ordnung $2-\abs\alpha$ von $\sigma^\alpha$ in $x$ um den Entwicklungspunkt $x^{j,k}$
\begin{align}
\spro {A^\delta_j u_\delta^{j,k}}{u_\delta^{j,k}}
=& \sum_{\abs\alpha\le 2} |\xi^j|^{(\abs\alpha-n)/2}\int \frac{a^\alpha(x^{j,k}+y|\xi^j|^{-1/2},\xi^j/\delta)}{\alpha!}\cc{v^{j,k}_\delta(y)}\D^\alpha v^{j,k}_\delta(y)\d y\nonumber\\ \label{termnachkoordtransf}
=& \sum_{\abs{\alpha+\beta}\le 2}|\xi^j|^{(\abs\alpha-\abs\beta-n)/2} \frac{a^\alpha_\beta(x^{j,k},\xi^j/\delta)}{\alpha!\beta!}\int\cc{v^{j,k}_\delta(y)}y^\beta\D^\alpha v^{j,k}_\delta(y)\d y\\
&+\sum_{\abs\alpha\le 2}|\xi^j|^{(\abs\alpha-n)/2}\int R_{(3-\abs\alpha)}[a^\alpha(\cdot,\xi^j/\delta)](x^{j,k}+y|\xi^j|^{-1/2},x^{j,k})\cc{v^{j,k}_\delta(y)}\D^\alpha v^{j,k}_\delta(y)\d y\nonumber
\end{align}
Das Taylorrestglied $R_d$, wobei $d=3-\abs\alpha$, kann durch einen Term $\mathcal O(1)\delta^{-1}|\xi^j|^{-(\abs\alpha+1)/2}$ außerhalb des Integrals ersetzt werden, da
\begin{align}
\begin{split}
&\Abs{R_d[\sigma^\alpha(\cdot,\xi^j/\delta)](x^{j,k}+y|\xi^j|^{-1/2},x^{j,k})} \\
&\le c\int \sum_{\abs\beta=d}\frac{|y^\beta||\xi^j|^{-d/2}}{\beta!}\Abs{\sigma^\alpha_\beta(x^{j,k}+ ty|\xi^j|^{-1/2},\xi^j/\delta) }\d t\\
&\le c_d \frac{\abs y ^d}{\abs{\xi^j}^{d/2}}\braket{\xi^j/\delta}^{1-\abs\alpha} \le\frac{\widetilde c_d\abs y ^d}{\abs{\xi^j}^{d/2}}|\xi^j/\delta|^{1-\abs\alpha} \le \widetilde c C ^d\delta^{-1}|\xi^j|^{-(1+\abs\alpha)/2}
\end{split}
\end{align}
für alle $y\in\supp\varphi_k$. Dabei kann $\widetilde c$ unabhängig von  $\alpha$, $u$, $j$, $k$ oder $\delta$ gewählt werden. Ferner ist der Träger aller $v_\delta^{j,k}$ in $B(0,C)$ enthalten, also hat die zweite Summe in \eqref{termnachkoordtransf} die Form
\begin{align}\label{restnachkoord}
\mathcal O(1) \delta^{-1}|\xi^j|^{-1/2}\sum_{\abs\alpha\le 2}\int  |\D^\alpha v^{j,k}_\delta(y)|^2\d y
\end{align}
Um das verbleibende Integral auszurechnen werden die neuen Funktionen $v^j_\delta$ durch die Relation
\begin{align}
\psi^j(\delta\D)u(x) = e^{ix\xi^j/\delta}v^j_\delta(x|\xi^j|^{1/2})
\end{align}
eingeführt. Aus der Relation folgt
\begin{align}
\widehat v^j_\delta ((\xi-\xi^j/\delta)|\xi^j|^{-1/2}) = \psi_j(\delta\xi)\widehat u(\xi)
\end{align}
und somit
\begin{align}
\xi \in\supp\widehat v^j_\delta \quad\Leftrightarrow\quad \xi|\xi^j|^{1/2} + \xi^j/\delta\in\supp \psi_j(\delta(\cdot))\widehat u
\end{align}
Außerdem gilt $|\delta\xi-\xi^j|\le C|\xi^j|^{1/2}$ für $\xi\in\supp \mathcal F[\psi_j(\delta\D)u]\subset\supp \psi_j(\delta (\cdot))$ und somit folgt
\begin{align}
|\xi|\le C/\delta \quad \text{für } \xi \in\supp\widehat v^j_\delta\,.
\end{align}
Darüber hinaus lassen sich die Funktionen $\varphi_k v^j_\delta$ durch Translation in die Funktionen $v^{j,k}_\delta$ überführen. In expliziter Form lautet diese Beziehung
\begin{align}
\varphi_k(y) v^j_\delta (y) = v^{j,k}_\delta (y-x^{j,k}|\xi^j|^{1/2})\quad\text{für }y\in\R^n\,.
\end{align}
Nun soll das über $k$ summierte verbleibende Integral aus \eqref{termnachkoordtransf} abgeschätzt werden. Dafür betrachten wir, den folgenden Ausdruck
\begin{align}
\label{polynomnachoben}
&\sum_{k\in\Z^n}\sum_{|\alpha+\beta|\le N} \int|y^\beta\D^\alpha v^{j,k}_\delta(y)|^2\d y \le C^{N} \sum_{k\in\Z^n} \sum_{|\alpha+\beta|\le N}\int |\D^\alpha(\varphi_k v^j_\delta)(y)|^2\d y \\
&\le C_N\sum_{k\in\Z^n}\| v^j_\delta \varphi_k\|_{(N)}^2 
= C_N\| v^j_\delta \|_{(N)}^2 \nonumber
\le \widetilde C_N \delta^{-2N}\| v^j_\delta \|^2 = \widetilde C_N \delta^{-2N}|\xi^j|^{n/2}\| \psi_j(\delta\D)u \|^2 
\end{align}
Wird nun über alle $j$ summiert, so dass $\xi^j>\delta^{-12}$ ist, erhält man den Ausdruck
\begin{align}
\sum_{j\colon \xi^j>\delta^{-12}} \delta^{-5} |\xi^j|^{-1/2}\int |\psi_j(\delta\D)u(y)|^2\d y = \mathcal O(1)\delta\norm u _{(0)}^2
\end{align}
Die restlichen endlich vielen Summanden ergeben einen Ausdruck $\mathcal O(1)C_\delta \norm u _{(s)}^2$. Somit folgt
\begin{align}
\sum_{j\in\Z^n}\delta^{-5} |\xi^j|^{-1/2}\int |\psi_j(\delta\D)u(y)|^2\d y = \mathcal O(1)\delta\norm u _{(0)}^2 + \mathcal O(1)C_\delta \norm u _{(s)}^2
\end{align}
Zusammengefasst ergibt dies
\begin{align}\label{fastfertig}
\spro {Au} u
=& \sum_{j,k\in\Z^n}\sum_{\abs{\alpha+\beta}\le 2}|\xi^j|^{(\abs\alpha-\abs\beta-n)/2} \frac{a^\alpha_\beta(x^{j,k},\xi^j/\delta)}{\alpha!\beta!}\int\cc{v^{j,k}_\delta(y)}y^\beta\D^\alpha v^{j,k}_\delta(y)\d y\\
&+\mathcal O(1)\delta\norm u _{(0)}^2 + \mathcal O(1)C_\delta \norm u _{(s)}^2\nonumber
\end{align}
In Zukunft soll $\delta$ so gewählt sein, dass der Realteil des $\mathcal O(1)\delta\norm u ^2_{(0)}$ Terms größer als $-(\epsilon/3)\norm u ^2_{(0)}$ ist. Wir wählen nun eine Abzählung für die Indizierung der $\psi_j$, $\varphi_j$ und $\xi^j$ so, dass $|\xi^j|$ eine monoton steigende Folge ist. Finden wir jetzt eine nullkonverente Folge $\rho_j$, so dass

\begin{align}
\Re\sum_{|\alpha+\beta|\le 2} \frac{a^\alpha_\beta(x^{j,k},\xi^j/\delta)}{\alpha!\beta!}|\xi^j|^{|\alpha-\beta|/2}\int \cc{v^{j,k}_\delta(y)}y^\beta\D^\alpha v^{j,k}_\delta(y)\d y + \frac{\epsilon}{2}\norm{v^{j,k}_\delta}^2 + \rho_j \norm{v^{j,k}_\delta}_{(4)}^2 \ge 0
\end{align}
dann folgt nach unter Verwendung von \eqref{polynomnachoben} und Addition über $k$,  dass 
\begin{align}
\sum_{k=0}^\infty \Re\sum_{|\alpha+\beta|\le 2} \frac{a^\alpha_\beta(x^{j,k},\xi^j/\delta)}{\alpha!\beta!}|\xi^j|^{|\alpha-\beta|/2}\int \cc{v^{j,k}_\delta(y)}y^\beta\D^\alpha v^{j,k}_\delta(y)\d y \ge -(\frac{\epsilon}{2}+C_4\delta^{-8} \rho_j)\norm{\psi_j({\delta\D})u}_{(0)}^2 
\end{align}
Wird dies über $j$ summiert, So folgt daraus zusammen mit \eqref{fastfertig} die gewünschte Ungleichung. Um zu zeigen, dass es so eine Folge $\rho_j$ tatsächlich gibt, brauchen wir das folgende Lemma.

\begin{lem}
Es sei $C>0$, $K_0\subset\R^n\setminus\{0\}$ kompakt und $\gamma\in\mathrm C(\R^n\times K_0)$ reellwertig, so dass $\gamma(x,\xi)$ konstant ist auf $B(0,R)^\mathrm{C}\times K_0$ für ein $R>0$. Ist die Aussage
\begin{align}\label{melinendlemma1}
(a,\xi)\in\R^n\times K_0\,,\quad a_1(x,\xi)=0 \quad \Rightarrow \quad 1/2 \mathrm{Tr}^+H_{a_1}(a,\xi) + \gamma(x,\xi)
\end{align}
wahr, dann existiert eine Funktion $\rho$, so dass $\rho(\lambda)\to 0$ für $\lambda\to\infty$ und
\begin{align}\label{melinendlemma2}
\Re \sum_{|\alpha+\beta|\le 2} \frac{{a_1}^\alpha_\beta(x,\xi)\lambda^{2-|\alpha+\beta|}}{\alpha!\beta!}\int\cc{v(y)}y^\beta\D^\alpha v(y)\d y + \gamma(x,\xi)\Norm{v}_{(0)}^2 + \rho(\lambda)\Norm v _{(4)}^2 \ge 0
\end{align}
für alle $(x,\xi)\in\R^n\times K_0$ und $v\in\mathrm C^\infty_0(B(0,C))$
\end{lem}

\begin{proof}
Wir nehmen an es gibt keine Funktion $\rho$ mit $\lim_{\lambda\to\infty}\rho(\lambda)=0$, so dass \eqref{melinendlemma2} erfüllt ist. Also existiert eine Zahl $\rho>0$, eine und Folgen $(\lambda_j)$ mit $\lambda_j\to\infty$, $(x_j,\xi_j)\in\R^n\times K_0$ und Funktionen $v_j\in\mathrm C^\infty_0(B(0,C))$ mit $\Norm {v_j} =1$, so dass
\begin{align}\label{melinendlemma3}
\Re \sum_{|\alpha+\beta|\le 2} \frac{{a_1}^\alpha_\beta(x_j,\xi_j)\lambda_j^{2-|\alpha+\beta|}}{\alpha!\beta!}\int\cc{v_j(y)}y^\beta\D^\alpha v_j(y)\d y + \gamma(x_j,\xi_j)\Norm{v_j}_{(0)}^2 + \rho\Norm {v_j} _{(4)}^2 \le 0
\end{align}
Für nichtnegative Funktionen $f\in\mathrm C^\infty_0 (\R^n)$,  existiert stets eine Konstante $C_f$, so dass
\begin{align}
|\operatorname{grad}f(y)|^2\le C_f f(y)
\end{align}
Da $a_1$ außerhalb einer kompakten Menge durch $\braket{\xi}$ gegeben ist, folgt mit obigem für $\abs\alpha=\abs\beta=1$
\begin{align}
|{a_1}^\alpha(x_j,\xi_j)\lambda_j\int \cc{v_j(y)}\D^\alpha v_j(y)\d y| &\le \lambda_j|{a_1}^\alpha(x_j,\xi_j)|\Norm{v_j}_{(0)}\Norm{v_j}_{(1)} \le \frac{\lambda_j^2}{3n}a_1(x_l,\xi_j) + \frac 1 2\Norm{v_j}_{(1)}^2\\
|{a_1}_\beta(x_j,\xi_j)\lambda_j\int \cc{v_j(y)}y^\beta v_j(y)\d y| &\le C \lambda_j|{a_1}^\alpha(x_j,\xi_j)|\Norm{v_j}_{(0)} \le \frac{\lambda_j^2}{3n}a_1(x_l,\xi_j) + \frac 1 2
\end{align}
%%%%%%%%%%%%%
für $j$ groß genug. Die quadratischen Terme in \eqref{melinendlemma3} erfüllen die Ungleichung
\begin{align}
\Abs{\sum_{|\alpha+\beta|= 2} \frac{{a_1}^\alpha_\beta(x_j,\xi_j)\lambda_j^{2-|\alpha+\beta|}}{\alpha!\beta!}\int\cc{v_j(y)}y^\beta\D^\alpha v_j(y)\d y}\le C'\Norm{v_j}_{(2)}^2\,.
\end{align}
Werden die eben aufgelisteten Ungleichungen in \eqref{melinendlemma3} verwendet, so folgt, da $\gamma$ beschränkt ist,
\begin{align}
\Norm{v_j}_{(4)}^2\le \frac{C''}{\rho}\Norm{v_j}_{(2)}^2\le \frac{C''}{\rho}\Norm{v_j}_{(0)}\Norm{v_j}_{(4)}^2
\end{align}
Somit ist die Folge $v_j$ $H^4(\R^n)$-beschränkt. Aus dem Satz von Rellich wissen wir, dass eine Teilfolge existiert, die in $H^3(\R^n)$ konvergiert. Der Grenzwert dieser Teilfolge sei durch $v_0$ gegeben. Da alle anderen Terme beschränkt sind, müssen außerdem auch die $\lambda_j^2a_1(x_j,\xi_j)$ beschränkt sein. Somit konvergiert $a_1(x_j,\xi_j)$ gegen Null. 
Da $a_1$ außerhalb von einer kompakten Menge größer als eine positive Konstante ist, können wir ohne Einschränkung annehmen, dass die Folge $(x_j,\xi_j)$ gegen ein Element $(x_0,\xi_0)$ konvergiert.
Wird $\rho$ verkleinert, so kann in \eqref{melinendlemma3} $v_j$ durch $v_0$ ersetzt werden, ohne die Ungleichung für große $j$ zu verletzen.
%%%%%%%%%%%%%%%%%%%%%%%%%%%%%%%%%%%%%%%%%%%%%%%%%%%%%%%%<
Wir können nun folgern, dass eine Funktion $v\in\mathrm C^\infty_0(\R^n)$ und $\sigma>0$ existiert, so dass $\Norm v =1$ und
%%%%%%%%%%%%%%%%%%%%%%%%%%%%%%%%%%%%%%%%%%%%%%%%%%%%%%%%%%%>
\begin{align}
\begin{split}
&\Re \sum_{|\alpha+\beta|\le 2} \frac{{a_1}^\alpha_\beta(x_j,\xi_j)\lambda_j^{2-|\alpha+\beta|}}{\alpha!\beta!}\int\cc{v(y)}y^\beta\D^\alpha v(y)\d y\\
& + \frac \sigma 2 \int \cc{v(y)}(y^2 + \D^2)v(y) \d y+ (\gamma(x_0,\xi_0)+ \frac\rho 2)\Norm {v} _{(0)}^2 < 0
\end{split}
\end{align}
Wir führen nun die Bezeichnungen
\begin{align}
F_j := \operatorname{grad}_{(x,\xi)}a_1(x_j,\xi_j)\,,\quad H_j:=H_{a_1}(x_j,\xi_j)\,,\quad H_j^\sigma:=H_j  + \sigma I
\end{align}
ein. Da die Matrix $H_0^\sigma$ positiv definit ist, ist auch $H_j^\sigma$ positiv definit, für $j$ groß genug. Das Anwenden von Satz \ref{satz:thm2.4} ergibt
\begin{align}\label{melinendlemma4}
\lambda_j^2\left[ a_1(x_j,\xi_j) - \frac{1}{2}\cspro{F_j}{{H_j^\delta}^{-1}}{F_j} \right] + \gamma(x_0,\xi_0) + \frac \rho 2 + \frac 1 2 \mathrm{Tr}^+ H_j^\sigma\le 0
\end{align}
Nun soll gezeigt werden, das die Ungleichung \eqref{melinendlemma4} im Widerspruch zu \eqref{melinendlemma2} steht. Eine Tayor-Entwicklung von $a_1$ liefert
\begin{align}
a_1((x_j,\xi_j) + h ) = a_1(x_j,\xi_j) + F_j\cdot h + \frac 1 2 \cspro{h}{H_j}{h} + \mathcal O(1)|h|^3
\end{align}
für $|h|\le C$ und $h\in\R^{2n}$. Da das Betragsquadrat des Gradienten $F_j$ durch $a_1(x_j,\xi_j)$ beschränkt ist, ist $F_j=\mathcal O(1)\lambda_j^{-1}$. Außerdem ist gilt $\cspro{w}{H_j^\sigma}{w}>\cspro{w}{H_j}{w}$ für alle $w\in\R^{2n}\setminus\{0\}$. Es folgt
\begin{align}
\begin{split}
0&\le a_1((x_j,\xi_j)- {H_j^\sigma}^{-1}F_j) \\&= a_1(x_j,\xi_j) - \cspro{F_j}{{H_j^\sigma}^{-1}}{F_j} + \frac 1 2\cspro{F_j}{{H_j^\sigma}^{-1}H_j{H_j^\sigma}^{-1}}{F_j} + \mathcal O(1)\lambda_j^{-3}\\
&\le a_1(x_j,\xi_j) - \frac 1 2\cspro{F_j}{{H_j^\sigma}^{-1}}{F_j}  + \mathcal O(1)\lambda_j^{-3}
\end{split}
\end{align}
Wird nun wieder $j$ groß genug gewählt, so gilt
\begin{align}
0\le \lambda_j^2\left[ a_1(x_j,\xi_j) - \frac{1}{2}\cspro{F_j}{{H_j^\delta}^{-1}}{F_j} \right]  +\frac\rho 4
\end{align}
Also muss wegen \eqref{melinendlemma4}
\begin{align}
\gamma(x_0,\xi_0) + \frac \rho 4 + \frac 1 2 \mathrm{Tr}^+ H_j^\sigma\ge 0
\end{align}
Da $\mathrm{Tr}^+$ stetig ist, folgt im Limes $j\to\infty$ und $\sigma\to 0$
\begin{align}
\gamma(x_0,\xi_0) + \frac \rho 4 + \frac 1 2 \mathrm{Tr}^+ H_{a_1}(x_0,\xi_0)\le 0
\end{align}
was im Widerspruch zu \eqref{melinendlemma2} steht.
\end{proof}

Um das Lemma zur Vervollständigung des Beweises nutzen zu können, fehlen noch einige Vorüberlegungen. Da
\begin{align}
{a_0}^\alpha_\beta(x^{j,k},\xi^j/\delta)|\xi^j|^{(|\alpha|-|\beta|)/2}\int \cc{v^{j,k}_\delta(y)}y^\beta\D^\alpha v^{j,k}_\delta(y) \le c \delta^{\abs\alpha}|\xi^j|^{-|\alpha+\beta|/2}\norm{v^{j,k}_\delta}_{(2)}^2
\end{align}
für $|\alpha+\beta|\le 2$ und $|\xi^j|$ bestimmt divergiert, können die Symbole $a^\alpha_\beta$ durch ${a_1}^\alpha_\beta$ ersetzt werden, wenn wir einen zusätzlichen Term
\begin{align}
{a_0}(x^{j,k},\xi^j/\delta)\norm{v^{j,k}_\delta}_{(0)}^2
\end{align}
dazu addieren. Nun wenden wir das Lemma mit $K_0=\delta^{-1}\mathbb{S}^{n-1}$, $\lambda_j=|\xi^j|^{1/2}$, $\rho_j(\lambda_j)$, $(x,\xi)=(x^{j,k},\xi^j/\delta|\xi^j|^{-1})$, $\gamma = \epsilon/2 + a_0$ und $v=v^{j,k}_\delta$  an.
 % Ungleichung von Melin :: Jonas Brinker

%
%

\chapter{Die Ungleichung von Feffermann und Phong}


% Literaturverzeichnis
\bibliographystyle{alpha}
\bibliography{seminar}
% Index
\printindex
\end{document}
