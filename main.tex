\documentclass[11pt,bibtotoc]{scrbook}
\usepackage{german}
\usepackage[utf8]{inputenc}
%
\usepackage{hyperref}
%
\usepackage{wasysym}\let\iint\relax\let\iiint\relax
\usepackage{amsfonts,amssymb,amsthm,amsmath,CJK,mathrsfs}
\usepackage[a4paper, scale=0.76]{geometry}
%
\usepackage{ifthen}
\usepackage{makeidx}
\makeindex
\newcommand{\eIndex}[2][]{%
\emph{#2}%
\ifthenelse{\equal{#1}{}}{\index{#2}}{\index{#1!#2}}%	
}%
\def\spro#1#2{\pmb(#1,#2\pmb)}
%
% Aufzaehlungen roemisch numeriert....
\renewcommand{\labelenumi}{{\rm\bf(\roman{enumi})}}
%
\newtheorem{thm}{Satz}[chapter]
\newtheorem{lem}[thm]{Lemma}
\newtheorem{cor}[thm]{Korollar}
\renewcommand{\proofname}{Beweis}
\theoremstyle{definition}
\newtheorem{df}[thm]{Definition}
\newtheorem{exa}{Beispiel}[chapter]
\newtheorem{rem}[thm]{Bemerkung}
%
%
% DEFINITIONEN
\def\N{\mathbb N}    % N
\def\R{\mathbb R}    % Koerper R
\def\C{\mathbb C}    % Koerper C
\def\e{\mathrm e}     % Eulersche Zahl e
\def\i{\mathrm i}       % imaginäre Einheit i
\def\D{\mathrm D}    % Differentialoperator D
\def\d{\,\mathrm d}   % Differential d für Integrale
\def\rmC{\mathrm C}% Raum C
\def\rmL{\mathrm L} % Raum L
\def\rmH{\mathrm H}% Raum H
\def\mcP{\mathcal P}   % curly P
\def\mcQ{\mathcal Q}  % curly Q  
\def\mcR{\mathcal R}  % curly R
%% von Tillmann
% NOT TO BE USED, WILL BE REMOVED LATER
\def\pr{\prime}
\def\tr{\top}%{\mathrm{T}}
\def\al{\alpha}
\def\be{\beta}
\def\de{\delta}
\def\ep{\epsilon}
\def\ga{\gamma}
\def\x{\xi}
\def\y{\eta}
\def\z{\zeta}
\def\Om{\Omega}
\def\pd{\partial}
\def\infi{\infty}
\def\til#1{\widetilde{#1}}
\def\mto{\mapsto}
\def\ti{\times}
\def\abs#1{\lvert#1\rvert}
\def\Abs#1{\left\lvert#1\right\rvert}
\def\norm#1{\lVert#1\rVert}
\def\Lap{\Delta}
\def\cc#1{\overline{#1}}
\def\ul#1{\underline{#1}}
\def\ska#1#2{\langle#1,#2\rangle}
\def\Ska#1#2{\left\langle#1,#2\right\rangle}
\def\subs{\subset}
%\def\Lin{\operatorname{span}}
%\def\Grad{\operatorname{deg}}
\def\gdw{\Leftrightarrow}
%%%%%%%%%%%%%%%%%%%%
%
\DeclareMathOperator{\supp}{supp}          % Traeger (supp)
\DeclareMathOperator{\diam}{diam}          % Durchmesser  (diam)
\DeclareMathOperator{\dist}{dist}               % Abstand (dist)
\let\div\relax
\DeclareMathOperator{\div}{div}                % Divergenz
\DeclareMathOperator{\spann}{span}        % Lineare Huelle (span)
\DeclareMathOperator{\Grad}{deg}           % Grad eines Polynoms
%
\begin{document}
\titlehead{Priv.-Doz. Dr. Jens Wirth\\Institut f\"ur Analysis, Dynamik und Modellierung\\Universit\"at Stuttgart}
\lowertitleback{\copyright 2015. Texte geschrieben von den Teilnehmern des Seminars.  Einzelne Beiträge sind nicht namentlich gekennzeichnet. Ausarbeitungen basieren auf Originalliteratur. }
\title{Mikrolokale Analysis}
\subtitle{Masterseminar}
\author{---}
\date{Sommersemester 2015}
\maketitle
\tableofcontents
%
\part{Analysis von Differentialoperatoren}
%
% !TEX root = main.tex
\chapter{Einleitung}
Die folgenden Abschnitte basieren im wesentlichen auf Hörmanders Arbeit \cite{Hormander:1955}. In dieser Einleitung sollen die verwendete Notation festgelegt und die wichtigsten operatortheoretischen Konzepte zusammengefaßt werden.

\section{Funktionalanalytische Grundlagen}

Seien $V$ und $W$ Banachräume und $\mathcal D_T\subseteq V$ ein linearer Teilraum. Ein (im allgemeinen unbeschr\"ankter) \eIndex{Operator} $T$ mit Definitionsbereich $\mathcal D_T$ ist eine lineare Abbildung $T:V\supset\mathcal D_T \to W$. Er heißt \eIndex[Operator]{beschr\"ankt}, falls es eine Konstante $C>0$ mit 
\begin{equation}
\forall v\in\mathcal D_T \quad:\quad \|Tv \|_W \le C \|v\|_V
\end{equation}
gibt. Weiter heißt
\begin{equation}
\mathcal R_T = \{ Tv : v\in \mathcal D_T \} \subseteq W
\end{equation}
der \eIndex[Operator]{Wertebereich} und 
\begin{equation}
\mathcal G_T = \{ (v,Tv) : v \in \mathcal D_T \} \subseteq V\times W 
\end{equation}
der \eIndex[Operator]{Graph} von $T$. Eine Teilmenge $\mathcal G\subseteq V\times W$ ist genau dann Graph eines Operators, wenn $\mathcal G$ linear ist und aus
$(0,w)\in\mathcal G$ stets $w=0$ folgt. Wir versehen $V\times W$ mit der Norm $\|(v,w)\|_{V\times W}^2 = \|v\|_V^2 + \|w\|_W^2$.


Der Operator $T$ wird als \eIndex[Operator]{abgeschlossen} bezeichnet, falls der Graph $\mathcal G_T$ ein abgeschlossener Teilraum des Produktraumes $V\times W$ ist. Weiter heißt $T$ \eIndex[Operator]{abschließbar}, falls der Abschluß von $\mathcal G_T$ in $V\times W$ Graph eines Operators ist. Dieser wird als Abschluß von $T$ bezeichnet.

\begin{thm}\label{thm:1:1.1}
Sei $T_1$ abgeschlossen und $T_2$ abschließbar und gelte $\mathcal D_{T_1}\subseteq \mathcal D_{T_2}$. Dann existiert eine Konstante $C$ mit  
\begin{equation}
   \| T_2 u\|_W^2 \le C ( \|T_1 u\|_W^2 + \|u\|_V^2 ). 
\end{equation}
\end{thm}
\begin{proof}
Wir betrachten die Abbildung $\mathcal G_{T_1} \ni (u,T_1u) \mapsto T_2 u \in W$. Diese ist linear und überall definiert. Wir zeigen, daß sie auch abgeschlossen ist. Angenommen, $u_n\to u$ und $T_1 u_n\to T_1 u$ und $T_2 u_n$ sei konvergent. Da $T_2$ abschließbar ist, konvergiert $T_2 u_n \to T_2 u$. Also ist die Abbildung abgeschlossen, nach dem Satz vom abgeschlossenen Graphen stetig und somit beschr\"ankt.
\end{proof}

Ist $T$ injektiv, so bestimmt $\mathcal G_{T^{-1}} = \{ (w,v) : (v,w)\in\mathcal G_T\}$ einen Graphen. Der zugehörige Operator wird als $T^{-1}$ bezeichnet. Dieser ist genau dann abgeschlossen, wenn $T$ abgeschlossen ist. Ist $T^{-1}$ beschränkt, gilt also
\begin{equation}\label{eq:1:1.6} 
   \forall u\in\mathcal D_T \quad:\quad \|u\|_V\le C \|Tu\|_W
\end{equation}
mit einer von $u$ unabhängigen Konstanten $C$, so sagen wir $T$ ist \eIndex[Operator]{beschränkt invertierbar}.

Im folgenden sei $V=W=H$ ein Hilbertraum. Zur Vereinfachung der Notation verwenden wir immer $\|\cdot\|$ f\"ur die Norm in $H$. Weiter sei das Innenprodukt in $H$ durch $\spro{u}{v}$ bezeichnet und $H\times H$ mit dem entsprechenden Innenprodukt $\spro{(u_0,u_1)}{(v_0,v_1)} = \spro{u_0}{v_0} + \spro{u_1}{v_1}$ versehen. Auf $H\times H$ sei der Operator $J$ mit $J(v,w) = (-w,v)$ definiert. Dann ist zu jedem dicht definierten Operator $T$ mit Graphen $\mathcal G_T$ durch $\mathcal G_{T^*}=(J\mathcal G_T)^\perp$ ein Graph gegeben. Der zugeh\"orige Operator $T^*$ wird als zu $T$ \eIndex[Operator]{adjungiert} bezeichnet. Er ist immer abgeschlossen und es gilt
\begin{equation}
   \spro{Tu}{v} = \spro{u}{T^*v}
\end{equation}
f\"ur $u\in\mathcal D_T$ und $v\in\mathcal D_{T^*}$.

\begin{thm}\label{thm:1:1.2}
Sei $T$ ein dicht definierter Operator. Dann gilt $\mathcal R_T=H$ genau dann, wenn 
$T^*$ beschränkt invertierbar ist.
\end{thm}
\begin{proof}
Sei $\mathcal R_T=H$. Dann existiert zu jedem $u\in H$ ein $w\in H$ mit $Tw=u$. Damit folgt
\begin{equation}
   \spro{u}{v} = \spro{Tw}{v} = \spro{w}{T^*v}, \qquad |\spro{u}{v}| \le C_u \|T^*v\|
\end{equation}
f\"ur alle $u\in H$ und $v\in\mathcal D_{T^*}$ und mit dem Satz über die gleichmäßige Beschränktheit die Behauptung.

Angenommen, $T^*$ erf\"ullt die Ungleichung \eqref{eq:1:1.6}. Dann ist der selbstadjungierte Operator $TT^*$ wegen
\begin{equation}
  \spro{TT^*v}{v} = \spro{T^*v}{T^*v} \ge C^{-2} \spro{v}{v}
\end{equation}
strikt positiv und stetig invertierbar. Dann ist aber $TT^* (TT^*)^{-1}  = I$ und somit $\mathcal R_T = H$.
\end{proof}

\begin{cor}\label{cor:1:1.3}
Ein dicht definierter Operator $T$ besitzt eine Rechtsinverse $S$ genau dann, wenn $T^*$ beschränkt invertierbar ist.
\end{cor}
\begin{proof}
Wenn $T$ eine Rechtsinverse $S$ besitzt, impliziert $RS=I$ schon $\mathcal R_T=H$ und mit obigem Satz folgt die beschränkte Invertierbarkeit von $T^*$. Gilt umgekehrt \eqref{eq:1:1.6}, so ist $S=T^*(TT^*)^{-1}$ das gesuchte. Da $T^*(TT^*)^{-1/2}$ eine Isometrie ist, ist $S$ stetig.
\end{proof}

\section{Differentialoperatoren} 
Wir nutzen im folgenden Multiindexschreibweise. Für Multiindices $\alpha,\beta\in\mathbb N_0^n$ sei
\begin{equation}
|\alpha|=\sum_{j=1}^n \alpha_j,\qquad \alpha! = \prod_{j=1}^n \alpha_j!,\qquad (\alpha+\beta)_j=\alpha_j+\beta_j.
\end{equation} 
Weiter sei zu $\zeta\in\C^n$ durch
\begin{equation}
   \zeta^\alpha = \prod_{j=1}^n \zeta_j^{\alpha_j}
\end{equation}
das Monom, sowie durch
\begin{equation}
   \D^\alpha = \prod_{j=1}^n \left(- \i \frac{\partial}{\partial x_j}\right)^{\alpha_j}
\end{equation}
ein formaler Differentialoperator definiert. Sei nun $\Omega\subseteq\R^n$ ein Gebiet. Ein Differentialoperator der Ordnung $m$ ist ein formaler Ausdruck der Form
\begin{equation}
   \mathcal P = \sum_{|\alpha|\le m} a_\alpha(x) \D^\alpha
\end{equation}
mit Koeffizientenfunktionen $a_\alpha\in \rmC^\infty(\Omega)$. Er agiert in nat\"urlicher Weise auf $u\in\rmC^\infty_0(\Omega)$ (oder auf $u\in\mathscr D'(\Omega)$ in distributionellem Sinne).

Versieht man $\rmC_0^\infty(\Omega)$ durch
\begin{equation}
    \spro{u}{v} = \int_{\Omega} u(x) \overline{v(x)} \d x
\end{equation}
mit einem $\rmL^2$-Innenprodukt, so kann man zu $\mathcal P$ den formal adjungierten Operator $\overline{\mathcal P}$ betrachten. Dieser erf\"ullt
\begin{equation}
   \forall u,v\in\rmC_0^\infty(\Omega)\quad:\quad \spro{\mathcal Pu}{v}=\spro{u}{\mathcal{\overline P}v}
\end{equation}
und ist durch
\begin{equation}
  \overline{ \mathcal P}v(x) = \sum_{|\alpha|\le m}  \D^\alpha\big(\overline{a_\alpha(x)}v(x)\big)
\end{equation}
gegeben.

\begin{lem}
Der Differentialoperator $\mathcal P$ versehen mit Definitionsbereich
\begin{equation}
  \mathcal D= \{ u\in\rmC^\infty(\Omega) \cap \rmL^2(\Omega) :  \mathcal Pu\in\rmL^2(\Omega) \}
\end{equation}
ist $\rmL^2$-$\rmL^2$-abschließbar.
\end{lem}
\begin{proof}
Sei $u_n\in\mathcal D$ eine Folge mit $u_n\to 0$ in $\rmL^2(\Omega)$ und $\mathcal Pu_n \to w\in\rmL^2(\Omega)$. Dann gilt f\"ur jedes 
$v\in\rmC^\infty_0(\Omega)$
\begin{equation}
 \spro{w}{v}=\lim_{n\to\infty}   \spro{\mathcal Pu_n}{v} = \lim_{n\to\infty} \spro{u_n}{\overline{\mathcal P}v} = 0
\end{equation}
somit $w=0$.
\end{proof}

Die Aussage gilt allgemeiner, es ergibt sich $\rmL^p$-$\rmL^q$-, $\rmC$-$\rmC$ sowie $\rmL^p$-$\rmC$-Abschließbarkeit bei entsprechend gewähltem $\mathcal D$. Insbesondere ist $\mathcal P$ mit Definitionsbereich $\rmC^\infty_0(\Omega)$ abschließbar als Operator auf $\rmL^2(\Omega)$. 

\begin{df} 
Sei $\mathcal P$ ein Differentialausdruck und $\Omega$ ein Gebiet. Dann wird der $\rmL^2$-$\rmL^2$-Abschluß des durch $\mathcal P$ auf $\rmC_0^\infty(\Omega)$ definierten Operators mit $P_0$ und als \eIndex[Differentialoperator]{minimaler Operator} zum Differentialausdruck $\mathcal P$ und Gebiet $\Omega$ bezeichnet. 
Weiterhin heißt $P := (\overline P_0)^*$ \eIndex[Differentialoperator]{maximaler Operator} zu $\mathcal P$ und $\Omega$.
\end{df}

Der so definierte maximale Operator besitzt den Definitionsbereich
\begin{equation}
   \mathcal D_P = \{ u\in\rmL^2(\Omega) : \mathcal Pu \in\rmL^2(\Omega)\},
\end{equation}
wobei die Anwendung von $\mathcal P$ im distributionellen Sinne zu verstehen ist. Wir betrachten ein einfaches Beispiel. Dazu sei $\Omega=(a,b)\subseteq\R$ ein Intervall und $\mathcal P = \D^2 = - \partial^2$ die Zuordnung der zweiten Ableitung. Dann besitzt der minimale Operator den Definitionsbereich
\begin{equation}
    \{ u\in\rmH^2(\Omega) :  u(a)=\partial u(a)=u(b)=\partial u(b) = 0\} = \rmH^2_0(\Omega),
\end{equation}
also den Sobolevraum $\rmH^2_0(\Omega)$ der am Rand verschwindenden Funktionen, und der maximale Operator gerade den gesamten Sobolevraum $\rmH^2(\Omega)$ als Definitionsbereich. Im Falle h\"oherer Raumdimensionen wird der Definitionsbereich des maximalen Operators in der Regel echt größer als der Sobolevraum passender Ordnung sein.
Der Unterschied zwischen minimalen und maximalen Operatoren besteht in der Wahl von Randbedingungen. 

\begin{thm}
Die Gleichung $Pu=f$ hat genau dann für jedes $f\in\rmL^2(\Omega)$ eine L\"osung $u\in\mathcal D_{P}$, wenn $\overline P_0$ beschränkt invertierbar ist, also wenn f\"ur den formal adjungierten Differentialausdruck
\begin{equation}
  \forall u\in\rmC^\infty_0(\Omega)\quad:\quad   \|u\| \le C \| \overline{\mathcal P} u \|
\end{equation}
gilt.
\end{thm}
\begin{proof}
Folgt aus Satz~\ref{thm:1:1.2} in Verbindung mit der Definition des maximalen Operators $P$.
\end{proof}

Solche und allgemeinere Ungleichungen stehen im Zusammenhang mit Enthaltenseinsbeziehungen zwischen Definitionsbereichen minimaler Operatoren. Dazu definieren wir zuerst: 
\begin{df}\label{df:1:1.2}
Seien $\mathcal P$ und $\mathcal Q$ Differentialausdrücke und $\Omega$ ein Gebiet. Gilt dann $\mathcal D_{P_0}\subseteq \mathcal D_{Q_0}$, so heiße $\mathcal P$ \eIndex[Differentialoperator]{stärker} als $\mathcal Q$ bzw. $\mathcal Q$ \eIndex[Differentialoperator]{schwächer} als $\mathcal P$. Gilt $\mathcal D_{P_0}=\mathcal D_{Q_0}$, so heißen beide \eIndex[Differentialoperator]{gleich stark}.
\end{df}
Die  Definition hängt im Falle konstanter Koeffizienten nicht von der Wahl des Gebietes ab. Im Falle variabler Koeffizienten (siehe später) sind die Gebiete hinreichend klein zu wählen um eine sinnvolle Definition erhalten.

\begin{lem}\label{lm: Qu stetig}
Seien $\mathcal P$ und $\mathcal Q$ Differentialausdr\"ucke auf einem Gebiet $\Omega$. 
\begin{enumerate}
\item
$\mathcal Q$ ist genau dann schwächer als $\mathcal P$, wenn
\begin{equation}\label{eq:1:1.21}
    \forall u\in\rmC^\infty_0(\Omega)\quad:\quad \|\mathcal Qu\|^2 \le C \big( \|\mathcal Pu\|^2 + \|u\|^2 \big)
\end{equation}
gilt.
\item
$Q_0 u$ besitzt für jedes $u\in\mathcal D_{P_0}$ genau dann einen stetigen Repräsentanten, wenn
\begin{equation}
    \forall u\in\rmC^\infty_0(\Omega)\quad:\quad  \sup_{x\in\Omega} |\mathcal Qu(x) |^2 \le C \big( \|\mathcal Pu\|^2 + \|u\|^2 \big)
\end{equation}
gilt.
\end{enumerate}
\end{lem}
\begin{proof}
{\bf (i)}\quad Die Hinrichtung folgt direkt aus Satz~\ref{thm:1:1.1}. Für die Rückrichtung nehmen wir an, die Ungleichung gilt. Dann existiert zu $u\in\mathcal D_{P_0}$ eine Folge $u_n\in\rmC_0^\infty(\Omega)$ mit $u_n\to u$ und $\mathcal P u_n\to P_0 u$ in $\rmL^2(\Omega)$. Dann ist aber $\mathcal Q(u_n-u_m)$ Cauchy und da $Q_0$ abgeschlossen ist folgt $u\in\mathcal D_{Q_0}$. $\bullet$\qquad
{\bf (ii)}\quad Für die Hinrichtung betrachtet man $\mathcal Q$ als Operator $\mathcal G_{P_0}\to\rmL^\infty(\Omega)$ und wendet Satz~\ref{thm:1:1.1} an. Für die Rückrichtung sei $u\in\mathcal D_{P_0}$ beliebig und $u_n\in\rmC_0^\infty(\Omega)$ eine Folge mit $u_n\to u$ und $\mathcal Pu_n\to P_0u$ in $\rmL^2(\Omega)$.
Dann konvergiert $\mathcal Qu_n$ gleichmäßig und Behauptung folgt.
\end{proof}

\section{Randwertprobleme}
Sei im folgenden $\mathcal P$ und $\Omega$ fixiert. Dann sind  $\mathcal G_P$ und $\mathcal G_{P_0}$ abgeschlossene Teilräume von $\rmL^2\times\rmL^2$. Da
$\mathcal G_{P_0}\subseteq \mathcal G_P$ gilt, kann man den Quotientenraum
\begin{equation}
   \mathcal C = \mathcal G_P / \mathcal G_{P_0}
\end{equation}
betrachten. Dieser wird als \eIndex[Differentialoperator]{Cauchyraum} des Differentialausdrucks $\mathcal P$ auf dem Gebiet $\Omega$ bezeichnet. Für $u\in\mathcal D_P$ sei
\begin{equation}
     \Gamma u := [ (u,Pu) ]_{\mathcal G_{P_0}} \in \mathcal C
\end{equation}
das zugeordnete Cauchydatum. Man kann sich $\Gamma u$ als eine Charakterisierung der Randwerte von $u$ auf $\Omega$ vorstellen, unterscheiden sich zwei Funktionen $u$ und $v$ nur in einer kompakten Teilmenge $\Omega' \Subset \Omega$, so gilt $\Gamma u=\Gamma v$.
Sei nun $B\subseteq\mathcal C$ ein linearer Unterraum. Dann heißt 
\begin{equation}
    Pu = f,\qquad \Gamma u\in B,
\end{equation}
f\"ur gegebenes $f\in\rmL^2(\Omega)$ das zugeordnete \eIndex{Randwertproblem} mit homogener \eIndex[Randwertproblem]{Randbedingung} $\Gamma u\in B$. Die Randbedingung 
$B$ bestimmt damit eine Einschr\"ankung $\widehat P$ des Operators $P$ auf den Definitionsbereich
\begin{equation}
    \mathcal D_{\widehat P} = \{ u \in \mathcal D_P : \Gamma u\in B \}.
\end{equation}
Der so definierte Operator ist abgeschlossen genau dann, wenn $B$ abgeschlossen ist. Die Menge der abgeschlossenen Operatoren $\widehat P$ mit $P_0\subseteq\widehat P\subseteq P$ steht in Bijektion zu den abgeschlossenen Teilräumen des Cauchyraumes $\mathcal C$.
Das Randwertproblem heißt \eIndex[Randwertproblem]{korrekt gestellt}, falls $\widehat P$ stetig invertierbar ist.

\begin{thm}[Vi\v{s}ik]
Zu einem Differentialausdruck $\mathcal P$ existieren auf einem Gebiet $\Omega$ genau dann korrekt gestellte Randwertprobleme, wenn $P_0$ und $\overline P_0$
beschr\"ankt invertierbar sind.
\end{thm} 
\begin{proof} 
Angenommen, es existiert ein korrekt gestelltes homogenes Randwertproblem $\widehat P$. Da $P_0\subseteq \widehat P$ gilt, muß dann $P_0^{-1}$ beschränkt sein. Weiterhin ist $\widehat P^{-1}$ rechtsinvers zu $P$, damit muß $\overline P_0^{-1}$ nach Korollar~\ref{cor:1:1.3} beschränkt sein.

Seien nun $P_0$ und $\overline P_0$ beschränkt invertierbar. Da $P_0^{-1}$ beschr\"ankt ist, ist $\mathcal R_{P_0}\subseteq \rmL^2(\Omega)$ abgeschlossen.  Bezeichne nun $\pi$ den Orthogonalprojektor auf $\mathcal R_{P_0}$. Ist nun $S$ eine Rechtsinverse zu $P$ (die nach  Korollar~\ref{cor:1:1.3} existiert), so gilt für
den durch  
\begin{equation}
  Tf = P_0^{-1} \pi f + S(I-\pi) f,\qquad f\in\rmL^2(\Omega) 
\end{equation}
definierten Operator $T$ nach Konstruktion $PT f = \pi f + (I-\pi) f = f$ und $T$ besitzt eine Inverse $\widehat P \subseteq P$. Weiter ist $T\supset P_0^{-1}$ und $\widehat P\supset P_0$ , so daß $P_0\subseteq \widehat P\subseteq P$. Da $\widehat P^{-1}$ beschränkt und auf ganz $\rmL^2(\Omega)$ definiert ist, ist der Satz bewiesen.
\end{proof}

Seien nun $P_0$ und $\overline P_0$ beschränkt invertierbar und bezeichne 
\begin{equation}
  U=\ker P = \{u\in\rmL^2(\Omega) : Pu =0\}
\end{equation}  
den Kern des maximalen Operators. Dann ist $\Gamma U\subseteq \mathcal C$ abgeschlossen und die Einschränkung $\gamma:=\Gamma|_U$ wegen
\begin{multline}
  \|u\|^2\ge \|\Gamma u\|^2 = \inf_{w\in \mathcal D_{P_0}} (\|u-w\|^2 + \|Pu - Pw\|^2) \\   \ge \inf_{w\in \rmL^2} (\|u-w\|^2 + C^{-2} \|w\|^2) = \frac{C^{-2}}{1+C^{-2}}\|u\|^2
\end{multline}
ein Isomorphismus. 
\begin{thm}
Seien $P_0$ und $\overline P_0$ beschränkt invertierbar und sei $B\subseteq\mathcal C$ abgeschlossener Teilraum. Das zugehörige homogene Randwertproblem $\widehat P$ ist genau dann korrekt gestellt, wenn $\mathcal C = B \dotplus \Gamma U $ als direkte Summe\footnote{Direkt, nicht notwendig orthogonal.} gilt.
\end{thm}
\begin{proof}
Angenommen, $\mathcal C = B \dotplus \Gamma U$. Dann ist der Lösungsoperator über eine Rechtsinverse $S$ zu $P$ und den Projektor $\pi\in\mathcal L(\mathcal C)$ auf  $\Gamma U$ entlang $B$ durch
\begin{equation}
     Tf = Sf-\gamma^{-1}\pi\Gamma Sf
\end{equation}
ausdrückbar, da für alle $f\in\rmL^2(\Omega)$ nach Konstruktion $PTf=f-P \gamma^{-1}\pi\Gamma S f=f-0=f$ sowie $\Gamma T f=\Gamma Sf - \Gamma\gamma^{-1}\pi \Gamma Sf=(I-\pi)\Gamma Sf\in B$ gilt und somit $T=\widehat P^{-1}$ stetig ist.  

Ist umgekehrt $\widehat P^{-1}$ die stetige Inverse zu $\widehat P$, so induziert
$\mathcal G_P\ni(v,Pv) \mapsto \Gamma(v-\widehat P^{-1}Pv)\in\Gamma U$ (was offenbar auf $\mathcal G_{P_0}$ verschwindet und auf $\Gamma U$ die Identität ist) einen Projektor $\pi\in\mathcal L(\mathcal C)$. Da weiterhin $v\in\mathcal D_{\widehat P}$ genau dann gilt, wenn $v=\widehat P^{-1}Pv$ erfüllt ist, folgt $B=\ker\pi$.
\end{proof}

\section{Differentialoperatoren mit konstanten Koeffizienten}
Im folgenden sollen vorerst nur Operatoren mit konstanten Koeffizienten betrachtet werden. Diese gehören zu Differentialausdrücken der Form
\begin{equation}
     P(\D) = \sum_{|\alpha|\le m} a_\alpha \D^\alpha
\end{equation}
mit $a_\alpha\in\C$. Für diese gilt 
\begin{equation}
     P(\D) \e^{\i x\cdot\zeta}  =  P(\zeta) \e^{\i x\cdot\zeta}
\end{equation}
für alle $\zeta\in\C^n$ und mit $x\cdot\zeta = \sum_{j=1}^n x_j\zeta_j$. Das Polynom $ P(\zeta)$ heißt das \eIndex[Differentialoperator]{Symbol} des Differentialausdrucks $\mathcal P(\D)$. Ist nun $\Omega\subseteq\R^n$ beschränkt und $u\in\rmC_0^\infty(\Omega)$, so ist die Fourier--Laplace-transformierte von $u$
\begin{equation}
    \widehat u (\zeta) = (2\pi)^{-n/2} \int_\Omega \e^{-\i x\cdot\zeta } u(x) \d x,\qquad \zeta\in\C^n
\end{equation}
eine ganze Funktion auf $\C^n$. Die Anwendung des Operators $ P(\D)$ entspricht im Fourierbild der Multiplikation mit $ P(\zeta)$. 

Eine erste Anwendung ist folgender Satz. Ein Beweis ergibt sich später nochmals in allgemeinerer Form.

\begin{thm}\label{thm:1:1.8}
Der Operator $P_0$ ist beschränkt invertierbar, es existiert also eine Konstante $C$ mit
\begin{equation}
   \forall u\in\rmC_0^\infty(\Omega)\quad:\quad \|u\|\le C\| P(\D) u\|.
\end{equation}
\end{thm}

Der Beweis basiert auf folgendem Lemma der Funktionentheorie. 
\begin{lem}
Sei $g\in\mathfrak A(\overline{\mathbb D})$ analytisch in einer Umgebung der Kreisscheibe $|z|\le 1$  und $r$ ein Polynom mit höchstem Koeffizienten $A$. Dann gilt
\begin{equation}
   |Ag(0)|^2 \le (2\pi)^{-1} \int_0^{2\pi} |g(\e^{\i\theta}) r(\e^{\i\theta})|^2 \d\theta.
\end{equation}
\end{lem}
\begin{proof}
Seien $z_j$ die Nullstellen von $r$ innerhalb der Kreisscheibe $|z|<1$ und
\begin{equation}
    r(z) = q(z) \prod_{j} \frac{z_j-z}{\overline{z_j}z-1}.
\end{equation} 
Dann gilt auf der Kreislinie $|r(z)|=|q(z)|$ und $q(z)$ ist analytisch in $\mathbb D$. Also folgt
\begin{equation}
  (2\pi)^{-1} \int  |g(\e^{\i\theta}) r(\e^{\i\theta})|^2 \d\theta = (2\pi)^{-1} \int  |g(\e^{\i\theta}) q(\e^{\i\theta})|^2 \d\theta \ge |g(0)q(0)|^2.
\end{equation}
Nun ist aber $q(0)/A$ gerade das Produkt der Nullstellen von $r(z)$ außerhalb des Einheitskreises und damit $|q(0)|\ge |A|$ und das Lemma folgt.
\end{proof}

\begin{proof}[Beweis zu Satz~\ref{thm:1:1.8}]
Bezeichne $p(\zeta)$ den homogenen Hauptteil von $P(\zeta)$. Sei weiter $\xi_0\in\R^n$ mit $p(\xi_0)\ne0$. Wendet man nun obiges Lemma auf die Funktion
$\widehat u(\zeta+ t\xi_0)$ und das Polynom $P(\zeta+t\xi_0)$ (verstanden als Funktionen von $t$) an, so erhält man
\begin{equation}
   |\widehat u(\zeta) p(\xi_0)|^2 \le (2\pi)^{-1} \int |\widehat u(\zeta+\e^{\i\theta} \xi_0) P(\zeta+\e^{\i\theta}\xi_0)|^2 \d\theta.
\end{equation}
Speziell mit $\zeta=\xi\in\R^n$ und Integration bezüglich $\xi$ ergibt sich
\begin{align}
   |p(\xi_0)|^2 \int |\widehat u(\xi)|^2 \d\xi &\le (2\pi)^{-1} \iint |\widehat u(\xi+\e^{\i\theta}\xi_0) P(\xi+\e^{\i\theta}\xi_0)|^2 \d\xi\d\theta \notag\\
   &= (2\pi)^{-1} \iint |\widehat u(\xi+\i\xi_0\sin\theta) P(\xi+\i\xi_0\sin\theta)|^2 \d\xi\d\theta
\end{align} 
und unter Ausnutzung der Parseval-Identität
\begin{equation}
   |p(\xi_0)|^2 \int |u(x)|^2\d x \le (2\pi)^{-1} \iint |P(D) u(x)|^2 \e^{2x\cdot\xi_0 \sin\theta} \d x\d\theta
\end{equation}
und mit $C=\sup_{x\in\Omega} \e^{|x\cdot\xi_0|} /|p(\xi_0)|$ folgt die Behauptung.
\end{proof}

\begin{cor}\label{cor:1:1.10}
\begin{enumerate}
\item
Maximale Differentialoperatoren mit konstanten Koeffizienten sind auf jedem beschränkten Gebiet surjektiv; zu jedem $f\in\rmL^2(\Omega)$ existiert ein $u\in\mathcal D_P$ mit $Pu=f$.
\item
Für Differentialoperatoren mit konstanten Koeffizienten existieren
zu jedem beschränkten Gebiet korrekt gestellte Randwertprobleme.
\end{enumerate}
\end{cor}
 % Einleitung, Teil 1
% !TEX root = main.tex
\chapter{Minimale Operatoren} % Minimale Operatoren :: Tilmann Kleiner, Andreas Bitter
% !TEX root = main.tex
\chapter{Maximale Operatoren}

% Voraussichtliche Struktur
\section{Differentialoperatoren von lokalem Typ} % bearbeitet durch Thomas Hamm


\section{Konstruktion von Fundamentallösungen eines vollständigen Operators von lokalen Typ} % bearbeitet von Matthias Hofmann
 % Maximale Operatoren :: Thomas Hamm, Matthias Hofmann
% !TEX root = main.tex
\chapter{Operatoren mit variablen Koeffizienten}

Zur Notation: Sei $\Omega\subset\R^n$ ein Gebiet und bezeichne weiterhin 
\begin{equation}
    \|u\|_{(m)} = \sum_{|\alpha|\le m} \|\D^\alpha u\|
\end{equation}
die $m$-te Sobolevnorm einer Funktion $u\in\rmC_0^\infty(\Omega)$. Im folgenden betrachten wir mininmale Operatoren zu Differentialausdrücken
\begin{equation}
   P(x,\D) = \sum_{|\alpha|\le m} a_\alpha(x) \D^\alpha
\end{equation}
mit Koeffizienten $a_\alpha\in\rmC^\infty(\Omega)$. Offensichtlich gilt $\|P(x,\D)u\|\le C\|u\|_{(m)}$ für alle $u\in\rmC_0^\infty(\Omega)$ mit einer nur von der Größe der Koeffizienten abhängenden Konstanten $C$.
Wir definieren \eIndex[Differentialoperator]{Symbol} und \eIndex[Differentialoperator]{Hauptsymbol}
\begin{equation}
   P(x,\zeta) = \sum_{|\alpha|\le m} a_\alpha(x)\zeta^\alpha,\qquad p(x,\zeta)=\sum_{|\alpha|=m} a_\alpha(x)\zeta^\alpha
\end{equation}
als Polynome auf $\C^n$ mit Koeffizienten aus $\rmC^\infty(\Omega)$. Analog zu den schon für den Fall konstanter Koeffizienten eingeführten Bezeichnungen nennen wir den Differentialausdruck $P(x,\D)$ im Punkt $x\in\Omega$
\begin{itemize}
\item \eIndex[Differentialoperator]{elliptisch}, falls für alle (reellen) $\xi\in\R^n\setminus\{0\}$ das Hauptsymbol $p(x,\xi)\ne0$ erfüllt;
\item \eIndex[Differentialoperator]{vom Haupttyp}, falls die Nullstellenmenge des Hauptsymbols 
\begin{equation}
    p(x,\xi) = 0 \quad \Longrightarrow \quad \nabla_\xi p(x,\xi)\ne0
\end{equation}
für alle $\xi\in\R^n\setminus\{0\}$ erfüllt.
\end{itemize}
Während die für Operatoren mit konstanten Koeffizienten gezeigten Aussagen und Abschätzungen im wesentlichen unabhängig vom Gebiet waren, 
treten bei Operatoren mit variablen Koeffizienten neue Effekte auf. Betrachtet man jedoch hinreichend {\em kleine} Gebiete, so kann man Abschätzungen auf den Fall konstanter Koeffizienten zurückführen. Das soll am Beispiel der Elliptizitätsabschätzung
\begin{equation}\label{eq:4:locEll}
\forall u\in\rmC_0^\infty(\omega)\quad:\quad    \| u\|_{(m)} \le C \| P(x,\D) u \|
\end{equation}
und der Haupttypabschätzung 
\begin{equation}\label{eq:4:locHT}
\forall u\in\rmC_0^\infty(\omega)\quad:\quad   \| u\|_{(m-1)} \le C \| P(x,\D) u \|
\end{equation}
diskutiert werden. Hierbei sei $x_0\in\Omega$ fest gewählt und $\omega\Subset\Omega$ eine hinreichend kleine Umgebung von $x_0$. Die Größe von $\omega$ hängt vom Verhalten der Koeffizienten ab. Die Gültigkeit einer solchen lokalen Abschätzung für jedes $x_0\in\Omega$ impliziert offenbar das entsprechende Resultat auf jeder relativ kompakten Teilmenge $\Omega'\Subset\Omega$. Konstanten hängen von der Wahl von $\omega$ beziehungsweise $\Omega'$ ab.



\section{Hinreichende Bedingungen}


\begin{thm}
Angenommen, $P(x,\D)$ ist elliptisch. Dann existiert zu jedem $x_0\in\Omega$ eine Umgebung $\omega\Subset\Omega$, so dass
\eqref{eq:4:locEll} mit einer von $\omega$ abhängigen Konstanten $C$ gilt. 
\end{thm}
\begin{proof}
Ohne Beschränkung der Allgemeinheit nehmen wir an der Ursprung liege im Gebiet und es gelte $x_0=0$. Sei weiter $P(\D)=P(0,\D)=\sum_\alpha a_\alpha(0)\D^\alpha$ der Operator der entsteht, wenn man in die Koeffizienten den Wert $0$ einsetzt. Dann gilt für alle $u\in\rmC_0^\infty(\Omega)$
\begin{equation}
 \|u\|_{(m)} \le C  \| P(\D) u\| 
\end{equation}
da $P(0,\D)$ elliptischer Operator mit konstanten Koeffizienten ist. Weiter existiert zu jedem $\epsilon>0$ ein $\delta>0$, so dass
$|a_\alpha(x)-a_\alpha(0)|\le \epsilon$ für $|x|\le\delta$. Damit impliziert die Dreiecksungleichung
\begin{equation}
  \| P(\D) u - P(x,\D) u\| \le \epsilon \sum_{|\alpha|\le m} \|\D^\alpha u\| = \epsilon \|u\|_{(m)}
\end{equation}
für alle $u\in\rmC_0^\infty(B_\delta)$. Für $\epsilon$ klein genug folgt damit die Behauptung.
\end{proof}

Ein Operator $P(x,\D)$ wird als  \eIndex[Differentialoperator]{von reellem Haupttyp} bezeichnet, falls sein Hauptsymbol $p(x,\xi)$ reellwertig ist. Unter der Voraussetzung stimmen $P(x,\D)$ und sein formal adjungierter ${}^tP(x,\D)$ bis auf einen Operator der Ordnung $m-1$ überein. Allgemeiner heißt $P(x,\D)$ \eIndex[Differentialoperator]{wesentlich normal}, falls die  \eIndex{Poissonklammer} des Hauptsymbols $p$ mit $\overline p$ definiert durch $\overline p(x, \overline\zeta) = \overline{p(x,\zeta)}$
\begin{equation}
    \{ p,\overline p\} (x,\xi) = \sum_{j=1}^n \bigg(\frac{\partial p(x,\xi)}{\partial \xi_j} \frac{\partial \overline p (x,\xi)}{\partial x_j} - \frac{\partial \overline p(x,\xi)}{\partial \xi_j}\frac{\partial  p(x,\xi)}{\partial x_j} \bigg)    = 0 
\end{equation}
für alle $\xi\in\R^n$ verschwindet. In diesem Fall ist der Kommutator von $P(x,\D)$ und ${}^tP(x,\D)$ von Ordnung $2m-2$ (statt nur $2m-1$).

Das folgende Resultat ergibt sich aus \cite[Theorem~4.1]{Hormander:1955} zusammen mit der Vorbemerkung zu diesem Kapitel. Die Umkehrung dieses Satzes gilt nicht.

\begin{thm}[{\cite[Theorem~4.1]{Hormander:1955}}]
Angenommen, $P(x,\D)$ ist vom Haupttyp und wesentlich normal. Dann existiert zu jedem $x_0\in\Omega$ eine Umgebung $\omega\Subset\Omega$, so dass
\eqref{eq:4:locHT} mit einer von $\omega$ abhängigen Konstanten $C$ gilt.
\end{thm}
\begin{proof}[Beweisskizze]
Für den Originalbeweis konstruiert man wiederum Energieintegrale und schätzt diese mittels partieller Integration ab.
\end{proof}

\section{Notwendige Bedingungen}
Im folgenden sei $\Omega \subset \R^n$ eine offene Umgebung der Null und $P(x,\D)$ ein Differentialausdruck auf $\Omega$ mit Hauptsymbol $p(x,\xi)$. 
\begin{lem}\label{lem1}
Angenommen, es existiert eine Funktion $u \in \rmC^\infty (\Omega)$ mit 
\begin{align}
\label{grad}
p(x, \nabla u) = \mathbf{o} \big(|x|^2\big), \qquad  x \rightarrow 0,
\end{align}
welche eine Taylorentwicklung 
\begin{align}\label{taylor}
u(x) = \i \xi\cdot x + x\cdot A x + \mathcal O\big(|x|^3\big),\qquad x\to0
%\sum_{j=1}^{n} x_j \xi_j + \dfrac{1}{2} \sum_{j,k=1}^{n} x_j x_k \alpha_{jk} + \mathcal O\left( |x|^3 \right), \qquad x\to0
\end{align}
mit $\xi\in\R^n$, einer symmetrischen Matrix $A=(\boldsymbol\alpha_{i,j})_{i,j}\in\C^{n\times n}$ und $B = (\Re \boldsymbol\alpha_{i,j})_{i,j} $ negativ definit  besitzt. 
Gilt weiterhin  
\begin{align}
\label{normpart}
\sum_{j=1}^{n}\bigg|\frac{\partial p(0,\xi)}{\partial \xi_j}\bigg|^2 \neq 0,
\end{align}
so folgt
\begin{align}
\label{lm 1 eq}
\sup_{v\in\rmC_0^\infty(\Omega)} \frac{ \lVert v \rVert_{(m-1)} }{ \lVert P(x,D)v \rVert} = \infty.
\end{align}
\end{lem}
\begin{proof}
Da die Matrix $B$ negativ definit ist, gilt
\begin{align*}
\Re u(x) = x\cdot Bx + \mathcal O\big(|x|^3\big) \le - 2 a |x|^2 + \mathcal O\big(|x|^3\big),\qquad x\to0
% \sum_{j,k=1}^{n} x^jx^k \Re \alpha_{jk} + \mathcal{O}\left( |x|^3 \right) \le -2a |x|^2 + \mathcal{O}\left( |x|^3 \right) (?)
\end{align*}
mit geeignet gewähltem $a>0$. Damit folgt
\begin{align}
\label{realteil}
\Re u(x) \le - a |x|^2 
\end{align}
für hinreichend kleine $|x|$. Nach Verkleinerung von $\Omega$ können wir also ohne Einschränkung annehmen, dass (\ref{realteil}) überall in $\Omega$ gilt. 

Sei nun $\phi \in \rmC_0^\infty(\Omega)$ und $t>0$. Dann erfüllt $v_t := \phi e^{tu} \in \rmC_0^\infty(\Omega)$ die Abschätzung $|v_t(x)| \le |\phi(x)| \e^{-at|x|^2}$. Wir zeigen nun
\[
\frac{\lVert v_t \rVert_{(m-1)}}{\lVert P(x,\D)v_t \rVert} \longrightarrow \infty, \qquad t \rightarrow \infty,
\]
woraus die Behauptung (\ref{lm 1 eq}) folgt.

Wir schreiben 
\[
P(x,D)v_t = e^{tu} \sum_{j=0}^{m} b_j(x) t^j.
\]
Es reicht, $b_m$ und $b_{m-1}$ zu berechnen. Mit der Leibnizformel erhalten wir
\[
P(x,D)(\phi e^{tu}) = \sum_{\alpha} \dfrac{D^\alpha \phi}{|\alpha|!} P^{(\alpha)}(x,D)e^{tu}.
\]
Wir zerlegen
\[
P(x,D) = p(x,D) + q(x,D) + r(x,D),
\]
wobei $p$ und $q$ homogen von Grad $m$ bzw. $m-1$ seien sowie $r$ von Grad $<m-1$. Dann sind die Koeffizienten von $t^m e^{tu}$ und $t^{m-1} e^{tu}$ in
\begin{align}
\phi p(x,D)e^{tu} + \left( \phi q(x,D) + \sum_{j=1}^{n} (D^j \phi) p^{(j)}(x,D) \right) e^{tu}
\end{align}
ebenfalls $b_m$ und $b_{m-1}$, denn alle anderen Summanden in der Leibnizformel haben Grad kleiner als $m-1$ in $t$. Nach Berechnung des ersten Summanden erhalten wir $b_m = (-i)^m \phi p(x,\nabla u)$ und insbesondere wegen (\ref{grad})
\begin{align}
b_m(x) = \textbf{o} (|x|^2), \hspace{0.5cm} x \rightarrow 0.
\end{align}
Weiter erhalten wir
\begin{align}
	\label{4.17}
b_{m-1} = (-i)^{m-1} \left( \phi (i^{m-1}g + q(x,\nabla u)) + \sum_{j=1}^{n} (D^j \phi) p^{(j)} (x,\nabla u) \right),
\end{align}
wobei $g$ eine glatte Funktion ist, die von $u$ und den Koeffizienten von $p$ abhängt. Wir wählen nun $\phi$ so, dass $\phi(0) =1$ gilt und sodass (\ref{4.17}) für $x=0$ verschwindet, was wegen (\ref{normpart}) möglich ist. Wegen Stetigkeit von $b_{m-1}$ ist dann $b_{m-1}(x) = \textbf{o}(1)$, wenn $x$ gegen Null strebt.

Nach diesen Überlegungen können wir für $\varepsilon >0$ eine Umgebung $U$ der Null wählen, sodass
\[
|b_m(x)| < \varepsilon |x|^2, \hspace{0.5cm} |b_{m-1}(x)| < \varepsilon, \hspace{0.5cm} \forall \, x \in U.
\]
Mit Gleichung (\ref{realteil}) erhalten wir für alle $x\in U$ die Abschätzung
\[
|P(x,D)v_t| \le t^{m-1} e^{-at|x|^2}(\varepsilon t |x|^2 + \varepsilon + C/t)
\]
und mit der Transformation $x \mapsto x/\sqrt{(t)}$ im Integral folgt
\[
\lVert P(x,D) v_t \rVert_{L^2(U)} ^2 \le t^{2m-2-n/2} \varepsilon^2 \int_U (|x|^2 + 1 +C/\varepsilon t)^2 e^{-2a|x|^2} dx,
\]
wobei das letzte Integral für $t \rightarrow \infty$ gegen
\[
B^2 = \int_U (|x|^2 + 1)^2 e^{-2a|x|^2} dx
\]
konvergiert. Für große $t$ liefert dies die Abschätzung
\[
\lVert P(x,D) v_t \rVert_{L^2(U)} ^2 \le 2 t^{2m-2-n/2} \varepsilon^2  B^2.
\]
Da aus (\ref{realteil}) ebenso
\[
\lVert P(x,D) v_t \rVert_{L^2(U^c)} ^2 = \mathcal{O}(t^{2m}e^{-2ct})
\]
für eine Konstante $c>0$ folgt, erhalten wir
\begin{align}
	\label{4.18}
\lVert P(x,D) v_t \rVert^2 \le 4 B^2 t^{2m-2-n/2} \varepsilon^2.
\end{align}
Als schätzen wir $\lVert v_t \rVert_{(m-1)}$ nach unten ab. Wir nehmen ohne Einschräung an, dass $\xi_1\neq 0$ gilt. Wir setzen $\alpha' = (m-1,0,\ldots,0)$ und erhalten
\[
D^{\alpha'} v_t = (\phi (D^1 u)^{m-1}t^{m-1}+ \ldots)e^{tu}.
\]
Wegen $\phi(0) (D^1u(0))^{m-1} = \xi_1^{m-1} \neq 0$ gilt $|\phi| |D^1u|^{m-1} \ge 2c > 0$ für eine Konstante $c > 0$ in einer Umgebung der Null und wegen $\Re u(x) = \mathcal{O} (|x|^2)$ existiert eine Konstante $A > 0$ sodass $\Re u(x) \ge -A |x|^2$. Für große $t$ erhalten wir
\[
|\phi| |D^1u|^{m-1} \ge ct^{m-1}e^{-tA|x|^2}.
\]
Für große $t$ folgt
\begin{align}
	\label{4.19}
\lVert v_t \rVert_{(m-1)}^2 \ge \lVert D^{\alpha'} v_t\rVert^2 \ge \int\limits_{t|x|^2 < 1}c^2t^{2m-21}e^{-2tA|x|^2} dx = C^2 t^{2m-2-n/2}
\end{align}
für eine Konstante $C > 0$ und eine Umgebung $V$ der Null. Kombiniert man (\ref{4.18}) und (\ref{4.19}), so erhält man für große $t$
\[
\dfrac{\lVert v_t \rVert_{(m-1)}}{\lVert P(x,D) v_t \rVert} \ge \dfrac{C}{2B\varepsilon}
\]
und da $\varepsilon>0$ beliebig war folgt die Behauptung des Lemmas.
\end{proof}

\begin{lem}\label{lem2}
Wir nehmen an, es gelte (\ref{normpart}). Dann gibt es eine Funktion $u \in \rmC^\infty(\Omega)$ mit (\ref{grad}) und (\ref{taylor}) genau dann, wenn
\begin{align}
	\label{4.15}
p(0, \xi) &= 0, \\ 	\label{4.16}
\sum_{k=1}^{n} \boldsymbol\alpha_{j,k}\frac{\partial p(0,\xi)}{\partial \xi_k} &= -\i\,  p_j(0,\xi), \hspace{0.5cm} j=1, \ldots, n, 
\end{align}
wobei $ p_j(x,\xi) = \partial_{x_j}  p(x,\xi)$ die partielle Ableitung des Hauptsymbols bezeichne.
\end{lem}

\begin{proof} [Beweisskizze]
Betrachten wir die Taylorentwicklung, so stellen wir fest, dass (\ref{grad}) genau dann erfüllt ist, wenn $p(x,\nabla u)$ und alle Ableitungen der Ordnung $\le 2$ im Nullpunkt verschwinden. Das Verschwinden von $p(x,\nabla u)$ wird gerade durch (\ref{4.15}) beschrieben. Wir berechnen weiter
\[
\dfrac{\partial}{\partial x_j} p(0,\nabla u) = 
p_j(0,i\xi) + \sum_{k=1}^{n} \boldsymbol\alpha_{j,k} \dfrac{\partial p(0,i\xi)}{\partial \xi_k}
\]
und bemerken unter Verwendung der Homogenität von $p$, dass die linke Seite genau dann für alle $j$ verschwindet, wenn (\ref{4.16}) gilt. Es bleibt zu zeigen, dass für geeignet gewähltes $u$ unter den gegebenen Bedingungen auch die zweiten Ableitungen verschwinden. Hierzu leitet man die obige Gleichung ein weiteres Mal ab und stellt unter Verwendung von (\ref{normpart}) fest, dass das resultierende System partieller Differentialgleichungen die Voraussetzungen des Satzes von Cauchy-Kovalevsky erfüllt, woraus die Existenz eines solchen $u$ folgt.
\end{proof}

\begin{lem}\label{lem3}
Seien $\zeta, f \in \C^n$ mit $\zeta \neq 0$. Dann gibt es genau dann eine symmetrische Matrix $A = (\boldsymbol\alpha_{i,j})_{i,j}$ mit negativ definitem Realteil
$B=(\Re\boldsymbol \alpha_{i,j})_{i,j}$ und
\begin{align}\label{laag1}
   A \zeta = f,
\end{align}
wenn die Vektoren $\zeta$ und $f$
\begin{align} \label{laag2}
\Re f\cdot \overline \zeta < 0
\end{align}
erfüllen.
\end{lem}
\begin{proof}
Es gelte (\ref{laag1}). Multiplikation beider Seiten mit $\overline \zeta_k$ und Aufaddieren ergibt unter Verwendung der Symmetrie von $A$
und mit $\zeta=\xi+\i\eta$
\[
 f\cdot \overline \zeta = \sum_{k=1}^{n}f_k \overline{\zeta_k} = \sum_{j,k=1}^{n} \boldsymbol\alpha_{kj} \zeta_j \overline \zeta_k
=  \sum_{j,k=1}^{n} \boldsymbol\alpha_{kj} \xi_j \xi_k +  \sum_{j,k=1}^{n} \boldsymbol\alpha_{kj} \eta_j \eta_k,
\]
Da für mindestens ein $j$ ein $\xi_j$ oder $\eta_j$ ungleich Null ist, folgt (\ref{laag2}) wegen der Definitheit von $B$. $\bullet$\qquad 
Es gelte nun umgekehrt (\ref{laag2}). Wir nehmen zunächst an, $\zeta$ ist proportional zu einem reellen Vektor. Nach Multiplikation von $f$ und $\zeta$ mit derselben komplexen Zahl können wir dann annehmen, dass $\zeta=\xi$ reell ist. Mit $A = B + \i C$, $f= g+\i h$ wird aus (\ref{laag1}) das Gleichungssystem bestehend aus $B \xi = g$ und $C \xi = h$. Dass eine reelle, symmetrische Matrix $C$ mit $C \xi = h$ existiert, ist offensichtlich. Schreiben wir weiter $g = g' + \xi (g\cdot \xi)/ 2 |\xi|^2$, so gilt $g'\cdot\xi = g\cdot\xi /2 < 0$. Man rechnet dann leicht nach, dass die Matrix $B$, gegeben durch
\[
B x = \dfrac{g\cdot\xi }{2|\xi|^2} x + \dfrac{x\cdot g'}{\xi\cdot g'} g',\qquad x\in\R^n,
\]
negativ definit und symmetrisch ist und $A=B+\i C$ erfüllt das gesuchte. Sei nun $\zeta$ nicht proportional zu einem reellen Vektor. Wir zeigen, dass 
\[
A = \dfrac{\Re( f\cdot\overline\zeta )}{|\zeta|^2} \mathrm I + \i C
\]
für ein reelles $C\in\R^{n\times n}$ die gewünschten Eigenschaften erfüllt. Die Bedingung an $C$ schreibt sich dann als
\begin{align}
	\i C \zeta = f',
\end{align}
wobei $f' = f - \zeta \frac{\Re( f\cdot\overline\zeta)}{|\zeta|^2}$ die Eigenschaft
\begin{align}
	\label{Ebene}
\Re (f'\cdot\overline\zeta) = 0
\end{align}
hat. Um die Existenz eines solchen $C$ zu zeigen, stellen wir zunächst fest, dass die Menge der Vektoren in $\C^n$, die als $\i C \zeta$, $C \in \R^{n \times n}$ symmetrisch, geschrieben werden können, einen reellen Vektorraum bilden. Dieser ist für ein $g \in \C^n$ in der Hyperebene $\{ z \in \C^n : \Re g\cdot \overline z = 0 \}$ enthalten. Für jedes $\xi \in \R^n$ ist die Matrix $C$, gegeben durch $C x =  (x\cdot\xi)\xi$, symmetrisch und es folgt
\[
\Re \i (\xi\cdot\overline g)(\zeta\cdot\xi) = 0.
\]
Insbesondere ist $(\xi\cdot\overline g)(\zeta\cdot\xi) $ immer reell. Unter Verwendung, dass $\zeta$ nicht proportional zu einem reellen Vektor ist, berechnet man schnell, dass $g$ ein reelles Vielfaches von $\zeta$ sein muss. Insbesondere ist $\Re(z\cdot\overline\zeta) = 0 \Leftrightarrow \Re(z\cdot \overline g) = 0$ und $f'$ lässt sich wegen (\ref{Ebene}) als $\i C \zeta$ schreiben für eine reelle, symmetrische Matrix $C$.  Es folgt die Behauptung.
\end{proof}

\begin{thm}
%Es seien die Koeffizienten von $P(x,\D)$ stetig und die Koeffizienten von $p$ seien $C^2$. Weiter 
Es gelte die Ungleichung (\ref{eq:4:locHT}), d.h.
\[
  \| u\|_{(m-1)} \le C \| P(x,\D) u \| \quad \forall u\in\rmC_0^\infty(\Omega).
\]
Dann folgt 
\begin{align}
	\label{Beh}
\{p,\overline p\} (x,\xi) = 0,
\end{align}
falls $p(x,\xi) = 0$, $x \in \Omega$ und $ \xi \in \R^n$.
\end{thm}
\begin{proof}
Wir nehmen ohne Einschränkung $x=0$ an. Weiterhin gelte
\[
\sum_{j=1}^{n}|\partial p(0,\xi)/\partial \xi_j|^2 \neq 0,
\]
denn sonst ist (\ref{Beh}) trivialerweise erfüllt. Die Lemmata \ref{lem1} und \ref{lem2} zeigen dann, dass die Gleichung (\ref{4.16}) für keine symmetrische Matrix mit negativ definitem Realteil erfüllt sein kann. Mit Lemma \ref{lem3} folgt deshalb $\{p,\overline p\}(x,\xi) \ge 0$. Das analoge Argument mit $-\xi$ anstatt $\xi$ liefert $\{p,\overline p\}(x,-\xi) \ge 0$. Weil $\{p,\overline p\}(x,\xi)$ eine ungerade Funktion in $\xi$ ist, folgt die Behauptung.
\end{proof} % Operatoren von reellem Haupttyp :: Julian Mauersberger
% !TEX root = main.tex
%\chapter{Ein unlösbarer Operator}
%\cite{Lewy:1957}
%\cite{Hormander:1960b}


\section{Das Beispiel von Lewy}


Bis Mitte des 20. Jahrhunderts ging man davon aus, dass die lokale Existenz glatter Lösungen für lineare Differentialgleichungen stets gegeben sei. Das Beispiel von Hans Lewy zeigt eindrucksvoll, dass dies nicht zwingend der Fall sein muss.  In diesem Abschnitt soll eine lineare, partielle Differentialgleichung erster Ordnung in drei Variablen mit komplexwertigen $\rmC^\infty$-Koeffizienten vorgestellt werden, welche in \emph{keiner} offenen Menge eine glatte/distributionelle Lösung besitzt. Hierfür betrachtet man den durch
\begin{equation}\label{lewy:differentialausdruck}
\mathscr L := -\frac{\partial}{\partial x_1} -\i\frac{\partial}{\partial x_2} +2\i(x_1+\i x_2)\frac{\partial}{\partial y_1}
\end{equation}
definierten Differentialausdruck auf $\R^3$. Das erste überraschende Resultat von Lewy ist in folgendem Lemma enthalten:
\begin{lem}[{\cite{Lewy:1957}}]\label{thm:1_lewy}
Zu einer reellwertigen Funktion $\psi\in \rmC^1(\mathbb{R})$ besitze das Problem
\begin{equation}\label{eq:1_lewy:gleichung}
\mathscr Lu=\psi'(y_1)
\end{equation}
in einer Umgebung $\Omega\subset\R^3$ von $(0,0,y_1^0)$ eine $\rmC^1$-Lösung $u$. Dann ist $\psi$ analytisch in $y_1=y_1^0$.
\end{lem}

\begin{proof}
Wir integrieren $(\partial_1+\i\partial_2) u$ für eine Lösung $u$ von \eqref{eq:1_lewy:gleichung} über einen Kreis in der $x_1$-$x_2$ Ebene um den Punkt $(0,0,y_1^0)$. Der Radius wird dabei so klein gewählt, dass der Kreis in $\Omega$ liegt. Sei dazu
\begin{equation}
x_1^2+x_2^2=y_2=\mathrm{const},\qquad y_1=\mathrm{const},
\end{equation}
$t=\log\sqrt{y_2}=\log\sqrt{x_1^2+x_2^2}$ und $\theta$ derjenige Winkel gegeben durch
\begin{equation}
x_1+\i x_2=\sqrt{y_2} \, \e^{\i\theta} =\e^{t+\i\theta}.
\end{equation}
Dann erhält man durch einfaches Nachrechnen $\overline x \overline\partial_x = \overline\partial_t$, also
\begin{equation}\label{thm:1_lewy:proof1}
\frac{\partial}{\partial x_1} +\i\frac{\partial}{\partial x_2}=
\e^{-t+\i\theta} \left(\frac{\partial}{\partial t}
+\i\frac{\partial}{\partial \theta}\right).
\end{equation}
Diese Identität zusammen mit partieller Integration liefert
\begin{align}\label{thm:1_lewy:abl_gleich}
\begin{split}
\int_0^{2\pi} \left(\frac{\partial}{\partial x_1} +\i\frac{\partial}{\partial x_2}\right)u\d\theta 
&= \int_0^{2\pi} \e^{-t+\i\theta} \left(\frac{\partial}{\partial \,t}+\i\frac{\partial}{\partial \theta}\right)u \d\theta \\
&= \int_0^{2\pi} \e^{-t+\i\theta} \left(\frac{\partial u}{\partial \,t} +\i\frac{\partial u}{\partial \theta}\right)\d\theta \\
&= \int_0^{2\pi} \e^{-t+\i\theta} \left(\frac{\partial u}{\partial \,t} +u\right)\d\theta .
\end{split}
\end{align}
Weiter impliziert $\sqrt{y_2}=\e^t$ für jede differenzierbare Funktion $w$
\begin{align}\label{thm:1_lewy:int_gleichheit1}
\begin{split}
\frac{\partial w}{\partial\, t}+ w  = 2\sqrt{y_2} \frac{\partial}{\partial y_2} \left( \sqrt{y_2} w(\log\sqrt{y_2})\right).
\end{split}
\end{align}
Eingesetzt in das letzte Integral aus \eqref{thm:1_lewy:abl_gleich} ergibt sich
\begin{align}\label{thm:1_lewy:abl_gleich_final}
\begin{split}
\int_0^{2\pi} \left(\frac{\partial}{\partial x_1} +\i\frac{\partial}{\partial x_2}\right)u\d\theta 
= 2\left(\frac{\partial}{\partial y_2}\right)\int_0^{2\pi} \e^{\i\theta}\sqrt{y_2}u\d\theta.
\end{split}
\end{align}
Setzen wir nun 
\begin{equation}
I(y_1,y_2)=\i\int_0^{2\pi} \e^{\i\theta} \sqrt{y_2} u\d\theta,
\end{equation}
so liefert \eqref{eq:1_lewy:gleichung} und danach \eqref{thm:1_lewy:abl_gleich_final}
\begin{align}
\begin{split}
\frac{\partial I}{\partial y_1} +\i\frac{\partial I}{\partial y_2} 
&= \i\int_{0}^{2\pi} \frac{\partial}{\partial y_1}\left(\e^{\i\theta}\sqrt{y_2}u\right)+\i\frac{\partial}{\partial y_2}\left(\e^{\i\theta}\sqrt{y_2}u\right)\d\theta \\
&= \i\int_{0}^{2\pi} \e^{\i\theta}\sqrt{y_2}\left(\frac{\partial}{\partial y_1}u\right) +\i\e^{\i\theta}\frac{\partial}{\partial y_2}(\sqrt{y_2}u)\d\theta\\
&= \i\int_{0}^{2\pi} \e^{\i\theta}\sqrt{y_2} \left(
	\frac{\psi'(y_1)}{2\i(x_1+\i x_2)}
	+ \frac{\left(\frac{\partial}{\partial x_1} + \i\frac{\partial}{\partial x_2}\right)u}{2\i(x_1+\i x_2)} 
	+ \frac{\i}{\sqrt{y_2}} \frac{\partial}{\partial y_2}(\sqrt{y_2} u)
\right)\d\theta \\
&= \i\int_{0}^{2\pi} \e^{\i\theta}\sqrt{y_2} \left( 
	\frac{\psi'(y_1)}{2\i\sqrt{y_2}\e^{\i\theta}} 
	+ \frac{\left(\frac{\partial}{\partial x_1} + \i\frac{\partial}{\partial x_2}\right)u}{2\i\sqrt{y_2}\e^{\i\theta}}
	+ \frac{\i}{\sqrt{y_2}} \frac{\partial}{\partial y_2}(\sqrt{y_2} u)
\right)\d\theta	 \\
&= \int_0^{2\pi} \frac{\psi'(y_1)}{2}\d\theta 
	+ \frac{1}{2}\int_0^{2\pi} \left(\frac{\partial}{\partial x_1} +\i \frac{\partial}{\partial x_2}\right)u\d\theta
	- \int_0^{2\pi} \e^{\i\theta}\frac{\partial}{\partial y_2}(\sqrt{y_2} u) \d\theta \\
&= 2\pi\frac{\psi'(y_1)}{2}
	+  \left(\frac{\partial}{\partial y_2}\right)\int_0^{2\pi} \e^{\i\theta}\sqrt{y_2}u\d\theta
	- \left(\frac{\partial}{\partial y_2}\right) \int_0^{2\pi} \e^{\i\theta}\sqrt{y_2}u\d\theta\\
&= \pi\psi'(y_1).
\end{split}
\end{align}
Ferner ist
\begin{equation}
J(y):=J(y_1,y_2):=I(y_1,y_2)-\pi\psi(y_1),
\end{equation}
eine $\rmC^1$-Funktion. Nach Konstruktion erfüllt diese die Cauchy-Riemannschen Differentialgleichungen
\begin{equation}
\frac{\partial J}{\partial y_1} +\i\frac{\partial J}{\partial y_2} = 0
\end{equation}
 und ist somit analytisch in $y=y_1+\i y_2$. Ihr Definitionsbereich enthält alle $y=(y_1,y_2)$ mit $y_2>0$ hinreichend klein und $y_1$ nahe $y_1^0$.
 Da weiter für $y_2=0$ nach Konstruktion $I(y_1,0)=0$ gilt, folgt
\begin{align*}
 J(y_1,0)=-\pi\psi'(y_1)
\end{align*}
mit nach Voraussetzung reellem $\psi$ und $J$ kann mithilfe des Spiegelungsprinzips in die untere komplexe Halbebene analytisch fortgesetzt werden. Somit ist $\psi'(y_1)$ 
und damit auch $\psi(y_1)$ analytisch in einer Umgebung des Punktes $y_1=y_1^0$ und die Behauptung ist gezeigt.
\end{proof}

Betrachtet man nun \eqref{eq:1_lewy:gleichung} mit reellwertigem $\psi\in \rmC^\infty(\R)$, welches \textit{nicht} analytisch in $y_1=y_1^0$ ist, so liefert die Kontraposition des obigen Lemmas, dass \eqref{eq:1_lewy:gleichung} in keiner Umgebung $\Omega$ von $(0,0,y_1^0)$ eine Lösung $u\in \rmC^1(\Omega)$ besitzen kann. Lewy nutzte obiges Lemma um eine Funktion $f\in\rmC^\infty(\R)$ zu konstruieren, so dass
\begin{equation} 
   \mathscr L u = f(y_1)
\end{equation}
in {\em keinem} Gebiet $\Omega\subset\R^3$ eine Lösung in einem Hölderraum $\rmC^{1,\alpha}(\Omega)$ besitzen kann.




\section{Hörmanders Unlösbarkeitskriterium}
Hörmander zeigte in \cite{Hormander:1960a}, dass für jedes Gebiet $\Omega\subset\R^n$ eine glatte Funktion $f\in\rmC^\infty(\R^n)$ existiert, für welche keine Distribution $u\in\mathscr{D}'(\Omega)$ mit $\mathscr Lu=f$ existiert. Dazu zeigte er eine notwendige Bedingung für die Lösbarkeit eines Differentialausdrucks erster Ordnung. In \cite{Hormander:1960b} verallgemeinerte er diese noch auf Operatoren höherer Ordnung.

Zuerst zeigen wir ein Analogon zu Satz~\ref{thm:4:Umkehrung}, welches es erlaubt von Lösbarkeitsaussagen auf das Verschwinden der Poissonklammer 
von $p$ und $\overline p$ zu schließen. 

\begin{thm}[{\cite[Theorem 1]{Hormander:1960b}}]\label{thm:3.1_hoer}
Sei $\Omega\subset\mathbb{R}^n$ ein Gebiet und $P(x,\D)$ ein Differentialausdruck der Ordnung $m$ mit Hauptsymbol $p(x,\xi)$.
Angenommen, für jedes $f\in \rmC_0^\infty(\Omega)$  existiert eine distributionelle Lösung $u\in\mathscr D'(\Omega)$ zu
\begin{equation}\label{eq:3.1_hoer}
P(x,\D)u=f.
\end{equation}
Dann gilt für alle $x\in\Omega$ und $\xi\in\mathbb{R}^n$ mit $p(x,\xi)=0$
\begin{equation}\label{eq:3.1_hoer_aussage}
 \{p,\overline{p}\}(x,\xi)=0.
\end{equation}
\end{thm}
\begin{proof}
Wir zeigen dies indirekt, beginnen mit einer Umformulierung der Voraussetzung als Ungleichung und konstruieren danach geeignete Testfunktionen um einen Widerspruch herzuleiten. 
\noindent {\sl Schritt 1.}  
 Bezeichne $\rmB^\infty_0(\omega)=\{f\in\rmC^\infty(\Omega) \mid \supp f\subseteq\overline\omega\}$ und 
\begin{equation}\label{eq:M_N}
   M_N = \{ f\in \rmB^\infty_0(\omega)\mid \exists_{u\in\mathscr D'(\omega)} \; \forall_{\psi\in\rmC_0^\infty(\omega)}\; |\langle u,\psi\rangle | \le N \|\psi\|_{(N)} \quad\text{und}\quad P(x,\D)u=f \}.
\end{equation}
Für jedes $u\in\mathscr D'(\Omega)$ und $\omega\Subset\Omega$ gilt $ |\langle u,\psi\rangle | \le N \|\psi\|_{(N)}$ für alle $\psi\in\rmC_0^\infty(\omega)$ und ein hinreichend großes $N$. Die Voraussetzung impliziert also
\begin{equation}
   \bigcup_{N=1}^\infty M_N = \rmB^\infty_0(\omega).
\end{equation} 
Die Mengen $M_N$ sind abgeschlossen (siehe nachfolgendes Lemma \ref{lm:closed}),  konvex und symmetrisch. Weiter ist $\rmB^\infty_0(\omega)$ metrisierbar, der Bairesche Kategoriensatz impliziert also, dass mindestens eine der Mengen $M_N$ einen inneren Punkt (wegen Symmetrie den Ursprung) besitzt. Es gibt also ein $k$ und $\epsilon>0$, so dass
\begin{equation}\label{eq:fepsInMN}
    \{ f\in \rmB^\infty_0(\omega)\mid \|f\|_{(k)}<\epsilon\} \subset M_N
\end{equation}
für ein $N$ gilt.
%
$\bullet$\qquad {\sl Schritt 2.} 
Angenommen, die Behauptung gilt nicht. Sei also ohne Beschränkung der Allgemeinheit $0\in\Omega$ und gelte für ein $\xi\in\R^n$ sowohl $p(0,\xi)=0$ als auch\footnote{Da $\{p,\overline p\}$ reellwertig und ungerade in $\xi$ ist, erfüllt für $\{p,\overline p\}(0,\xi)\ne0$ entweder $\xi$ oder $-\xi$ die Bedingung $\{p,\overline p\}(0,\xi)<0$.}   $\{p,\overline p\}(0,\xi)<0$. Analog zu Lemma~\ref{thm:4:lem2} folgt die Existenz einer Funktion $u\in\rmC^\infty(\Omega)$ mit 
\begin{equation}
    p(x,\nabla u(x)) = \mathcal O(|x|^q),\qquad x\to0
\end{equation}
sowie
\begin{equation}
   u(x) = \i \xi\cdot x + x\cdot Ax + \mathcal O(|x|^3),\qquad x\to0 
\end{equation}
zu vorgegebener Ordnung $q$, obigem Vektor $\xi$ und geeignet (mit Lemma~\ref{thm:4:lem3}) gewählter symmetrischer Matrix $A$ mit negativ definitem Realteil. 
%
$\bullet$\qquad {\sl Schritt~3.} Wir definieren die Hilfsfunktionen
\begin{equation}
   f_{\tau, k}(x) = \tau^{-k} \chi(\tau x),\qquad \text{wobei $\chi\in\rmC_0^\infty(\R)$ mit}\quad \widehat \chi(-\xi) = (2\pi)^{-n/2}  \int \e^{\i x\cdot\xi} \chi(x) \d x \ne 0,
\end{equation}
sowie für $q=2(r+1)$, $r=n+k+m+N$ und dem oben konstruierten $u(x)$ die Hilfsfunktionen
\begin{equation}
   v_{\tau,k,N} (x) = \tau^{n+1+k} \e^{\tau u(x)} \sum_{j=0}^{r-1} \tau^{-j} \varphi_j(x) 
\end{equation}
mit noch zu bestimmenden Koeffizienten $\varphi_j\in\rmC_0^\infty(\Omega)$. Nach Konstruktion gilt für die Sobolevnormen
\begin{equation}\label{eq:cond1}
    \limsup_{\tau\to\infty} \| f_{k,\tau}\|_{(k)} < \infty,
\end{equation}
und für $\varphi_0(0)=1$ folgt
\begin{equation}\label{eq:cond3}
   \frac1\tau \int f_{\tau,k}(x) v_{\tau,k,N}(x)\d x = \int \e^{\tau u(\frac x\tau)} \chi(x) \sum_{j=0}^{r-1} \varphi_j(\frac x\tau) \tau^{-j} \d x \longrightarrow \int \e^{\i x\cdot\xi}\chi(x)\d x\ne0
\end{equation}
für $\tau\to\infty$. Die verbleibenden $\varphi_j$ werden so gewählt, dass für den transponierten Differentialausdruck
\begin{equation}\label{eq:cond2}
   \limsup_{\tau\to\infty} \|{}^tP(x,\D) v_{\tau,k,N}\|_{(N)} <\infty 
\end{equation}
gilt. Dazu nutzt man, dass 
\begin{equation}
{}^tP(x,\D) v_{\tau,k,N}(x)=\tau^{n+1+k+m}  \e^{\tau u(x)} \sum_{j=0}^m \tau^{-j} a_j(x) 
\end{equation}
mit Koeffizienten $a_j\in\rmC_0^\infty(\Omega)$ gilt und wählt $\varphi_j$ so, dass $a_j(x)=\mathcal O(|x|^{q-2j})$ (was auf eine erneute Anwendung des Satzes von Cauchy--Kowalewskaja hinausläuft). Danach zeigt eine einfache Rechnung, dass für jedes $\psi\in\rmC_0^\infty(\omega)$ mit $\psi(x)=\mathcal O(|x|^{2s})$, $x\to0$, $s\geq 0$
und $\omega\Subset\Omega$ hinreichend klein stets die Normabschätzung $\limsup_{\tau\to\infty} \|\tau^{s-N}\psi \e^{\tau u}\|_{(N)}<\infty$ gilt und \eqref{eq:cond2} folgt.~$\bullet$ \qquad {\sl Schritt~4.} Wir zeigen, dass \eqref{eq:fepsInMN} im Widerspruch zur Existenz der gerade konstruierten Hilfsfunktionen steht. 
Zum Einen impliziert \eqref{eq:fepsInMN} die Existenz einer Distribution $u$ mit $P(x,\D)u=f$ für $f\in \rmB^\infty_0(\omega)$ mit $\|f\|_{(k)}<\epsilon$. Damit gilt für jede Testfunktion $v\in\rmC_0^\infty(\omega)$ die Abschätzung
\begin{equation}\label{eq:4.74}
   \bigg| \int f(x) v(x) \d x\bigg| = |\langle u, {}^t P(x,\D) v \rangle| \le N \| {}^t P(x,\D) v\|_{(N)}.
\end{equation}
Speziell mit $f=c f_{k,\tau}$ und $v=v_{k,N,\tau}$ folgt für $\tau\to\infty$ ein Widerspruch: Für $c>0$ klein genug impliziert \eqref{eq:fepsInMN} zusammen mit \eqref{eq:cond1}, dass $f\in M_N$. Also gilt \eqref{eq:4.74}. Nun ist die rechte Seite aber wegen \eqref{eq:cond2} gleichmäßig in $\tau$ beschränkt, die linke strebt mit \eqref{eq:cond3} für $\tau\to\infty$  gegen Unendlich. Widerspruch.
\end{proof}


\begin{rem}
Die im obigen Beweis verwendete Menge $\rmB^\infty_0(\omega)$ ist gerade der Abschluss von $\rmC_0^\infty(\omega)$ in $\rmC^\infty(\Omega)$
(also auch in $\rmC^\infty(\R^n)$). Damit kann man für jedes Gebiet $\Omega\subset\R^n$ den entsprechenden Raum auch als
\begin{equation}
\rmB_0^\infty(\Omega) =\{f\in \rmC^\infty(\Omega) \mid \forall_{\alpha\in\N_0^n}\;\forall_{\epsilon >0} \;\exists_{K_{\alpha,\epsilon}\Subset\Omega} \;:\; \sup\nolimits_{x\in \Omega\setminus K_{\alpha,\epsilon}}|\D^\alpha f(x)|<\epsilon \}
\end{equation}
charakterisieren. Weiter sei 
\begin{equation}
\rmB^\infty(\Omega) = \{ f\in\rmC^\infty(\Omega) \mid  \forall_{\alpha\in\N_0^n} \;:\; \sup\nolimits_{x\in\Omega} |\D^\alpha f(x)|<\infty \}.
\end{equation}
Nach Konstruktion ist $\rmB^\infty_0(\Omega)$ abgeschlossener Teilraum von $\rmB^\infty(\Omega)$ und für $\omega\Subset\Omega$ ist die Einschränkung auf $\omega$ surjektiv von $\rmB^\infty_0(\Omega)$ auf $\rmB^\infty(\omega)$. Das nachfolgende Lemma liefert damit insbesondere die im Beweis verwendete Abgeschlossenheit der Mengen $M_N$.
\end{rem}


\begin{lem}\label{lm:closed}
Sei $\omega\Subset\Omega$ und
\begin{equation}
M_N:= \{f\in \rmB^\infty(\omega)\mid \exists_{u\in\mathscr D'(\omega)}\; \forall_{\psi\in\rmC_0^\infty(\omega)}\; |\langle u,\psi\rangle | \le N \|\psi\|_{(N)} \quad\text{und}\quad P(x,\mathrm{D})u=f\quad \text{in $\omega$} \}.
\end{equation}
Dann ist $M_N\subset \rmB^\infty(\omega)$ abgeschlossen.
\end{lem}

\begin{proof}
Die Menge der Distributionen $u\in\mathscr D'(\omega)$, welche die  Bedingung 
\begin{equation}\label{lewy:uglSchwartz}
|\langle u,\psi\rangle|\leq N\|\psi\|_{(N)} 
\end{equation}
für alle $\psi\in\rmC_0^\infty(\omega)$ erfüllen, ist folgenkompakt. Wir zeigen dies mit einem Diagonalfolgenargument.
Sei dazu $(u_n)_{n\in\N}$ eine Folge in $\mathscr D'(\omega)$ mit dieser Schranke. Sei weiter $(\psi_j)_{j\in\N}$ eine Folge
von $\rmC_0^\infty(\omega)$ Funktionen, die in $\rmH^N_0(\omega)$ (also dem Abschluss von $\rmC_0^\infty(\omega)$ in der $\|\cdot\|_{(N)}$-Norm) dicht ist.
Diese Existiert wegen der Separabilität von $\rmH_0^N(\omega)$.

Da die Folge $\langle u_n,\psi_1\rangle$ in $\C$ beschränkt ist, existiert eine Teilfolge $(u_{n_k}^{(1)})_{k\in\mathbb{N}}$ derart, dass
$\langle u_{n_k}^{(1)},\psi_1\rangle$ konvergiert. Weiter finden wir auch eine Teilfolge $(u_{n_k}^{(2)})_{k\in\mathbb{N}}$ von $u_{n_k}^{(1)}$, so dass $\langle u_{n_k}^{(2)},\psi_2\rangle$ konvergiert, etc. Wählen wir nun die Diagonalfolge $v_k = u_{n_k}^{(k)}$, so konvergiert $\langle v_k,\psi_j\rangle$ nach Konstruktion für alle $j$. 
Also konvergiert wegen der vorausgesetzten Schranke \eqref{lewy:uglSchwartz} die Folge   $\langle v_k,\psi\rangle$  für alle $\psi\in\rmH^N_0(\omega)$ und da $\rmC_0^\infty(\omega)\subset \rmH_0^N(\omega)$ gilt in $\mathscr D'(\omega)$. Der Grenzwert erfüllt offenbar ebenfalls \eqref{lewy:uglSchwartz}.

Sei also $(f_n)_{n\in\mathbb{N}}$ eine gegen $f\in \rmB^\infty(\omega)$ konvergente Folge aus $M_N$. Dann gilt  $f_n=P(x,\D)u_n$ für gewisse $u_n\in\mathscr{D}'(\omega)$ mit  $|\langle u,\psi\rangle | \le N \|\psi\|_{(N)}$ für alle $\psi\in\rmC_0^\infty(\omega)$. Auf Grund der gerade gezeigten Folgenkompaktheit existiert eine in $\mathscr D'(\omega)$  konvergente Teilfolge $(u_{n_k})_{k\in\N}$. Sei nun $u=\lim_{k\rightarrow\infty}u_{n_k}$. Dann erfüllt $u$ ebenfalls  
 \eqref{lewy:uglSchwartz} und da $P(x,\D)u_{n_k}\to P(x,\D)u$ in $\mathscr D'(\omega)$ konvergiert, folgt $P(x,\D)u=f$.
\end{proof}
\begin{thm}[{\cite[Theorem 2]{Hormander:1960b}}]\label{thm:2_hoer}
Sei $P(x,\D)$ ein Differentialausdruck der Ordnung $m$ mit  Hauptsymbol $p(x,\xi)$ und existiere zu jedem $\omega\Subset\Omega$ 
ein $x\in\omega$ und ein $\xi\in\R^n$ mit
\begin{equation}
 \{p,\overline p\}(x,\xi)\ne 0.
\end{equation}
Dann existieren Funktionen $f\in \rmB_0^\infty(\Omega)$, so dass
\begin{equation}\label{lewy:pxDu=f}
P(x,\mathrm D)u=f
\end{equation}
in keiner der Mengen $\omega\subseteq\Omega$ eine Lösung $u\in\mathscr D'(\omega)$ besitzt. Die Menge dieser Funktionen $f$ ist von zweiter Kategorie\footnote{Eine Teilmenge $A$ eines topologischen Raumes $B$ heißt von erster Kategorie, falls eine abzählbare Menge nirgends dichter Teilmengen aus $B$ existiert, deren Vereinigung $A$ ergibt. Ist dies nicht der Fall, so heißt $A$ von zweiter Kategorie.}.
\end{thm}

\begin{proof} Der Beweis folgt {\cite[Theorem 3.2]{Hormander:1960a}}.
{\sl Schritt 1.}
Sei zunächst $\omega\subseteq\Omega$ eine feste, nichtleere Menge und $M$ definiert durch
\begin{equation}
M=\{f\in \rmB_0^\infty(\Omega)\mid \exists\,u\in\mathscr{D}'(\omega) : P(x,\mathrm{D})u=f\quad \text{auf $\omega$}\}.
\end{equation}
Wir zeigen, dass $M$ von erster Kategorie ist.  Sei hierzu $\omega_1\Subset\omega$ offen und nichtleer. Dann existiert für jede Distribution $u\in\mathscr{D}'(\omega)$ ein $N\in\mathbb{N}$, so dass $u$ die Ungleichung \eqref{lewy:uglSchwartz}
für alle $\psi\in \rmC_0^\infty(\omega_1)$ erfüllt ist. Seien nun wieder Mengen $M_N$ durch
\begin{equation}
M_N:= \{f\in \rmB^\infty(\omega_1)\mid \exists\,u\in\mathscr D'(\omega_1): P(x,\mathrm{D})u=f\;\mathrm{in}\;\omega_1\;\wedge\; u\mathrm{\;erf"ullt\;}\eqref{lewy:uglSchwartz}\},
\end{equation}
gegeben. Diese sind nach Lemma~\ref{lm:closed} abgeschlossen, offenbar konvex und auch symmetrisch. 
Keines der $M_N$ besitzt einen inneren Punkt. 
Anderenfalls gäbe es wiederum ein $N$, ein $k$ und ein $\epsilon>0$, so dass alle $f\in \rmB^\infty(\omega_1)$ mit
$\|f\|_{(k)}<\epsilon$ in $M_N$ liegen. Also gäbe es zu jedem $f\in\rmC_0^\infty(\omega_1)$ eine Konstante $\delta\ne0$, so dass $P(x,\D)u=\delta f$ in $\mathscr D'(\omega_1)$ eine L\"osung besitzt. Da aber dann $\delta^{-1}u$ schon $P(x,\D)u=f$ löst und $f$ beliebig war, widerspricht dies Satz~\ref{thm:3.1_hoer}. 

Gleiches trifft auch auf ihre Urbilder $\widetilde M_N$ unter der stetigen Einschränkung $\rmB^\infty_0(\Omega)\to \rmB^\infty(\omega_1)$ zu. Insbesondere sind diese Mengen abgeschlossen und besitzen keine inneren Punkte. Also ist $\widetilde M_N$ von erster Kategorie und ebenso die abzählbare Vereinigung dieser Mengen. Da aber $M\subset \bigcup_N \widetilde M_N$ gilt, folgt die  Behauptung.
$\bullet$\qquad {\sl Schritt 2.}
Wir zeigen nun, dass \eqref{lewy:pxDu=f} tatsächlich keine Lösung in $\Omega$ besitzt. Sei dazu $(\omega_j)_{j\in\mathbb{N}}$ eine abzählbare Basis der Topologie von $\Omega$ und bezeichne
\begin{equation}
M^{(j)}:=\{f\in \rmB_0^\infty(\Omega)\mid \exists u\in\mathscr{D}'(\omega_j): P(x,\mathrm{D})u=f\;\mathrm{auf\;}\omega_j\}.
\end{equation}
Dann folgt aus Schritt 1, dass alle $M^{(j)}$ und folglich auch $\bigcup M^{(j)}$ von erster Kategorie sind. Nach Definition der $M^{(j)}$ kann aber $P(x,\mathrm{D})u=f$ für $f\not\in  \bigcup M^{(j)}$ auf keinem $\omega_j$ gelöst werden. Da für jede beliebige, offene, nichtleere Menge $\omega\subseteq\Omega$ ein Index $j_0$ existiert, so dass $\omega_{j_0}\subset\omega$ gilt, besitzt $P(x,\mathrm{D})u=f$ auf keiner solchen Menge $\omega$ eine Lösung. 
\end{proof}



\begin{exa}
Für Lewys Beispiel 
\begin{equation}
p(x,\xi)=-i\xi_1+\xi_2-2(x_1+ix_2)\xi_3
\end{equation}
in $n=3$ Dimensionen aus dem ersten Teil dieses Kapitels erhalten wir $\{p,\overline{p}\}(x,\xi)=-8\xi_3$. Wegen
\begin{equation}
\xi_3 =1,\quad \xi_1=-2x_2,\quad \xi_2=2x_1\qquad\Rightarrow\qquad p(x,\xi)=0
\end{equation}
ist $\{p,\overline{p}\}(x,\xi)=-8\neq 0$ und \eqref{eq:3.1_hoer_aussage} gilt nicht für alle $x\in\mathbb{R}^n$ womit folglich die Voraussetzungen von Satz \ref{thm:2_hoer} erfüllt sind.
\end{exa} % Ein unloesbarer Operator :: Robin Lang
%
%
\part{Untere Schranken an Pseudodifferentialoperatoren}
%
% !TEX root = main.tex
\chapter{Einleitung}
Hier sollen die Grundlagen dafür gelegt werden, auch Operatoren mit variablen Koeffizienten richtig behandeln zu können. Die Darstellung basiert auf \cite{Hormander:1965}, \cite{Hormander:1966}, sowie \cite[Kapitel 18]{Hormander:1985}. Zuerst verallgemeinern wir den Begriff des Differentialoperators soweit, daß wir auch Inverse und allgemeinere Funktionen von solchen Operatoren behandeln können. 

\section{Operatoren und Symbole}
Sei $\Omega$ eine Mannigfaltigkeit. Differentialoperatoren auf $\Omega$ können dann in lokalen Karten definiert werden oder global durch ihre Eigenschaften charakterisiert werden. Die bekannteste davon ist, daß eine stetige lineare Abbildung $P:\rmC_0^\infty(\Omega)\to\rmC^\infty(\Omega)$ genau dann Differentialoperator ist, wenn $P$ lokal ist, also wenn
\begin{equation}
  \forall f\in\rmC_0^\infty(\Omega)\quad:\quad \supp Pf \subseteq \supp f
\end{equation}
gilt. Für Verallgemeinerungen brauchbarer ist folgende Charakterisierung:

\begin{lem}
Eine  lineare Abbildung   $P:\rmC_0^\infty(\Omega)\to\rmC^\infty(\Omega)$ ist genau dann ein Differentialoperator der Ordnung $m$, wenn für alle $f\in\rmC_0^\infty(\Omega)$ und alle $g\in\rmC^\infty(\Omega)$ die Funktion 
\begin{equation}\label{eq:6:6.2}
\e^{-\i\lambda g} P(f\e^{\i\lambda g}) = \sum_{j=0}^m P_j(f,g) \lambda^j
\end{equation}
ein Polynom vom Grad $m$ in $\lambda$ ist.
\end{lem}
\begin{proof}
Die Hinrichtung ist klar. Zum Beweis der Rückrichtung sei $x'\in\Omega$ ein Punkt und $f\in\rmC_0^\infty(\Omega)$ mit $f=1$  in einer (in einem Kartengebiet liegenden) Umgebung $\omega$ von $x'$. Sei weiter $\xi\in\R^n$ und $g(x) = x\cdot\xi$. Dann folgt aus \eqref{eq:6:6.2}
\begin{equation}
    \e^{-\i \lambda x\cdot\xi} P ( f\e^{\i \lambda x\cdot\xi}) = \sum_{j=0}^m p_j(f;x,\xi) \lambda^j
\end{equation}
mit $\rmC^\infty$-Funktionen auf $\omega\times\R^n$ als Koeffizienten $p_j(f;x,\xi)$. Da auch
$p_j(f;x,\lambda\xi)=p_j(f;x,\xi)\lambda^j$ gilt, ist $p_j(f;x,\xi)$ homogen vom Grad $j$. Die einzigen auf ganz $\R^n$ glatten homogenen Funktionen sind Polynome,
$p_j(f;x,\xi) = \sum_{|\alpha|=j} a_\alpha(f;x) \xi^\alpha$. Sei nun $u\in\rmC_0^\infty(\omega)$. Dann gilt mit der Fourierschen Inversionsformel
\begin{equation}
    u(x) = (2\pi)^{-n/2} \int \e^{\i x\cdot\xi} \widehat u(\xi)\d\xi
\end{equation}
und da $u=fu$ folgt insbesondere 
\begin{align}
    Pu(x) &= P(fu)(x) = (2\pi)^{-n/2} \int P(f \e^{\i x\cdot\xi}) \widehat u(\xi)\d\xi \notag \\&= \sum_{j=0}^m (2\pi)^{-n/2} \int \e^{\i x\cdot\xi} p_j(f;x,\xi) \widehat u(\xi)\d\xi
    = \sum_{|\alpha|\le m} a_\alpha(f;x)\D^\alpha u.
\end{align}
Mit Linearität folgt die Behauptung.
\end{proof}

Sei nun $\Omega$ ein Gebiet.
Wir betrachten im folgenden allgemeiner Operatoren $P : \rmC_0^\infty(\Omega) \to \rmC^\infty(\Omega)$ und deren Beschreibung durch ein \eIndex[Pseudodifferentialoperator]{Symbol}
\begin{equation}\label{eq:6:6.6}
    \sigma_P(f; x,\xi) = \e^{-\i x\cdot\xi} P( f(x) \e^{\i x\cdot\xi}).
\end{equation}  
Dabei sei $f\in\rmC_0^\infty(\Omega)$ kompakt getragen mit $f(x)=1$ auf einer kompakten Teilmenge $\omega\Subset\Omega$. 
Dann gilt für jedes $u\in\rmC_0^\infty(\omega)$ 
\begin{equation}
   P u (x) = (2\pi)^{-n/2} \int \e^{\i x\cdot\xi} \sigma_P(f; x,\xi) \widehat u(\xi) \d\xi.
\end{equation}
Der Operator $P$ wird als \eIndex{Pseudodifferentialoperator} der \eIndex[Pseudodifferentialoperator]{Ordnung} $m$ bezeichnet, falls jedes so entstehende Symbol 
$\sigma_P(f;\cdot,\cdot)$ zur nachfolgend definierten \eIndex{Symbolklasse} $S^m_{\rm loc}(\Omega\times\R^n)$ gehört.

\begin{df}
Eine Funktion $\sigma\in\rmC^\infty(\Omega\times\R^n)$ gehört zur Symbolklasse $S^m_{\rm loc}(\Omega\times\R^n)$ falls
\begin{equation} 
  \sup_{x\in K}  |  \partial_x^\beta\partial_\xi^\alpha \sigma(x,\xi) | \le C_{K,\alpha,\beta} \langle\xi\rangle^{m-|\alpha|}
\end{equation}
für alle Kompakta $K\Subset\Omega$ und alle Multiindices $\alpha,\beta\in\N^n_0$ gilt. Dabei bezeichnet $\langle\xi\rangle = \sqrt{1+|\xi|^2}$.
\end{df}

Ist nun $\sigma\in S^m_{\rm loc}(\Omega\times\R^n)$, so konvergiert für jedes $u\in\rmC_0^\infty(\Omega)$ das Integral
\begin{equation}\label{eq:6:6.9}
   P u (x) = (2\pi)^{-n/2} \int \e^{\i x\cdot\xi} \sigma(x,\xi) \widehat u(\xi) \d\xi
\end{equation}
und definiert eine glatte Funktion $Pu\in\rmC^\infty(\Omega)$. Es gilt sogar mehr

\begin{lem}
Sei $\sigma\in S^m_{\rm loc}(\Omega\times\R^n)$ und $Pu$ für $u\in\rmC_0^\infty(\Omega)$ durch \eqref{eq:6:6.9} definiert. Dann gilt für jedes $f\in\rmC_0^\infty(\Omega)$
und das durch \eqref{eq:6:6.6} zugeordnete Symbol
\begin{equation}
    \sigma_P(f;\cdot,\cdot)\in S^m_{\rm loc}(\Omega\times\R^n). 
\end{equation}
Weiterhin gilt
\begin{equation}
   \sigma_P(f;x,\xi) \sim \sum_{\alpha}  \frac{1}{\alpha!} \big( \partial_\xi^\alpha \sigma(x,\xi) \big) \big(\D_x^\alpha f(x)\big)
\end{equation}
und, falls $f(x)=1$ auf $\omega\Subset\Omega$ gilt, auch $\sigma_P(f;x,\xi)=\sigma(x,\xi) \mod\mathscr S(\overline{\omega}\times\R^n)$.
\end{lem}
\begin{proof}
Nach Definition gilt
\begin{align}
   \sigma_P(f;x,\xi) &=(2\pi)^{-n}  \e^{-\i x\cdot\xi} \int \e^{\i x\cdot\eta} \sigma(x,\eta) \int_\Omega \e^{-\i y\cdot(\eta-\xi)} f(y) \d y \d\eta \notag\\
   & = (2\pi)^{-n/2}  \int \e^{\i x\cdot(\eta-\xi)} \sigma(x,\eta) \widehat f(\eta-\xi)\d\eta \notag\\
   & = (2\pi)^{-n/2}  \int \e^{\i x\cdot\eta} \sigma(x,\eta+\xi) \widehat f(\eta)\d\eta \notag\\
   & = (2\pi)^{-n/2}  \int \e^{\i x\cdot\eta} \left( \sum_{|\alpha|<N} \frac{\eta^\alpha}{\alpha!} \partial_\xi^\alpha \sigma(x,\xi) + R_N(x,\xi,\eta) \right) \widehat f(\eta)\d\eta \notag\\  
   & =  \sum_{|\alpha|<N} \frac{1}{\alpha!} \big( \partial_\xi^\alpha \sigma(x,\xi) \big) \big(\D_x^\alpha f(x)\big)  +
   \rho_N(x,\xi)
\end{align}
unter Ausnutzung des taylorschen Satzes mit Entwicklungspunkt $\eta=0$. Weiter gilt unter Nutzung der Integraldarstellung des Restgliedes $R_N(x,\xi,\eta)$ 
\begin{equation}
   R_N(x,\xi,\eta) = N \int_0^1 \sum_{|\alpha|=N} \frac{(1-t)^N \eta^\alpha}{\alpha!} \partial_\xi^\alpha \sigma(x,\xi+t\eta) \d t 
\end{equation}
und somit für beliebige Kompakta $K\Subset\Omega$ und $N\ge m$
\begin{align}
   |\langle\xi\rangle^{N-m} \rho_N(x,\xi)|& \lesssim 
   \left|\langle\xi\rangle^{N-m}  \int \e^{\i x\cdot\eta} \int_0^1 \sum_{|\alpha|=N} \frac{(1-t)^N \eta^\alpha}{\alpha!} \partial_\xi^\alpha \sigma(x,\xi+t\eta) \d t  \widehat f(\eta) \d\eta   \right| \notag\\
&\lesssim  \int \int_0^1 \langle\eta\rangle^N \langle\xi+t\eta\rangle^{m-N} \langle\xi\rangle^{N-m} \d t | \widehat f(\eta) |\d\eta   
\lesssim \int \langle\eta\rangle^{2N} |\widehat f(\eta)|\d\eta
\end{align}
gleichmäßig in $x\in K$ (und unter Ausnutzung der Ungleichung von Peetre $\langle \xi\rangle^s \lesssim \langle\eta\rangle^s \langle\xi-\eta\rangle^s$ für $s\ge0$).
Weiter gilt für Multiindices $\beta,\gamma\in\N_0^n$
\begin{multline}
   \partial_x^\gamma \partial_\xi^\beta \rho_N(x,\xi) \\
   =  (2\pi)^{-n/2} N \sum_{\delta\le\gamma} \binom{\gamma}{\delta} \i^{|\delta|} \int \e^{\i x\cdot\eta} \int_0^1 \sum_{|\alpha|=N} \frac{(1-t)^N \eta^{\alpha+\delta}}{\alpha!} \partial_x^{\gamma-\delta} \partial_\xi^{\alpha+\beta} \sigma(x,\xi+t\eta) \d t  \widehat f(\eta) \d\eta 
\end{multline}
und obige Abschätzung liefert $\rho_N\in S^{m-N}_{\rm loc}(\Omega\times\R^n)$. Das impliziert die Behauptung.
\end{proof}

Pseudodifferentialoperatoren auf Gebieten kann man nicht notwendigerweise verketten. Wir bezeichnen einen Pseudodifferentialoperator $P$ als \eIndex[Pseudodifferentialoperator]{eigentlich getragen}, falls er $\rmC_0^\infty(\Omega)$ nach $\rmC_0^\infty(\Omega)$ abbildet. Dies gilt zum Beispiel dann, wenn das Symbol in $x$ kompakt getragen ist. 

%
%Analog ergibt sich eine erste Kompositionsformel für Pseudodifferentialoperatoren. Es gilt für zwei solche Operatoren $P$ und $Q$ und deren Symbole im Sinne von \eqref{eq:6:6.6}
%\begin{equation}
%  \sigma_Q(gPf; x,\xi) \sim \sum_{\alpha} \frac1{\alpha!} \big(\partial_\xi^\alpha\sigma_Q(g; x,\xi) \big) \big(\D_x^\alpha \sigma_P(f;x,\xi)\big).  
%\end{equation}

BLA

\section{Hörmanders globales Kalkül}

Im folgenden betrachten wir globale Symbole $\sigma(x,\xi)$, welche die Bedingung
\begin{equation}
    \sup_{x\in\R^n} |\partial_x^\beta\partial_\xi^\alpha \sigma(x,\xi)| \le C_{\alpha,\beta} \langle\xi\rangle^{m-|\alpha|}
\end{equation}
für alle Multiindices $\alpha,\beta\in\N_0^n$ erfüllen. Die Symbolabschätzungen sind also global gleichmäßig in $x$.
Wir bezeichnen die Menge dieser Symbole mit $S^m_{\rm unif}(\R^n\times\R^n)$. Gilt $\sigma\in S^m_{\rm unif}(\R^n\times\R^n)$
und $u\in\mathscr S(\R^n)$ eine Schwartzfunktion, so definiert
\begin{equation}\label{eq:6:6.17}
   Pu (x) = (2\pi)^{-n/2} \int \e^{\i x\cdot\xi} \sigma(x,\xi) \widehat u(\xi)\d\xi
\end{equation}
wiederum eine Schwartzfunktion. Das folgt durch geeignetes partielles integrieren. Interessanter für uns ist folgender Satz; Operatoren der Ordnung $0$ sind $\rmL^2$--$\rmL^2$-beschränkt.

\begin{thm}
Sei $\sigma\in S^0_{\rm unif}(\R^n\times\R^n)$ und $P$ durch \eqref{eq:6:6.17} definiert. Dann gibt es eine Konstante $C>0$, so daß für alle $u\in\mathscr S(\R^n)$
\begin{equation}
   \| Pu\| \le C \|u\|
\end{equation}
gilt.
\end{thm}
\begin{proof}
\end{proof}


\begin{thm}
Seien $A$ und $B$ globale Pseudodifferentialoperatoren mit Symbolen $\sigma_A\in S^{m_1}_{\rm unif}(\R^n\times\R^n)$ und $\sigma_B\in S^{m_2}_{\rm unif}(\R^n\times\R^n)$. Dann gilt
\begin{enumerate}
\item die Verkettung $A\circ B$ ist ein globaler Pseudodifferentialoperator der Ordnung $m_1+m_2$ und für das zugehörige Symbol gilt
\begin{equation}
   \sigma_{A\circ B}(x,\xi) \sim \sum_\alpha \frac1{\alpha!} \big(\partial_\xi^\alpha \sigma_A(x,\xi) \big) \big(\D_x^\alpha \sigma_B(x,\xi)\big);
\end{equation}
\item der formal adjungierte Operator $A^*$ ist ein Pseudodifferentialoperator der Ordnung $m_1$ und für das zugehörige Symbol gilt
\begin{equation}
   \sigma_{A^*}(x,\xi) = \sum_{\alpha} \frac1{\alpha!} \partial_\xi^\alpha \D_x^\alpha \overline{\sigma_A(x,\xi)}.
\end{equation}
\end{enumerate}
\end{thm}
\begin{proof}
\end{proof}
 % Einleitung, Teil 2
% !TEX root = main.tex
\chapter{Die Ungleichung von G\r{a}rding}

\section{Dirichletformen}
Im folgenden sollen Dirichletformen 
\begin{equation}
   P(x,\D,\overline\D)[u,v] = \sum_{|\alpha|,|\beta|\le m} \int_\Omega p_{\alpha,\beta}(x) \big(\D^\alpha u(x)\big) \overline{\big(\D^\beta v(x)\big)} \d x,\qquad u,v\in\rmC_0^\infty(\Omega)
\end{equation}
zu einem gegebenen Gebiet $\Omega\subset\R^n$ und für Koeffizienten $p_{\alpha,\beta}\in\rmC^\infty(\overline{\Omega})$ mit $p_{\beta,\alpha}(x)=\overline{p_{\alpha,\beta}(x)}$  betrachtet werden. Wir schreiben kurz $P(x,\D,\overline\D)[u]=P(x,\D,\overline\D)[u,u]$ und wenn klar ist, welche Form wir betrachten, nur $P[u,v]$ beziehungsweise $P[u]$. Zugeordnet zu einer solchen Form betrachten wir das Symbol
\begin{equation}
   P(x,\zeta,\overline\zeta) = \sum_{\alpha,\beta} p_{\alpha,\beta}(x) \zeta^\alpha\overline\zeta^\beta,\qquad \zeta\in\C^n,
\end{equation}
und das zugeordnete Hauptsymbol
\begin{equation}
   p(x,\zeta,\overline\zeta) = \sum_{|\alpha|=|\beta|=m} p_{\alpha,\beta}(x) \zeta^\alpha\overline\zeta^\beta,\qquad \zeta\in\C^n.
\end{equation}
Auf Grund der Symmetriebedingung ist das Symbol $P(x,\xi,\xi)$ reellwertig f\"ur alle $\xi\in\R^n$, ebenso ist 
$P(x,\D,\overline \D)[u]$ reell. Spezialfälle solcher Dirichletformen sind die Innenprodukte des $\rmH^m_0(\Omega)$. Für diese schreiben wir kurz
\begin{equation}
  \spro{u}{v}_{(m)} = \sum_{|\alpha|\le m} \spro{\D^\alpha u}{\D^\alpha v} = \sum_{|\alpha|\le m} \int_\Omega   \big(D^\alpha u(x)\big) \overline{\big(\D^\alpha v(x)\big)} \d x.
\end{equation}
Es stellt sich die Frage, unter welchen Voraussetzungen eine Dirichletform äquivalent zu einem solchen Innenprodukt ist. Gilt $a_{\alpha,\beta}\in \rmC^\infty(\overline\Omega)$, so ergibt sich stets eine obere Schranke der Form
\begin{equation}
    P[u] \le C \spro{u}{u}_{(m)}.
\end{equation}
Untere Schranken sind komplizierter. Wir nennen die Form (gleichmäßig) elliptisch, falls 
\begin{equation}
  \inf_{x\in\Omega}  \inf_{|\xi|=1} |p(x,\xi,\xi)| > 0.
\end{equation}
Unter dieser Voraussetzung zeigen wir, dass es Konstanten $a,b\in\R$ gibt, für welche
\begin{equation} 
   \spro{u}{u}_{(m)} \le a\, P[u] + b\, \spro{u}{u}
\end{equation}
gilt. Dies ist dazu äquivalent, dass der Quotient $P [u]  / \spro{u}{u}_{(m)}$ nach unten beschränkt ist.

\begin{thm}[{\cite[Theorem 2.1]{Garding:1953}}]
Sei $P[u]$ eine gleichmäßig elliptische Dirichletform der Ordnung $m$ mit positivem Hauptsymbol auf einem beschränkten Gebiet $\Omega\Subset\R^n$. Dann gilt
\begin{equation}
  \inf_{u\in\rmH_0^m(\Omega)} \frac{P[u]}{\spro{u}{u}_{(m)}} > -\infty.
\end{equation}
\end{thm}
\begin{proof}
{\sl Schritt 1.} Wir beginnen mit dem Spezialfall, dass $P(x,\xi,\xi) = P(\xi)$ unabhängig von $x$ ist. Dann folgt mit dem Satz von Plancherel
\begin{equation}
 P[u] =  \sum_{|\alpha|,|\beta|\le m} \int_\Omega  p_{\alpha,\beta} \big(\D^\alpha u(x)\big)\overline{\big(\D^\beta u(x)\big)} \d x
 =   \sum_{|\alpha|,|\beta|\le m} \int  p_{\alpha,\beta} \xi^\alpha \xi^\beta  |\widehat u(\xi)|^2 \d\xi 
\end{equation}
mit $p_{\alpha,\beta}\in\C$ den Koeffizienten der Form und für beliebiges $u\in\rmC_0^\infty(\Omega)$. Da weiterhin die Elliptizitätsbedingung $p(\xi)\ge \tilde c |\xi|^{2m}$ 
vorausgesetzt ist und $\Omega$ beschränkt ist, gilt
\begin{equation}
  P[u] \ge \tilde c \int |\xi|^{2m} |\widehat u(\xi)|^2 \d\xi  -  \tilde c' \int \langle \xi\rangle^{2m-1} |\widehat u(\xi)|^2 \d\xi  \ge  \tilde c'\; \spro{u}{u}_{(m)}
\end{equation}
mit einer geeigneten Konstanten $\tilde c'$. $\bullet$\qquad{\sl Schritt 2.}  Wir führen den allgemeinen Fall auf den gerade gezeigten Spezialfall zurück. Dazu bezeichne im Folgenden
\begin{equation}
	w(\rho) = \sup\left\{| p_{\alpha,\beta}(x)-p_{\alpha,\beta}(y)| : |\alpha|, |\beta| \le m, | x-y| \le \rho\right\}.
\end{equation} 
Da nach Voraussetzung die Koeffizienten aus $\rmC^\infty(\overline\Omega)$ sind, sind sie insbesondere gleichmäßig stetig und somit gilt
$\lim_{\rho\to0} w(\rho)=0$.  Sei nun $x_0\in \Omega$ beliebig, $u\in \rmH_0^m(\Omega\cap \partial B_\rho(x_0))$ und bezeichne $P_0(x,\xi, \xi):=P(x_0,\xi,\xi)$ die Form mit in $x_0$ eingefrorenen Koeffizienten. Dann gilt
\begin{equation}
\begin{split}
\big| P[u]-P_0[u] \big| &=\left| \sum\limits_{|\alpha|, |\beta| \le m} \int_\Omega \left(p_{\alpha,\beta}(x)-p_{\alpha,\beta}(x_0)\right){\D^\alpha(u(x)}\overline{\D^\beta u(x))}\mathrm{d}x\right|\\
	&\le w(\rho) \sum\limits_{|\alpha|,|\beta| \le m} \spro{\D^\alpha u}{\D^\beta u} \le  w(\rho) \sum\limits_{|\alpha|,|\beta| \le m}\|\D^\alpha u\|\, \|\D^\beta u\| \le
	 w(\rho) \|u\|_{(m)}^2
\end{split}
\end{equation}
unter zweifacher Ausnutzung der Ungleichung von Cauchy--Schwarz. Also folgt unter Verwendung des oben betrachteten Spezialfalls
\begin{equation}
	P[u] \ge P_0[u] - \big| P[u] - P_0[u] \big| \ge \big( \tilde c' - w(\rho) \big) \|u\|_{(m)}^2 \ge  c \| u\|_{(m)}^2
\end{equation} 
für ein $c\in\R$ und alle $u\in \rmH_0^m(\Omega \cap \partial B_\rho(x_0))$,  $\rho$ hinreichend klein. Die Abschätzung ist gleichmäßig in $x_0\in\Omega$.
$\bullet$\qquad {\sl Schritt 3.} Da $\Omega$ beschränkt ist, finden wir zu jedem $\epsilon>0$ eine endliche Überdeckung $\Omega\subset\bigcup_{k=1}^N B_{{\epsilon}/{2}}(x_k)$ und eine dieser Überdeckung untergeordnete Zerlegung der Eins $\psi_k\in\rmC_0^\infty(B_{\epsilon/2}(x_k))$ mit
\begin{equation}
 \sum\limits_{k=1}^N |\psi_k(x)|^2 = 1,\qquad \ x\in \Omega.
\end{equation}
Sei nun $u\in\rmH_0^m(\Omega)$ beliebig. Dann gilt
\begin{equation}
\begin{split}
	P[u]&= \sum_{k=1}^N \sum\limits_{|\alpha|,|\beta|\le m} \int_\Omega |\psi_k(x)|^2p_{\alpha\beta}(x)\big(\D^\alpha u(x)\big)\overline{\big(\D^\beta u(x)\big)}\d x\\
	&=  \sum_{k=1}^N \bigg( \sum\limits_{|\alpha|,|\beta|\le m} \int_\Omega  p_{\alpha\beta}(x)\big(\D^\alpha \psi_k(x) u(x)\big)\overline{\big(\D^\beta \psi_k(x) u(x)\big)}\d x + R_k[u]\bigg),
\end{split}
\end{equation}
wobei der Restterm $R_k[u]$ eine Form der Ordnung $m$ mit verschwindendem Hauptsymbol\footnote{Zur Erinnerung: $r_k(x,\xi,\xi)=0$ heißt, daß im Integral nie ein Produkt von zwei Ableitungen der Ordnung $m$ steht.} ist. Es gilt also mit Cauchy--Schwarz $|R_k[u]|\le c_k \|u\|_{(m)} \|u\|_{(m-1)}$, während jeder Summand der äußeren Summe mit Schritt 2 abgeschätzt werden kann. Also folgt 
\begin{equation}\label{eq:6.16}
P[u] \ge \sum_{k=1}^N  \bigg( c \|\psi_k u\|_{(m)}^2 - c_k \|u\|_{(m)} \|u\|_{(m-1)} \bigg) .
\end{equation}
Weiter gilt 
\begin{equation}
   \|\psi_k u\|_{(m)}^2 =  \sum_{|\alpha|\le m} \int_\Omega |\D^\alpha (\psi_k(x) u(x)) |^2 \d x
   =   \sum_{|\alpha|\le m} \int_\Omega  |\psi_k(x)|^2  |\D^\alpha u(x) |^2 \d x + \tilde R_k[u]
\end{equation}
mit einer weiteren Form $\tilde R_k[u]$  der Ordnung $m$ und verschwindendem Hauptsymbol. Also gilt wiederum $|\tilde R_k[u]|\le b_k \|u\|_{(m)}\|u\|_{(m-1)}$ und  \eqref{eq:6.16} liefert mit $b= \sum_{k=1}^N (c_k + c b_k  )$
\begin{equation}\label{eq:6.18}
P[u] \ge c \|u\|_{(m)}^2 - b \|u\|_{(m)}\|u\|_{(m-1)} .
\end{equation}
Mit $\|u\|_{(m-1)}\le \|u\|_{(m)}$ folgt die Behauptung.
\end{proof}

\begin{rem} 
Durch Interpolation folgt aus \eqref{eq:6.18}
\begin{equation}
P[u] \ge c \|u\|_{(m)}^2 - \tilde b \|u\|_{(m-1/2)}^2
\end{equation}
mit den in Definiton~\ref{df:sob-norm} definierten Sobolevnormen. Ist die Form $P[u]$ homogen, das Polynom $P(x,\zeta,\overline \zeta)$ also gleich dem entsprechenden Hauptteil $p(x,\zeta,\overline\zeta)$, so gilt die Abschätzung aus Schritt~1 des Beweises mit einer positiven Konstanten $\tilde c>0$. Das erlaubt es, die Abschätzung auch auf nichtelliptische Formen mit $p(x,\xi,\xi)\ge0$ auszudehnen. Die Formen werden elliptisch, wenn man zu $p(x,\zeta,\overline\zeta)$
noch $\epsilon ( \zeta\cdot\overline\zeta )^{m}$ addiert. Auf diese Weise erhält man für jedes $\epsilon>0$ 
die Existenz einer Konstanten $K(\epsilon)>0$ mit
\begin{equation}
P[u] \ge -\epsilon \|u\|_{(m)}^2 - K(\epsilon) \|u\|_{(m-1/2)}^2.
\end{equation}
Es stellt sich die Frage, was mit $K(\epsilon)$ für $\epsilon\to0$ passiert. 
\end{rem}

\section{Die scharfe G\r{a}rdingungleichung}

Nun soll die gerade aufgeworfene Frage beantwortet werden. Wir betrachten dazu Pseudodifferentialoperatoren $A$ zu Symbolen $\sigma_A(x,\xi)$.

\begin{thm}[Hörmander, {\cite[Theorem 1.3.3]{Hormander:1966}}]
Sei $A\in {\rm OP}S^{2m}_{\rm loc}(\Omega\times\R^n)$ Pseudodifferentialoperator mit Hauptsymbol\footnote{Es gilt also $\sigma_A-a\in S^{2m-1}_{\rm loc}(\Omega\times\R^n)$.} $a(x,\xi)\ge0$. Dann gibt es  für alle $\omega\Subset \Omega$ ein $K>0$ mit
 \begin{equation}
    \Re \spro{Au}{u} \ge - K \|u\|_{(m-1/2)}^2 
 \end{equation}
 für alle $u\in\rmC_0^\infty(\omega)$.
\end{thm}

Es genügt, eine solche Aussage für Operatoren der Ordnung Null zu beweisen. Operatoren anderer Ordnungen können mit dem Pseudodifferentialkalkül auf diesen Fall zurückgeführt werden. Im folgenden beziehen wir uns auf  Lax und Nirenberg \cite{Lax:1966} und zeigen eine entsprechende Aussage für (matrixwertige) Operatoren. 

Zuerst benötigen wir einige Definitionen und Hilfsmittel.

\medskip\noindent
{\em Symbolklassen:} Für $ a\in S^0_{\rm comp}(\R^n\times\R^n)$ definieren wir die Normen
\begin{equation}
\begin{split}&	\vert \widehat{a}(\eta,\cdot)\vert_k := \sum\limits_{\vert \beta\vert \le k} \sup\limits_{\xi} \langle \xi\rangle^{\vert \beta\vert}\vert\D_\xi^\beta  \widehat{a}(\eta,\xi)\vert,\\
&   | a|_{k,\ell} := (2\pi)^{-n/2} \int \langle\eta\rangle^\ell   \vert \widehat{a}(\eta,\cdot)\vert_k \d \eta.
\end{split}
\end{equation}
Hier bezeichnet $\widehat a(\eta,\xi)$ wiederum die Fouriertransformierte des Symbols bezüglich $x$. Weiter sei $\mathcal{C}_{k,\ell}$ der Abschluß der Symbolklasse in der $ |\cdot |_{k,\ell}$-Norm.
Im folgenden werden Symbole matrixwertig sein.

\medskip\noindent
{\em Zerlegung der Eins:} Sei im folgenden $\Theta\in\rmC_0^\infty(\R^n)$ mit $\supp\Theta\subset [-3/4,3/4]^n$ sowie $\Theta>0$ auf $[-1/2,1/2]^n$. Dann sind die Funktionen
\begin{equation}
\varphi_k(x) = \Theta(x-k) \bigg(\sum_{j\in\Z^n} \Theta(x-j)^2\bigg)^{-1/2}
\end{equation}
Elemente von $\rmC_0^\infty(\R^n)$ mit $\sum_{k} \varphi_k(x)^2=1$ und es existiert für jeden Multiindex $\alpha\in\N_0^n$ eine Konstante $C_\alpha$ mit
\begin{equation}
    \sum_{k\in\Z^n} | \D^\alpha \varphi_k(x)|^2 \le C_\alpha.
\end{equation}
Weiterhin impliziert $x,y\in\supp\varphi_k$ stets $|x-y|\le 3\sqrt n/2$. Ausgehend von dieser Partition der Eins definieren wir
\begin{equation}
    \psi_k(\xi)= \varphi_k(\xi / \sqrt{|\xi|}), 
\end{equation}
so daß
\begin{equation}\label{eq:6:6.35} 
    \sum_{k\in\Z^n} \psi_k(\xi)^2 = 1\qquad\text{und}\qquad |\xi|^\alpha \sum_{k\in\Z^n} |\partial_\xi^\alpha \psi_k(\xi)|^2 \le C_\alpha
\end{equation}
gilt. Weiter impliziert $\xi,\eta\in\supp\psi_k$ stets $|\sqrt{|\xi|}-\sqrt{|\eta|}|\le C$ und damit 
\begin{equation}\label{eq:6:6.36}
 |\xi-\eta|\le C \langle\xi\rangle^{1/2} \qquad\text{für alle $\xi,\eta\in\supp\psi_k$}.
\end{equation}
Weiterhin nützlich ist
\begin{equation}\label{eq:6:6.37}
  \sum_{k\in\Z^n} |\psi_k(\xi)-\psi_k(\eta)|^2 \le C \frac{|\xi-\eta|^2}{ \langle\xi\rangle^{1/2} \langle\eta\rangle^{1/2}},
\end{equation} 
was nur für $|\xi-\eta|\le \langle\xi\rangle^{1/2}$ nichttrivial ist und dann aus \eqref{eq:6:6.35} für $|\alpha|=1$ per Integration über die Strecke von $\xi$ zu $\eta$ folgt.

\medskip\noindent
{\em Positiv semidefinite Matrizen.}
Sei $A\in\C^{d\times d}$ selbstadjungiert und positiv semidefinit. Dann betrachten wir die zugeordneten quadratischen Formen\footnote{Zur Notation: Wir definieren $\cspro{u}{A}{v} = u^* A v=\overline u^\top A v$ für $A\in\C^{d\times d}$ und $u,v\in\C^{d}$.} $\cspro{v}A{v}$ auf $\C^d$ und  schreiben $A \preceq B$ für zwei solche Matrizen, falls 
\begin{equation}
   \forall_{v\in\C^d} \quad:\quad \cspro{v}A{v}\le \cspro{v}B{v}
\end{equation}
gilt. 

Eine erste Anwendung ist folgendes Lemma.

\begin{lem}[{\cite[Lemma 2.2]{Lax:1966}}]\label{Matrizenlemma}
	Seien $A,B\in\C^{d\times d}$ selbstadjungiert und es gelte $A\pm B \succeq 0$. Dann gilt
	\begin{equation}
		\forall_{v,w \in \mathbb{C}^d} \quad: \quad \vert \cspro{v}B{w} \vert \le \cspro{v}A{v}^{{1}/{2}}\cspro{w}A{w}^{{1}/{2}}. 
	\end{equation}
\end{lem}
\begin{proof}
	Nach Voraussetzung gilt
	\begin{equation}\begin{split}
		&\cspro{v+w}{A+B}{v+w} \ge 0,\\
		&\cspro{v-w}{A-B}{v-w} \ge 0.
		\end{split}
	\end{equation}
	Addieren der beiden Gleichungen liefert
	\begin{equation}
		\cspro{v}A{v}+2\Re \cspro{w}B{v} +\cspro{w}Aw \ge 0
	\end{equation}
	und damit nach Multiplikation von $w$ mit einer komplexen Zahl vom Betrag Eins (so dass $\Re \cspro{\lambda w}B{v} = - | \cspro{w}B{v}|$ gilt) die Behauptung. 
\end{proof}

\medskip\noindent
{\em Ein Hilfslemma.} Wir beginnen mit einer Hilfsaussage, welche auch für sich genommen von Interesse ist. 

\begin{lem}[{\cite[Lemma 3.1]{Lax:1966}}]\label{lem3.1:Lax}
Sei $p\in \mathcal{C}_{0,2}$ eine von $\xi$ unabhängige selbstadjungierte Matrix und sei weiter $\Phi: \mathbb{R}^n\rightarrow\mathbb{R}$ Lipschitz-stetig mit Lipschitzkonstante $K$ auf $\supp \widehat{u}$. Dann gilt
\begin{equation}
		| \Re \spro {\Phi(\D)pu}{\Phi(\D)u}-\Re\spro{p\Phi(\D)u}{\Phi(\D)u}| \le\frac{1}{2}| p|_{0,2}K^2|| u||^2.
\end{equation}	 
\end{lem}
\begin{proof}
	Die Rechenregeln der Fouriertransformation liefern
	\begin{equation}\label{lem3.1:Lax;1}
		\begin{split} J[u]&=  \Re \spro {\Phi(\D)pu}{\Phi(\D)u}-\Re\spro{p\Phi(\D)u}{\Phi(\D)u} \\
		&= \left(2\pi\right)^{-\frac{n}{2}}\Re\iint\cspro {\widehat{u}(\xi)}{\widehat{p}(\xi-\eta)}{\widehat{u}(\eta)}\left(\Phi(\xi)-\Phi(\eta)\right)\Phi(\xi)\d \xi\d\eta.
		\end{split}
	\end{equation}
	Weil $p$ selbstadjungiert ist, folgt $\widehat{p}(\xi) = {\widehat{p}}(-\xi)^*$. Durch Vertauschen der Integrationsvariablen erhalten wir
	\begin{equation}\label{lem3.1:Lax;2}
	\begin{split}	
	J[u]&= \left(2\pi\right)^{-\frac{n}{2}}\Re\iint \cspro{ \widehat{u}(\eta)}{\widehat{p}(\eta-\xi)}{\widehat{u}(\xi)}\left(\Phi(\eta)-\Phi(\xi)\right)\Phi(\eta)\d \xi\d\eta\\
		&= \left(2\pi\right)^{-\frac{n}{2}}\Re\iint\cspro{\widehat{u}(\xi)}{\widehat{p}(\xi-\eta)}{\widehat{u}(\eta)}\left(\Phi(\eta)-\Phi(\xi)\right)\Phi(\eta)\d \xi\d\eta.
	\end{split}	
	\end{equation}
	und Addition von $(\ref{lem3.1:Lax;1})$ und $(\ref{lem3.1:Lax;2})$ liefert
	\begin{equation}
		J[u]= \frac{1}{2}\left(2\pi\right)^{-\frac{n}{2}}\Re\iint \cspro{\widehat{u}(\xi)}{\widehat{p}(\xi-\eta)}{\widehat{u}(\eta)}\left(\Phi(\xi)-\Phi(\eta)\right)^2\d\xi\d\eta.
	\end{equation}
	Mit  $\left(\Phi(\xi)-\Phi(\eta)\right)^2\le K^2 |\xi-\eta|^2$ und $| \widehat{p}(\xi-\eta,\cdot)|_{0,2} = (2\pi)^{-\frac{n}{2}}\int \langle\xi-\eta\rangle^2 |\widehat{p}(\xi-\eta)|\d\eta$ sowie der Ungleichung von Cauchy--Schwarz folgt die Behauptung. 
\end{proof}

Damit kommen wir zum Hauptresultat dieses Abschnitts, der scharfen G\r{a}rdingungleichung für Systeme nach Lax und Nirenberg.

\begin{thm}[{\cite[Theorem 3.1]{Lax:1966}}]
Sei $a\in\mathcal{C}_{0,2}\cap\mathcal{C}_{2,0}$ selbstadjungiert und positiv-semidefinit. Dann erfüllt der zugehörige Operator $A$ die Ungleichung
\begin{equation}
	\Re \spro{Au}{u} \ge - K|| u||_{\left(-{1}/{2}\right)}^2.
\end{equation}
\end{thm}
\begin{proof}
Wir machen Gebrauch von der zu Beginn des Kapitels konstruierten Zerlegung der Eins. Sei dazu $	u_k:= \psi_k(\D)u$. Dann gilt $\widehat{u_k}(\xi) = \psi_k(\xi) \widehat{u}(\xi)$ und mit der Darstellung 
\begin{equation}
	\widehat{Au}(\xi) = \left(2\pi\right)^{-\frac{n}{2}}\int \widehat{a}(\xi-\eta,\eta)\widehat{u}(\eta)\d \eta
\end{equation}	 
folgt für die Differenz $J[u]$ der quadratischen Formen $\spro{Au}{u}$ und $\sum_k \spro{Au_k}{u_k}$
\begin{align}
 J[u]&=\spro{Au}{u}-\sum\limits_k \spro{Au_k}{u_k} \notag\\
& = \left(2\pi\right)^{-\frac{n}{2}}\iint \cspro{\widehat{u}(\xi)}{\widehat{a}(\xi-\eta,\eta)}{\widehat{u}(\eta)}\left(1-\sum\limits_k \psi_k(\eta)\psi_k(\xi)\right)\d \xi\d\eta\notag\\
	&= \frac{1}{2}\left(2\pi\right)^{-\frac{n}{2}} \iint \cspro{\widehat{u}(\xi)}{\widehat{a}(\xi-\eta,\eta)}{\widehat{u}(\eta)}\sum\limits_k\vert\psi_k(\eta)-\psi_k(\xi)\vert^2\d \xi\d\eta.
\end{align}
Damit folgt mit \eqref{eq:6:6.37} und erneutem Anwenden von Cauchy--Schwarz
\begin{equation}
\begin{split}	
| J[u] | &\le \frac{C}{2}\left(2\pi\right)^{-\frac{n}{2}}\iint| \cspro{\widehat{u}(\xi)}{\widehat{a}(\xi-\eta,\eta)}{\widehat{u}(\eta)}| \, |\xi-\eta|^2\langle \xi\rangle^{-\frac{1}{2}} \langle\eta\rangle^{-\frac{1}{2}}\d \xi \d \eta\\ 
& \le \frac{C}{2}| a|_{0,2}|| u||_{(-{1}/{2})}^2.
\end{split}
\end{equation}
Weiterhin gilt $\sum_k \|u_k\|^2_{(-1/2)} = \|u\|^2_{(-1/2)}$.
Damit ist es ausreichend, die gewünschte Ungleichung für die Funktionen $u_k$ einzeln (und mit von $k$ unabhängiger Konstanten) zu zeigen. Wir halten ein beliebiges $k$ fest und setzen im Folgenden
 $v= u_k$. Sei $\tau\in \supp \widehat{v}$ fixiert. Für $\mu, \eta \in \supp \widehat{v}$ gilt mit \eqref{eq:6:6.36}
\begin{equation}
	1+|\mu|^2 \le 1+\left(| \eta|+| \eta-\mu|\right)^2 \le 1+|\eta|^2+ c_1 \langle \mu \rangle,
\end{equation}
also
\begin{equation}
	\langle \mu\rangle \le c_2\langle \eta\rangle
\end{equation}
und somit
\begin{equation}\label{Abschaetzung 0-Norm}
	\langle \tau\rangle^{-1}|| v||^2 \lesssim|| v||_{\left(-{1}/{2}\right)}^2.
\end{equation}
Für $l=1,\ldots, n$ definieren wir $\widehat{a}^l(\xi-\eta,\tau)=\frac{\partial}{\partial \tau_l}\widehat{a}(\xi-\eta,\tau)$ und wenden den Mittelwertsatz an. Dies liefert für $\eta\in \supp \psi_k$ 
\begin{equation}
\begin{split}	&| \widehat{a}(\xi-\eta,\eta)-\widehat{a}(\xi-\eta,\tau) -\sum\limits_{l=1}^n (\eta_l-\tau_l)\widehat{a}^l\left(\xi-\eta,\tau\right)| \lesssim| \eta-\tau|^2| \widehat{a}(\xi-\eta,\cdot)|_2\langle\eta\rangle^{-2} \\&
\lesssim | \widehat{a}(\xi-\eta,\cdot)|_2 \langle\eta\rangle^{-1} \lesssim| \widehat{a}(\xi-\eta,\cdot)|_2\langle\eta\rangle^{-\frac{1}{2}}\langle\xi\rangle^{-\frac{1}{2}}.
\end{split}
\end{equation}
Setzen wir nun $p(x)=a(x,\tau)$, $p^l(x)= a^l(x,\tau) $ und $v^l:=(\D_l-\tau_l)v$ (also $\widehat{v}^l(\xi)=\left(\xi_l-\tau_l\right)\widehat{v}(\xi)$), dann folgt
\begin{equation}
	|\spro{Av}{v}-\spro{pv}{v}-\sum\spro{p^lv^l}{v}| \le c | a|_{2,0}|| u||_{\left(-{1}/{2}\right)}^2.
\end{equation}
Es verbleibt also, die Ungleichung 
\begin{equation}
	Q[v] \ge -C|| v||_{\left(-{1}/{2}\right)}^2
\end{equation}
für die quatratische Form
\begin{equation}
	Q[v]=\spro{pv}{v}+\Re\sum\spro{p^lv^l}{v} 
\end{equation}
zu zeigen. Diese entspricht der in $\xi$ linearen Approximation an das Symbol $a(x,\xi)$ in $\xi=\tau$.

Für jedes $l\in\{1,\dots, n\}$ gilt für $\xi = \pm c\langle\tau\rangle^{\frac{1}{2}}\e_l$ mit einem später geschickt gewählten $c$, dass
\begin{equation}
	a(x,\xi+\tau) = a(x,\tau)\pm\langle\tau\rangle^{\frac{1}{2}}a^l(x,\tau) + R(x,\xi,\tau). 
\end{equation}
Weil $a(x,\xi+\tau)$ positiv semi-definit ist, ist für eine geeignete Konstante $C$ auch
\begin{equation}
	a(x,\tau) \pm c\langle\tau\rangle^{\frac{1}{2}}a^l(x,\tau)+C\langle\tau\rangle \frac{| a(x,\cdot)|_{2}}{\langle\tau\rangle^2}\succeq 0
\end{equation}
positiv semi-definit. Da Addieren eines Terms der Form $c \langle\tau\rangle^{-1}$ zu $p(x)$ gemäß \eqref{Abschaetzung 0-Norm} einen Fehler der Ordnung $\mathcal O\left(\vert\vert v\vert\vert_{\left(-{1}/{2}\right)}^2\right)$ verursacht, können wir ohne Einschränkung annehmen, dass  
\begin{equation}
	p(x) \pm \langle\tau\rangle^{\frac{1}{2}}p^l(x) \succeq 0
\end{equation}
positiv semi-definit ist. Wir wenden darauf Lemma \ref{Matrizenlemma} an und erhalten
\begin{equation}
	c\langle\tau\rangle^{\frac{1}{2}} |\spro{p^lv}{v}|\le\spro{pv}{v}^{\frac{1}{2}}\spro{pv^l}{v^l}^{\frac{1}{2}}.
\end{equation}
Unter Ausnutzen der Ungleichung $\sqrt{ab}\le \frac{n}{2}a+\frac{1}{2n}b$ folgt
\begin{equation}
	|\spro{p^lv^l}{v}|\le \frac{1}{2n}\spro{pv}{v}+\frac{n}{2c^2}\frac{1}{\langle\tau\rangle}\spro{pv^l}{v^l},
\end{equation}
was
\begin{equation}
	Q[ v]=\spro{pv}{v}+\Re\sum\nolimits_l \spro{p^lv^l}{v} \ge \frac{1}{2}\spro{pv}{v} -\frac{n}{2c^2}\frac{1}{\langle\tau\rangle}\sum\nolimits_l \spro{pv^l}{v^l}
\end{equation}
impliziert. Lemma \ref{lem3.1:Lax}, angewandt auf $\Phi(\xi) = \xi_l-\tau_l$, liefert in Verbindung mit $(\ref{Abschaetzung 0-Norm})$
\begin{equation}
\sum\nolimits_l \spro{pv^l}{v^l}=\Re\sum\nolimits_l \spro{pv}{\left(\D_l-\tau_l\right)v}+\mathcal{O}\left(|| v||_{\left(-{1}/{2}\right)}^2\right) 
\end{equation}
und damit
\begin{equation}
	Q[v] \ge \frac{1}{2}\spro{pv}{v}-\frac{n}{2c^2}\frac{1}{\langle\tau\rangle}\Re\spro{pv}{| D-\tau|^2v)}+\mathcal{O}\left(|| v||_{\left(-{1}/{2}\right)}^2\right),
\end{equation}
was wir äquivalent ausdrücken durch
\begin{equation}
	2Q[v] \ge \Re\spro{pv}{\left(1-\frac{n}{c^2}\frac{| \D-\tau|^2}{\langle\tau\rangle}\right)v}+\mathcal{O}\left(|| v||_{\left(-{1}/{2}\right)}^2\right).
\end{equation}
Auf $\supp\widehat{v}$ gilt
\begin{equation}
	\frac{|\xi-\tau|^2}{\langle\tau\rangle}\le \tilde{C}^2 
\end{equation}
für eine Konstante $\tilde{C}>0$. Setzen wir nun $c:=\tilde{C}\sqrt{2n}$ und 
\begin{equation}
	\Phi(\xi):=\begin{cases}
		\left(1-\frac{n}{c^2}\frac{| \xi-\tau|^2}{\langle\tau\rangle}\right)^{\frac{1}{2}}, &| \xi-\tau| \le \tilde C(\langle\tau\rangle)^{\frac{1}{2}},\\
		\sqrt{\frac{1}{2}}, & \text{sonst},
	\end{cases}
\end{equation}
dann ist $\Phi(\xi)$ Lipschitz-stetig mit Lipschitz-Konstante $K=\mathcal O(\langle\tau\rangle)^{-\frac{1}{2}}$, erneute Anwendung von Lemma \ref{lem3.1:Lax} liefert
\begin{equation}
	2Q[v] \ge \Re\spro{pv}{\Phi(\D)^2v}+\mathcal{O}\left(|| v||_{\left(-{1}/{2}\right)}^2\right) = \Re\spro{p\Phi(\D)v}{\Phi(\D)v}+\mathcal{O}\left(|| v||_{\left(-{1}/{2}\right)}^2\right).
\end{equation}
Weil $a(x,\tau)$ positiv definit ist, folgt $\Re\spro{p\Phi(\D)v}{\Phi(\D)v}\ge 0$ und damit die gewünschte Ungleichung. 
\end{proof}

  % Gardingsche Ungleichung :: Simon Barth
% !TEX root = main.tex
\chapter{Die Ungleichung von Melin}
\cite{Melin:1971} % Ungleichung von Melin :: Jonas Brinker

%
%

\chapter{Die Ungleichung von Feffermann und Phong}


% Literaturverzeichnis
\bibliographystyle{alpha}
\bibliography{seminar}
% Index
\printindex
\end{document}
