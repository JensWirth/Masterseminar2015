% !TEX root = main.tex
\chapter{Die Ungleichung von Melin}
%\cite{Melin:1971}

\section{Eine spezialle Klasse quadratischer Dirichletformen}
Wir betrachten zuerst Dirichletformen der speziellen Struktur
\begin{equation}
   P[v] = \Re  \int \overline{v(x)}p(x,\D)v(x)\d x 
\end{equation}
für ein Polynom zweiten Grades
\begin{equation}
    p(x,\xi) = \sum_{|\alpha+\beta|\le 2} \frac{ a_{\alpha,\beta}}{\alpha!\beta!} x^\beta \xi^\alpha,\qquad a_{\alpha,\beta}\in\C
\end{equation}
und Funktionen $v\in\mathscr S(\R^n)$. Diese speziellen Formen werden später als Taylorapproximanten auftreten und sollen zuerst untersucht werden. Verschiedene Wahlen der Koeffizienten $a_{\alpha,\beta}$ können dabei dieselbe Form erzeugen, wir wählen die Koeffizienten im folgenden reell.

Zugeordnet zu $P$ betrachten wir die (reelle) quadratische Form
\begin{equation}
  h(x,\xi) = \Re  \sum_{|\alpha+\beta|= 2} \frac{a_{\alpha,\beta}}{\alpha!\beta!} x^\beta\xi^\alpha
\end{equation}
auf $\R^n\times\R^n$ und die symmetrische Grammatrix $H$ mit $\cspro{X}{H}{X}=h(X)$, $X=(x,\xi)\in\R^n\times\R^n$.  Das folgende Lemma legt nahe, die Dirichletform $P$ mit Hilfe der kanonischen symplektischen Struktur\footnote{Wir versehen den $\R^{2n}$ mit der symplektischen Form
$\varsigma(X,Y) = x\cdot \eta - \xi\cdot y$, wobei $X=(x,\xi)$ und $Y=(y,\eta)$. Eine Matrix aus $\R^{2n\times 2n}$ heißt symplektisch, falls sie die symplektische Form erhält. Invertierbare symplektische Matrizen bilden die symplektische Gruppe $\mathrm{SP}_{2n}(\R)$.} auf $\R^n\times\R^n$ zu untersuchen.
\begin{lem}
Es sei $p$ ein Polynom zweiten Grades auf $\R^n\times\R^n$ und $\chi\in\mathrm{SP}_{2n}(\R)$ ein Element der symplektischen Gruppe. Dann gilt
\begin{align}
\forall_{v\in\mathscr S(\R^n)} \quad:\quad \Re \int \overline{v(x)}p(x,\D)v(x)\d x\ge 0
\end{align}
genau dann für das Polynom $p$, wenn es für das Polynom $p\circ \chi$ gilt.
\end{lem}
Im folgenden bezeichne
\begin{align}
J=\begin{pmatrix}
0 & I\\
-I& 0 
\end{pmatrix},\qquad\text{ $ I \in\R^{n\times n}$  Einheitsmatrix }
\end{align}
die symplektische Matrix. Dann ist die symplektische Form gerade durch $\cspro{X}{J}{Y}$ gegeben. Für eine reelle symmetrische Matrix $M$ hat  die Matrix $JM$ ein symmetrisch um den Ursprung angeordnetes, rein imaginäres Spektrum. Die positive Spur von $H$ durch
\begin{equation}
   {\mathrm{Tr}}^+ H = \sum_j\lambda_j 
\end{equation}
bezeichnet, wobei die Folge $(\lambda_j)$ eine ihrer Vielfachheit entsprechenden Aufzählung der positiven Elemente von $i\sigma(JM)$ ist. Wir stellen fest, dass die positive Spur
der Matrix $H$ unter symplektischen Transformationen der quadratischen Form $h$ invariant ist. Also gilt für alle $\chi\in\mathrm{Sp}_{2n}(\R)$
\begin{align*}
\mathrm{Tr}^+(\chi^\mathrm{T}H\chi) = \mathrm{Tr}^+ H\,.
\end{align*}

Der nachfolgende Satz ist von entscheidender Bedeutung um notwendige und hinreichende Bedingungen für die Ungleichung von Melin zu formulieren.

\begin{thm}[{\cite[Theorem 2.4.]{Melin:1971}}]
Es gilt $P[u]\ge0$ genau dann für alle $u\in\mathscr S(\R^n)$, wenn die Bedingungen
\begin{enumerate}
\item  $h(x,\xi)\ge0$ auf $\R^n\times\R^n$;
\item $F \cdot X = \sum_{|\alpha+\beta|=1} \Re a_{\alpha,\beta} x^\beta\xi^\alpha = 0$ für $X=(x,\xi)\in\R^{2n}$ mit $h(x,\xi)=0$;
\item $\Re a_{0,0} - \frac12 \sum_{|\beta|=1} \Im a_{\beta,\beta} -\frac14  \cspro{F}{H^{-1}}{F} + {\mathrm{Tr}}^+ H \ge 0$
\end{enumerate}
erfüllt sind.
\end{thm}


\section{Die Ungleichung von Melin}

Die Ungleichung von Melin verallgemeinert die Ungleichung von G\r ardingen. Der nachfolgende Satz gibt scharfe Bedingungen für die Gültigkeit dieser Ungleichung.

\begin{thm}[{\cite[Theorem 3.1]{Melin:1971}}]
Sei $A\in\mathrm{OP}S^m_{\rm loc}(\Omega\times\R^n)$ mit homogenem Hauptsymbol $a_m(x,\xi)\ge0$ und $\mu=(m-1)/2$. Dann sind \"aquivalent:
\begin{enumerate}
\item F\"ur jedes $\epsilon>0$, jedes $K\Subset\Omega$ und jedes $s\in\R$ existiert ein $C$, so dass
\begin{equation}
    \Re \spro{Au}{u} + \epsilon \|u\|^2_{(\mu)}  \ge C\|u\|_{(s)}^2\label{ungleichungvonmelin}
\end{equation}
für alle $u\in\rmC_0^\infty(K)$.
\item
Für alle $(x,\xi)\in\Omega\times\R^n$ mit $a_m(x,\xi)=0$ gilt
\begin{equation}
   \Re a_{m-1} +  \frac 1 2 {\mathrm{Tr}}^+ H_{a_m} \ge 0,
\end{equation}
wobei $H_{a_m}$ die Hessematrix zu $a_m$ bezeichne.
\end{enumerate}
\end{thm}

%Die Hessematrix $H_{a_m}$ kann in Blöcke aufgeteilt werden, die homogen vom Grad $m$, $m-1$ oder $m-2$ sind. Für $\lambda>0$ gilt
%\begin{align*}
%H_{a_m}(x,\lambda\xi) = \lambda^{m-1}\chi_\lambda^\mathrm{T}H_{a_m}(x,\xi)\chi_\lambda \quad \text{mit } \chi_\lambda=\left(\begin{matrix}
%\lambda^{1/2}E_n& 0\\
%0& \lambda^{-1/2}E_n
%\end{matrix}\right)\,.
%\end{align*}
%Da $\chi_\lambda\in\mathrm{Sp}_{2n}(\R)$ bleibt die positive Spur von $H_{a_m}$ bei dieser Transformation erhalten und somit ist $\mathrm{Tr}^+H_{a_m}(x,\xi)$ homogen vom Grad $m-1$ in $\xi$.

Zuerst soll die allgemeine Situation im Satz auf einen einfacheren Fall zurückgeführt werden. Der Operator $\braket{\D}^{-\rho}$ besitzt auf $\mathrm C^\infty_0(\R^n)$ die Inverse $\braket{\D}^{\rho}$, ist eigentlich getragen und $L^2$-symmetrisch, außerdem gilt
\begin{align*}
\braket{\D}^\rho\in\mathrm{OP} S_\mathrm{unif}^\rho(\R^n\times\R^n)\text{ und }\Norm{\braket{\D}^\rho u}_{(s)}=\Norm{u}_{(s+\rho)}\,.
\end{align*}
Es sei nun $A_\rho:=\braket{\D}^{-\rho}A\braket{\D}^{-\rho}$. Wenn man in \eqref{ungleichungvonmelin} $u$ durch $\braket{D}^{-\rho}u$ ersetzt, so wird ersichtlich, dass die Ungleichung für $A$ und $\mu=(m-1)/2$ genau dann erfüllt ist, wenn sie für $A_\rho$ mit $\mu=(m-\rho-1)/2$ und $s$ ersetzt durch $s-\rho$ erfüllt ist. Desweiteren gilt $A_\rho\in \mathrm{OP}S^{m-2\rho}(\Omega\times\R^n)$. Da für das Symbol von $A_\rho$ die lokale  Entwicklung
\begin{align*}
\sigma_{A_\rho}(x,\xi) \sim \sum_{\alpha}\frac 1 {\alpha!} (\partial_\xi^\alpha\braket{\xi}^{-\rho})(\D_x^\alpha \sigma_{A}(x,\xi)\braket{\xi}^{-\rho}) \sim \sum_{k=0}^\infty\sum_{\alpha} c_\alpha a_{m-k}(x,\xi)\braket{\xi}^{-2\rho-\abs\alpha}\partial^\alpha_\xi\braket{\xi}
\end{align*}
existiert, ist das Hauptsymbol von $A_\rho$ durch $\braket{\xi}^{-2\rho}a_m(x,\xi)$ gegeben. Der Entwicklungsterm der Ordnung $m-1-2\rho$ ist $\braket{\xi}^{-2\rho}a_{m-1}(x,\xi)$

\section{Beweis der Ungleichung von Melin}

\subsection{Beweis der Hinrichtung}

Es sei $(x_0,\xi_0)\in\Omega\times\R^n$ mit $a_1(x_0,\xi_0)=0$ und $K\Subset \Omega$ so gewählt, dass  $(x_0,\xi_0)\in \overset{\circ} {K}\times\R^n$. Wir können ohne Beschränkung der Allgemeinheit annehmen, dass $x_0=0$. 
Nun sei 
\begin{align*}
 v\in\mathrm C^\infty_0(\R^n) \quad \text{und} \quad v_\lambda(x):=\lambda^{n/2}v(\lambda x)\, e^{i\lambda^2x\xi_0}\, ,\quad\lambda>0
\end{align*}
Es gilt $\|v\|=\|v_\lambda\|$ und $\supp v_\lambda \subset K$ für genügend große $\lambda$. Außerdem folgt mit entsprechenden Koordinatentransformationen
\begin{align*}
\spro {A v_\lambda} {v_\lambda} = {2\pi}^{-n/2}\int\int e^{ix\xi} \sigma(x/\lambda,\lambda^2\xi_0 + \lambda\xi)\widehat v(\xi)\cc{v(\xi)}\d\xi\d x\,.
\end{align*}
Für homogenen Entwicklungsterm $a_k\in\S^k_\mathrm{lok}(\Omega\times\R^n)$ gilt
\begin{align*}
&{a_k}^\alpha_\beta(\widetilde x, \lambda^2 \widetilde \xi)(x/\lambda)^\beta(\lambda\xi)^\alpha = \mathcal O(1)\braket{\widetilde\xi}^{k-\abs\alpha}\braket{\xi}^{\abs\alpha}\lambda^{2k-\abs{\beta+\alpha}} = \mathcal O(1)\braket{\xi}^{k}\lambda^{2k-\abs{\beta+\alpha}}
\end{align*}
mit $(\widetilde x,\widetilde\xi)\in[0,x/\lambda]\times[\xi_0,\xi_0 + \xi/\lambda]$ beliebig. Wobei $\mathcal O(1)$ eine in $\xi$ und $\lambda\in[1,\infty)$ beschränkte Funktion darstellt. Da der Ausdruck nach Annahme $a_1\ge 0$ müssen bei diesem Minimum die Ableitungen ersten Grades verschwinden müssen und die Terme niedriger Ordnung wie oben zusammengefasst werden können, kann die Taylor Entwicklung von $\sigma$ in $\xi$ und $x$ um den Entwicklungspunkt $(0,\lambda^2\xi_0)$ also durch
\begin{align*}
\sigma(x/\lambda, \lambda^2\xi_0+\lambda\xi) = a_0(0,\xi_0) + \sum_{\abs{\alpha+\beta}=2} {a_1}^\alpha_\beta(0,\xi_0)\frac{x^\beta\xi^\alpha}{\alpha!\beta!} + \mathcal O(1) \lambda^{-1}\braket{\xi}^4
\end{align*}
beschrieben werden. Wird dies nun in die Ungleichung von Melin mit $u=v_\lambda$ und $s=-1$ eingesetzt, so folgt
\begin{align*}
\Re\int \cc{v(x)}\sum_{\abs{\alpha+\beta}=2}\frac{{a_1}^\alpha_\beta(0,\xi_0)}{\alpha!\beta!}x^\beta\D^\alpha v(x)\d x + (\epsilon +\Re a_0(0,\xi_0))\|v\|_{(0)}^2 + \mathcal O(1)\lambda^{-1}\ge C\|v_\lambda\|_{(-1)}^2
\end{align*}
Wegen $\widehat{v}_\lambda(\xi)=\widehat{v}(\xi/\lambda-\xi_0\lambda)$ und \eqref{oentwrueck} folgt außerdem
\begin{align*}
\|v_\lambda\|_{(-1)} = \int \braket{\lambda^2\xi_0+\lambda \xi}^{-1}|\widehat{v}(\xi)|\d\xi = \mathcal{O}(1)\lambda^{-1} 
\end{align*}
Mit $\lambda\to\infty$ und anschließend $\epsilon \to 0$, bekommen wir somit
\begin{align*}
\Re\int \cc{v(x)}\sum_{\abs{\alpha+\beta}=2}\frac{{a_1}^\alpha_\beta(0,\xi_0)}{\alpha!\beta!}x^\beta\D^\alpha v(x)\d x + \Re a_0(0,\xi_0))\int|v(x)|^2\d x\ge 0
\end{align*}
Die symmetrische Matrix $H$, die die obige quadratische Form erzeugt, hat die Koeffizienten $1/2\,\Re {a_1}^\alpha_\beta(0,\xi_0)$. Mit $H=1/2\, H_{a_1}(0,\xi_0)$ und Satz \ref{satz:thm2.4} folgen nun die Eigenschaften \eqref{bedfuerungleichung}.

\subsection{Beweis der Rückrichtung}

Es sei $P$ ein Pseudo-Differentialoperator auf dem Gebiet $\Omega$ mit Symbol $\sigma$. Wir untersuchen, wie gut die zur Taylorentwicklung des Symbols zugehörigen Differentialoperatoren
\begin{align}
 P_j^\delta v (x):=\sum_{\abs \alpha \le 2} \frac{\partial ^\alpha\sigma(x,\xi^j/\delta)}{\alpha!}\left( \mathrm D - \xi^j/\delta\right)^\alpha v(x)
\end{align}
für $v\in\mathrm {C}^\infty(\R^n)$ und $\delta\in(0,1]$ den Operator $P$ approximiert.

\begin{lem}
Sei $P$ Pseudodifferentialoperator mit Symbol $\sigma\in S^1_{\rm comp}(\R^n\times\R^n)$. Dann gilt
%Ist der Pseudo-Differentialoperator $ P$ von erster Ordnung und verschwindet dessen Symbol außerhalb einer kompakten Menge $K$, also $\sigma(x,\xi)=0$ für $x\in\R^n\setminus K$, dann gilt die Ungleichung
\begin{align}\label{hilfsungl1}
\abs {
\spro {P u}  u - \sum_{j\in\Z^n}\spro{ P_j^\delta\psi_j(\delta\mathrm D )u} { \psi(\delta\mathrm D )u }
}
\le C\delta \Norm u ^2_{(0)} + C_\delta \Norm u ^2_{(-1/4)}\ ,\quad u\in\mathrm {C}^\infty_0(\R^n)\ .
\end{align}
\end{lem}

\begin{proof}
Es sei $u\in\mathrm C^\infty_0(\R^n)$. Da das Symbol $\sigma$ für $x\in\R^n\setminus$ verschwindet, existiert die Fouriertransformation bezüglich $x$, sie wird mit $\widehat\sigma (\eta,\xi)$ bezeichnet. Da außerdem $ P u\in\mathrm C^\infty_0(\R^n)$, gilt
\begin{align*}
\widehat{ P u}(\eta) &= (2\pi)^{-n}\int e^{-i\eta x} \int e^{i\xi x}\sigma(x,\xi) \widehat u (\xi) \d\xi\d x = (2\pi)^{-n}\int \widehat\sigma(\eta - \xi,\xi)\widehat u(\xi)\d\xi \\
&= (2\pi)^ {-n}\int \sum_{j\in\Z^n} \frac 1 2\left[(\psi_j(\delta \xi) - \psi_j(\delta \eta))^2 + 2 \psi_j(\delta\xi)\psi(\delta \eta)\right] \widehat\sigma(\eta - \xi,\xi)\widehat u(\xi)\d\xi\,.
\end{align*}
Des weiteren gilt $\widehat{\partial^\alpha_\xi\sigma(\cdot,\xi)} = \partial_\xi^\alpha\widehat \sigma(\cdot,\xi)$ und somit
\begin{align*}
\mathcal{F}\left[\partial^\alpha\sigma(\cdot,\xi^j/\delta)\left( \mathrm D - \xi^j/\delta\right)^\alpha\psi_j(\delta\mathrm D)u\right](\eta) = \int \partial ^\alpha\widehat \sigma(\eta-\xi,\xi^j/\delta)\left( \xi - \xi^j/\delta\right)^\alpha\psi_j(\delta\xi)\widehat u(\xi)\d\xi\,.
\end{align*}
Unter Ausnutzung des Satzes von Plancherel folgt nun
\begin{align}
\spro {Pu} u  - \sum_{j\in\Z^n}\spro{ P_j^\delta\psi_j(\delta\mathrm D )u} { \psi(\delta\mathrm D )u } = (2\pi)^{-n} \int\int H_\delta (\xi,\eta) \widehat u(\xi)\widehat u(\eta)\d\xi\d\eta\,,
\end{align}
dabei ist die Funktion $H_\delta$ durch
\begin{align}\label{eq:hdeltadef}
H_\delta(\xi,\eta) := &\frac 1 2 \sum_{j\in \N}(\psi_j(\delta \xi) - \psi_j(\delta \eta))^2\widehat b(\eta-\xi,\xi)\\
&+ \sum_{j\in\Z^n} \psi_j(\delta\xi)\psi_j(\delta \eta)\left[
\widehat \sigma(\eta-\xi,\xi) - \sum_{\abs \alpha \le 2} \frac{\partial ^\alpha\widehat\sigma(\eta-\xi,\xi^j/\delta)}{\alpha!}\left( \xi - \xi^j/\delta\right)^\alpha
\right]
\end{align}
definiert. Nun soll die Funktion $H_\delta$ nach oben abgeschätzt werden. Da $\sigma\in S^1_{\text{unif}}(\R^n\times\R^n)$ folgt
\begin{align*}
\abs{\braket{\eta}^{2N}\partial^\alpha\widehat\sigma(\eta,\xi)}\le (2\pi)^{-n/2}\int_K |\braket{\mathrm D}^{2N}\partial^\alpha\sigma(x,\xi)|\d x \le C_{\alpha,2N} \braket{\xi}^{1 - \abs \alpha}
\end{align*}
und somit existiert für jedes $N\in\N_0$ und jeden Multiindex $\alpha$ eine Konstante $C_{\alpha,N}> 0 $, so dass
\begin{align}\label{eq:fourierordnungsabsch}
\abs {\partial ^\alpha \widehat \sigma(\eta,\xi)} \le C_{\alpha,N}\braket{\eta}^{-N}\braket{\xi}^{1-\abs\alpha}
\end{align}
Mit Hilfe des Taylorrestglieds
\begin{align*}
R_N[\widehat \sigma(\eta,\cdot)](\xi,a):= N \int_0^1\sum_{\abs \alpha = N} \frac{(1-t)^N(\xi-a)^\alpha}{\alpha!}\partial^\alpha\widehat \sigma(\eta,\xi+t(a-\xi))\d t
\end{align*}
und einer Taylorentwicklung im zweiten Argument von $\widehat \sigma$ um den Entwicklungspunkt $\xi^j/\delta$ folgt mit der Abschätzung \eqref{eq:fourierordnungsabsch}
\begin{align*}
\Abs{\widehat \sigma(\eta-\xi,\xi) - \sum_{\abs \alpha \le 2} \frac{\partial ^\alpha\widehat\sigma(\eta-\xi,\xi^j/\delta)}{\alpha!}\left( \xi - \xi^j/\delta\right)^\alpha} = \Abs{R_3[\widehat \sigma(\eta-\xi,\cdot)](\xi,\xi^j/\delta)}\\ \le
C_N\braket{\eta-\xi}^{-N}\sup_{\zeta\in[\xi,\xi^j/\delta]}\frac{|\xi-\xi^j/\delta|^3}{\braket{\zeta}^2}\,.
\end{align*}

%%%%%%%%%%%%%%%%%%%%%%%%%%%%%%%%%%%%%%%%%%%%%%%%%%%%%%%%%%%%%%%%%%%%%%%%%%%%%%%%%%%%%%%%%%%%%%%%%%%%%%%%%%%%%%%
Aus den Eigenschaften der Partition der Eins $(\psi_j)_{j\in\Z^n}$ folgt nun $|\xi\delta-\eta\delta|\le C\braket{\eta\delta}^{1/2}$ für $\psi_j(\delta\eta)\psi_j(\delta\xi)\neq 0$. Für alle $\delta\xi,\delta\eta\in\supp\psi_j$ mit $j\neq 0$ folgt nun
\begin{align*}
|\xi -\eta|\le \frac{1}{\delta}C\braket{\delta\eta}^{1/2}\le \widetilde C \delta^{-1/2}|\eta|^{1/2}
\end{align*}
sofern $\widetilde C$ groß genug gewählt wird. Weiter gilt $\widetilde C\delta^{-1/2}|\eta|^{-1/2}\le|\eta|/2$ für $|\eta|\ge 4\widetilde C^2\delta^{-1}$
\begin{align*}
\Abs{\sum_{j\in\Z^n} \psi_j(\delta\xi)\psi_j(\delta \eta)\left[
\widehat \sigma(\eta-\xi,\xi) - \sum_{\abs \alpha \le 2} \frac{\partial ^\alpha\widehat\sigma(\eta-\xi,\xi^j/\delta)}{\alpha!}\left( \xi - \xi^j/\delta\right)^\alpha
\right]}\le \\
\end{align*}
%%%%%%%%%%%%%%%%%%%%%%%%%%%%%%%%%%%%%%%%%%%%%%%%%%%%%%%%%%%%%%%%%%%%%%%%%%%%%%%%%%%%%%%%%%%%%%%%%%%%%%%%%%%%%%%%

Es gilt
\begin{align*}
\int\int\braket{\xi-\eta}^{-(n+1)}|\widehat u(\xi)\widehat u(\eta)|\d\eta\d\xi &= \int\int \braket{\eta}^{-(n+1)}|\widehat{u}(\xi||\widehat u(\eta-\xi)|\d\eta\d\xi\\
&\le \Norm{\abs{\widehat u} * \abs{\widehat u}}_\infty \norm{\braket{\cdot}^{-(n+1)}}_1\le C_{n+1} \Norm{\widehat u}_2^2
\end{align*}
Mit $|H_\delta(\xi,\eta)|\le C_N\delta\braket{\xi-\eta}^{-N} +\phi_{\delta,N}(\xi)\phi_{\delta,N}(\eta)$ und $N$ groß genug folgt nun
\begin{align*}
\int\int|H_\delta(\xi,\eta)\widehat u(\xi)\widehat u(\eta)|\d\xi\d\eta &\le \delta C_N \Norm{u}^2_{(0)} + \Norm{\phi_{\delta,N} \widehat u}_1^2 \le \delta C_N \Norm{u}^2_{(0)} + \|\phi_{\delta,N}\|_{(1/4)}^2 \Norm{\widehat u}_{(-1/4)}^2 \\
&= \delta C_N \Norm{u}^2_{(0)} +  C_{\delta,N} \Norm{ u}_{(-1/4)}^2\,.
\end{align*}

\end{proof}

Das Lemma zeigt, dass der Fehler, der durch die Partitionierung von $u$ im Fourierraum und die Taylorentwicklung des Symbols entsteht, durch in $u$ und $\delta$ beschränkten Funktionen und den Konstanten und Normen in \eqref{hilfsungl1} beschrieben werden kann. Solche Funktionen werden in Zukunft mit dem Symbol $\mathcal O(1)$ bezeichnet. Wir schreiben also
\begin{align*}
\spro {Au}{u} = \sum_{j\in\Z^n} \spro {A_j^\delta \psi_j(\delta\D)u}{\psi_j(\delta\D)u} + C\delta \mathcal{O}(1) \Norm{u}_{(0)} + C_\delta \mathcal O(1)\Norm{u}_{(-1/4)}\,.
\end{align*}

Nun soll eine weitere Lokalisation der Funktion $u$ vorgenommen werden. Dafür sei $\varphi_{j,k}(x):=\varphi_k(x|\xi^j|^{1/2})$. Für alle $v\in\C^\infty(\R^n)$ folgt nach Ausmultiplizieren und der Regel von Leibniz
\begin{align*}
\sum_{k\in\Z^n}\spro {P^\delta_j\varphi_{j,k}v} {\varphi_{j,k}v} = \sum_{k\in\Z^n}\sum_{|\alpha|\le 2}\sum_{\gamma\le\beta\le\alpha} \left({\frac{\partial^\alpha\sigma(\cdot,\xi^j/\delta)(-\xi^j/\delta)^{(\alpha-\beta)}}{(\alpha-\beta)!(\beta-\gamma)!\gamma!}(\D^\gamma\varphi_{j,k})(\D^{\beta-\gamma}v)}\ , {\varphi_{j,k}v}\right)
\end{align*}
Da außerdem $0=\D^\gamma\sum_k\phi_{j,k}^2 = 2 \sum_k\phi_{j,k}\D^\gamma\phi_{j,k}$ für $|\gamma|=1$ fallen in der oberen Summe alle Terme mit $|\gamma|=1$ weg und es folgt
\begin{align}\label{eq:vorueberlegung1}
\sum_{k\in\Z^n}\spro {P^\delta_j\varphi_{j,k}v} {\varphi_{j,k}v} = &\sum_{k\in\Z^n}\spro {\varphi_{j,k}P^\delta_jv} {\varphi_{j,k}v} \\&+ \sum_{k\in\Z^n}\sum_{|\alpha|=2}\int \frac{\partial^\alpha\sigma(x,\xi^j/\delta)}{\alpha!} |\xi^j|(\varphi_k\D^\alpha\varphi_k)(x|\xi^j|^{1/2})|v(x)|^2\d x\nonumber
\end{align}
Da $\sigma\in S^1_\text{unif}(\R^n\times\R^n)$ und somit $|\partial^\alpha\sigma(x,\xi^j/\delta)|\le\braket{\xi^j/\delta}^{-1}\le \delta|\xi^j|$ für $|\alpha|=2$ und außerdem
\begin{align*}
\sum_{k\in\Z^n} |\varphi_k(x)\D^\alpha\varphi_k(x)|\le \left(\sum_{k\in\Z^n}|\phi_k(x)|^2\right)^{1/2}\left(\sum_{k\in\Z^n}|\D^\alpha\phi_k(x)|^2\right)^{1/2}\le C_\alpha^{1/2}
\end{align*}
kann die zweite Summe aus \eqref{eq:vorueberlegung1} durch $\|v\|^2$ abgeschätzt werden. Wird nun für $v$ $\psi_j(\delta\D)$ eingesetzt und über $j$ summiert, so wird ersichtlich, dass die zweite Summe wieder von der Form $C\delta \mathcal{O}(1) \Norm{u}_{(0)} + C_\delta \mathcal O(1)\Norm{u}_{(-1/4)}$ ist, da
\begin{align*}
\left|\sum_{j,k\in\Z^n}\sum_{|\alpha|=2}\int \frac{\partial^\alpha\sigma(x,\xi^j/\delta)}{\alpha!} |\xi^j|(\varphi_k\D^\alpha\varphi_k)(x|\xi^j|^{1/2})|v(x)|^2\d x\right|\\ \le \delta C \sum_{j\in\Z^n} \norm {\psi_j(\delta\D)u}^2 = \delta C \norm {\sum_{j\in\Z^n} \psi_j(\delta \cdot)\widehat u}^2 = C \delta \Norm u ^2\,.
\end{align*}
Mit der Bezeichnung $u_\delta^{j,k} = \varphi_{j,k} \psi_j(\delta\D)u$ bedeutet dies
\begin{align*}
\spro {Pu} u = \sum_{j,k\in\Z^n} \spro {P^\delta_j u_\delta^{j,k}}{u_\delta^{j,k}} + C\delta\mathcal O(1)\norm u _{(0)} + C_\delta \Norm u _{(-1/4)}
\end{align*}
Um den verbleibenden Term abzuschätzen wird nun eine Koordinatentransformation durchgeführt. Hierfür sei $x^k = \xi^k |\xi^k|^{-1/2}\in\supp \phi_k$, $x^{j,k}=x^k |\xi^j|^{-1/2}$ und weiterhin sei $v^{j,k}_\delta$ durch die Relation
\begin{align*}
u^{j,k}_\delta(x) = v^{j,k}_\delta ((x-x^{j,k})|\xi^j|^{1/2}) e^{ix\xi^j/\delta}
\end{align*}
definiert. Es folgt
\begin{align*}
|v^{j,k}_\delta(y)| = \abs{e^{-i(y+x^k)x^j}u^{j,k}_\delta((y+x^k)|\xi^j|^{(-1/2)})} = |\varphi_k(x^k+y)\psi_j(\delta\D)u((y+x^k)|\xi^j|^{(-1/2)})|
\end{align*}
Da aus den Eigenschaften der Zerlegung der Eins $\supp \phi_k \subset B(x^k,C)$ folgt, gilt auch
\begin{align*}
\supp v^{j,k}_\delta \subset B(0,C)
\end{align*}
Außerdem impliziert die Wahl der Transformation für $y=(x-x^{j,k})|\xi^j|^{1/2}$
\begin{align*}
(\D - \xi^j/\delta)^\alpha u^{j,k}_\delta(x)=  e^{i(y+x^k)x^j}|\xi^j|^{\abs\alpha/2}\D^\alpha v^{j,k}_\delta(y)
\end{align*}
Somit folgt mit der Koordinatentransformation und einer Taylorentwicklung der Ordnung $2-\abs\alpha$ von $\sigma^\alpha$ in $x$ um den Entwicklungspunkt $x^{j,k}$
\begin{align}
\spro {P^\delta_j u_\delta^{j,k}}{u_\delta^{j,k}}
=& \sum_{\abs\alpha\le 2} |\xi^j|^{(\abs\alpha-n)/2}\int \frac{\sigma^\alpha(x^{j,k}+y|\xi^j|^{-1/2},\xi^j/\delta)}{\alpha!}\cc{v^{j,k}_\delta(y)}\D^\alpha v^{j,k}_\delta(y)\d y\nonumber\\ \label{termnachkoordtransf}
=& \sum_{\abs{\alpha+\beta}\le 2}|\xi^j|^{(\abs\alpha-\abs\beta-n)/2} \frac{\sigma^\alpha_\beta(x^{j,k},\xi^j/\delta)}{\alpha!\beta!}\int\cc{v^{j,k}_\delta(y)}y^\beta\D^\alpha v^{j,k}_\delta(y)\d y\\
&+\sum_{\abs\alpha\le 2}|\xi^j|^{(\abs\alpha-n)/2}\int R_{d}[\sigma^\alpha(\cdot,\xi^j/\delta)](x^{j,k}+y|\xi^j|^{-1/2},x^{j,k})\cc{v^{j,k}_\delta(y)}\D^\alpha v^{j,k}_\delta(y)\d y\nonumber
\end{align}
Das Taylorrestglied $R_d$, wobei $d=2-\abs\alpha+1$, kann durch einen Term $\mathcal O(1)\delta^{-1}|\xi^j|^{-(\abs\alpha+1)/2}$ außerhalb des Integrals ersetzt werden, da
\begin{align*}
&\Abs{R_3[\sigma^\alpha(\cdot,\xi^j/\delta)](x^{j,k}+y|\xi^j|^{-1/2},x^{j,k})} \\
&\le d\sum_{\abs\beta=3}\frac{|y^\beta||\xi^j|^{-3/2}}{\beta!}\Abs{\sigma^\alpha_\beta(R_3[\sigma^\alpha(\cdot,\xi^j/\delta)](x^{j,k}+y|\xi^j|^{-1/2},x^{j,k}),\xi^j/\delta) }\\
&\le d\ n^d\frac{\abs y ^d}{\abs{\xi^j}^{d/2}}\braket{\xi^j/\delta}^{1-\abs\alpha} \le d\ n^d\frac{\widetilde C\abs y ^d}{\abs{\xi^j}^{d/2}}|\xi^j/\delta|^{1-\abs\alpha} \le d\ n^d\widetilde C C ^d\delta^{-1}|\xi^j/|^{-(1+\abs\alpha)/2}
\end{align*}
%%%%%%%%%%%%%%%%%%%%%%%%%%%%%%%%%%%%%%%%%%%%%%%%%%%%%%%%%%%%%%%%%%%%%%%%%%%%%%%%%%%%%%%%%%%%%%%%%%%%%%%%%%%%%%%
für alle $y\in\supp\varphi_k$. Dabei kannd $\widetilde C = (\min_j|\xi^j|)^{-1} + 1$ gewählt werden und ist somit genau wie die Konstante $C$ unabhängig von $\alpha$, $u$, $j$, $k$ oder $\delta$. Ferner ist der Träger aller $v_\delta^{j,k}$ in $B(0,C)$ enthalten, also ...
%\begin{align*}
%\int  \abs{\sum_{\abs\alpha\le 2}\cc{v^{j,k}_\delta(y)}\D^\alpha v^{j,k}_\delta(y)}\d y \le \norm{v^{j,k}_\delta}\ \norm{\sum_{\abs\alpha\le 2}D^\alpha v^{j,k}_\delta}\le c \norm{\sum_{\abs\alpha\le 2}D^\alpha v^{j,k}_\delta}^2
%\end{align*}
%%%%%%%%%%%%%%%%%%%%%%%%%%%%%%%%%%%%%%%%%%%%%%%%%%%%%%%%%%%%%%%%%%%%%%%%%%%%%%%%%%%%%%%%%%%%%%%%%%%%%%%%%%%%%%%

Also hat die zweite Summe in \eqref{termnachkoordtransf} die Form
\begin{align*}\label{restnachkoord}
\mathcal O(1) \delta^{-1}|\xi^j|^{-1/2}\sum_{\abs\alpha\le 2}\int  |\D^\alpha v^{j,k}_\delta(y)|^2\d y
\end{align*}
Um das verbleibende Integral auszurechnen werden die neuen Funktionen $v^j_\delta$ durch die Relation
\begin{align*}
\psi^j(\delta\D)u(x) = e^{ix\xi^j/\delta}v^j_\delta(x|\xi^j|^{1/2})
\end{align*}
eingeführt. Aus der Relation folgt
\begin{align*}
\widehat v^j_\delta ((\xi-\xi^j/\delta)|\xi^j|^{-1/2}) = \psi_j(\delta\xi)\widehat u(\xi)
\end{align*}
und somit
\begin{align*}
\xi \in\supp\widehat v^j_\delta \quad\Leftrightarrow\quad \xi|\xi^j|^{1/2} + \xi^j/\delta\in\supp \psi_j(\delta(\cdot))\widehat u
\end{align*}
Außerdem gilt $|\delta\xi-\xi^j|\le C|\xi^j|^{1/2}$ für $\xi\in\supp \mathcal F[\psi_j(\delta\D)u]\subset\supp \psi_j(\delta (\cdot))$ und somit folgt
\begin{align*}
|\xi|\le C/\delta \quad \text{für } \xi \in\supp\widehat v^j_\delta\,.
\end{align*}
Darüber hinaus lassen sich die Funktionen $\varphi_k v^j_\delta$ durch Translation in die Funktionen $v^{j,k}_\delta$ überführen. In expliziter Form lautet diese Beziehung
\begin{align*}
\varphi_k(y) v^j_\delta (y) = v^{j,k}_\delta (y-x^{j,k}|\xi^j|^{1/2})\quad\text{für }y\in\R^n\,.
\end{align*}
Nun soll das über $k$ summierte verbleibende Integral aus \eqref{termnachkoordtransf} abgeschätzt werden. Dafür betrachten wir, den folgenden Ausdruck
\begin{align*}
\sum_{k\in\Z^n} \int|y^\beta\D^\alpha v^{j,k}_\delta(y)|^2\d y &\le C^{\abs\beta} \sum_{k\in\Z^n} \int |\D^\alpha(\varphi_k v^j_\delta)(y)|^2\d y \\
&\le C^{\abs\beta} \sum_{k\in\Z^n}\sum_{\gamma\le\alpha}\binom{\alpha}{\gamma}\int |(\D^\gamma v^j_\delta)(\D^{\alpha-\gamma})\varphi_k|^2\d y\\
&\le C^{\abs\beta}\sum_{\gamma\le\alpha}C_{\alpha-\gamma}\int |\D^\gamma v^j_\delta|^2\d y\\
&\le C^{\abs\beta}\sum_{\gamma\le\alpha}C_{\alpha-\gamma}C^{\abs\gamma}\delta^{-\abs\gamma}\int |\widehat v^j_\delta|^2\d y\\
&= C^{\abs\beta}\sum_{\gamma\le\alpha}C_{\alpha-\gamma}C^{\abs\gamma}\delta^{-\abs\gamma}|\xi^j|^{n/2}\int |\psi_j(\delta\D)u(y)|^2\d y
\end{align*}
Wird nun über alle $j$ summiert, so dass $\xi^j>\delta^{-12}$ ist, erhält man den Ausdruck
\begin{align*}
\sum_{j\colon \xi^j>\delta^{-12}} \delta^{-5} |\xi^j|^{-1/2}\int |\psi_j(\delta\D)u(y)|^2\d y = \mathcal O(1)\delta\norm u _{(0)}^2
\end{align*}
Die restlichen endlich vielen Summanden ergeben einen Ausdruck $\mathcal O(1)C_\delta \norm u _{(-1/4)}^2$. Somit folgt
\begin{align*}
\sum_{j\in\Z^n}\delta^{-5} |\xi^j|^{-1/2}\int |\psi_j(\delta\D)u(y)|^2\d y = \mathcal O(1)\delta\norm u _{(0)}^2 + \mathcal O(1)C_\delta \norm u _{(-1/4)}^2
\end{align*}
Zusammengefasst ergibt dies
\begin{align*}
\spro {Pu} u
=& \sum_{j,k\in\Z^n}\sum_{\abs{\alpha+\beta}\le 2}|\xi^j|^{(\abs\alpha-\abs\beta-n)/2} \frac{\sigma^\alpha_\beta(x^{j,k},\xi^j/\delta)}{\alpha!\beta!}\int\cc{v^{j,k}_\delta(y)}y^\beta\D^\alpha v^{j,k}_\delta(y)\d y\\
&+\mathcal O(1)\delta\norm u _{(0)}^2 + \mathcal O(1)C_\delta \norm u _{(-1/4)}^2\nonumber
\end{align*}
In Zukunft soll $\delta$ so gewählt sein, dass der $\mathcal O(1)\delta\norm u ^2_{(0)}$ Term größer als $-(\epsilon/3)\norm u ^2_{(0)}$ ist.
