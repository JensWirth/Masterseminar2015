% !TEX root = main.tex
\chapter{Operatoren mit variablen Koeffizienten}

Zur Notation: Sei $\Omega\subset\R^n$ ein Gebiet und bezeichne weiterhin 
\begin{equation}
    \|u\|_{(m)} = \sum_{|\alpha|\le m} \|\D^\alpha u\|
\end{equation}
die $m$-te Sobolevnorm einer Funktion $u\in\rmC_0^\infty(\Omega)$. Im folgenden betrachten wir mininmale Operatoren zu Differentialausdrücken
\begin{equation}
   P(x,\D) = \sum_{|\alpha|\le m} a_\alpha(x) \D^\alpha
\end{equation}
mit Koeffizienten $a_\alpha\in\rmC^\infty(\Omega)$. Offensichtlich gilt $\|P(x,\D)u\|\le C\|u\|_{(m)}$ für alle $u\in\rmC_0^\infty(\Omega)$ mit einer nur von der Größe der Koeffizienten abhängenden Konstanten $C$.
Wir definieren \eIndex[Differentialoperator]{Symbol} und \eIndex[Differentialoperator]{Hauptsymbol}
\begin{equation}
   P(x,\zeta) = \sum_{|\alpha|\le m} a_\alpha(x)\zeta^\alpha,\qquad p(x,\zeta)=\sum_{|\alpha|=m} a_\alpha(x)\zeta^\alpha
\end{equation}
als Polynome auf $\C^n$ mit Koeffizienten aus $\rmC^\infty(\Omega)$. Analog zu den schon für den Fall konstanter Koeffizienten eingeführten Bezeichnungen nennen wir den Differentialausdruck $P(x,\D)$ im Punkt $x\in\Omega$
\begin{itemize}
\item \eIndex[Differentialoperator]{elliptisch}, falls für alle (reellen) $\xi\in\R^n\setminus\{0\}$ das Hauptsymbol $p(x,\xi)\ne0$ erfüllt;
\item \eIndex[Differentialoperator]{vom Haupttyp}, falls die Nullstellenmenge des Hauptsymbols 
\begin{equation}
    p(x,\xi) = 0 \quad \Longrightarrow \quad \nabla_\xi p(x,\xi)\ne0
\end{equation}
für alle $\xi\in\R^n\setminus\{0\}$ erfüllt.
\end{itemize}
Während die für Operatoren mit konstanten Koeffizienten gezeigten Aussagen und Abschätzungen im wesentlichen unabhängig vom Gebiet waren, 
treten bei Operatoren mit variablen Koeffizienten neue Effekte auf. Betrachtet man jedoch hinreichend {\em kleine} Gebiete, so kann man Abschätzungen auf den Fall konstanter Koeffizienten zurückführen. Das soll am Beispiel der Elliptizitätsabschätzung
\begin{equation}\label{eq:4:locEll}
\forall u\in\rmC_0^\infty(\omega)\quad:\quad    \| u\|_{(m)} \le C \| P(x,\D) u \|
\end{equation}
und der Haupttypabschätzung 
\begin{equation}\label{eq:4:locHT}
\forall u\in\rmC_0^\infty(\omega)\quad:\quad   \| u\|_{(m-1)} \le C \| P(x,\D) u \|
\end{equation}
diskutiert werden. Hierbei sei $x_0\in\Omega$ fest gewählt und $\omega\Subset\Omega$ eine hinreichend kleine Umgebung von $x_0$. Die Größe von $\omega$ hängt vom Verhalten der Koeffizienten ab. Die Gültigkeit einer solchen lokalen Abschätzung für jedes $x_0\in\Omega$ impliziert offenbar das entsprechende Resultat auf jeder relativ kompakten Teilmenge $\Omega'\Subset\Omega$. Konstanten hängen von der Wahl von $\omega$ beziehungsweise $\Omega'$ ab.



\section{Hinreichende Bedingungen}


\begin{thm}
Angenommen, $P(x,\D)$ ist elliptisch. Dann existiert zu jedem $x_0\in\Omega$ eine Umgebung $\omega\Subset\Omega$, so dass
\eqref{eq:4:locEll} mit einer von $\omega$ abhängigen Konstanten $C$ gilt. 
\end{thm}
\begin{proof}
Ohne Beschränkung der Allgemeinheit nehmen wir an der Ursprung liege im Gebiet und es gelte $x_0=0$. Sei weiter $P(\D)=P(0,\D)=\sum_\alpha a_\alpha(0)\D^\alpha$ der Operator der entsteht, wenn man in die Koeffizienten den Wert $0$ einsetzt. Dann gilt für alle $u\in\rmC_0^\infty(\Omega)$
\begin{equation}
 \|u\|_{(m)} \le C  \| P(\D) u\| 
\end{equation}
da $P(0,\D)$ elliptischer Operator mit konstanten Koeffizienten ist. Weiter existiert zu jedem $\epsilon>0$ ein $\delta>0$, so dass
$|a_\alpha(x)-a_\alpha(0)|\le \epsilon$ für $|x|\le\delta$. Damit impliziert die Dreiecksungleichung
\begin{equation}
  \| P(\D) u - P(x,\D) u\| \le \epsilon \sum_{|\alpha|\le m} \|\D^\alpha u\| = \epsilon \|u\|_{(m)}
\end{equation}
für alle $u\in\rmC_0^\infty(B_\epsilon)$. Für $\epsilon$ klein genug folgt damit die Behauptung.
\end{proof}

Ein Operator $P(x,\D)$ wird als  \eIndex[Differentialoperator]{von reellem Haupttyp}, falls Hauptsymbol $p(x,\xi)$ reellwertig ist. Unter der Voraussetzung stimmen $P(x,\D)$ und sein formal adjungierter ${}^tP(x,\D)$ bis auf einen Operator der Ordnung $m-1$ überein. Allgemeiner heißt $P(x,\D)$ \eIndex[Differentialoperator]{wesentlich normal}, falls die  \eIndex{Poissonklammer} des Hauptsymbols $p$ mit $\overline p$ definiert durch $\overline p(x, \overline\zeta) = \overline{p(x,\zeta)}$
\begin{equation}
    \{ p,\overline p\} (x,\xi) = \sum_{j=1}^n \bigg(\frac{\partial p(x,\xi)}{\partial \xi_j} \frac{\partial \overline p (x,\xi)}{\partial x_j} - \frac{\partial \overline p(x,\xi)}{\partial \xi_j}\frac{\partial  p(x,\xi)}{\partial x_j} \bigg)    = 0 
\end{equation}
für alle $\xi\in\R^n$ verschwindet. In diesem Fall ist der Kommutator von $P(x,\D)$ und ${}^tP(x,\D)$ von Ordnung $2m-2$ (statt nur $2m-1$).

Das folgende Resultat ergibt sich aus \cite[Theorem~4.1]{Hormander:1955} zusammen mit der Vorbemerkung zu diesem Kapitel. Die Umkehrung dieses Satzes gilt nicht.

\begin{thm}[{\cite[Theorem~4.1]{Hormander:1955}}]
Angenommen, $P(x,\D)$ ist vom Haupttyp und wesentlich normal. Dann existiert zu jedem $x_0\in\Omega$ eine Umgebung $\omega\Subset\Omega$, so dass
\eqref{eq:4:locHT} mit einer von $\omega$ abhängigen Konstanten $C$ gilt.
\end{thm}
\begin{proof}[Beweisskizze]
Für den Originalbeweis konstruiert man wiederum Energieintegrale und schätzt diese mittels partieller Integration ab.
\marginpar{zu Ergänzen}
\end{proof}

\section{Notwendige Bedingungen}
Im folgenden sei $\Omega \subset \R^n$ eine offene Umgebung der Null. 
\begin{lem}
Wir nehmen an, es existiert eine Funktion $u \in \rmC^\infty (\Omega)$, sodass 
\begin{align}
\label{grad}
p(x, \nabla u) = o \left(|x|^2\right), \qquad  x \rightarrow 0
\end{align}
und $u$ eine Taylorentwicklung 
\begin{align}
	\label{taylor}
u(x) = \i \sum_{j=1}^{n} x_j \xi_j + \dfrac{1}{2} \sum_{j,k=1}^{n} x_j x_k \alpha_{jk} + \mathcal O\left( |x|^3 \right), \qquad x\to0
\end{align}
mit reellen $\xi_j$, einer symmetrischen Matrix $(\alpha_{jk})$ und $\Re \alpha_{jk}$ positiv definit besitzt. Gilt weiterhin  
\begin{align}
\label{normpart}
\sum_{j=1}^{n}|\partial p(0,\xi)/\partial \xi_j|^2 \neq 0,
\end{align}
so folgt
\begin{align}
\label{lm 1 eq}
\sup \frac{ \lVert v \rVert_{m-1} }{ \lVert P(x,D)v \rVert_{L^2}} = \infty
\end{align}
für alle $v \in C_0^\infty(\Omega)$.
\end{lem}
\begin{proof}
Weil $\Re \alpha_{jk}$ negativ definit ist, gilt\begin{align*}
\Re u(x) = \sum_{j,k=1}^{n} x^jx^k \Re \alpha_{jk} + O\left( |x|^3 \right) \le -2a |x|^2 + O\left( |x|^3 \right) (?)
\end{align*}
mit $a<0$. Es folgt
\begin{align}
\label{realteil}
\Re u(x) \le -a |x|^2 (?)
\end{align}
für kleine $|x|$. Nach Verkleinerung von $\Omega$ nehmen wir an, dass (\ref{realteil}) überall gilt. Für $\phi \in C_0^\infty(\Omega)$, $t>0$ ist $v_t := \phi e^{tu} \in C_0^\infty(\Omega)$ und es gilt $|v_t| \le |\phi| e^{-at|x|^2}$. Wir zeigen nun
\[
\frac{\lVert v_t \rVert_{m-1}}{\lVert P(x,D)v_t \rVert_{L^2}} \rightarrow \infty, \, t \rightarrow \infty,
\]
woraus die Behauptung (\ref{lm 1 eq}) folgt.
\end{proof}
\begin{lem}
Wir nehmen an, die Koeffizienten von $p$ seien $C^2$ im Ursprung und es gelte (\ref{normpart}). Dann gibt es eine Funktion $u \in C^\infty(\Omega)$ mit (\ref{grad}) und (\ref{taylor}) genau dann, wenn
\begin{align}
p(0, \xi) = 0, \\
\sum_{k=1}^{n} \alpha_{jk}p^{(k)}(0,\xi) = -i \cdot p_j(0,\xi), \hspace{0.5cm} j=1, \ldots, n,
\end{align}
wobei
\begin{align}
p_j(x,\xi) = \dfrac{\partial p(x,\xi)}{\partial x_j}.
\end{align}
\end{lem}

\begin{proof}
Für $u \in C^\infty(\Omega)$ ist klar, dass (\ref{grad}) genau dann erfüllt ist, wenn $p(x, \nabla u)$ und alle Ableitungen der Ordnung $\le 2$ im Ursprung verschwinden. (?)
\end{proof}