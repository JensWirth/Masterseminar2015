% !TEX root = main.tex
%\newcommand{\A}{\forall}
%\newcommand{\E}{\exists}
%\newcommand{\imp}{\Rightarrow}
%\renewcommand{\labelenumi}{\roman{enumi})}
%
%\newcommand{\Sph}{\mathbb S}
%\renewcommand{\P}{\mathcal{P}}
%\newcommand{\Pin}{\textbf{Pin}}
%\newcommand{\Q}{\mathbb Q}
%\renewcommand{\H}{\mathbb H}
%\newcommand{\Ri}{\widetilde{R}}
%\newcommand{\K}{\mathbb{K}}
%\newcommand{\lle}{\preccurlyeq}
%\newcommand{\G}{\mathcal{G}}
%\newcommand{\M}{\widetilde{M}}
%\newcommand{\Or}{\mathcal{O}}
%\renewcommand{\le}{\leqslant}
%\renewcommand{\ge}{\geqslant}
%\renewcommand{\O}{\textbf{O}}
%\newcommand{\SO}{\textbf{SO}}
%\newcommand{\U}{\textbf{U}}
%\newcommand{\Ad}{\text{Ad}}
%\newcommand{\SU}{\textbf{SU}}
%\newcommand{\Symp}{\textbf{Sp}}
%\newcommand{\id}{\text{id}}
%\newcommand{\Cl}{\textbf{Cl}}
%\newcommand{\Ind}{\text{Ind}}
%\newcommand{\proofed}{$\mbox{}$\hfill $\blacksquare{}$\\} 
%\newcommand{\g}{\mathfrak{g}}
%\renewcommand{\k}{\mathfrak{k}}
%\newcommand{\m}{\mathfrak{m}}
%\renewcommand{\o}{\overline}
\chapter{Operatoren von reellem Haupttyp}
%<<<<<<< HEAD
%%\cite{Hormander:1960a}
%\section{Notation und Vorbereitung}
%
%Im folgenden sei $\P$ ein Differentialoperator auf einem Gebiet $\Omega \subset \R^n$, gegeben durch
%\[
%\P = \sum_{|\alpha| \le m} a_\alpha (x) D^\alpha.
%\]
%
%
%\begin{df}
%	Für $\xi \in \C^n$ bezeichnen wir mit
%	\[
%	p(x,\xi) = \sum_{|\alpha| = m} a_\alpha (x) \xi^\alpha
%	\]
%	das \emph{Hauptsymbol} oder das \emph{charakteristische Polynom} von $\P$. $\P$ heißt von \emph{reellem Haupttyp}, wenn für reelle $\xi$ auch $p(x,\xi)$ reell ist und für reelle $\xi \neq 0$ nicht alle partiellen Ableitungen $\partial p(x,\xi) / \partial \xi_i$ gleichzeitig verschwinden.
%\end{df} 
%Wir verwenden für Vektoren $\xi^1, \ldots, \xi^m$ die Notation
%\[
%p(x,\xi_1, \ldots, \xi_m) = \sum_{|\alpha|=m} a_\alpha(x)\xi_1^\alpha \cdots \xi_m^\alpha (?)
%\]
%und für $k_1 + \ldots + k_p=m$ schreiben wir auch $p(x,\xi_1^{k_1}, \ldots, \xi_p^{k_p})$, wenn $k_i$ Argumente gleich $\xi_i$ sein sollen.
%\begin{lem}
%Für den Tensor $T^{ik}$ gilt die Gleichung
%\[
%\sum_{i=1}^{n} (\zeta_i - \overline \zeta_i) T^{ik}(\zeta, \o \zeta)  =
%p(\zeta) \dfrac{\partial p(\o \zeta)}{\partial \o \zeta_k} - p(\o \zeta) \dfrac{\partial p( \zeta)}{\partial  \zeta_k}.
%\]
%\end{lem}
%\begin{proof}
%
%\end{proof}
%\begin{df}
%	Für variable $\xi, \eta$ definieren wir die Tensoren
%	\begin{align*}
%	\sum_{i,k=1}^n R^{ik}(\zeta, \overline \zeta) \xi_i \eta_k &= m \sum_{j=0}^{m-1} p(\zeta^j, \overline \zeta^{m-1-j},\xi) p(\zeta^{m-1-j},\overline \zeta^j ,\eta), \\
%	\sum_{i,k=1}^n S^{ik}(\zeta, \overline \zeta) \xi_i \eta_k &= m \sum_{j=0}^{m-1} p(\zeta^j, \overline \zeta^{m-j}) p(\zeta^{m-1-j},\overline \zeta^{j-1} ,\xi,\eta)
%	\end{align*}
%	sowie den symmetrischen Tensor $T^{ik} = R^{ik}-S^{ik}$, wobei wir den Punkt $x$ in der Notation zur Übersichtlichkeit unterschlagen.
%\end{df}
%
%\section{a}
%Wir nehmen im Folgenden an, dass $\P$ in einer Umgebung einer Sphäre $|x| \le A$ in $\R^n$ definiert ist.
%
%\begin{thm}
%	Sei $\P$ von reellem Haupttyp, die Koeffizienten von $p(x,\xi)$ seien stetig differenzierbar und die anderen Koeffizienten von $\P$ stetig. Dann existiert eine Umgebung $\Omega$ der Null, sodass
%	\begin{align}
%	\sum_{|\alpha| \le m} \lVert D^\alpha u \rVert^2 \le C \lVert \P u \rVert^2, \hspace{0.5cm} u \in C_0^\infty (\Omega).
%	\end{align}
%\end{thm}
%
%\begin{proof}
%	
%\end{proof}
%=======
%



Zur Notation: Sei $\Omega\subset\R^n$ ein Gebiet und 
\begin{equation}
   P(x,\D) = \sum_{|\alpha|\le m} a_\alpha(x) \D^\alpha
\end{equation}
ein Differentialausdruck mit Koeffizienten $a_\alpha\in\rmC^\infty(\Omega)$. Wir definieren \eIndex[Differentialoperator]{Symbol} und \eIndex[Differentialoperator]{Hauptsymbol}
\begin{equation}
   P(x,\zeta) = \sum_{|\alpha|\le m} a_\alpha(x)\zeta^\alpha,\qquad p(x,\zeta)=\sum_{|\alpha|=m} a_\alpha(x)\zeta^\alpha
\end{equation}
als Polynome auf $\C^n$ mit Koeffizienten aus $\rmC^\infty(\Omega)$. Analog zu den schon für den Fall konstanter Koeffizienten eingeführten Bezeichnungen nennen wir den Differentialausdruck $P(x,\D)$ im Punkt $x\in\Omega$
\begin{itemize}
\item \eIndex[Differentialoperator]{elliptisch}, falls für alle (reellen) $\xi\in\R^n\setminus\{0\}$ das Hauptsymbol $p(x,\xi)\ne0$ erfüllt;
\item \eIndex[Differentialoperator]{vom Haupttyp}, falls die Nullstellenmenge des Hauptsymbols 
\begin{equation}
    p(x,\xi) = 0 \quad \Longrightarrow \quad \nabla_\xi p(x,\xi)\ne0
\end{equation}
für alle $\xi\in\R^n\setminus\{0\}$ erfüllt.
\end{itemize}
Interessant sind Operatoren \eIndex[Differentialoperator]{von reellem Haupttyp}, für diese ist zusätzlich $p(x,\xi)$ reellwertig. Dann stimmt $P(x,\D)$ bis auf einen Operator der Ordnung $m-1$ mit seinem formal adjungierten ${}^t{P}(x,\D)$ \"uberein. Für komplexwertige Hauptsymbole fordert man oft allgemeiner, dass
die \eIndex{Poissonklammer} des Hauptsymbols $p$ mit $\overline p$ definiert durch $\overline p(x, \overline\zeta) = \overline{p(x,\zeta)}$.
\begin{equation}
    \{ p,\overline p\} (x,\xi) = \sum_{j=1}^n \bigg(\frac{\partial p(x,\xi)}{\partial \xi_j} \frac{\partial \overline p (x,\xi)}{\partial x_j} - \frac{\partial \overline p(x,\xi)}{\partial \xi_j}\frac{\partial  p(x,\xi)}{\partial x_j} \bigg)    = 0 
\end{equation}
für alle $\xi\in\R^n$ verschwindet. Für solche Operatoren ist der Kommutator zwischen $P(x,\D)$ und ${}^t P(x,\D)$ ein Operator der Ordnung $2m-2$ (statt trivialerweise nur $2m-1$).

Ist $P(\D)$ ein Operator vom Haupttyp mit konstanten Koeffizienten, so existiert insbesondere eine Konstante $C$, so dass
\begin{equation}
  \forall u\in\rmC_0^\infty(\Omega)\quad:\quad  \sum_{|\beta|<m} \| \D^\beta u\| \le C \| P(\D) u \|
\end{equation}
gilt. Im Falle variabler Koeffizienten suchen wir eine solche Abschätzung lokal und fragen, ob für $u$ mit hinreichend kleinem kompaktem Träger entsprechend
\begin{equation}
\forall u\in\rmC_0^\infty(\Omega')\quad:\quad      \sum_{|\beta|<m} \| \D^\beta u\| \le C \| P(x,\D) u \|
\end{equation}
gilt.

\section{Hinreichende Bedingungen}
\cite{Hormander:1955}


\section{Notwendige Bedingungen}
Im folgenden sei $\Omega \subset \R^n$ eine offene Umgebung der Null. Zur Übersichtlichkeit schreiben wir
\begin{align*}
\lVert u \rVert_k := \left( \sum_{|\alpha| \le k} \lVert D^\alpha u \rVert^2 \right)^{\frac{1}{2}}.
\end{align*}
\begin{lem}
Wir nehmen an, es existiert eine Funktion $u \in C^\infty (\Omega)$, sodass 
\begin{align}
\label{grad}
p(x, \nabla u) = o \left(|x|^2\right), \, x \rightarrow 0.
\end{align}
Weiter habe $u$ die Taylorentwicklung \begin{align}
	\label{taylor}
u(x) = i \sum_{j=1}^{n} x_i \xi_j + \dfrac{1}{2} \sum_{j,k=1}^{n} x^j x^k \alpha_{jk} + O\left( |x|^3 \right),
\end{align}

wobei $\xi_j$ reell sei mit 
\begin{align}
\label{normpart}
\sum_{j=1}^{n}|\partial p(0,\xi)/\partial \xi_j|^2 \neq 0.
\end{align}
Zusätzlich sei die Matrix $\alpha_{jk}$ symmetrisch und die Matrix $\Re \alpha_{jk}$ negativ definit. Hat $P$ stetige Koeffizienten, so gilt
\begin{align}
\label{lm 1 eq}
\sup \frac{ \lVert v \rVert_{m-1} }{ \lVert P(x,D)v \rVert_{L^2}} = \infty
\end{align}
für alle $v \in C_0^\infty(\Omega)$.
\end{lem}
\begin{proof}
Weil $\Re \alpha_{jk}$ negativ definit ist, gilt\begin{align*}
\Re u(x) = \sum_{j,k=1}^{n} x^jx^k \Re \alpha_{jk} + O\left( |x|^3 \right) \le -2a |x|^2 + O\left( |x|^3 \right) (?)
\end{align*}
mit $a<0$. Es folgt
\begin{align}
\label{realteil}
\Re u(x) \le -a |x|^2 (?)
\end{align}
für kleine $|x|$. Nach Verkleinerung von $\Omega$ nehmen wir an, dass (\ref{realteil}) überall gilt. Für $\phi \in C_0^\infty(\Omega)$, $t>0$ ist $v_t := \phi e^{tu} \in C_0^\infty(\Omega)$ und es gilt $|v_t| \le |\phi| e^{-at|x|^2}$. Wir zeigen nun
\[
\frac{\lVert v_t \rVert_{m-1}}{\lVert P(x,D)v_t \rVert_{L^2}} \rightarrow \infty, \, t \rightarrow \infty,
\]
woraus die Behauptung (\ref{lm 1 eq}) folgt.
\end{proof}
\begin{lem}
Wir nehmen an, die Koeffizienten von $p$ seien $C^2$ im Ursprung und es gelte (\ref{normpart}). Dann gibt es eine Funktion $u \in C^\infty(\Omega)$ mit (\ref{grad}) und (\ref{taylor}) genau dann, wenn
\begin{align}
p(0, \xi) = 0, \\
\sum_{k=1}^{n} \alpha_{jk}p^{(k)}(0,\xi) = -i \cdot p_j(0,\xi), \hspace{0.5cm} j=1, \ldots, n,
\end{align}
wobei
\begin{align}
p_j(x,\xi) = \dfrac{\partial p(x,\xi)}{\partial x_j}.
\end{align}
\end{lem}

\begin{proof}
Für $u \in C^\infty(\Omega)$ ist klar, dass (\ref{grad}) genau dann erfüllt ist, wenn $p(x, \nabla u)$ und alle Ableitungen der Ordnung $\le 2$ im Ursprung verschwinden. (?)
\end{proof}