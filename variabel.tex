% !TEX root = main.tex
\chapter{Operatoren mit variablen Koeffizienten}

Zur Notation: Sei $\Omega\subset\R^n$ ein Gebiet und bezeichne weiterhin 
\begin{equation}
    \|u\|_{(m)} = \sum_{|\alpha|\le m} \|\D^\alpha u\|
\end{equation}
die $m$-te Sobolevnorm einer Funktion $u\in\rmC_0^\infty(\Omega)$. Im folgenden betrachten wir mininmale Operatoren zu Differentialausdrücken
\begin{equation}
   P(x,\D) = \sum_{|\alpha|\le m} a_\alpha(x) \D^\alpha
\end{equation}
mit Koeffizienten $a_\alpha\in\rmC^\infty(\Omega)$. Offensichtlich gilt $\|P(x,\D)u\|\le C\|u\|_{(m)}$ für alle $u\in\rmC_0^\infty(\Omega)$ mit einer nur von der Größe der Koeffizienten abhängenden Konstanten $C$.
Wir definieren \eIndex[Differentialoperator]{Symbol} und \eIndex[Differentialoperator]{Hauptsymbol}
\begin{equation}
   P(x,\zeta) = \sum_{|\alpha|\le m} a_\alpha(x)\zeta^\alpha,\qquad p(x,\zeta)=\sum_{|\alpha|=m} a_\alpha(x)\zeta^\alpha
\end{equation}
als Polynome auf $\C^n$ mit Koeffizienten aus $\rmC^\infty(\Omega)$. Analog zu den schon für den Fall konstanter Koeffizienten eingeführten Bezeichnungen nennen wir den Differentialausdruck $P(x,\D)$ im Punkt $x\in\Omega$
\begin{itemize}
\item \eIndex[Differentialoperator]{elliptisch}, falls für alle (reellen) $\xi\in\R^n\setminus\{0\}$ das Hauptsymbol $p(x,\xi)\ne0$ erfüllt;
\item \eIndex[Differentialoperator]{vom Haupttyp}, falls die Nullstellenmenge des Hauptsymbols 
\begin{equation}
    p(x,\xi) = 0 \quad \Longrightarrow \quad \nabla_\xi p(x,\xi)\ne0
\end{equation}
für alle $\xi\in\R^n\setminus\{0\}$ erfüllt.
\end{itemize}
Während die für Operatoren mit konstanten Koeffizienten gezeigten Aussagen und Abschätzungen im wesentlichen unabhängig vom Gebiet waren, 
treten bei Operatoren mit variablen Koeffizienten neue Effekte auf. Betrachtet man jedoch hinreichend {\em kleine} Gebiete, so kann man Abschätzungen auf den Fall konstanter Koeffizienten zurückführen. Das soll am Beispiel der Elliptizitätsabschätzung
\begin{equation}\label{eq:4:locEll}
\forall u\in\rmC_0^\infty(\omega)\quad:\quad    \| u\|_{(m)} \le C \| P(x,\D) u \|
\end{equation}
und der Haupttypabschätzung 
\begin{equation}\label{eq:4:locHT}
\forall u\in\rmC_0^\infty(\omega)\quad:\quad   \| u\|_{(m-1)} \le C \| P(x,\D) u \|
\end{equation}
diskutiert werden. Hierbei sei $x_0\in\Omega$ fest gewählt und $\omega\Subset\Omega$ eine hinreichend kleine Umgebung von $x_0$. Die Größe von $\omega$ hängt vom Verhalten der Koeffizienten ab. Die Gültigkeit einer solchen lokalen Abschätzung für jedes $x_0\in\Omega$ impliziert offenbar das entsprechende Resultat auf jeder relativ kompakten Teilmenge $\Omega'\Subset\Omega$. Konstanten hängen von der Wahl von $\omega$ beziehungsweise $\Omega'$ ab.



\section{Hinreichende Bedingungen für untere Schranken}


\begin{thm}\label{thm:4:4.1}
Angenommen, $P(x,\D)$ ist elliptisch. Dann existiert zu jedem $x_0\in\Omega$ eine Umgebung $\omega\Subset\Omega$, so dass
\eqref{eq:4:locEll} mit einer von $\omega$ abhängigen Konstanten $C$ gilt. 
\end{thm}
\begin{proof}
Ohne Beschränkung der Allgemeinheit nehmen wir an der Ursprung liege im Gebiet und es gelte $x_0=0$. Sei weiter $P(\D)=P(0,\D)=\sum_\alpha a_\alpha(0)\D^\alpha$ der Operator der entsteht, wenn man in die Koeffizienten den Wert $0$ einsetzt. Dann gilt für alle $u\in\rmC_0^\infty(\Omega)$
\begin{equation}
 \|u\|_{(m)} \le C  \| P(\D) u\| 
\end{equation}
da $P(0,\D)$ elliptischer Operator mit konstanten Koeffizienten ist. Weiter existiert zu jedem $\epsilon>0$ ein $\delta>0$, so dass
$|a_\alpha(x)-a_\alpha(0)|\le \epsilon$ für $|x|\le\delta$. Damit impliziert die Dreiecksungleichung
\begin{equation}
  \| P(\D) u - P(x,\D) u\| \le \epsilon \sum_{|\alpha|\le m} \|\D^\alpha u\| = \epsilon \|u\|_{(m)}
\end{equation}
für alle $u\in\rmC_0^\infty(B_\delta)$. Für $\epsilon$ klein genug folgt damit die Behauptung.
\end{proof}

Ein Operator $P(x,\D)$ wird als  \eIndex[Differentialoperator]{von reellem Haupttyp} bezeichnet, falls sein Hauptsymbol $p(x,\xi)$ reellwertig ist. Unter der Voraussetzung stimmen $P(x,\D)$ und sein formal adjungierter ${}^tP(x,\D)$ bis auf einen Operator der Ordnung $m-1$ überein. Allgemeiner heißt $P(x,\D)$ \eIndex[Differentialoperator]{wesentlich normal}, falls die  \eIndex{Poissonklammer} des Hauptsymbols $p$ mit $\overline p$ definiert durch $\overline p(x, \overline\zeta) = \overline{p(x,\zeta)}$
\begin{equation}
    \{ p,\overline p\} (x,\xi) = \sum_{j=1}^n \bigg(\frac{\partial p(x,\xi)}{\partial \xi_j} \frac{\partial \overline p (x,\xi)}{\partial x_j} - \frac{\partial \overline p(x,\xi)}{\partial \xi_j}\frac{\partial  p(x,\xi)}{\partial x_j} \bigg)    = 0 
\end{equation}
für alle $\xi\in\R^n$ verschwindet. In diesem Fall ist der Kommutator von $P(x,\D)$ und ${}^tP(x,\D)$ von Ordnung $2m-2$ (statt nur $2m-1$).

Das folgende Resultat ergibt sich aus \cite[Theorem~4.1]{Hormander:1955} zusammen mit der Vorbemerkung zu diesem Kapitel. Die Umkehrung dieses Satzes gilt nicht.

\begin{thm}[{\cite[Theorem~4.1]{Hormander:1955}}]\label{thm:4:4.2}
Angenommen, $P(x,\D)$ ist vom Haupttyp und wesentlich normal. Dann existiert zu jedem $x_0\in\Omega$ eine Umgebung $\omega\Subset\Omega$, so dass
\eqref{eq:4:locHT} mit einer von $\omega$ abhängigen Konstanten $C$ gilt.
\end{thm}
\begin{proof}[Beweisidee]
%%Für den Beweis nutzt man quadratische Energieformen der Ordnung $(m,\mu)$
%%\begin{equation}
%%   F(x,\D,\overline\D) u\overline v = \sum_{\alpha,\beta} a_{\alpha,\beta}(x) \big(\D^\alpha u(x)\big) \overline{\big(\D^\beta v(x)\big)},\qquad F(x,\zeta,\overline\zeta)=\sum_{\alpha,\beta} a_{\alpha,\beta}(x) \zeta^\alpha \overline\zeta^\beta,
%%\end{equation}
%%wobei die Summe über alle Multiindices mit $|\alpha|,|\beta|\le m$ und $|\alpha+\beta|\le \mu$ läuft. In Analogie zu Lemma~\ref{lem:2:qfsym} und \ref{lem:2:2.2}
%%impliziert $F(x,\xi,\xi)=0$ für alle $\xi\in\R^n$ die Existenz einer Energieform $G$ der Ordnung $(m-1,\mu)$ mit
%%\begin{equation}
%%    \int_\Omega F(x,\D,\overline\D) u\overline u \d x = \int_\Omega G(x,\D,\overline\D) u\overline u \d x
%%\end{equation}
%%für alle $u\in\rmC_0^\infty(\Omega)$. 
Der Beweis beruht wiederum auf geeignet konstruierten Energieformen und partieller Integration. Dazu verweisen wir auf die Originalarbeit \cite[Section 4.2]{Hormander:1955} für den Spezialfall mit reellwertigem Hauptsymbol oder auf \cite[Section 8.5]{Hormander:1963} für das allgemeine Resultat.
\end{proof}

\begin{cor}
Angenommen, $P(x,\D)$ erfüllt die Voraussetzungen von Satz~\ref{thm:4:4.1} oder Satz~\ref{thm:4:4.2}. Dann existiert zu jedem $\Omega'\Subset\Omega$ und jedem
$f\in\rmL^2(\Omega')$ ein $u\in\mathscr D'(\Omega')$ mit $P(x,\D)u=f$ in $\Omega'$.
\end{cor}
\begin{proof}
Mit $P(x,\D)$ erfüllt auch ${}^tP(x,\D)$ die Voraussetzungen des Satzes. Insbesondere ist damit der zu ${}^tP(x,\D)$ auf $\Omega'$ assoziierte minimale Operator beschränkt invertierbar und damit nach Satz~\ref{thm:1:loesbarkeit} der zu $P(x,\D)$ assoziierte maximale Operator surjektiv.
\end{proof}

\section{Notwendige Bedingungen für untere Schranken}
Im folgenden sei $\Omega \subset \R^n$ eine offene Umgebung der Null und $P(x,\D)$ ein Differentialausdruck auf $\Omega$ mit Hauptsymbol $p(x,\xi)$. 
\begin{lem}[{\cite[Lemma 2.1]{Hormander:1960a}}]\label{lem1:hoer2.2}
Angenommen, es existiert eine Funktion $u \in \rmC^\infty (\Omega)$ mit 
\begin{equation}
\label{grad}
p(x, \nabla u(x)) = \mathbf{o} \big(|x|^2\big), \qquad  x \rightarrow 0,
\end{equation}
welche eine Taylorentwicklung 
\begin{equation}\label{taylor}
u(x) = \i \xi\cdot x + x\cdot A x + \mathcal O\big(|x|^3\big),\qquad x\to0
\end{equation}
mit $\xi\in\R^n$, einer symmetrischen Matrix $A=(\boldsymbol\alpha_{i,j})_{i,j}\in\C^{n\times n}$ und $B = (\Re \boldsymbol\alpha_{i,j})_{i,j} $ negativ definit  besitzt. 
Gilt weiterhin  
\begin{equation}
\label{normpart}
\sum_{j=1}^{n}\bigg|\frac{\partial p(0,\xi)}{\partial \xi_j}\bigg|^2 \neq 0,
\end{equation}
so folgt
\begin{equation}
\label{lm 1 eq}
\sup_{v\in\rmC_0^\infty(\Omega)} \frac{ \lVert v \rVert_{(m-1)} }{ \lVert P(x,\D)v \rVert} = \infty.
\end{equation}
\end{lem}
\begin{proof}
Da die Matrix $B$ negativ definit ist, gilt
\begin{equation}
\Re u(x) = x\cdot Bx + \mathcal O\big(|x|^3\big) \le - 2 a |x|^2 + \mathcal O\big(|x|^3\big),\qquad x\to0
\end{equation}
mit geeignet gewähltem $a>0$. Damit folgt
\begin{equation}
\label{realteil}
\Re u(x) \le - a |x|^2 
\end{equation}
für hinreichend kleine $|x|$. Nach Verkleinerung von $\Omega$ können wir also ohne Einschränkung annehmen, dass (\ref{realteil}) überall in $\Omega$ gilt. 
%
Sei nun $\phi \in \rmC_0^\infty(\Omega)$ und $t>0$. Dann erfüllt $v_t := \phi \e^{tu} \in \rmC_0^\infty(\Omega)$ die Abschätzung $|v_t(x)| \le |\phi(x)| \e^{-at|x|^2}$. Wir zeigen nun
\begin{equation}
\frac{\lVert v_t \rVert_{(m-1)}}{\lVert P(x,\D)v_t \rVert} \longrightarrow \infty, \qquad t \rightarrow \infty,
\end{equation}
woraus die Behauptung (\ref{lm 1 eq}) folgt. Dazu nutzen wir
\begin{equation}
P(x,\D)v_t(x) = \e^{tu(x)} \sum_{j=0}^{m} b_j(x) t^j
\end{equation}
mit Koeffizienten $b_j\in\rmC^\infty(\Omega)$. Es reicht, $b_m$ und $b_{m-1}$ zu berechnen. Mit der Leibnizformel erhalten wir
\begin{equation}
P(x,\D)(\phi \e^{tu}) = \sum_{\alpha} \dfrac{\D^\alpha \phi}{\alpha!} P^{(\alpha)}(x,\D)\e^{tu}.
\end{equation}
Wir zerlegen nun
\begin{equation}
P(x,\D) = p(x,\D) + q(x,\D) + r(x,\D),
\end{equation}
wobei $p$ und $q$ homogen der Ordnung $m$ bzw. $m-1$ seien sowie $r$ Ordnung $<m-1$ habe. Dann sind die Koeffizienten von $t^m \e^{tu}$ und $t^{m-1} \e^{tu}$ in
\begin{equation}
\phi p(x,\D)\e^{tu} + \bigg( \phi q(x,\D) + \sum_{j=1}^{n} (\D_j \phi) p^{(j)}(x,\D) \bigg) \e^{tu}
\end{equation}
ebenfalls $b_m$ und $b_{m-1}$, denn alle anderen Summanden in der Leibnizformel haben Grad kleiner als $m-1$ in $t$. Nach Berechnung des ersten Summanden erhalten wir $b_m = (-\i)^m \phi p(x,\nabla u)$ und insbesondere wegen (\ref{grad})
\begin{equation}
b_m(x) = \mathbf{o} (|x|^2), \quad x \rightarrow 0.
\end{equation}
Weiter erhalten wir
\begin{equation}
	\label{4.17}
b_{m-1}(x) = (-\i)^{m-1} \bigg( \phi(x) (\i^{m-1}g(x) + q(x,\nabla u(x))) + \sum_{j=1}^{n} (\D_j \phi(x)) p^{(j)} (x,\nabla u(x)) \bigg),
\end{equation}
wobei $g$ eine glatte Funktion ist, die von $u$ und den Koeffizienten von $p$ abhängt. Wir wählen nun $\phi$ so, dass $\phi(0) =1$ gilt und sodass (\ref{4.17}) für $x=0$ verschwindet, was wegen (\ref{normpart}) möglich ist. Wegen Stetigkeit von $b_{m-1}$ ist dann $b_{m-1}(x) = \mathbf{o}(1)$, wenn $x$ gegen Null strebt.

Nach diesen Überlegungen können wir für $\varepsilon >0$ eine Umgebung $U$ der Null wählen, sodass
\begin{equation}
|b_m(x)| < \varepsilon |x|^2, \hspace{0.5cm} |b_{m-1}(x)| < \varepsilon, \hspace{0.5cm} \forall \, x \in U.
\end{equation}
Mit Gleichung (\ref{realteil}) erhalten wir für alle $x\in U$ die Abschätzung
\begin{equation}
|P(x,\D)v_t| \le t^{m-1} \e^{-at|x|^2}(\varepsilon t |x|^2 + \varepsilon + C/t)
\end{equation}
und mit der Transformation $x \mapsto x/\sqrt{t}$ im Integral und der Vergrößerung des Integrationsgebietes folgt
\begin{equation}
\lVert P(x,\D) v_t \rVert_{L^2(U)} ^2 \le t^{2m-2-n/2} \varepsilon^2 \int (|x|^2 + 1 +C/\varepsilon t)^2 \e^{-2a|x|^2} \d x,
\end{equation}
wobei das letzte Integral für $t \rightarrow \infty$ gegen
\begin{equation}
B^2 = \int (|x|^2 + 1)^2 \e^{-2a|x|^2} \d x
\end{equation}
konvergiert. Für große $t$ liefert dies die Abschätzung
\begin{equation}
\lVert P(x,\D) v_t \rVert_{L^2(U)} ^2 \le 2 t^{2m-2-n/2} \varepsilon^2  B^2.
\end{equation}
Da aus (\ref{realteil}) ebenso
\begin{equation}
\lVert P(x,\D) v_t \rVert_{L^2(U^c)} ^2 = \mathcal{O}(t^{2m}e^{-2ct})
\end{equation}
für eine Konstante $c>0$ folgt, erhalten wir
\begin{equation}
	\label{4.18}
\lVert P(x,\D) v_t \rVert^2 \le 4 B^2 t^{2m-2-n/2} \varepsilon^2.
\end{equation}
Als schätzen wir $\lVert v_t \rVert_{(m-1)}$ nach unten ab. Wir nehmen ohne Einschräung an, dass $\xi_1\neq 0$ gilt. Wir setzen $\alpha' = (m-1,0,\ldots,0)$ und erhalten
\begin{equation}
\D^{\alpha'} v_t = (\phi(x) (\D_1 u(x))^{m-1}t^{m-1}+ \ldots)\e^{tu}.
\end{equation}
Wegen $\phi(0) (\D_1u(0))^{m-1} = \xi_1^{m-1} \neq 0$ gilt $|\phi(x)| |\D_1u(x)|^{m-1} \ge 2c > 0$ für eine Konstante $c > 0$ in einer Umgebung der Null und wegen $\Re u(x) = \mathcal{O} (|x|^2)$ existiert eine Konstante $a' > 0$ sodass $\Re u(x) \ge -a' |x|^2$. Für große $t$ erhalten wir
\begin{equation}
 |\D^{\alpha'} v_t(x)| \ge |\phi(x)|\, |\D_1 u(x)|^{m-1} \,\e^{t \Re u(x)} \ge ct^{m-1}\e^{-ta'|x|^2}.
\end{equation}
Weiter folgt für große $t$    
\begin{equation}
	\label{4.19}
\lVert v_t \rVert_{(m-1)}^2 \ge \lVert \D^{\alpha'} v_t\rVert^2 \ge \int\limits_{t|x|^2 < 1}c^2t^{2m-2}\e^{-2ta'|x|^2} \d x = C^2 t^{2m-2-n/2}
\end{equation}
für eine Konstante $C > 0$. Kombiniert man (\ref{4.18}) und (\ref{4.19}), so erhält man für große $t$
\begin{equation}
\dfrac{\lVert v_t \rVert_{(m-1)}}{\lVert P(x,\D) v_t \rVert} \ge \dfrac{C}{2B\varepsilon}
\end{equation}
und da $\varepsilon>0$ beliebig war folgt die Behauptung des Lemmas.
\end{proof}

\begin{lem}[{\cite[Lemma 2.2]{Hormander:1960a}}]\label{thm:4:lem2}
Wir nehmen an, es gelte \eqref{normpart}. Dann gibt es eine Funktion $u \in \rmC^\infty(\Omega)$ mit \eqref{grad} und \eqref{taylor} genau dann, wenn
\begin{align}
	\label{4.15}
p(0, \xi) &= 0, \\ 	\label{4.16}
\sum_{k=1}^{n} \boldsymbol\alpha_{j,k}\frac{\partial p(0,\xi)}{\partial \xi_k} &= -\i\,  p_j(0,\xi), \hspace{0.5cm} j=1, \ldots, n, 
\end{align}
wobei $ p_j(x,\xi) = \partial_{x_j}  p(x,\xi)$ die partielle Ableitung des Hauptsymbols bezeichne.
\end{lem}

\begin{proof}
Betrachten wir die Taylorentwicklung, so stellen wir fest, dass (\ref{grad}) genau dann erfüllt ist, wenn $p(x,\nabla u(x))$ und alle Ableitungen der Ordnung $\le 2$ im Nullpunkt verschwinden. Das Verschwinden von $p(0,\nabla u(0))$ wird gerade durch (\ref{4.15}) beschrieben. Wir berechnen weiter
\begin{equation}
\dfrac{\partial}{\partial x_j} p(0,\nabla u(0)) = 
p_j(0,\i\xi) + \sum_{k=1}^{n} \boldsymbol\alpha_{j,k} \dfrac{\partial p(0,\i\xi)}{\partial \xi_k}
\end{equation}
und bemerken unter Verwendung der Homogenität von $p$, dass die linke Seite genau dann für alle $j$ verschwindet, wenn (\ref{4.16}) gilt. Es bleibt zu zeigen, dass für geeignet gewähltes $u$ unter den gegebenen Bedingungen auch die zweiten Ableitungen verschwinden. Dazu schreiben wir $\tilde p(x,\xi)$ für die Taylorapproximation von $p(x,\xi)$ zur Ordnung $3$ in $x=0$ und nutzen den Satz von Cauchy-Kowalewskaja um ein $u$ mit $\tilde p(x,\nabla u(x))=0$ zu konstruieren.
Da $\nabla_\xi \tilde p(0,\xi) = \nabla_\xi p(0,\xi)\ne0$ gilt, existiert insbesondere eine Richtung $\eta\in\R^n$ in welche die Richtungsableitung $\eta\cdot \nabla_\xi \tilde p(0,\xi)$ nicht verschwindet. Die 
Hyperebene $\{ x\in\R^n : x\cdot\eta = 0\}$ ist dann nichtcharakteristisch in einer Umgebung des Ursprungs und die Gleichung $\tilde p(x,\nabla u(x))=0$ mit der Nebenbedingung $u(x) = \i x\cdot\xi+x\cdot Ax$ für $x\cdot\eta=0$ in einer Umgebung des Ursprungs eindeutig lösbar. Die so konstruierte Funktion erfüllt \eqref{grad}und \eqref{taylor}.
\end{proof}

\begin{lem}[{\cite[Lemma 2.3]{Hormander:1960a}}]\label{thm:4:lem3}
Seien $\zeta, f \in \C^n$ mit $\zeta \neq 0$. Dann gibt es genau dann eine symmetrische Matrix $A = (\boldsymbol\alpha_{i,j})_{i,j}$ mit negativ definitem Realteil
$B=(\Re\boldsymbol \alpha_{i,j})_{i,j}$ und
\begin{equation}\label{laag1}
   A \zeta = f,
\end{equation}
wenn die Vektoren $\zeta$ und $f$
\begin{equation} \label{laag2}
\Re f\cdot \overline \zeta < 0
\end{equation}
erfüllen.
\end{lem}
\begin{proof}
Es gelte (\ref{laag1}). Dann folgt mit $A\zeta =f$ und $\zeta=\xi+\i\eta$
\begin{equation}
 f\cdot \overline \zeta = \sum_{k=1}^{n}f_k \overline{\zeta_k} = \sum_{j,k=1}^{n} \boldsymbol\alpha_{kj} \zeta_j \overline \zeta_k
=  \sum_{j,k=1}^{n} \boldsymbol\alpha_{kj} \xi_j \xi_k +  \sum_{j,k=1}^{n} \boldsymbol\alpha_{kj} \eta_j \eta_k,
\end{equation}
Da für mindestens ein $j$ ein $\xi_j$ oder $\eta_j$ ungleich Null ist, folgt (\ref{laag2}) wegen der Definitheit von $B$. $\bullet$\qquad 
Es gelte nun umgekehrt (\ref{laag2}). Wir nehmen zunächst an, $\zeta$ ist proportional zu einem reellen Vektor. Nach Multiplikation von $f$ und $\zeta$ mit derselben komplexen Zahl können wir dann annehmen, dass $\zeta=\xi$ reell ist. Mit $A = B + \i C$, $f= g+\i h$ wird aus (\ref{laag1}) das Gleichungssystem bestehend aus $B \xi = g$ und $C \xi = h$. Dass eine reelle, symmetrische Matrix $C$ mit $C \xi = h$ existiert, ist offensichtlich. Schreiben wir weiter $g = g' + \xi (g\cdot \xi)/ 2 |\xi|^2$, so gilt $g'\cdot\xi = g\cdot\xi /2 < 0$. Man rechnet dann leicht nach, dass die Matrix $B$, gegeben durch
\begin{equation}
B x = \dfrac{g\cdot\xi }{2|\xi|^2} x + \dfrac{x\cdot g'}{\xi\cdot g'} g',\qquad x\in\R^n,
\end{equation}
negativ definit und symmetrisch ist und $A=B+\i C$ erfüllt das gesuchte. Sei nun $\zeta$ nicht proportional zu einem reellen Vektor. Wir zeigen, dass 
\begin{equation}
A = \dfrac{\Re( f\cdot\overline\zeta )}{|\zeta|^2} \mathrm I + \i C
\end{equation}
für ein reelles $C\in\R^{n\times n}$ die gewünschten Eigenschaften erfüllt. Die Bedingung an $C$ schreibt sich dann als
\begin{equation}
	\i C \zeta = f',
\end{equation}
wobei $f' = f - \zeta \frac{\Re( f\cdot\overline\zeta)}{|\zeta|^2}$ die Eigenschaft
\begin{equation}
	\label{Ebene}
\Re (f'\cdot\overline\zeta) = 0
\end{equation}
hat. Um die Existenz eines solchen $C$ zu zeigen, stellen wir zunächst fest, dass die Menge der Vektoren in $\C^n$, die als $\i C \zeta$, $C \in \R^{n \times n}$ symmetrisch, geschrieben werden können, einen reellen Vektorraum bilden. Dieser ist für ein $g \in \C^n$ in der Hyperebene $\{ z \in \C^n : \Re g\cdot \overline z = 0 \}$ enthalten. Für jedes $\xi \in \R^n$ ist die Matrix $C$, gegeben durch $C x =  (x\cdot\xi)\xi$, symmetrisch und es folgt
\begin{equation}
\Re \i (\xi\cdot\overline g)(\zeta\cdot\xi) = 0.
\end{equation}
Insbesondere ist $(\xi\cdot\overline g)(\zeta\cdot\xi) $ immer reell. Unter Verwendung, dass $\zeta$ nicht proportional zu einem reellen Vektor ist, berechnet man schnell, dass $g$ ein reelles Vielfaches von $\zeta$ sein muss. Insbesondere ist $\Re(z\cdot\overline\zeta) = 0 \Leftrightarrow \Re(z\cdot \overline g) = 0$ und $f'$ lässt sich wegen (\ref{Ebene}) als $\i C \zeta$ schreiben für eine reelle, symmetrische Matrix $C$.  Es folgt die Behauptung.
\end{proof}

\begin{thm}[{\cite[Theorem 2.1]{Hormander:1960a}}]
\label{thm:4:Umkehrung}
Es gelte die Ungleichung (\ref{eq:4:locHT}), d.h. für ein $\Omega'\Subset\Omega$
\begin{equation}
  \forall u\in\rmC_0^\infty(\Omega') \qquad:\qquad  \| u\|_{(m-1)} \le C \| P(x,\D) u \|.
\end{equation}
Dann folgt 
\begin{equation}
	\label{Beh}
\{p,\overline p\} (x,\xi) = 0,
\end{equation}
falls $p(x,\xi) = 0$, $x \in \Omega'$ und $ \xi \in \R^n$.
\end{thm}
\begin{proof}
Wir nehmen ohne Einschränkung $x=0$ an. Weiterhin gelte
\begin{equation}
\sum_{j=1}^{n}|\partial p(0,\xi)/\partial \xi_j|^2 \neq 0,
\end{equation}
denn sonst ist (\ref{Beh}) trivialerweise erfüllt. Die Lemmata \ref{lem1:hoer2.2} und \ref{thm:4:lem2} zeigen dann, dass die Gleichung (\ref{4.16}) für keine symmetrische Matrix mit negativ definitem Realteil erfüllt sein kann. Mit Lemma \ref{thm:4:lem3} folgt deshalb $\{p,\overline p\}(x,\xi) \ge 0$. Das analoge Argument mit $-\xi$ anstatt $\xi$ liefert $\{p,\overline p\}(x,-\xi) \ge 0$. Weil $\{p,\overline p\}(x,\xi)$ eine ungerade Funktion in $\xi$ ist, folgt die Behauptung.
\end{proof}
\begin{rem}
In Satz \ref{thm:4:Umkehrung} folgt nur das Verschwinden der Poissonklammer auf der Nullstellenmenge von $p(x,\xi)$ und insbesondere nicht die wesentliche Normalität von $P(x,\D)$. Folglich handelt es sich nicht um die Umkehrung von Satz \ref{thm:4:4.2}, wo wesentliche Normalität vorausgesetzt wurde.
\end{rem}

% !TEX root = main.tex
%\chapter{Ein unlösbarer Operator}
%\cite{Lewy:1957}
%\cite{Hormander:1960b}


\section{Das Beispiel von Lewy}


Bis Mitte des 20. Jahrhunderts ging man davon aus, dass die lokale Existenz glatter Lösungen für lineare Differentialgleichungen stets gegeben sei. Das Beispiel von Hans Lewy zeigt eindrucksvoll, dass dies nicht zwingend der Fall sein muss.  In diesem Abschnitt soll eine lineare, partielle Differentialgleichung erster Ordnung in drei Variablen mit komplexwertigen $\rmC^\infty$-Koeffizienten vorgestellt werden, welche in \emph{keiner} offenen Menge eine glatte/distributionelle Lösung besitzt. Hierfür betrachtet man den durch
\begin{equation}\label{lewy:differentialausdruck}
\mathscr L := -\frac{\partial}{\partial x_1} -\i\frac{\partial}{\partial x_2} +2\i(x_1+\i x_2)\frac{\partial}{\partial y_1}
\end{equation}
definierten Differentialausdruck auf $\R^3$. Das erste überraschende Resultat von Lewy ist in folgendem Lemma enthalten:
\begin{lem}[{\cite{Lewy:1957}}]\label{thm:1_lewy}
Zu einer reellwertigen Funktion $\psi\in \rmC^1(\mathbb{R})$ besitze das Problem
\begin{equation}\label{eq:1_lewy:gleichung}
\mathscr Lu=\psi'(y_1)
\end{equation}
in einer Umgebung $\Omega\subset\R^3$ von $(0,0,y_1^0)$ eine $\rmC^1$-Lösung $u$. Dann ist $\psi$ analytisch in $y_1=y_1^0$.
\end{lem}

\begin{proof}
Wir integrieren $(\partial_1+\i\partial_2) u$ für eine Lösung $u$ von \eqref{eq:1_lewy:gleichung} über einen Kreis in der $x_1$-$x_2$ Ebene um den Punkt $(0,0,y_1^0)$. Der Radius wird dabei so klein gewählt, dass der Kreis in $\Omega$ liegt. Sei dazu
\begin{equation}
x_1^2+x_2^2=y_2=\mathrm{const},\qquad y_1=\mathrm{const},
\end{equation}
$t=\log\sqrt{y_2}=\log\sqrt{x_1^2+x_2^2}$ und $\theta$ derjenige Winkel gegeben durch
\begin{equation}
x_1+\i x_2=\sqrt{y_2} \, \e^{\i\theta} =\e^{t+\i\theta}.
\end{equation}
Dann erhält man durch einfaches Nachrechnen $\overline x \overline\partial_x = \overline\partial_t$, also
\begin{equation}\label{thm:1_lewy:proof1}
\frac{\partial}{\partial x_1} +\i\frac{\partial}{\partial x_2}=
\e^{-t+\i\theta} \left(\frac{\partial}{\partial t}
+\i\frac{\partial}{\partial \theta}\right).
\end{equation}
Diese Identität zusammen mit partieller Integration liefert
\begin{align}\label{thm:1_lewy:abl_gleich}
\begin{split}
\int_0^{2\pi} \left(\frac{\partial}{\partial x_1} +\i\frac{\partial}{\partial x_2}\right)u\d\theta 
&= \int_0^{2\pi} \e^{-t+\i\theta} \left(\frac{\partial}{\partial \,t}+\i\frac{\partial}{\partial \theta}\right)u \d\theta \\
&= \int_0^{2\pi} \e^{-t+\i\theta} \left(\frac{\partial u}{\partial \,t} +\i\frac{\partial u}{\partial \theta}\right)\d\theta \\
&= \int_0^{2\pi} \e^{-t+\i\theta} \left(\frac{\partial u}{\partial \,t} +u\right)\d\theta .
\end{split}
\end{align}
Weiter impliziert $\sqrt{y_2}=\e^t$ für jede differenzierbare Funktion $w$
\begin{align}\label{thm:1_lewy:int_gleichheit1}
\begin{split}
\frac{\partial w}{\partial\, t}+ w  = 2\sqrt{y_2} \frac{\partial}{\partial y_2} \left( \sqrt{y_2} w(\log\sqrt{y_2})\right).
\end{split}
\end{align}
Eingesetzt in das letzte Integral aus \eqref{thm:1_lewy:abl_gleich} ergibt sich
\begin{align}\label{thm:1_lewy:abl_gleich_final}
\begin{split}
\int_0^{2\pi} \left(\frac{\partial}{\partial x_1} +\i\frac{\partial}{\partial x_2}\right)u\d\theta 
= 2\left(\frac{\partial}{\partial y_2}\right)\int_0^{2\pi} \e^{\i\theta}\sqrt{y_2}u\d\theta.
\end{split}
\end{align}
Setzen wir nun 
\begin{equation}
I(y_1,y_2)=\i\int_0^{2\pi} \e^{\i\theta} \sqrt{y_2} u\d\theta,
\end{equation}
so liefert \eqref{eq:1_lewy:gleichung} und danach \eqref{thm:1_lewy:abl_gleich_final}
\begin{align}
\begin{split}
\frac{\partial I}{\partial y_1} +\i\frac{\partial I}{\partial y_2} 
&= \i\int_{0}^{2\pi} \frac{\partial}{\partial y_1}\left(\e^{\i\theta}\sqrt{y_2}u\right)+\i\frac{\partial}{\partial y_2}\left(\e^{\i\theta}\sqrt{y_2}u\right)\d\theta \\
&= \i\int_{0}^{2\pi} \e^{\i\theta}\sqrt{y_2}\left(\frac{\partial}{\partial y_1}u\right) +\i\e^{\i\theta}\frac{\partial}{\partial y_2}(\sqrt{y_2}u)\d\theta\\
&= \i\int_{0}^{2\pi} \e^{\i\theta}\sqrt{y_2} \left(
	\frac{\psi'(y_1)}{2\i(x_1+\i x_2)}
	+ \frac{\left(\frac{\partial}{\partial x_1} + \i\frac{\partial}{\partial x_2}\right)u}{2\i(x_1+\i x_2)} 
	+ \frac{\i}{\sqrt{y_2}} \frac{\partial}{\partial y_2}(\sqrt{y_2} u)
\right)\d\theta \\
&= \i\int_{0}^{2\pi} \e^{\i\theta}\sqrt{y_2} \left( 
	\frac{\psi'(y_1)}{2\i\sqrt{y_2}\e^{\i\theta}} 
	+ \frac{\left(\frac{\partial}{\partial x_1} + \i\frac{\partial}{\partial x_2}\right)u}{2\i\sqrt{y_2}\e^{\i\theta}}
	+ \frac{\i}{\sqrt{y_2}} \frac{\partial}{\partial y_2}(\sqrt{y_2} u)
\right)\d\theta	 \\
&= \int_0^{2\pi} \frac{\psi'(y_1)}{2}\d\theta 
	+ \frac{1}{2}\int_0^{2\pi} \left(\frac{\partial}{\partial x_1} +\i \frac{\partial}{\partial x_2}\right)u\d\theta
	- \int_0^{2\pi} \e^{\i\theta}\frac{\partial}{\partial y_2}(\sqrt{y_2} u) \d\theta \\
&= 2\pi\frac{\psi'(y_1)}{2}
	+  \left(\frac{\partial}{\partial y_2}\right)\int_0^{2\pi} \e^{\i\theta}\sqrt{y_2}u\d\theta
	- \left(\frac{\partial}{\partial y_2}\right) \int_0^{2\pi} \e^{\i\theta}\sqrt{y_2}u\d\theta\\
&= \pi\psi'(y_1).
\end{split}
\end{align}
Ferner ist
\begin{equation}
J(y):=J(y_1,y_2):=I(y_1,y_2)-\pi\psi(y_1),
\end{equation}
eine $\rmC^1$-Funktion. Nach Konstruktion erfüllt diese die Cauchy-Riemannschen Differentialgleichungen
\begin{equation}
\frac{\partial J}{\partial y_1} +\i\frac{\partial J}{\partial y_2} = 0
\end{equation}
 und ist somit analytisch in $y=y_1+\i y_2$. Ihr Definitionsbereich enthält alle $y=(y_1,y_2)$ mit $y_2>0$ hinreichend klein und $y_1$ nahe $y_1^0$.
 Da weiter für $y_2=0$ nach Konstruktion $I(y_1,0)=0$ gilt, folgt
\begin{align*}
 J(y_1,0)=-\pi\psi'(y_1)
\end{align*}
mit nach Voraussetzung reellem $\psi$ und $J$ kann mithilfe des Spiegelungsprinzips in die untere komplexe Halbebene analytisch fortgesetzt werden. Somit ist $\psi'(y_1)$ 
und damit auch $\psi(y_1)$ analytisch in einer Umgebung des Punktes $y_1=y_1^0$ und die Behauptung ist gezeigt.
\end{proof}

Betrachtet man nun \eqref{eq:1_lewy:gleichung} mit reellwertigem $\psi\in \rmC^\infty(\R)$, welches \textit{nicht} analytisch in $y_1=y_1^0$ ist, so liefert die Kontraposition des obigen Lemmas, dass \eqref{eq:1_lewy:gleichung} in keiner Umgebung $\Omega$ von $(0,0,y_1^0)$ eine Lösung $u\in \rmC^1(\Omega)$ besitzen kann. Lewy nutzte obiges Lemma um eine Funktion $f\in\rmC^\infty(\R)$ zu konstruieren, so dass
\begin{equation} 
   \mathscr L u = f(y_1)
\end{equation}
in {\em keinem} Gebiet $\Omega\subset\R^3$ eine Lösung in einem Hölderraum $\rmC^{1,\alpha}(\Omega)$ besitzen kann.




\section{Hörmanders Unlösbarkeitskriterium}
Hörmander zeigte in \cite{Hormander:1960a}, dass für jedes Gebiet $\Omega\subset\R^n$ eine glatte Funktion $f\in\rmC^\infty(\R^n)$ existiert, für welche keine Distribution $u\in\mathscr{D}'(\Omega)$ mit $\mathscr Lu=f$ existiert. Dazu zeigte er eine notwendige Bedingung für die Lösbarkeit eines Differentialausdrucks erster Ordnung. In \cite{Hormander:1960b} verallgemeinerte er diese noch auf Operatoren höherer Ordnung.

Zuerst zeigen wir ein Analogon zu Satz~\ref{thm:4:Umkehrung}, welches es erlaubt von Lösbarkeitsaussagen auf das Verschwinden der Poissonklammer 
von $p$ und $\overline p$ zu schließen. 

\begin{thm}[{\cite[Theorem 1]{Hormander:1960b}}]\label{thm:3.1_hoer}
Sei $\Omega\subset\mathbb{R}^n$ ein Gebiet und $P(x,\D)$ ein Differentialausdruck der Ordnung $m$ mit Hauptsymbol $p(x,\xi)$.
Angenommen, für jedes $f\in \rmC_0^\infty(\Omega)$  existiert eine distributionelle Lösung $u\in\mathscr D'(\Omega)$ zu
\begin{equation}\label{eq:3.1_hoer}
P(x,\D)u=f.
\end{equation}
Dann gilt für alle $x\in\Omega$ und $\xi\in\mathbb{R}^n$ mit $p(x,\xi)=0$
\begin{equation}\label{eq:3.1_hoer_aussage}
 \{p,\overline{p}\}(x,\xi)=0.
\end{equation}
\end{thm}
\begin{proof}
Wir zeigen dies indirekt, beginnen mit einer Umformulierung der Voraussetzung als Ungleichung und konstruieren danach geeignete Testfunktionen um einen Widerspruch herzuleiten. 
\noindent {\sl Schritt 1.}  
 Bezeichne $\rmB^\infty_0(\omega)=\{f\in\rmC^\infty(\Omega) \mid \supp f\subseteq\overline\omega\}$ und 
\begin{equation}\label{eq:M_N}
   M_N = \{ f\in \rmB^\infty_0(\omega)\mid \exists_{u\in\mathscr D'(\omega)} \; \forall_{\psi\in\rmC_0^\infty(\omega)}\; |\langle u,\psi\rangle | \le N \|\psi\|_{(N)} \quad\text{und}\quad P(x,\D)u=f \}.
\end{equation}
Für jedes $u\in\mathscr D'(\Omega)$ und $\omega\Subset\Omega$ gilt $ |\langle u,\psi\rangle | \le N \|\psi\|_{(N)}$ für alle $\psi\in\rmC_0^\infty(\omega)$ und ein hinreichend großes $N$. Die Voraussetzung impliziert also
\begin{equation}
   \bigcup_{N=1}^\infty M_N = \rmB^\infty_0(\omega).
\end{equation} 
Die Mengen $M_N$ sind abgeschlossen (siehe nachfolgendes Lemma \ref{lm:closed}),  konvex und symmetrisch. Weiter ist $\rmB^\infty_0(\omega)$ metrisierbar, der Bairesche Kategoriensatz impliziert also, dass mindestens eine der Mengen $M_N$ einen inneren Punkt (wegen Symmetrie den Ursprung) besitzt. Es gibt also ein $k$ und $\epsilon>0$, so dass
\begin{equation}\label{eq:fepsInMN}
    \{ f\in \rmB^\infty_0(\omega)\mid \|f\|_{(k)}<\epsilon\} \subset M_N
\end{equation}
für ein $N$ gilt.
%
$\bullet$\qquad {\sl Schritt 2.} 
Angenommen, die Behauptung gilt nicht. Sei also ohne Beschränkung der Allgemeinheit $0\in\Omega$ und gelte für ein $\xi\in\R^n$ sowohl $p(0,\xi)=0$ als auch\footnote{Da $\{p,\overline p\}$ reellwertig und ungerade in $\xi$ ist, erfüllt für $\{p,\overline p\}(0,\xi)\ne0$ entweder $\xi$ oder $-\xi$ die Bedingung $\{p,\overline p\}(0,\xi)<0$.}   $\{p,\overline p\}(0,\xi)<0$. Analog zu Lemma~\ref{thm:4:lem2} folgt die Existenz einer Funktion $u\in\rmC^\infty(\Omega)$ mit 
\begin{equation}
    p(x,\nabla u(x)) = \mathcal O(|x|^q),\qquad x\to0
\end{equation}
sowie
\begin{equation}
   u(x) = \i \xi\cdot x + x\cdot Ax + \mathcal O(|x|^3),\qquad x\to0 
\end{equation}
zu vorgegebener Ordnung $q$, obigem Vektor $\xi$ und geeignet (mit Lemma~\ref{thm:4:lem3}) gewählter symmetrischer Matrix $A$ mit negativ definitem Realteil. 
%
$\bullet$\qquad {\sl Schritt~3.} Wir definieren die Hilfsfunktionen
\begin{equation}
   f_{\tau, k}(x) = \tau^{-k} \chi(\tau x),\qquad \text{wobei $\chi\in\rmC_0^\infty(\R)$ mit}\quad \widehat \chi(-\xi) = (2\pi)^{-n/2}  \int \e^{\i x\cdot\xi} \chi(x) \d x \ne 0,
\end{equation}
sowie für $q=2(r+1)$, $r=n+k+m+N$ und dem oben konstruierten $u(x)$ die Hilfsfunktionen
\begin{equation}
   v_{\tau,k,N} (x) = \tau^{n+1+k} \e^{\tau u(x)} \sum_{j=0}^{r-1} \tau^{-j} \varphi_j(x) 
\end{equation}
mit noch zu bestimmenden Koeffizienten $\varphi_j\in\rmC_0^\infty(\Omega)$. Nach Konstruktion gilt für die Sobolevnormen
\begin{equation}\label{eq:cond1}
    \limsup_{\tau\to\infty} \| f_{k,\tau}\|_{(k)} < \infty,
\end{equation}
und für $\varphi_0(0)=1$ folgt
\begin{equation}\label{eq:cond3}
   \frac1\tau \int f_{\tau,k}(x) v_{\tau,k,N}(x)\d x = \int \e^{\tau u(\frac x\tau)} \chi(x) \sum_{j=0}^{r-1} \varphi_j(\frac x\tau) \tau^{-j} \d x \longrightarrow \int \e^{\i x\cdot\xi}\chi(x)\d x\ne0
\end{equation}
für $\tau\to\infty$. Die verbleibenden $\varphi_j$ werden so gewählt, dass 
\begin{equation}\label{eq:cond2}
   \limsup_{\tau\to\infty} \|{}^tP(x,\D) v_{\tau,k,N}\|_{(N)} <\infty 
\end{equation}
gilt. Dazu nutzt man, dass 
\begin{equation}
{}^tP(x,\D) v_{\tau,k,N}(x)=\tau^{n+1+k+m}  \e^{\tau u(x)} \sum_{j=0}^m \tau^{-j} a_j(x) 
\end{equation}
mit Koeffizienten $a_j\in\rmC_0^\infty(\Omega)$ gilt und wählt $\varphi_j$ so, dass $a_j(x)=\mathcal O(|x|^{q-2j})$ (was auf eine erneute Anwendung des Satzes von Cauchy--Kowalewskaja hinausläuft). Danach zeigt eine einfache Rechnung, dass für jedes $\psi\in\rmC_0^\infty(\omega)$ mit $\psi(x)=\mathcal O(|x|^{2s})$, $x\to0$, $s\geq 0$
und $\omega\Subset\Omega$ hinreichend klein stets die Normabschätzung $\limsup_{\tau\to\infty} \|\tau^{s-N}\psi \e^{\tau u}\|_{(N)}<\infty$ gilt und \eqref{eq:cond2} folgt.~$\bullet$ \qquad {\sl Schritt~4.} Wir zeigen, dass \eqref{eq:fepsInMN} im Widerspruch zur Existenz der gerade konstruierten Hilfsfunktionen steht. 
Zum Einen impliziert \eqref{eq:fepsInMN} die Existenz einer Distribution $u$ mit $P(x,\D)u=f$ für $f\in \rmB^\infty_0(\omega)$ mit $\|f\|_{(k)}<\epsilon$. Damit gilt für jede Testfunktion $v\in\rmC_0^\infty(\omega)$ die Abschätzung
\begin{equation}\label{eq:4.74}
   \bigg| \int f(x) v(x) \d x\bigg| = |\langle u, {}^t P(x,\D) v \rangle| \le N \| {}^t P(x,\D) v\|_{(N)}.
\end{equation}
Speziell mit $f=c f_{k,\tau}$ und $v=v_{k,N,\tau}$ folgt für $\tau\to\infty$ ein Widerspruch: Für $c>0$ klein genug impliziert \eqref{eq:fepsInMN} zusammen mit \eqref{eq:cond1}, dass $f\in M_N$. Also gilt \eqref{eq:4.74}. Nun ist die rechte Seite aber wegen \eqref{eq:cond2} gleichmäßig in $\tau$ beschränkt, die linke strebt mit \eqref{eq:cond3} für $\tau\to\infty$  gegen Unendlich. Widerspruch.
\end{proof}


\begin{rem}
Die im obigen Beweis verwendete Menge $\rmB^\infty_0(\omega)$ ist gerade der Abschluss von $\rmC_0^\infty(\omega)$ in $\rmC^\infty(\Omega)$
(also auch in $\rmC^\infty(\R^n)$). Damit kann man für jedes Gebiet $\Omega\subset\R^n$ den entsprechenden Raum auch als
\begin{equation}
\rmB_0^\infty(\Omega) =\{f\in \rmC^\infty(\Omega) \mid \forall_{\alpha\in\N_0^n}\;\forall_{\epsilon >0} \;\exists_{K_{\alpha,\epsilon}\Subset\Omega} \;:\; \sup\nolimits_{x\in \Omega\setminus K_{\alpha,\epsilon}}|\D^\alpha f(x)|<\epsilon \}
\end{equation}
charakterisieren. Weiter sei 
\begin{equation}
\rmB^\infty(\Omega) = \{ f\in\rmC^\infty(\Omega) \mid  \forall_{\alpha\in\N_0^n} \;:\; \sup\nolimits_{x\in\Omega} |\D^\alpha f(x)|<\infty \}.
\end{equation}
Nach Konstruktion ist $\rmB^\infty_0(\Omega)$ abgeschlossener Teilraum von $\rmB^\infty(\Omega)$ und für $\omega\Subset\Omega$ ist die Einschränkung auf $\omega$ surjektiv von $\rmB^\infty_0(\Omega)$ auf $\rmB^\infty(\omega)$. Das nachfolgende Lemma liefert damit insbesondere die im Beweis verwendete Abgeschlossenheit der Mengen $M_N$.
\end{rem}


\begin{lem}\label{lm:closed}
Sei $\omega\Subset\Omega$ und
\begin{equation}
M_N:= \{f\in \rmB^\infty(\omega)\mid \exists_{u\in\mathscr D'(\omega)}\; \forall_{\psi\in\rmC_0^\infty(\omega)}\; |\langle u,\psi\rangle | \le N \|\psi\|_{(N)} \quad\text{und}\quad P(x,\mathrm{D})u=f\quad \text{in $\omega$} \}.
\end{equation}
Dann ist $M_N\subset \rmB^\infty(\omega)$ abgeschlossen.
\end{lem}

\begin{proof}
Die Menge der Distributionen $u\in\mathscr D'(\omega)$, welche die  Bedingung 
\begin{equation}\label{lewy:uglSchwartz}
|\langle u,\psi\rangle|\leq N\|\psi\|_{(N)} 
\end{equation}
für alle $\psi\in\rmC_0^\infty(\omega)$ erfüllen, ist folgenkompakt. Wir zeigen dies mit einem Diagonalfolgenargument.
Sei dazu $(u_n)_{n\in\N}$ eine Folge in $\mathscr D'(\omega)$ mit dieser Schranke. Sei weiter $(\psi_j)_{j\in\N}$ eine Folge
von $\rmC_0^\infty(\omega)$ Funktionen, die in $\rmH^N_0(\omega)$ (also dem Abschluss von $\rmC_0^\infty(\omega)$ in der $\|\cdot\|_{(N)}$-Norm) dicht ist.
Diese Existiert wegen der Separabilität von $\rmH_0^N(\omega)$.

Da die Folge $\langle u_n,\psi_1\rangle$ in $\C$ beschränkt ist, existiert eine Teilfolge $(u_{n_k}^{(1)})_{k\in\mathbb{N}}$ derart, dass
$\langle u_{n_k}^{(1)},\psi_1\rangle$ konvergiert. Weiter finden wir auch eine Teilfolge $(u_{n_k}^{(2)})_{k\in\mathbb{N}}$ von $u_{n_k}^{(1)}$, so dass $\langle u_{n_k}^{(2)},\psi_2\rangle$ konvergiert, etc. Wählen wir nun die Diagonalfolge $v_k = u_{n_k}^{(k)}$, so konvergiert $\langle v_k,\psi_j\rangle$ nach Konstruktion für alle $j$. 
Also konvergiert wegen der vorausgesetzten Schranke \eqref{lewy:uglSchwartz} die Folge   $\langle v_k,\psi\rangle$  für alle $\psi\in\rmH^N_0(\omega)$ und da $\rmC_0^\infty(\omega)\subset \rmH_0^N(\omega)$ gilt in $\mathscr D'(\omega)$. Der Grenzwert erfüllt offenbar ebenfalls \eqref{lewy:uglSchwartz}.

Sei also $(f_n)_{n\in\mathbb{N}}$ eine gegen $f\in \rmB^\infty(\omega)$ konvergente Folge aus $M_N$. Dann gilt  $f_n=P(x,\D)u_n$ für gewisse $u_n\in\mathscr{D}'(\omega)$ mit  $|\langle u,\psi\rangle | \le N \|\psi\|_{(N)}$ für alle $\psi\in\rmC_0^\infty(\omega)$. Auf Grund der gerade gezeigten Folgenkompaktheit existiert eine in $\mathscr D'(\omega)$  konvergente Teilfolge $(u_{n_k})_{k\in\N}$. Sei nun $u=\lim_{k\rightarrow\infty}u_{n_k}$. Dann erfüllt $u$ ebenfalls  
 \eqref{lewy:uglSchwartz} und da $P(x,\D)u_{n_k}\to P(x,\D)u$ in $\mathscr D'(\omega)$ konvergiert, folgt $P(x,\D)u=f$.
\end{proof}
\begin{thm}[{\cite[Theorem 2]{Hormander:1960b}}]\label{thm:2_hoer}
Sei $P(x,\D)$ ein Differentialausdruck der Ordnung $m$ mit  Hauptsymbol $p(x,\xi)$ und existiere zu jedem $\omega\Subset\Omega$ 
ein $x\in\omega$ und ein $\xi\in\R^n$ mit
\begin{equation}
 \{p,\overline p\}(x,\xi)\ne 0.
\end{equation}
Dann existieren Funktionen $f\in \rmB_0^\infty(\Omega)$, so dass
\begin{equation}\label{lewy:pxDu=f}
P(x,\mathrm D)u=f
\end{equation}
in keiner der Mengen $\omega\subseteq\Omega$ eine Lösung $u\in\mathscr D'(\omega)$ besitzt. Die Menge dieser Funktionen $f$ ist von zweiter Kategorie\footnote{Eine Teilmenge $A$ eines topologischen Raumes $B$ heißt von erster Kategorie, falls eine abzählbare Menge nirgends dichter Teilmengen aus $B$ existiert, deren Vereinigung $A$ ergibt. Ist dies nicht der Fall, so heißt $A$ von zweiter Kategorie.}.
\end{thm}

\begin{proof} Der Beweis folgt {\cite[Theorem 3.2]{Hormander:1960a}}.
{\sl Schritt 1.}
Sei zunächst $\omega\subseteq\Omega$ eine feste, nichtleere Menge und $M$ definiert durch
\begin{equation}
M=\{f\in \rmB_0^\infty(\Omega)\mid \exists\,u\in\mathscr{D}'(\omega) : P(x,\mathrm{D})u=f\quad \text{auf $\omega$}\}.
\end{equation}
Wir zeigen, dass $M$ von erster Kategorie ist.  Sei hierzu $\omega_1\Subset\omega$ offen und nichtleer, d.h. insbesondere ist $\overline{\omega_1}\subset\omega$ kompakt. Dann existiert für jede Distribution $u\in\mathscr{D}'(\omega)$ ein $N\in\mathbb{N}$, so dass $u$ die Ungleichung \eqref{lewy:uglSchwartz}
für alle $\psi\in \rmC_0^\infty(\omega_1)$ erfüllt ist. Seien nun wieder Mengen $M_N$ durch
\begin{equation}
M_N:= \{f\in \rmB^\infty(\omega_1)\mid \exists\,u\in\mathscr D'(\omega_1): P(x,\mathrm{D})u=f\;\mathrm{in}\;\omega_1\;\wedge\; u\mathrm{\;erf"ullt\;}\eqref{lewy:uglSchwartz}\},
\end{equation}
gegeben. Diese sind nach Lemma~\ref{lm:closed} abgeschlossen, offenbar konvex und auch symmetrisch. 
Keines der $M_N$ besitzt einen inneren Punkt. 
Anderenfalls gäbe es wiederum ein $N$, ein $k$ und ein $\epsilon>0$, so dass alle $f\in \rmB^\infty(\omega_1)$ mit
$\|f\|_{(k)}<\epsilon$ in $M_N$ liegen. Also gäbe es zu jedem $f\in\rmC_0^\infty(\omega_1)$ eine Konstante $\delta\ne0$, so dass $P(x,\D)u=\delta f$ in $\mathscr D'(\omega_1)$ eine L\"osung besitzt. Da aber dann $\delta^{-1}u$ schon $P(x,\D)u=f$ löst und $f$ beliebig war, widerspricht dies Satz~\ref{thm:3.1_hoer}. 

Gleiches trifft auch auf ihre Urbilder $\widetilde M_N$ unter der stetigen Einschränkung $\rmB^\infty_0(\Omega)\to \rmB^\infty(\omega_1)$ zu. Insbesondere sind diese Mengen abgeschlossen und besitzen keine inneren Punkte. Also ist $\widetilde M_N$ von erster Kategorie und ebenso die abzählbare Vereinigung dieser Mengen. Da aber $M\subset \bigcup_N \widetilde M_N$ gilt, folgt die  Behauptung.
$\bullet$\qquad {\sl Schritt 2.}
Wir zeigen nun, dass \eqref{lewy:pxDu=f} tatsächlich keine Lösung in $\Omega$ besitzt. Sei dazu $(\omega_j)_{j\in\mathbb{N}}$ eine abzählbare Basis der Topologie von $\Omega$ und bezeichne
\begin{equation}
M^{(j)}:=\{f\in \rmB_0^\infty(\Omega)\mid \exists u\in\mathscr{D}'(\omega_j): P(x,\mathrm{D})u=f\;\mathrm{auf\;}\omega_j\}.
\end{equation}
Dann folgt aus Schritt 1, dass alle $M^{(j)}$ und folglich auch $\bigcup M^{(j)}$ von erster Kategorie sind. Nach Definition der $M^{(j)}$ kann aber $P(x,\mathrm{D})u=f$ für $f\not\in  \bigcup M^{(j)}$ auf keinem $\omega_j$ gelöst werden. Da für jede beliebige, offene, nichtleere Menge $\omega\subseteq\Omega$ ein Index $j_0$ existiert, so dass $\omega_{j_0}\subset\omega$ gilt, besitzt $P(x,\mathrm{D})u=f$ auf keiner solchen Menge $\omega$ eine Lösung. 
\end{proof}



\begin{exa}
Für Lewys Beispiel 
\begin{equation}
p(x,\xi)=-i\xi_1+\xi_2-2(x_1+ix_2)\xi_3
\end{equation}
in $n=3$ Dimensionen aus dem ersten Teil dieses Kapitels erhalten wir $\{p,\overline{p}\}(x,\xi)=-8\xi_3$. Wegen
\begin{equation}
\xi_3 =1,\quad \xi_1=-2x_2,\quad \xi_2=2x_1\qquad\Rightarrow\qquad p(x,\xi)=0
\end{equation}
ist $\{p,\overline{p}\}(x,\xi)=-8\neq 0$ und \eqref{eq:3.1_hoer_aussage} gilt nicht für alle $x\in\mathbb{R}^n$ womit folglich die Voraussetzungen von Satz \ref{thm:2_hoer} erfüllt sind.
\end{exa} % Ein unloesbarer Operator :: Robin Lang