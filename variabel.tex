% !TEX root = main.tex
\newcommand{\A}{\forall}
\newcommand{\E}{\exists}
\newcommand{\imp}{\Rightarrow}
\renewcommand{\labelenumi}{\roman{enumi})}

\newcommand{\Sph}{\mathbb S}
\renewcommand{\P}{\mathcal{P}}
\newcommand{\Pin}{\textbf{Pin}}
\newcommand{\Q}{\mathbb Q}
\renewcommand{\H}{\mathbb H}
\newcommand{\Ri}{\widetilde{R}}
\newcommand{\K}{\mathbb{K}}
\newcommand{\lle}{\preccurlyeq}
\newcommand{\G}{\mathcal{G}}
\newcommand{\M}{\widetilde{M}}
\newcommand{\Or}{\mathcal{O}}
\renewcommand{\le}{\leqslant}
\renewcommand{\ge}{\geqslant}
\renewcommand{\O}{\textbf{O}}
\newcommand{\SO}{\textbf{SO}}
\newcommand{\U}{\textbf{U}}
\newcommand{\Ad}{\text{Ad}}
\newcommand{\SU}{\textbf{SU}}
\newcommand{\Symp}{\textbf{Sp}}
\newcommand{\id}{\text{id}}
\newcommand{\Cl}{\textbf{Cl}}
\newcommand{\Ind}{\text{Ind}}
\newcommand{\proofed}{$\mbox{}$\hfill $\blacksquare{}$\\} 
\newcommand{\g}{\mathfrak{g}}
\renewcommand{\k}{\mathfrak{k}}
\newcommand{\m}{\mathfrak{m}}
\renewcommand{\o}{\overline}
\chapter{Operatoren von reellem Haupttyp}
%\cite{Hormander:1960a}
\section{Notation und Vorbereitung}

Im folgenden sei $\P$ ein Differentialoperator auf einem Gebiet $\Omega \subset \R^n$, gegeben durch
\[
\P = \sum_{|\alpha| \le m} a_\alpha (x) D^\alpha.
\]


\begin{df}
	Für $\xi \in \C^n$ bezeichnen wir mit
	\[
	p(x,\xi) = \sum_{|\alpha| = m} a_\alpha (x) \xi^\alpha
	\]
	das \emph{Hauptsymbol} oder das \emph{charakteristische Polynom} von $\P$. $\P$ heißt von \emph{reellem Haupttyp}, wenn für reelle $\xi$ auch $p(x,\xi)$ reell ist und für reelle $\xi \neq 0$ nicht alle partiellen Ableitungen $\partial p(x,\xi) / \partial \xi_i$ gleichzeitig verschwinden.
\end{df} 
Wir verwenden für Vektoren $\xi^1, \ldots, \xi^m$ die Notation
\[
p(x,\xi_1, \ldots, \xi_m) = \sum_{|\alpha|=m} a_\alpha(x)\xi_1^\alpha \cdots \xi_m^\alpha (?)
\]
und für $k_1 + \ldots + k_p=m$ schreiben wir auch $p(x,\xi_1^{k_1}, \ldots, \xi_p^{k_p})$, wenn $k_i$ Argumente gleich $\xi_i$ sein sollen.
\begin{lem}
Für den Tensor $T^{ik}$ gilt die Gleichung
\[
\sum_{i=1}^{n} (\zeta_i - \overline \zeta_i) T^{ik}(\zeta, \o \zeta)  =
p(\zeta) \dfrac{\partial p(\o \zeta)}{\partial \o \zeta_k} - p(\o \zeta) \dfrac{\partial p( \zeta)}{\partial  \zeta_k}.
\]
\end{lem}
\begin{proof}

\end{proof}
\begin{df}
	Für variable $\xi, \eta$ definieren wir die Tensoren
	\begin{align*}
	\sum_{i,k=1}^n R^{ik}(\zeta, \overline \zeta) \xi_i \eta_k &= m \sum_{j=0}^{m-1} p(\zeta^j, \overline \zeta^{m-1-j},\xi) p(\zeta^{m-1-j},\overline \zeta^j ,\eta), \\
	\sum_{i,k=1}^n S^{ik}(\zeta, \overline \zeta) \xi_i \eta_k &= m \sum_{j=0}^{m-1} p(\zeta^j, \overline \zeta^{m-j}) p(\zeta^{m-1-j},\overline \zeta^{j-1} ,\xi,\eta)
	\end{align*}
	sowie den symmetrischen Tensor $T^{ik} = R^{ik}-S^{ik}$, wobei wir den Punkt $x$ in der Notation zur Übersichtlichkeit unterschlagen.
\end{df}

\section{a}
Wir nehmen im Folgenden an, dass $\P$ in einer Umgebung einer Sphäre $|x| \le A$ in $\R^n$ definiert ist.

\begin{thm}
	Sei $\P$ von reellem Haupttyp, die Koeffizienten von $p(x,\xi)$ seien stetig differenzierbar und die anderen Koeffizienten von $\P$ stetig. Dann existiert eine Umgebung $\Omega$ der Null, sodass
	\begin{align}
	\sum_{|\alpha| \le m} \lVert D^\alpha u \rVert^2 \le C \lVert \P u \rVert^2, \hspace{0.5cm} u \in C_0^\infty (\Omega).
	\end{align}
\end{thm}

\begin{proof}
	
\end{proof}