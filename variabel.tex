% !TEX root = main.tex
\chapter{Operatoren mit variablen Koeffizienten}

Zur Notation: Sei $\Omega\subset\R^n$ ein Gebiet und bezeichne weiterhin 
\begin{equation}
    \|u\|_{(m)} = \sum_{|\alpha|\le m} \|\D^\alpha u\|
\end{equation}
die $m$-te Sobolevnorm einer Funktion $u\in\rmC_0^\infty(\Omega)$. Im folgenden betrachten wir mininmale Operatoren zu Differentialausdrücken
\begin{equation}
   P(x,\D) = \sum_{|\alpha|\le m} a_\alpha(x) \D^\alpha
\end{equation}
mit Koeffizienten $a_\alpha\in\rmC^\infty(\Omega)$. Offensichtlich gilt $\|P(x,\D)u\|\le C\|u\|_{(m)}$ für alle $u\in\rmC_0^\infty(\Omega)$ mit einer nur von der Größe der Koeffizienten abhängenden Konstanten $C$.
Wir definieren \eIndex[Differentialoperator]{Symbol} und \eIndex[Differentialoperator]{Hauptsymbol}
\begin{equation}
   P(x,\zeta) = \sum_{|\alpha|\le m} a_\alpha(x)\zeta^\alpha,\qquad p(x,\zeta)=\sum_{|\alpha|=m} a_\alpha(x)\zeta^\alpha
\end{equation}
als Polynome auf $\C^n$ mit Koeffizienten aus $\rmC^\infty(\Omega)$. Analog zu den schon für den Fall konstanter Koeffizienten eingeführten Bezeichnungen nennen wir den Differentialausdruck $P(x,\D)$ im Punkt $x\in\Omega$
\begin{itemize}
\item \eIndex[Differentialoperator]{elliptisch}, falls für alle (reellen) $\xi\in\R^n\setminus\{0\}$ das Hauptsymbol $p(x,\xi)\ne0$ erfüllt;
\item \eIndex[Differentialoperator]{vom Haupttyp}, falls die Nullstellenmenge des Hauptsymbols 
\begin{equation}
    p(x,\xi) = 0 \quad \Longrightarrow \quad \nabla_\xi p(x,\xi)\ne0
\end{equation}
für alle $\xi\in\R^n\setminus\{0\}$ erfüllt.
\end{itemize}
Während die für Operatoren mit konstanten Koeffizienten gezeigten Aussagen und Abschätzungen im wesentlichen unabhängig vom Gebiet waren, 
treten bei Operatoren mit variablen Koeffizienten neue Effekte auf. Betrachtet man jedoch hinreichend {\em kleine} Gebiete, so kann man Abschätzungen auf den Fall konstanter Koeffizienten zurückführen. Das soll am Beispiel der Elliptizitätsabschätzung
\begin{equation}\label{eq:4:locEll}
\forall u\in\rmC_0^\infty(\omega)\quad:\quad    \| u\|_{(m)} \le C \| P(x,\D) u \|
\end{equation}
und der Haupttypabschätzung 
\begin{equation}\label{eq:4:locHT}
\forall u\in\rmC_0^\infty(\omega)\quad:\quad   \| u\|_{(m-1)} \le C \| P(x,\D) u \|
\end{equation}
diskutiert werden. Hierbei sei $x_0\in\Omega$ fest gewählt und $\omega\Subset\Omega$ eine hinreichend kleine Umgebung von $x_0$. Die Größe von $\omega$ hängt vom Verhalten der Koeffizienten ab. Die Gültigkeit einer solchen lokalen Abschätzung für jedes $x_0\in\Omega$ impliziert offenbar das entsprechende Resultat auf jeder relativ kompakten Teilmenge $\Omega'\Subset\Omega$. Konstanten hängen von der Wahl von $\omega$ beziehungsweise $\Omega'$ ab.



\section{Hinreichende Bedingungen}


\begin{thm}
Angenommen, $P(x,\D)$ ist elliptisch. Dann existiert zu jedem $x_0\in\Omega$ eine Umgebung $\omega\Subset\Omega$, so dass
\eqref{eq:4:locEll} mit einer von $\omega$ abhängigen Konstanten $C$ gilt. 
\end{thm}
\begin{proof}
Ohne Beschränkung der Allgemeinheit nehmen wir an der Ursprung liege im Gebiet und es gelte $x_0=0$. Sei weiter $P(\D)=P(0,\D)=\sum_\alpha a_\alpha(0)\D^\alpha$ der Operator der entsteht, wenn man in die Koeffizienten den Wert $0$ einsetzt. Dann gilt für alle $u\in\rmC_0^\infty(\Omega)$
\begin{equation}
 \|u\|_{(m)} \le C  \| P(\D) u\| 
\end{equation}
da $P(0,\D)$ elliptischer Operator mit konstanten Koeffizienten ist. Weiter existiert zu jedem $\epsilon>0$ ein $\delta>0$, so dass
$|a_\alpha(x)-a_\alpha(0)|\le \epsilon$ für $|x|\le\delta$. Damit impliziert die Dreiecksungleichung
\begin{equation}
  \| P(\D) u - P(x,\D) u\| \le \epsilon \sum_{|\alpha|\le m} \|\D^\alpha u\| = \epsilon \|u\|_{(m)}
\end{equation}
für alle $u\in\rmC_0^\infty(B_\delta)$. Für $\epsilon$ klein genug folgt damit die Behauptung.
\end{proof}

Ein Operator $P(x,\D)$ wird als  \eIndex[Differentialoperator]{von reellem Haupttyp} bezeichnet, falls sein Hauptsymbol $p(x,\xi)$ reellwertig ist. Unter der Voraussetzung stimmen $P(x,\D)$ und sein formal adjungierter ${}^tP(x,\D)$ bis auf einen Operator der Ordnung $m-1$ überein. Allgemeiner heißt $P(x,\D)$ \eIndex[Differentialoperator]{wesentlich normal}, falls die  \eIndex{Poissonklammer} des Hauptsymbols $p$ mit $\overline p$ definiert durch $\overline p(x, \overline\zeta) = \overline{p(x,\zeta)}$
\begin{equation}
    \{ p,\overline p\} (x,\xi) = \sum_{j=1}^n \bigg(\frac{\partial p(x,\xi)}{\partial \xi_j} \frac{\partial \overline p (x,\xi)}{\partial x_j} - \frac{\partial \overline p(x,\xi)}{\partial \xi_j}\frac{\partial  p(x,\xi)}{\partial x_j} \bigg)    = 0 
\end{equation}
für alle $\xi\in\R^n$ verschwindet. In diesem Fall ist der Kommutator von $P(x,\D)$ und ${}^tP(x,\D)$ von Ordnung $2m-2$ (statt nur $2m-1$).

Das folgende Resultat ergibt sich aus \cite[Theorem~4.1]{Hormander:1955} zusammen mit der Vorbemerkung zu diesem Kapitel. Die Umkehrung dieses Satzes gilt nicht.

\begin{thm}[{\cite[Theorem~4.1]{Hormander:1955}}]
Angenommen, $P(x,\D)$ ist vom Haupttyp und wesentlich normal. Dann existiert zu jedem $x_0\in\Omega$ eine Umgebung $\omega\Subset\Omega$, so dass
\eqref{eq:4:locHT} mit einer von $\omega$ abhängigen Konstanten $C$ gilt.
\end{thm}
\begin{proof}[Beweisskizze]
Für den Originalbeweis konstruiert man wiederum Energieintegrale und schätzt diese mittels partieller Integration ab.
\end{proof}

\section{Notwendige Bedingungen}
Im folgenden sei $\Omega \subset \R^n$ eine offene Umgebung der Null und $P(x,\D)$ ein Differentialausdruck auf $\Omega$ mit Hauptsymbol $p(x,\xi)$. 
\begin{lem}\label{lem1}
Angenommen, es existiert eine Funktion $u \in \rmC^\infty (\Omega)$ mit 
\begin{align}
\label{grad}
p(x, \nabla u) = \mathbf{o} \big(|x|^2\big), \qquad  x \rightarrow 0,
\end{align}
welche eine Taylorentwicklung 
\begin{align}\label{taylor}
u(x) = \i \xi\cdot x + x\cdot A x + \mathcal O\big(|x|^3\big),\qquad x\to0
%\sum_{j=1}^{n} x_j \xi_j + \dfrac{1}{2} \sum_{j,k=1}^{n} x_j x_k \alpha_{jk} + \mathcal O\left( |x|^3 \right), \qquad x\to0
\end{align}
mit $\xi\in\R^n$, einer symmetrischen Matrix $A=(\boldsymbol\alpha_{i,j})_{i,j}\in\C^{n\times n}$ und $B = (\Re \boldsymbol\alpha_{i,j})_{i,j} $ negativ definit  besitzt. 
Gilt weiterhin  
\begin{align}
\label{normpart}
\sum_{j=1}^{n}\bigg|\frac{\partial p(0,\xi)}{\partial \xi_j}\bigg|^2 \neq 0,
\end{align}
so folgt
\begin{align}
\label{lm 1 eq}
\sup_{v\in\rmC_0^\infty(\Omega)} \frac{ \lVert v \rVert_{(m-1)} }{ \lVert P(x,D)v \rVert} = \infty.
\end{align}
\end{lem}
\begin{proof}
Da die Matrix $B$ negativ definit ist, gilt
\begin{align*}
\Re u(x) = x\cdot Bx + \mathcal O\big(|x|^3\big) \le - 2 a |x|^2 + \mathcal O\big(|x|^3\big),\qquad x\to0
% \sum_{j,k=1}^{n} x^jx^k \Re \alpha_{jk} + \mathcal{O}\left( |x|^3 \right) \le -2a |x|^2 + \mathcal{O}\left( |x|^3 \right) (?)
\end{align*}
mit geeignet gewähltem $a>0$. Damit folgt
\begin{align}
\label{realteil}
\Re u(x) \le - a |x|^2 
\end{align}
für hinreichend kleine $|x|$. Nach Verkleinerung von $\Omega$ können wir also ohne Einschränkung annehmen, dass (\ref{realteil}) überall in $\Omega$ gilt. 

Sei nun $\phi \in \rmC_0^\infty(\Omega)$ und $t>0$. Dann erfüllt $v_t := \phi e^{tu} \in \rmC_0^\infty(\Omega)$ die Abschätzung $|v_t(x)| \le |\phi(x)| \e^{-at|x|^2}$. Wir zeigen nun
\[
\frac{\lVert v_t \rVert_{(m-1)}}{\lVert P(x,\D)v_t \rVert} \longrightarrow \infty, \qquad t \rightarrow \infty,
\]
woraus die Behauptung (\ref{lm 1 eq}) folgt.

Wir schreiben 
\[
P(x,D)v_t = e^{tu} \sum_{j=0}^{m} b_j(x) t^j.
\]
Es reicht, $b_m$ und $b_{m-1}$ zu berechnen. Mit der Leibnizformel erhalten wir
\[
P(x,D)(\phi e^{tu}) = \sum_{\alpha} \dfrac{D^\alpha \phi}{|\alpha|!} P^{(\alpha)}(x,D)e^{tu}.
\]
Wir zerlegen
\[
P(x,D) = p(x,D) + q(x,D) + r(x,D),
\]
wobei $p$ und $q$ homogen von Grad $m$ bzw. $m-1$ seien sowie $r$ von Grad $<m-1$. Dann sind die Koeffizienten von $t^m e^{tu}$ und $t^{m-1} e^{tu}$ in
\begin{align}
\phi p(x,D)e^{tu} + \left( \phi q(x,D) + \sum_{j=1}^{n} (D^j \phi) p^{(j)}(x,D) \right) e^{tu}
\end{align}
ebenfalls $b_m$ und $b_{m-1}$, denn alle anderen Summanden in der Leibnizformel haben Grad kleiner als $m-1$ in $t$. Nach Berechnung des ersten Summanden erhalten wir $b_m = (-i)^m \phi p(x,\nabla u)$ und insbesondere wegen (\ref{grad})
\begin{align}
b_m(x) = \textbf{o} (|x|^2), \hspace{0.5cm} x \rightarrow 0.
\end{align}
Weiter erhalten wir
\begin{align}
	\label{4.17}
b_{m-1} = (-i)^{m-1} \left( \phi (i^{m-1}g + q(x,\nabla u)) + \sum_{j=1}^{n} (D^j \phi) p^{(j)} (x,\nabla u) \right),
\end{align}
wobei $g$ eine glatte Funktion ist, die von $u$ und den Koeffizienten von $p$ abhängt. Wir wählen nun $\phi$ so, dass $\phi(0) =1$ gilt und sodass (\ref{4.17}) für $x=0$ verschwindet, was wegen (\ref{normpart}) möglich ist. Wegen Stetigkeit von $b_{m-1}$ ist dann $b_{m-1}(x) = \textbf{o}(1)$, wenn $x$ gegen Null strebt.

Nach diesen Überlegungen können wir für $\varepsilon >0$ eine Umgebung $U$ der Null wählen, sodass
\[
|b_m(x)| < \varepsilon |x|^2, \hspace{0.5cm} |b_{m-1}(x)| < \varepsilon, \hspace{0.5cm} \forall \, x \in U.
\]
Mit Gleichung (\ref{realteil}) erhalten wir für alle $x\in U$ die Abschätzung
\[
|P(x,D)v_t| \le t^{m-1} e^{-at|x|^2}(\varepsilon t |x|^2 + \varepsilon + C/t)
\]
und mit der Transformation $x \mapsto x/\sqrt{(t)}$ im Integral folgt
\[
\lVert P(x,D) v_t \rVert_{L^2(U)} ^2 \le t^{2m-2-n/2} \varepsilon^2 \int_U (|x|^2 + 1 +C/\varepsilon t)^2 e^{-2a|x|^2} dx,
\]
wobei das letzte Integral für $t \rightarrow \infty$ gegen
\[
B^2 = \int_U (|x|^2 + 1)^2 e^{-2a|x|^2} dx
\]
konvergiert. Für große $t$ liefert dies die Abschätzung
\[
\lVert P(x,D) v_t \rVert_{L^2(U)} ^2 \le 2 t^{2m-2-n/2} \varepsilon^2  B^2.
\]
Da aus (\ref{realteil}) ebenso
\[
\lVert P(x,D) v_t \rVert_{L^2(U^c)} ^2 = \mathcal{O}(t^{2m}e^{-2ct})
\]
für eine Konstante $c>0$ folgt, erhalten wir
\begin{align}
	\label{4.18}
\lVert P(x,D) v_t \rVert^2 \le 4 B^2 t^{2m-2-n/2} \varepsilon^2.
\end{align}
Als schätzen wir $\lVert v_t \rVert_{(m-1)}$ nach unten ab. Wir nehmen ohne Einschräung an, dass $\xi_1\neq 0$ gilt. Wir setzen $\alpha' = (m-1,0,\ldots,0)$ und erhalten
\[
D^{\alpha'} v_t = (\phi (D^1 u)^{m-1}t^{m-1}+ \ldots)e^{tu}.
\]
Wegen $\phi(0) (D^1u(0))^{m-1} = \xi_1^{m-1} \neq 0$ gilt $|\phi| |D^1u|^{m-1} \ge 2c > 0$ für eine Konstante $c > 0$ in einer Umgebung der Null und wegen $\Re u(x) = \mathcal{O} (|x|^2)$ existiert eine Konstante $A > 0$ sodass $\Re u(x) \ge -A |x|^2$. Für große $t$ erhalten wir
\[
|\phi| |D^1u|^{m-1} \ge ct^{m-1}e^{-tA|x|^2}.
\]
Für große $t$ folgt
\begin{align}
	\label{4.19}
\lVert v_t \rVert_{(m-1)}^2 \ge \lVert D^{\alpha'} v_t\rVert^2 \ge \int\limits_{t|x|^2 < 1}c^2t^{2m-21}e^{-2tA|x|^2} dx = C^2 t^{2m-2-n/2}
\end{align}
für eine Konstante $C > 0$ und eine Umgebung $V$ der Null. Kombiniert man (\ref{4.18}) und (\ref{4.19}), so erhält man für große $t$
\[
\dfrac{\lVert v_t \rVert_{(m-1)}}{\lVert P(x,D) v_t \rVert} \ge \dfrac{C}{2B\varepsilon}
\]
und da $\varepsilon>0$ beliebig war folgt die Behauptung des Lemmas.
\end{proof}

\begin{lem}\label{lem2}
Wir nehmen an, es gelte (\ref{normpart}). Dann gibt es eine Funktion $u \in \rmC^\infty(\Omega)$ mit (\ref{grad}) und (\ref{taylor}) genau dann, wenn
\begin{align}
	\label{4.15}
p(0, \xi) &= 0, \\ 	\label{4.16}
\sum_{k=1}^{n} \boldsymbol\alpha_{j,k}\frac{\partial p(0,\xi)}{\partial \xi_k} &= -\i\,  p_j(0,\xi), \hspace{0.5cm} j=1, \ldots, n, 
\end{align}
wobei $ p_j(x,\xi) = \partial_{x_j}  p(x,\xi)$ die partielle Ableitung des Hauptsymbols bezeichne.
\end{lem}

\begin{proof} [Beweisskizze]
Betrachten wir die Taylorentwicklung, so stellen wir fest, dass (\ref{grad}) genau dann erfüllt ist, wenn $p(x,\nabla u)$ und alle Ableitungen der Ordnung $\le 2$ im Nullpunkt verschwinden. Das Verschwinden von $p(x,\nabla u)$ wird gerade durch (\ref{4.15}) beschrieben. Wir berechnen weiter
\[
\dfrac{\partial}{\partial x_j} p(0,\nabla u) = 
p_j(0,i\xi) + \sum_{k=1}^{n} \boldsymbol\alpha_{j,k} \dfrac{\partial p(0,i\xi)}{\partial \xi_k}
\]
und bemerken unter Verwendung der Homogenität von $p$, dass die linke Seite genau dann für alle $j$ verschwindet, wenn (\ref{4.16}) gilt. Es bleibt zu zeigen, dass für geeignet gewähltes $u$ unter den gegebenen Bedingungen auch die zweiten Ableitungen verschwinden. Hierzu leitet man die obige Gleichung ein weiteres Mal ab und stellt unter Verwendung von (\ref{normpart}) fest, dass das resultierende System partieller Differentialgleichungen die Voraussetzungen des Satzes von Cauchy-Kovalevsky erfüllt, woraus die Existenz eines solchen $u$ folgt.
\end{proof}

\begin{lem}\label{lem3}
Seien $\zeta, f \in \C^n$ mit $\zeta \neq 0$. Dann gibt es genau dann eine symmetrische Matrix $A = (\boldsymbol\alpha_{i,j})_{i,j}$ mit negativ definitem Realteil
$B=(\Re\boldsymbol \alpha_{i,j})_{i,j}$ und
\begin{align}\label{laag1}
   A \zeta = f,
\end{align}
wenn die Vektoren $\zeta$ und $f$
\begin{align} \label{laag2}
\Re f\cdot \overline \zeta < 0
\end{align}
erfüllen.
\end{lem}
\begin{proof}
Es gelte (\ref{laag1}). Multiplikation beider Seiten mit $\overline \zeta_k$ und Aufaddieren ergibt unter Verwendung der Symmetrie von $A$
und mit $\zeta=\xi+\i\eta$
\[
 f\cdot \overline \zeta = \sum_{k=1}^{n}f_k \overline{\zeta_k} = \sum_{j,k=1}^{n} \boldsymbol\alpha_{kj} \zeta_j \overline \zeta_k
=  \sum_{j,k=1}^{n} \boldsymbol\alpha_{kj} \xi_j \xi_k +  \sum_{j,k=1}^{n} \boldsymbol\alpha_{kj} \eta_j \eta_k,
\]
Da für mindestens ein $j$ ein $\xi_j$ oder $\eta_j$ ungleich Null ist, folgt (\ref{laag2}) wegen der Definitheit von $B$. $\bullet$\qquad 
Es gelte nun umgekehrt (\ref{laag2}). Wir nehmen zunächst an, $\zeta$ ist proportional zu einem reellen Vektor. Nach Multiplikation von $f$ und $\zeta$ mit derselben komplexen Zahl können wir dann annehmen, dass $\zeta=\xi$ reell ist. Mit $A = B + \i C$, $f= g+\i h$ wird aus (\ref{laag1}) das Gleichungssystem bestehend aus $B \xi = g$ und $C \xi = h$. Dass eine reelle, symmetrische Matrix $C$ mit $C \xi = h$ existiert, ist offensichtlich. Schreiben wir weiter $g = g' + \xi (g\cdot \xi)/ 2 |\xi|^2$, so gilt $g'\cdot\xi = g\cdot\xi /2 < 0$. Man rechnet dann leicht nach, dass die Matrix $B$, gegeben durch
\[
B x = \dfrac{g\cdot\xi }{2|\xi|^2} x + \dfrac{x\cdot g'}{\xi\cdot g'} g',\qquad x\in\R^n,
\]
negativ definit und symmetrisch ist und $A=B+\i C$ erfüllt das gesuchte. Sei nun $\zeta$ nicht proportional zu einem reellen Vektor. Wir zeigen, dass 
\[
A = \dfrac{\Re( f\cdot\overline\zeta )}{|\zeta|^2} \mathrm I + \i C
\]
für ein reelles $C\in\R^{n\times n}$ die gewünschten Eigenschaften erfüllt. Die Bedingung an $C$ schreibt sich dann als
\begin{align}
	\i C \zeta = f',
\end{align}
wobei $f' = f - \zeta \frac{\Re( f\cdot\overline\zeta)}{|\zeta|^2}$ die Eigenschaft
\begin{align}
	\label{Ebene}
\Re (f'\cdot\overline\zeta) = 0
\end{align}
hat. Um die Existenz eines solchen $C$ zu zeigen, stellen wir zunächst fest, dass die Menge der Vektoren in $\C^n$, die als $\i C \zeta$, $C \in \R^{n \times n}$ symmetrisch, geschrieben werden können, einen reellen Vektorraum bilden. Dieser ist für ein $g \in \C^n$ in der Hyperebene $\{ z \in \C^n : \Re g\cdot \overline z = 0 \}$ enthalten. Für jedes $\xi \in \R^n$ ist die Matrix $C$, gegeben durch $C x =  (x\cdot\xi)\xi$, symmetrisch und es folgt
\[
\Re \i (\xi\cdot\overline g)(\zeta\cdot\xi) = 0.
\]
Insbesondere ist $(\xi\cdot\overline g)(\zeta\cdot\xi) $ immer reell. Unter Verwendung, dass $\zeta$ nicht proportional zu einem reellen Vektor ist, berechnet man schnell, dass $g$ ein reelles Vielfaches von $\zeta$ sein muss. Insbesondere ist $\Re(z\cdot\overline\zeta) = 0 \Leftrightarrow \Re(z\cdot \overline g) = 0$ und $f'$ lässt sich wegen (\ref{Ebene}) als $\i C \zeta$ schreiben für eine reelle, symmetrische Matrix $C$.  Es folgt die Behauptung.
\end{proof}

\begin{thm}
%Es seien die Koeffizienten von $P(x,\D)$ stetig und die Koeffizienten von $p$ seien $C^2$. Weiter 
Es gelte die Ungleichung (\ref{eq:4:locHT}), d.h.
\[
  \| u\|_{(m-1)} \le C \| P(x,\D) u \| \quad \forall u\in\rmC_0^\infty(\Omega).
\]
Dann folgt 
\begin{align}
	\label{Beh}
\{p,\overline p\} (x,\xi) = 0,
\end{align}
falls $p(x,\xi) = 0$, $x \in \Omega$ und $ \xi \in \R^n$.
\end{thm}
\begin{proof}
Wir nehmen ohne Einschränkung $x=0$ an. Weiterhin gelte
\[
\sum_{j=1}^{n}|\partial p(0,\xi)/\partial \xi_j|^2 \neq 0,
\]
denn sonst ist (\ref{Beh}) trivialerweise erfüllt. Die Lemmata \ref{lem1} und \ref{lem2} zeigen dann, dass die Gleichung (\ref{4.16}) für keine symmetrische Matrix mit negativ definitem Realteil erfüllt sein kann. Mit Lemma \ref{lem3} folgt deshalb $\{p,\overline p\}(x,\xi) \ge 0$. Das analoge Argument mit $-\xi$ anstatt $\xi$ liefert $\{p,\overline p\}(x,-\xi) \ge 0$. Weil $\{p,\overline p\}(x,\xi)$ eine ungerade Funktion in $\xi$ ist, folgt die Behauptung.
\end{proof}