% !TEX root = main.tex
\chapter{Maximale Operatoren}

% Voraussichtliche Struktur

Sei $P(\D)$ eine Differentialausdruck und $\Omega\subset\R^n$ ein Gebiet. Wir erinnern an die Definition des maximalen Operators $P$, dieser versteht sich als größtmögliche distributionelle Fortsetzung von $P(D)$ und ist auf dem Definitionsbereich
\begin{equation}
   \mathcal D_P = \{ u\in\rmL^2(\Omega)\;:\; P(\D)u\in\rmL^2(\Omega) \}
\end{equation}
gegeben. Maximale Operatoren sind im wesentlichen durch ihren Definitionsbereich bestimmt. Es gilt
\begin{thm}
Angenommen, für zwei Differentialausdrücke $P(\D)$ und $Q(\D)$ und die zugeordneten maximalen Operatoren gilt $\mathcal D_P\subseteq\mathcal D_Q$. Dann gilt
entweder $P(\xi)=a Q(\xi) + b$ mit Konstanten $a,b\in\C$ oder es gibt einen Vektor $\nu\in\R^n$ und Polynome $p$ und $q$ in einer Variablen mit
$P(\xi)=p(\xi\cdot\nu)$ und $Q(\xi)=q(\xi\cdot\nu)$ sowie $\deg q\le\deg p$.
\end{thm}
\begin{proof} Vgl. \cite{Hormander:1955}, beruht auf strukturellen Argumenten zu Polynomringen über $\C$. Wesentlich ist, daß $\mathcal D_P\subseteq \mathcal D_Q$ Abschätzungen der Form
\begin{equation}
   |Q(\zeta)|^2 \le C \big( |P(\zeta)|^2 + 1\big)%,\qquad |Q^{(\alpha)}(\zeta)|^2 \le C \widetilde P(\zeta)^2
\end{equation}
für alle {\em komplexen} $\zeta\in\C^n$ impliziert.
\end{proof}




\section{Differentialoperatoren von lokalem Typ} % bearbeitet durch Thomas Hamm
\begin{df}
Der Differentialoperator $P(\D)$ heißt \eIndex[Differentialoperator]{von lokalem Typ}, wenn $u\varphi\in\mathcal D_{P_0}$ für alle $u\in\mathcal D_{P}$ und alle $\varphi\in C_0^\infty(\Omega)$.
\end{df}

\begin{lem}
$P(\D)$ ist von lokalem Typ genau dann, wenn für jedes $u\in\mathcal{D}_P$ und jedes Kompaktum $K\subset\Omega$ eine Funktion $v\in\mathcal{D}_{P_0}$ existiert mit $u\vert_K=v\vert_K$.
\end{lem}
\begin{proof}
Sei zunächst $P(\D)$ von lokalem Typ und $K\subset\Omega$ kompakt. Sei weiter $\varphi\in C_0^\infty(\Omega)$ mit $\varphi(x)=1$ für $x\in K$ und setze $v=u\varphi$. Dann ist nach Voraussetzung $v\in\mathcal{D}_P$ und darüber hinaus $\supp v$ kompakt. Wir zeigen, dass daraus bereits $v\in\mathcal{D}_{P_0}$ folgt. Sei dazu $\psi \in C_0^\infty(\R^n)$ mit $\int\psi(x)\d x=1$ und $\psi_\epsilon(x)=\epsilon^{-n} \psi(x/\epsilon)$ für $\epsilon>0$. Für alle hinreichend kleinen $\epsilon$ ist dann $v_\epsilon=v*\psi_\epsilon\in C_0^\infty(\Omega)$ und $v_\epsilon \to v$ in $\rmL^2(\Omega)$ für $\epsilon\to 0$. Da dann außerdem $P(\D)v_\epsilon=(P(\D)v)*\psi_\epsilon \to P(\D)v$ in $\rmL^2(\Omega)$ für $\epsilon\to 0$ folgt daraus $v\in \mathcal{D}_{P_0}$ gerade aus der Definition von $P_0$ als $\rmL^2$-$\rmL^2$-Abschluß auf $C_0^\infty(\Omega)$.\\
Für die Umkehrung sei $u\in\mathcal{D}_P$ und $\varphi\in C_0^\infty(\Omega)$. Dann existiert für $K=\supp \varphi$ ein $v\in\mathcal{D}_{P_0}$ mit $u\vert_K=v\vert_K$. Dann ist $u\varphi=v\varphi $ und $v\varphi\in\mathcal{D}_{P_0}$. %klar?
\end{proof}
\begin{lem}[{\cite[Lemma~3.4]{Hormander:1955}}]\label{lem : loktyp}
$P(\D)$ ist von lokalem Typ genau dann, wenn $P^{(\alpha)}(\D)u \in \rmL^2_{\text{loc}}(\Omega)$ für alle Multiindices $\alpha$ und $u\in\mathcal{D}_P$. In diesem Fall existiert für ein beliebiges Gebiet $\Omega'\subset \Omega$ mit kompaktem Abschluss in $\Omega$ ein $C>0$ mit
\begin{equation}
\int_{\Omega'} \abs{P^{(\alpha)}(\D)u(x)}^2\d x\leq C\left( \int_{\Omega} \abs{P(\D)u(x)}^2\d x + \int_{\Omega} \abs{u(x)}^2 \d x \right) \label{eq:lm3.4}
\end{equation}
für alle $u\in\mathcal{D}_P$.
\end{lem}
\begin{proof}
Sei zuerst $P(\D)$ von lokalem Typ, $u\in\mathcal{D}_P$ und $K\subset\Omega$ kompakt. Nach Lemma \ref{lem : loktyp} existiert $v\in\mathcal{D}_{P_0}$ mit $v|_K=u|_K$. Daraus folgt $P^{(\alpha)}(\D)u|_K=P^{(\alpha)}(\D)v|_K$ und es ist $P^{(\alpha)}(\D)v\in\rmL^2(\Omega)$ nach Korollar \ref{cor:2:2.6}. Also ist $P^{(\alpha)}(\D)u \in \rmL^2_{\text{loc}}(\Omega)$.\\
Sei umgekehrt $P^{(\alpha)}(\D)u \in \rmL^2_{\text{loc}}(\Omega)$ für alle Multiindices $\alpha$ und $K\subset\Omega$ kompakt. Sei $\varphi\in C_0^\infty(\Omega)$ mit $\varphi(x)=1$ für $x\in K$. Dann ist wegen
\begin{equation}
P(\D)v=\sum_\alpha \dfrac{\D^\alpha \varphi}{{\alpha}!}P^{(\alpha)}(\D)u
\end{equation}
$P(\D)v\in L^2(\Omega)$, also $v\in\mathcal{D}_P$. Da außerdem $\supp v$ kompakt, ist $v\in \mathcal{D}_{P_0}$, wie wir im Beweis von Lemma \ref{lem : loktyp} gesehen haben. Also ist $P(\D)$ nach Lemma \ref{lem : loktyp} von lokalem Typ.\\
Die Ungleichung \ref{eq:lm3.4} folgt aus Satz \ref{thm:1:1.1} angewandt auf die Abbildung
\begin{equation}
\mathcal{D}_P \ni u \mapsto P^{(\alpha)}u\in \rmL^2(\Omega').
\end{equation}
\end{proof}

%
%
%

\begin{thm}[{\cite[Theorem~3.12]{Hormander:1955}}]
   Angenommen, $P(\D)$ ist von lokalem Typ. Dann ist $P$ der Abschluß seiner Einschränkung auf $\mathcal D_P\cap \rmC^\infty(\Omega)$.
\end{thm}
\begin{proof}

\end{proof}

%%
%
%

Im folgenden sollen Kriterien dafür gefunden werden, daß ein gegebener Differentialausdruck $P(\D)$ von lokalem Typ ist. Ein erstes notwendiges Kriterium liefert folgendes Lemma.

\begin{lem}[{\cite[Lemma~3.5]{Hormander:1955}}]
Sei $\Omega$ ein beschränktes Gebiet und $P(\D)$ von lokalem Typ. Dann existiert für jedes $A>0$ ein $C>0$, sodass
\begin{equation}
\widetilde{P}(\zeta)^2=\sum_\alpha \abs{P^{(\alpha)}(\zeta)}^2\leq C(1+\abs{P(\zeta)}^2)
\end{equation}
für alle $\zeta\in\C^n$ mit $\abs{\Im \zeta} < A$.
\end{lem}
\begin{proof}
Setze $u(x)=\e^{\i x\cdot\zeta}$. Dann ist $P^{(\alpha)}(\D)u(x)=P^{(\alpha)}(\zeta)\e^{\i x\cdot\zeta}$ und mit Lemma \ref{lem : loktyp} folgt
\begin{equation}
\abs{P^{(\alpha)}(\zeta)}^2 \int_{\Omega} \e^{-2x\cdot \eta}\d x \leq C(1+\abs{P(\zeta)}^2)\int_\Omega \e^{-2x\cdot \eta}\d x
\end{equation}
wobei $\eta = \Im \zeta$. Setzen wir weiter $\delta = \sup \{2\abs{x}:x\in\Omega\}$, so folgt
\begin{equation}
\int_\Omega \e^{-2x\cdot \eta}\d x \geq \int_\Omega \e^{-\delta \abs{\eta}}=\abs{\Omega}\e^{-\delta \abs{\eta}}
\end{equation}
und
\begin{equation}
\int_\Omega \e^{-2x\cdot \eta}\d x \leq \int_\Omega \e^{\delta \abs{\eta}}=\abs{\Omega}\e^{\delta \abs{\eta}}.
\end{equation}
Insgesamt also $\abs{P^{(\alpha)}(\zeta)}^2\leq C\e^{2\delta \abs{\eta}}(1+\abs{P(\zeta)}^2)\leq \widetilde{C}(1+\abs{P(\zeta)}^2) $, wenn $\abs{\eta}=\abs{\Im \zeta}$ beschränkt. Summation über alle $\alpha$ liefert die Behauptung.
\end{proof}


%
% Satz 3.3.
%

Es zeigt sich, dass das gerade gefundene Kriterium auch hinreichend ist. Wir formulieren den Satz, der Beweis wird sich über mehrere Abschnitte hinziehen und nachfolgend in einzelnen Teilschritten gezeigt werden.
\begin{thm}[{\cite[Theorem~3.3]{Hormander:1955}}]
Sei $P(\D)$ ein Differentialausdruck mit Symbol $P(\xi)$. Dann sind \"aquivalent:
\begin{enumerate}
\item Für jedes $A>0$ existiert ein $R>0$, so daß $P(\xi+\i\eta)\ne0$ für alle $|\eta|<A$ und $\dist(\xi,\Lambda(P))>R$.
\item Für jedes $\vartheta\in\R^n$ gilt $\lim_{\xi\to\infty} P(\xi+\vartheta)/P(\xi)=1$ modulo $\Lambda(P)$.
\item Für jeden Multiindex $\alpha\in\N_0^n$, $\alpha\ne0$, gilt $\lim_{\xi\to\infty} P^{(\alpha)}(\xi)/P(\xi)=0$ modulo $\Lambda(P)$.
\item Für jedes $A>0$ existiert ein $C>0$, so daß $\widetilde P(\xi+\i\eta)^2\le C\big(|P(\xi+\i\eta)|^2+1\big)$ für $|\eta|\le A$.
\item Für jedes $A>0$ gilt $\lim_{\xi\to\infty} P(\xi+\i\eta)=\infty$ modulo $\Lambda(P)$ gleichmäßig in $|\eta|\le A$.
\item $P(\D)$ ist von lokalem Typ.
\end{enumerate}
\end{thm} 





\begin{lem}[{\cite[Lemma~3.7]{Hormander:1955}}]
Sei $P\in \C[X_1,\ldots,X_n]$. Angenommen für jedes $A>0$ existiert ein $B>0$, sodass $P(\xi+\i\eta)\neq 0$ für alle $\xi,\eta\in\R^n$ mit $\abs{\eta}<A$ und $\abs{\xi}>B$. Dann ist $P$ vollständig und
\begin{equation}
\lim_{\xi\to\infty}\dfrac{P(\xi+\vartheta)}{P(\xi)}=1 \label{eq:lm3.7}
\end{equation}
für alle $ \vartheta\in\R^n$.
\end{lem}
\begin{proof}
Angenommen $P$ wäre nicht vollständig. Dann sei $0\neq\nu\in\Lambda(P)$. Sei außerdem $\zeta=\xi+\i\eta$ eine Nullstelle von $P$. Dann ist $P(\xi+t\nu+\i\eta)=0$ für alle $t\in\R$, also für $\abs{\xi+t\nu}$ beliebig groß. Ein Widerspruch.\\
Durch Verschieben des Koordinatensystems zeigen wir \ref{eq:lm3.7} ohne Einschränkung für den Vektor $\vartheta=(1,0,\ldots,0)$.
\end{proof}

\section{Konstruktion von Fundamentallösungen eines vollständigen Operators von lokalen Typ} % bearbeitet von Matthias Hofmann

Im folgenden konstruieren wir eine Fundamentallösung, also eine Distribution $E\in\mathscr D'(\R^n)$ mit
\begin{equation}
E*(P(\D) u) = u, \quad u \in \rmC_0^\infty(\R^n),
\end{equation}
welche zusätzliche Regularitätseigenschaften besitzt.
Um genauer zu sein,  wollen wir zeigen:
\begin{thm}\label{fundamental_exist}
Sei $P$ vollständig und von lokalem Typ, dann besitzt $P(\D)$ eine Fundamentallösung $E$, welche 
\begin{enumerate}
\item im Gebiet $\R^n\setminus\{0\}$  durch eine unendlich differenzierbare Funktion $E(x)$ dargestellt wird und
\item für die $P^{(\alpha)}(\D) E*u$ für jedes $u\in \rmL^2(\R^n)$ mit kompakten Träger $\supp u\Subset\R^n$ lokal quadrat-integrierbar ist. 
\end{enumerate}
\end{thm}
\begin{rem}
Ist $P(\D)$ vollständig und von lokalem Typ so besitzt jede Fundamentallösung diese Eigenschaften, da die Differenz zweier Fundamentallösungen unendlich oft differenzierbar ist. Dies werden wir später sehen.
\end{rem}

Zunächst einige Worte zur Beweisidee.  Falls $P(\xi)\neq 0$ für alle $\xi\in \mathbb R$, so ließe sich $E$ bestimmen aus
\begin{equation}\label{distr1}
E*u(x) = (2\pi)^{-n/2}\int \e^{\i x\cdot \xi} \frac{\widehat u(\xi)}{P(\xi)}\, \mathrm d\xi, \quad u \in \rmC_0^\infty,
\end{equation}
oder äquivalent
\begin{equation}\label{distr2}
\langle E,\check u\rangle = (2\pi)^{-n/2} \int \frac{\widehat u(\xi)}{P(\xi)}\, \mathrm d\xi, \quad u \in\rmC_0^\infty,
\end{equation}
wobei $\check u$ durch $\check u(x)=u(-x)$ gegeben ist. Wir überlegen uns also eine Verallgemeinerung von \eqref{distr2}. 

\begin{proof}[Beweis von Satz \ref{fundamental_exist}]
Wir werden diesen Beweis in einigen Schritten führen.
\vspace{1mm}\\
\textbf{Schritt 1:} \emph{Konstruktion einer Fundamentallösung.}
Nach Theorem ?? ({\bf v}) gilt $|P(\xi)|\ge 1$ falls $\xi\in \mathbb R$ und $|\xi|\ge C$ mit geeigneter Konstante $C$.  Damit folgt $|P(\xi)\ge 1|$ falls $\xi_2^2+ \ldots + \xi_n^2 \ge C^2$.  

Wir können annehmen, dass der Koeffizient vor der höchsten Potenz von $\xi_1$ von $P(\xi)$ konstant ist (vgl. Lemma ??). 
%Da die Nullstellen des Polynoms stetig von den Parametern abhängen, die nicht im führenden Koeffizienten stehen.  
So finden wir außerdem leicht eine zweite Konstante $C'$ sodass $|P(\xi_1, \xi_2, \cdots, \xi_nu)|\ge 1$ für $\xi_2^2+\ldots + \xi_n^2< C$ and $|\xi_1|\ge C'$.  

Jetzt setzen wir für $u\in\rmC_0^\infty(\mathbb R^n)$
\begin{equation}\label{distr3}
\langle E, \check u\rangle  = (2\pi)^{-n/2} \int \, \mathrm d\xi_2 \cdots \, \mathrm d\xi_n \oint \frac{\widehat u(\xi)}{P(\xi)} \, \mathrm d\xi_1= (2\pi)^{-n/2} \oint \frac{\widehat u(\xi)}{P(\xi)} \, \mathrm d\xi.
\end{equation}
wobei das Integral über $\xi_1$ über die reelle Achse gehe, falls $\xi_2^2+ \cdots + \xi_n^2 \ge C^2$, und über die reelle Achse mit dem Intervall $(-C', C')$ ersetzt durch einen Halbkreis in der unteren Halbebene, falls $\xi_2^2+ \cdots+ \xi_n^2 < C^2$. Damit gilt $|P(\xi)|\ge 1$  in dem Integral.  Damit ergibt sich analog zu \eqref{distr1}
\begin{equation}
E*u(x) = (2\pi)^{-n/2}\oint e^{\i x\cdot \xi } \frac{\widehat u(\xi)}{P(\xi)}\, \mathrm d\xi, \quad u \in \rmC_0^\infty.
\end{equation} 
Damit folgt für $u\in\rmC_0^\infty(\mathbb R^n)$
\begin{equation}
E*(P(\D)u) = (2\pi)^{-n/2} \oint \e^{\i x\cdot\xi} P(\xi)  \frac{\widehat{u} (\xi)}{P(\xi)}\, \mathrm d\xi=(2\pi)^{-n/2} \oint \e^{\i x\cdot \xi} \widehat u(\xi) \, \mathrm d\xi
\end{equation}
und da der Integrand holomorph in $\xi_1$ ist, können wir den Integrationsweg als reale Achse wählen. Damit folgt
\begin{equation}
E*(P(\D) u)(x) = (2\pi)^{-n/2} \int \e^{\i x\cdot \xi} \widehat u(\xi) \, \mathrm d\xi = u(x),
\end{equation}
und $E$ ist somit eine Fundamentallösung.

\textbf{Schritt 2:} \emph{Beweis der Quadratintegrierbarkeit von $P^{(\alpha)}(\D) E * u$.}
Wir teilen das Integral \eqref{distr3} in zwei Teile auf. Falls $R=\sqrt{C^2 + C'^2}$, haben wir $|\xi<<R$ und der Teil des Integrals von \eqref{distr3}, wo $\xi$ nicht real ist. Schreibe also 
\begin{equation}\label{distr4}
E=E_1 + E_2,
\end{equation}
\begin{equation}\label{distr5}
\langle E_1,\check u\rangle = (2\pi)^{-n/2} \int_{|\xi|\ge R} \frac{\widehat u(\xi)}{P(\xi)}\, \mathrm d\xi, \quad E_2(\check u)= (2\pi)^{-n/2} \oint_{|\xi|\le R} \frac{\widehat u(\xi)}{P(\xi)}\, \mathrm d\xi
\end{equation}
Die Variable $\xi$ in der Definition von $E_1$ kann nur reelle Werte im Integral annehmen. Die Distribution $E_2$ ist eine ganze Funktion, da für $u\in\rmC_0^\infty$ erhält man mit der Definition von $\widehat u$
\begin{equation}
\langle E_2, \check u\rangle = (2\pi)^{-\nu} \oint_{|\xi |\le R} \frac{\mathrm d\xi}{P(\xi)} \int u(x) \e^{-\i x\cdot\xi}\, \mathrm dx= (2\pi)^{-n} \int \check u(x) \, \oint_{|\xi|\le R} \frac{e^{\i x\cdot \xi}}{P(\xi)}\, \mathrm d\xi \, \mathrm dx
\end{equation}
erhält. Damit ist $E_2$ regulär und
\begin{equation}
E_2(x) = (2\pi)^{-n} \oint_{|\xi|\le R} \frac{\e^{\i x\cdot \xi}}{P(\xi)}\, \mathrm d\xi,
\end{equation}
die analytisch ist, da das Integral gleichmäßig konvergent auf Kompakta ist. 

Sei $u\in L^2$ mit kompaktem Träger. Die Faltung $P^{\alpha}(\D)E_2 *u$ ist dann eine analytische Funktion.  Die Aussage ({\bf ii}) folgt dann, wenn wir zeigen, dass $P^{(\alpha)}(\D) E_1 * u$ quadratintegrierbar ist.  Sei $\phi \in\rmC_0^\infty$. Dann ist die Funktion $U*\phi$ auch in $C_0^\infty$ und im Sinne von \eqref{distr5} folgt
\begin{equation}
\langle P^{(\alpha)} (\D) E_1 * u, \check \phi\rangle  = P^{(\alpha)}(\D) E_1 * u * \phi(0) = \int_{|\xi| \ge R} \frac{P^{(\alpha)}(\xi)}{P(\xi)} \widehat u(\xi) \widehat \phi(\xi) \, \mathrm d\xi,
\end{equation}     
so dass die Fouriertransformierte von $P^{(\alpha)}(\D) E_1 * u$ eine Funktion ist die für $|\xi|<R$ verschwindet und gleich $\widehat u(\xi)P^{(\alpha)}(\xi)/P(\xi)$ falls $|\xi|\ge R$. Damit ist $P^{(\alpha)}(\D) E_1 * u$ auch quadratintegrierbar und ({\bf ii}) folgt.

\textbf{Schritt 3:} \emph{Beweis der Differenzierbarkeit in $\mathbb R^n\setminus{0}$.}
Im Folgenden wenden wir uns dem Beweis von ({\bf i}) zu. Da $E_2$ eine ganze Funktion ist, genügt es zu zeigen, dass $E_1$ unendlich oft differenzierbar für $x\neq 0$ ist.  Wir beginnen mit folgenden Lemma:
\begin{lem}\label{3.2lem1}
Sei $0\neq y \in \mathbb R^n$ fest und sei
\begin{equation}\label{3.2inf}
M(\tau) = \inf_{\zeta, \xi} |\zeta-\xi|,
\end{equation}
wobei $\xi\in \mathbb C^n$ und $P(\xi)=0$, und $\xi\in \mathbb R^n$ mit $|y, \xi|= \tau$. Dann existieren positive Zahlen $a$ und $b$ mit
\begin{equation}\label{3.2to}
M(\tau) \tau^{-b} \to a\quad \text{mit } \tau \to \infty.
\end{equation}
\end{lem}
\begin{proof}
Nach Theorem ?? ({\bf i}) ist der Restriktionsbereich für $\zeta$ kompakt und das Infimum in \ref{3.2inf} wird angenommen. Weiter folgt leicht, dass $M(\tau)$ stetig von $\tau$ abhängt.  
\end{proof}
\end{proof}
