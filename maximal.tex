% !TEX root = main.tex
\chapter{Maximale Operatoren}

% Voraussichtliche Struktur
\section{Differentialoperatoren von lokalem Typ} % bearbeitet durch Thomas Hamm


\section{Konstruktion von Fundamentallösungen eines vollständigen Operators von lokalen Typ} % bearbeitet von Matthias Hofmann

Im folgenden konstruieren wir eine Fundamentallösung, also eine Distribution $E\in\mathscr D'(\R^n)$ mit
\begin{equation}
E*(P(\D) u) = u, \quad u \in \rmC_0^\infty(\R^n),
\end{equation}
welche zusätzliche Regularitätseigenschaften besitzt.
Um genauer zu sein,  wollen wir zeigen:
\begin{thm}\label{fundamental_exist}
Sei $P$ vollständig und von lokalem Typ, dann besitzt $P(\D)$ eine Fundamentallösung $E$, welche 
\begin{enumerate}
\item im Gebiet $\R^n\setminus\{0\}$  durch eine unendlich differenzierbare Funktion $E(x)$ dargestellt wird und
\item für die $P^{(\alpha)}(\D) E*u$ für jedes $u\in \rmL^2(\R^n)$ mit kompakten Träger $\supp u\Subset\R^n$ lokal quadrat-integrierbar ist. 
\end{enumerate}
\end{thm}
\begin{rem}
Ist $P(\D)$ vollständig und von lokalem Typ so besitzt jede Fundamentallösung diese Eigenschaften, da die Differenz zweier Fundamentallösungen unendlich oft differenzierbar ist. Dies werden wir später sehen.
\end{rem}

Zunächst einige Worte zur Beweisidee.  Falls $P(\xi)\neq 0$ für alle $\xi\in \mathbb R$, so ließe sich $E$ bestimmen aus
\begin{equation}\label{distr1}
E*u(x) = (2\pi)^{-n/2}\int \e^{\i x\cdot \xi} \frac{\hat u(\xi)}{P(\xi)}\, \mathrm d\xi, \quad u \in \rmC_0^\infty,
\end{equation}
oder äquivalent
\begin{equation}\label{distr2}
\langle E,\check u\rangle = (2\pi)^{-n/2} \int \frac{\hat u(\xi)}{P(\xi)}\, \mathrm d\xi, \quad u \in\rmC_0^\infty,
\end{equation}
wobei $\check u$ durch $\check u(x)=u(-x)$ gegeben ist\footnote{Die Integrabilität von $\hat u$ folgt im Wesentlichen aus dem Satz von Paley-Wiener.}. Wir überlegen uns also eine Verallgemeinerung von \eqref{distr2}. 

\begin{proof}[Beweis von Satz \ref{fundamental_exist}]
Wir werden diesen Beweis in einigen Schritten führen.
\vspace{1mm}\\
\textbf{Schritt 1:} \emph{Konstruktion einer Fundamentallösung.}
Nach Theorem ?? ({\bf v}) gilt $|P(\xi)|\ge 1$ falls $\xi\in \mathbb R$ und $|\xi|\ge C$ mit geeigneter Konstante $C$.  Damit folgt $|P(\xi)\ge 1|$ falls $\xi_2^2+ \ldots + \xi_n^2 \ge C^2$.  

Wir können annehmen, dass der Koeffizient vor der höchsten Potenz von $\xi_1$ von $P(\xi)$ konstant ist (vgl. Lemma ??). 
%Da die Nullstellen des Polynoms stetig von den Parametern abhängen, die nicht im führenden Koeffizienten stehen.  
So finden wir außerdem leicht eine zweite Konstante $C'$ sodass $|P(\xi_1, \xi_2, \cdots, \xi_nu)|\ge 1$ für $\xi_2^2+\ldots + \xi_n^2< C$ and $|\xi_1|\ge C'$.  

Jetzt setzen wir für $u\in\rmC_0^\infty(\mathbb R^n)$
\begin{equation}\label{distr3}
\langle E, \check u\rangle  = (2\pi)^{-n/2} \int \, \mathrm d\xi_2 \cdots \, \mathrm d\xi_n \oint \frac{\hat u(\xi)}{P(\xi)} \, \mathrm d\xi_1= (2\pi)^{-n/2} \oint \frac{\hat u(\xi)}{P(\xi)} \, \mathrm d\xi.
\end{equation}
wobei das Integral über $\xi_1$ über die reelle Achse gehe, falls $\xi_2^2+ \cdots + \xi_n^2 \ge C^2$, und über die reelle Achse mit dem Intervall $(-C', C')$ ersetzt durch einen Halbkreis in der unteren Halbebene, falls $\xi_2^2+ \cdots+ \xi_n^2 < C^2$. Damit gilt $|P(\xi)|\ge 1$  in dem Integral.  Damit ergibt sich analog zu \eqref{distr1}
\begin{equation}
E*u(x) = (2\pi)^{-n/2}\oint e^{\i x\cdot \xi } \frac{\hat u(\xi)}{P(\xi)}\, \mathrm d\xi, \quad u \in \rmC_0^\infty.
\end{equation} 
Damit folgt für $u\in\rmC_0^\infty(\mathbb R^n)$
\begin{equation}
E*(P(\D)u) = (2\pi)^{-n/2} \oint \e^{\i x\cdot\xi} P(\xi)  \frac{\hat{u} (\xi)}{P(\xi)}\, \mathrm d\xi=(2\pi)^{-n/2} \oint \e^{\i x\cdot \xi} \hat u(\xi) \, \mathrm d\xi
\end{equation}
und da der Integrand analytisch in $\xi_1$ ist, können wir den Integrationsweg als reale Achse wählen. Damit folgt
\begin{equation}
E*(P(\D) U)(x) = (2\pi)^{-n/2} \int \e^{\i x\cdot \xi} \hat u(\xi) \, \mathrm d\xi = u(x),
\end{equation}
und $E$ ist somit eine Fundamentallösung.

\textbf{Schritt 2:} \emph{Beweis der Quadratintegrierbarkeit von $P^{(\alpha)}(\D) E * u$.}
Wir teilen das Integral \eqref{distr3} in zwei Teile auf. Falls $R=\sqrt{C^2 + C'^2}$, haben wir $|\xi<<R$ und der Teil des Integrals von \eqref{distr3}, wo $\xi$ nicht real ist. Schreibe also 
\begin{equation}\label{distr4}
E=E_1 + E_2,
\end{equation}
\begin{equation}\label{distr5}
E_1(\check u) = (2\pi)^{-n/2} \int_{|\xi|\ge R} \frac{\hat u(\xi)}{P(\xi)}\, \mathrm d\xi, \quad E_2(\check u)= (2\pi)^{-n/2} \oint_{|\xi|\le R} \frac{\hat u(\xi)}{P(\xi)}\, \mathrm d\xi
\end{equation}
Die Variable $\xi$ in der Definition von $E_1$ kann nur reelle Werte im Integral annehmen. Die Distribution $E_2$ ist eine ganze Funktion, da für $u\in\rmC_0^\infty$ erhält man mit der Definition von $\hat u$
\begin{equation}
E_2(\check u) = (2\pi)^{-\nu} \oint_{|\xi |\le R} \frac{\mathrm d\xi}{P(\xi)} \int u(x) \e^{-\i x\cdot\xi}\, \mathrm dx= (2\pi)^{-n} \int \check u(x) \, \oint_{|\xi|\le R} \frac{e^{\i x\cdot \xi}}{P(\xi)}\, \mathrm d\xi \, \mathrm dx
\end{equation}
erhält. Damit ist $E_2$ regulär und
\begin{equation}
E_2(x) = (2\pi)^{-n} \oint_{|\xi|\le R} \frac{\e^{\i x\cdot \xi}}{P(\xi)}\, \mathrm d\xi,
\end{equation}
die analytisch ist, da das Integral gleichmäßig konvergent auf Kompakta ist. 

Sei $u\in L^2$ mit kompaktem Träger. Die Faltung $P^{\alpha}(\D)E_2 *u$ ist dann eine analytische Funktion.  Die Aussage ({\bf ii}) folgt dann, wenn wir zeigen, dass $P^{(\alpha)}(\D) E_1 * u$ quadratintegrierbar ist.  Sei $\phi \in\rmC_0^\infty$. Dann ist die Funktion $U*\phi$ auch in $C_0^\infty$ und im Sinne von \eqref{distr5} folgt
\begin{equation}
(P^{(\alpha)} (\D) e_1 * u) (\check \phi) = P^{(\alpha)}(\D) E_1 * u * \phi(0) = \int_{|\xi| \ge R} \frac{P^{(\alpha)}(\xi)}{P(\xi)} \hat u(\xi) \hat \phi(\xi) \, \mathrm d\xi,
\end{equation}     
so dass die Fouriertransformierte von $P^{(\alpha)}(\D) E_1 * u$ eine Funktion ist die für $|\xi|<R$ verschwindet. 
\end{proof}
