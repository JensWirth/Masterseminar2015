% !TEX root = main.tex
\chapter{Maximale Operatoren}

% Voraussichtliche Struktur

Sei $P(\D)$ ein Differentialausdruck und $\Omega\subset\R^n$ ein Gebiet. Wir erinnern an die Definition des maximalen Operators $P$, dieser versteht sich als größtmögliche distributionelle Fortsetzung von $P(\D)$ und ist auf dem Definitionsbereich
\begin{equation}
   \mathcal D_P = \{ u\in\rmL^2(\Omega)\;:\; P(\D)u\in\rmL^2(\Omega) \}
\end{equation}
gegeben. Maximale Operatoren sind im wesentlichen durch ihren Definitionsbereich bestimmt. Es gilt
\begin{thm}
Angenommen, für zwei Differentialausdrücke $P(\D)$ und $Q(\D)$ und die zugeordneten maximalen Operatoren auf einem beschränkten Gebiet $\Omega$ gilt $\mathcal D_P\subseteq\mathcal D_Q$. Dann gilt
entweder $Q(\xi)=a P(\xi) + b$ mit Konstanten $a,b\in\C$ oder es gibt einen Vektor $\nu\in\R^n$ und Polynome $p$ und $q$ in einer Variablen mit
$P(\xi)=p(\xi\cdot\nu)$ und $Q(\xi)=q(\xi\cdot\nu)$ sowie $\deg q\le\deg p$.
\end{thm}
\begin{proof}[Beweisidee] Vgl. \cite{Hormander:1955}, der Beweis beruht auf strukturellen Argumenten zu Polynomringen über $\C$. Wesentlich ist, daß $\mathcal D_P\subseteq \mathcal D_Q$ Abschätzungen der Form
\begin{equation}
   |Q(\zeta)|^2 \le C \big( |P(\zeta)|^2 + 1\big)
\end{equation}
für alle {\em komplexen} $\zeta\in\C^n$ impliziert. Wir zeigen dies kurz. Aus Satz \ref{thm:1:1.1} folgt
\begin{equation}
\norm{Qu}^2\leq C(\norm{Pu}^2+\norm{u}^2)
\end{equation}
für alle $u\in\mathcal{D}_P$. Da wir das Gebiert $\Omega$ als beschränkt vorausgesetzt haben, gilt $\rmC^\infty(\overline{\Omega})\subset\mathcal{D}_P$. Also folgt mit $u(x)=\e^{\i x\cdot \zeta}$ für $\zeta\in\C^n$
\begin{equation}
\abs{Q(\zeta)}^2\int_\Omega \big|\e^{\i x\cdot \zeta}\big|^2\d x \leq C \left( \abs{P(\zeta)}^2\int_\Omega \big|\e^{\i x\cdot \zeta}\big|^2\d x+\int_\Omega \big|\e^{\i x\cdot \zeta}\big|^2\d x  \right)
\end{equation}
und damit nach Division beider Seiten durch $ \int_\Omega \big|\e^{\i x\cdot \zeta}\big|^2\d x$ die gewünschte Ungleichung.
\end{proof}

Nichttriviale Vergleichsresultate, wie wir sie in Kapitel~\ref{chap2} für minimale Operatoren erhalten haben, sind damit für maximale Operatoren von vornherein ausgeschlossen. 



\section{Differentialoperatoren von lokalem Typ} % bearbeitet durch Thomas Hamm
\begin{df}
Der Differentialausdruck $P(\D)$ heißt \eIndex[Differentialoperator]{von lokalem Typ}\footnote{Die Bezeichnung hat sich so nicht durchgesetzt; vollständige Operatoren von lokalem Typ entsprechen den heute als \eIndex[Differentialoperator]{(lokal) hypoelliptisch} bezeichneten Operatoren.} (auf einem beschränkten Gebiet $\Omega$), falls $u\varphi\in\mathcal D_{P_0}$ für alle $u\in\mathcal D_{P}$ und alle $\varphi\in \rmC_0^\infty(\Omega)$ gilt.
\end{df}
Es wird sich zeigen, daß die Definition unabhängig vom Gebiet $\Omega$ ist. Gilt die definierende Bedingung für ein beschränktes Gebiet, so gilt sie für jedes. Im folgenden sei $\Omega$ fest gewählt. Identifiziert man man den Graphen der Operatoren $P$ und $P_0$ mit den Definitionsbereichen versehen mit der Graphennorm, dann charakterisiert der \eIndex[Differentialoperator]{Cauchyraum} $\mathcal C = \mathcal D_{P}/\mathcal D_{P_0}$ eines Operators von lokalem Typ das Randverhalten der Funktionen aus $\mathcal D_{P}$ in der Umgebung von $\partial\Omega$.

\begin{lem}[{\cite[Lemma~3.4]{Hormander:1955}}]\label{lem : loktyp}
Der Differentialausdruck $P(\D)$ ist genau dann von lokalem Typ, wenn für jedes $u\in\mathcal D_P$ und alle Multiindices $\alpha\in\N_0^n$ stets $P^{(\alpha)}(\D)u \in \rmL^2_{\rm loc}(\Omega)$ gilt. In diesem Fall existiert für jedes Teilebiet $\omega\Subset \Omega$ mit kompaktem Abschluss in $\Omega$ ein $C>0$ mit
\begin{equation}
\int_{\omega} \abs{P^{(\alpha)}(\D)u(x)}^2\d x\leq C\left( \int_{\Omega} \abs{P(\D)u(x)}^2\d x + \int_{\Omega} \abs{u(x)}^2 \d x \right) \label{eq:lm3.4}
\end{equation}
für alle $u\in\mathcal{D}_P$.
\end{lem}
\begin{proof}
Sei zuerst $P(\D)$ von lokalem Typ, $u\in\mathcal{D}_P$ und $\omega\Subset\Omega$ relativ kompakt. Sei weiter $\varphi\in\rmC_0^\infty(\Omega)$ so gewählt, dass $\varphi(x)=1$ auf $\omega$ gilt. Dann gilt $v=u\varphi\in\mathcal D_{P_0}$ und $u=v$ auf $\omega$. Daraus folgt $P^{(\alpha)}(\D)u|_\omega=P^{(\alpha)}(\D)v|_\omega$ und es ist $P^{(\alpha)}(\D)v\in\rmL^2(\Omega)$ nach Korollar \ref{cor:2:2.6} und damit $P^{(\alpha)}(\D)u|_\omega \in\rmL^2(\omega)$. 
Also ist $P^{(\alpha)}(\D)u \in \rmL^2_{\rm loc}(\Omega)$. $\bullet$\qquad
Sei umgekehrt $P^{(\alpha)}(\D)u \in \rmL^2_{\rm loc}(\Omega)$ für alle Multiindices $\alpha\in\N_0^n$. Sei weiter $\varphi\in C_0^\infty(\Omega)$. Dann ist wegen Leibniz
\begin{equation}
P(\D)v=\sum_\alpha \dfrac{\D^\alpha \varphi}{{\alpha}!}P^{(\alpha)}(\D)u
\end{equation}
und damit nach Voraussetzung $P(\D)v\in \rmL^2(\Omega)$ und somit $v\in\mathcal{D}_P$. Da außerdem $\supp v\Subset\Omega$ kompakt ist gilt $\dist(\supp v,\partial\Omega)=\delta>0$. Wir wählen eine Funktion $\psi\in\rmC_0^\infty(B_{\delta/2})$ mit $\psi\ge0$ und $\int\psi(x)\d x=1$ und setzen $\psi_k(x)=k^{n} \psi(k x)$. Dann ist
$v_k=v*\psi_k\in\rmC_0^\infty(\Omega)$ und es gilt $\partial^\alpha v_k\to \partial^\alpha v$, $k\to\infty$, in $\rmL^2(\Omega)$ 
und somit $P(\D)v_k\to P(\D)v$. Damit ist $v\in \mathcal{D}_{P_0}$ und $P(\D)$ von lokalem Typ. $\bullet$ \qquad 
Die Ungleichung \eqref{eq:lm3.4} folgt aus Satz \ref{thm:1:1.1} angewandt auf die Abbildung
\begin{equation}
\mathcal{D}_P \ni u \mapsto P^{(\alpha)}u\in \rmL^2(\omega).
\end{equation}
Insbesondere gilt \eqref{eq:lm3.4} genau dann für alle $\omega\Subset\Omega$, wenn $P(\D)$ von lokalem Typ ist.
\end{proof}

%
%
%

\begin{thm}[{\cite[Theorem~3.12]{Hormander:1955}}]
   Angenommen, $P(\D)$ ist von lokalem Typ. Dann ist $P$ der Abschluß seiner Einschränkung auf $\mathcal D_P\cap \rmC^\infty(\Omega)$.
\end{thm}
\begin{proof}
Sei $\{\Omega_i\}_{i\in \N}$ eine abzählbare offene Überdeckung von $\Omega$ mit $\Omega_i$ beschränkt  und $\{\varphi_i\}_{i\in \N}$ mit $\varphi_i\in \rmC_0^\infty (\Omega_i)$ eine untergeordnete lokal endliche Zerlegung der Eins. Zu einem beliebigen $u\in\mathcal{D}_P$ setze nun $u_i=u\varphi_i$. Dann ist $\sum_{i\geq 1} u_i=u$, $\sum_{i\geq 1} P(\D)u_i=P(\D)u$ und da $P$ von lokalem Typ ist, ist $u_i\in\mathcal{D}_{P_0}$. Da $P_0$ gerade der $\rmL^2$-$\rmL^2$-Abschluß auf $\rmC_0^\infty(\Omega)$ ist, können wir für gegebenes $\epsilon>0$ Funktionen $v_i\in \rmC_0^\infty(\Omega)$ wählen mit $\norm{u_i-v_i}<\epsilon 2^{-i}$ und $\norm{P(\D)u_i-P(\D)v_i}<\epsilon 2^{-i}$. Weiterhin können wir ohne Einschränkung davon ausgehen, dass $\supp v_i \subset \Omega_i$. Damit folgt
\begin{equation}
\norm{v-u}\leq \sum_{i=1}^\infty \norm{v_i-u_i} < \epsilon
\end{equation}
und
\begin{equation}
\norm{P(\D)v-P(\D)u}\leq \sum_{i=1}^\infty \norm{P(\D)v_i-P(\D)u_i} < \epsilon
\end{equation}
und $v,P(\D)v\in \mathcal D_P\cap \rmC^\infty(\Omega)$ und damit die Behauptung.
\end{proof}

%%
%
%

Im folgenden sollen Kriterien dafür gefunden werden, daß ein gegebener Differentialausdruck $P(\D)$ von lokalem Typ ist. Ein erstes notwendiges Kriterium liefert folgendes Lemma.

\begin{lem}[{\cite[Lemma~3.5]{Hormander:1955}}]
Sei $\Omega$ ein beschränktes Gebiet und $P(\D)$ von lokalem Typ. Dann existiert für jedes $A>0$ ein $C>0$, sodass
\begin{equation}
\widetilde{P}(\zeta)^2=\sum_\alpha \abs{P^{(\alpha)}(\zeta)}^2\leq C(1+\abs{P(\zeta)}^2)
\end{equation}
für alle $\zeta\in\C^n$ mit $\abs{\Im \zeta} < A$.
\end{lem}
\begin{proof}
Setze $u(x)=\e^{\i x\cdot\zeta}$. Dann ist $P^{(\alpha)}(\D)u(x)=P^{(\alpha)}(\zeta)\e^{\i x\cdot\zeta}$ und mit Lemma \ref{lem : loktyp} folgt
\begin{equation}
\abs{P^{(\alpha)}(\zeta)}^2 \int_{\omega} \e^{-2x\cdot \eta}\d x \leq C(1+\abs{P(\zeta)}^2)\int_\Omega \e^{-2x\cdot \eta}\d x
\end{equation}
wobei $\eta = \Im \zeta$ und $\omega\Subset\Omega$. Setzen wir weiter $\delta = \sup \{2\abs{x}:x\in\Omega\}$, so folgt
\begin{equation}
\int_\omega \e^{-2x\cdot \eta}\d x \geq \int_\omega \e^{-\delta \abs{\eta}}=\abs{\omega}\e^{-\delta \abs{\eta}}
\end{equation}
und
\begin{equation}
\int_\Omega \e^{-2x\cdot \eta}\d x \leq \int_\Omega \e^{\delta \abs{\eta}}=\abs{\Omega}\e^{\delta \abs{\eta}}.
\end{equation}
Insgesamt also $\abs{P^{(\alpha)}(\zeta)}^2\leq C\e^{2\delta \abs{\eta}}(1+\abs{P(\zeta)}^2)\leq \widetilde{C}(1+\abs{P(\zeta)}^2) $, wenn $\abs{\eta}=\abs{\Im \zeta}$ beschränkt. Summation über alle $\alpha$ liefert die Behauptung.
\end{proof}


%
% Satz 3.3.
%

Es zeigt sich, dass das gerade gefundene Kriterium auch hinreichend ist. Wir formulieren den Satz, der Beweis wird sich über mehrere Abschnitte hinziehen und nachfolgend in einzelnen Teilschritten gezeigt werden.

\begin{thm}[{\cite[Theorem~3.3]{Hormander:1955}}]\label{thm:3:3.3}
Sei $P(\D)$ ein Differentialausdruck mit Symbol $P(\xi)$ und gelte $\Lambda(P)=\{0\}$. Dann sind \"aquivalent:
\begin{enumerate}
\item Für jedes $A>0$ existiert ein $R>0$, so daß $P(\xi+\i\eta)\ne0$ für alle $|\eta|<A$ und $|\xi|>R$.
\item Für jedes $\vartheta\in\R^n$ gilt $\lim_{\xi\to\infty} P(\xi+\vartheta)/P(\xi)=1$.
\item Für jeden Multiindex $\alpha\in\N_0^n$, $\alpha\ne0$, gilt $\lim_{\xi\to\infty} P^{(\alpha)}(\xi)/P(\xi)=0$.
\item Für jedes $A>0$ existiert ein $C>0$, so daß $\widetilde P(\xi+\i\eta)^2\le C\big(|P(\xi+\i\eta)|^2+1\big)$ für $|\eta|\le A$.
\item Für jedes $A>0$ gilt $\lim_{\xi\to\infty} P(\xi+\i\eta)=\infty$ gleichmäßig in $|\eta|\le A$.
\item $P(\D)$ ist von lokalem Typ.
\end{enumerate}
\end{thm} 

 Der Schritt {\bf (vi)} $\Rightarrow$ {\bf(iv)} wurde gerade gezeigt. Die Voraussetzung $\Lambda(P)=\{0\}$ kann entfallen, wenn die Aussagen {\bf (i)}, {\bf (ii)}, {\bf (iii)}
 und {\bf (v)} jeweils modulo $\Lambda(P)$ gefordert werden.


\begin{lem}[{\bf(i) $\Rightarrow$ (ii)};{\cite[Lemma~3.7]{Hormander:1955}}]
Angenommen für jedes $A>0$ existiert ein $B>0$, sodass $P(\xi+\i\eta)\neq 0$ für alle $\xi,\eta\in\R^n$ mit $\abs{\eta}<A$ und $\abs{\xi}>B$. Dann gilt
\begin{equation}
\lim_{\xi\to\infty}\dfrac{P(\xi+\vartheta)}{P(\xi)}=1 \label{eq:lm3.7}
\end{equation}
für alle $ \vartheta\in\R^n$.
\end{lem}
\begin{proof}
Durch Verschieben des Koordinatensystems zeigen wir \ref{eq:lm3.7} ohne Einschränkung für den Vektor $\vartheta=(1,0,\ldots,0)$. Sei $\epsilon > 0$ beliebig. Dann existiert nach Voraussetzung ein $B>0$, sodass $P(\xi+\i\eta)\neq 0$ für alle $\abs{\eta} < \epsilon^{-1}$ und $\abs{\xi}>B$. Sei nun $\xi=(\xi_1,\ldots,\xi_n)\in\R^n$ mit $\abs{\xi}>B+\epsilon^{-1}$ fest und $t_1,\ldots,t_m\in \C$ die Nullstellen des Polynoms $P(\cdot,\xi_2,\ldots,\xi_n)$. Für $\zeta=(t_k,\xi_2,\ldots,\xi_n)$ gilt dann wegen $P(\zeta)=0$, dass $\abs{\Re \zeta} \leq B$ oder $\abs{\Im \zeta} \geq \epsilon^{-1}$. Für $\abs{\xi-\zeta}= (\abs{\xi-\Re \zeta}^2+\abs{\Im \zeta})^{1/2}$ folgt dann in beiden Fällen $\abs{\xi_1-t_k}=\abs{\xi-\zeta}\geq \epsilon^{-1}$. Weiterhin haben wir für $P$ die Darstellung
\begin{equation}
P(\xi)=c\prod_{k=1}^m (\xi_1-t_k).
\end{equation}
und damit
\begin{equation}
\dfrac{P(\xi+\vartheta)}{P(\xi)}=\prod_{k=1}^m \dfrac{\xi_1+1-t_k}{\xi_1-t_k}=\prod_{k=1}^m\left( 1+\dfrac{1}{\xi_1-t_k} \right).
\end{equation}
Zusammen mit der Abschätzung für die Linearfaktoren folgt
\begin{equation}
\left|\dfrac{P(\xi+\vartheta)}{P(\xi)}-1\right|\leq \epsilon\sum_{k=1}^m \binom{m}{k}\epsilon^{k-1} \leq m\epsilon  \sum_{k=0}^{m-1}\binom{m-1}{k}\epsilon^k = m\epsilon(1+\epsilon)^{m-1}
\end{equation}
für alle $\xi \in\R^n$ mit $\abs{\xi}>B+\epsilon^{-1}$.
\end{proof}
\begin{lem}[{\cite[Lemma~2.10]{Hormander:1955}}]
Sei $P$ ein Polynom. Dann gilt 
\begin{equation}
\spann\{P^{(\alpha)}:\abs{\alpha}\leq \deg P\}=\spann\{P(\cdot+\vartheta):\vartheta\in\R^n\}.
\end{equation}
\end{lem}
\begin{proof}
\relax[$\supseteq$] Da $P$ ein Polynom ist, folgt aus der Taylorformel
\begin{equation}
P(\xi+\vartheta)=\sum_\alpha \dfrac{P^{(\alpha)}(\xi)}{\alpha!}\vartheta^{\alpha}
\end{equation}
für jedes feste $\vartheta\in\R^n$. $\bullet$\qquad [$\subseteq$] Betrachten wir nun den Unterraum, der von allen Translaten von $P$ aufgespannt wird. Dieser enthält alle Linearkombinationen der Form
\begin{equation}
\sum_{i=1}^m t_i P(\xi+\vartheta_i)=\sum_{i=1}^mt_i\sum_\alpha \dfrac{P^{(\alpha)}(\xi)}{\alpha!}\vartheta_i^\alpha =\sum_\alpha \dfrac{P^{(\alpha)}(\xi)}{\alpha!} \sum_{i=1}^mt_i\vartheta_i^\alpha=\sum_\alpha c_\alpha \dfrac{P^{(\alpha)}(\xi)}{\alpha!}
\end{equation}
mit Koeffizienten $t_i\in\C$ und Vektoren $\vartheta_i\in\R^n$. Für geeignet gewählte $ m,t_i,\vartheta_i $ nehmen dabei die Koeffizienten $c_\alpha$ beliebige Werte an. Also enthält dieser Unterraum auch alle Linearkombinationen der $P^{(\alpha)}$.
\end{proof}
\begin{lem}[{\bf(ii) $\Rightarrow$ (iii),(iv)};{\cite[Lemma~3.8]{Hormander:1955}}]
Gilt
\begin{equation}
\lim_{\xi\to\infty}\dfrac{P(\xi+\vartheta)}{P(\xi)}=1
\end{equation}
für alle reellen $\vartheta$, so auch für alle komplexen und die Konvergenz ist gleichmäßig in $\vartheta$, wenn $\abs{\vartheta}$ beschränkt. Außerdem gilt dann für alle Multiindices $\alpha\neq0$
\begin{equation}
\lim_{\xi\to\infty}\dfrac{P^{(\alpha)}(\xi+\vartheta)}{P(\xi+\vartheta)}=0
\end{equation}
ebenfalls gleichmäßig in $\vartheta$, wenn $\abs{\vartheta}$ beschränkt.
\end{lem}
\begin{proof}
Nach obigem Lemma existiert für $P^{(\alpha)}$ eine Darstellung der Form
\begin{equation}
P^{(\alpha)}(\xi)=\sum_{i=1}^m t_i P(\xi+\vartheta_i).
\end{equation}
Für $\alpha\neq0$ enthält der linke Term keine Monome der Ordnung $\deg P$ und durch Koeffizientenvergleich folgt $\sum t_i=0$. Damit folgt
\begin{equation}
\dfrac{P^{(\alpha)}(\xi)}{P(\xi)}=\sum_{i=1}^m t_i\dfrac{P(\xi+\vartheta_i)}{P(\xi)} \rightarrow \sum_{i=1}^m t_i=0
\end{equation}
nach Voraussetzung für $\xi\to\infty$. Weiterhin folgt mit der Taylorformel
\begin{equation}
\dfrac{P(\xi+\vartheta)}{P(\xi)}=1+\sum_{\alpha\neq 0} \dfrac{P^{(\alpha)}(\xi)}{P(\xi)}\dfrac{\vartheta^\alpha}{\alpha!}
\end{equation}
und damit die Konvergenz für komplexe $\vartheta$, sowie die gleichmäßige Konvergenz. Ebenso folgt damit für
\begin{equation}
\dfrac{P^{(\alpha)}(\xi+\vartheta)}{P(\xi+\vartheta)}=\sum_{i=1}^mt_i\dfrac{P(\xi+\vartheta+\vartheta_i)}{P(\xi)}\dfrac{P(\xi)}{P(\xi+\vartheta)} \rightarrow 0
\end{equation}
die gleichmäßige Konvergenz für $\xi\to\infty$, $\alpha\neq0$, wenn $\abs{\vartheta}$ beschränkt.
\end{proof}
Der Beweis der Implikation {\bf (iv) $\Rightarrow$ (v)} verläuft mit Mitteln aus dem Beweis von Satz \ref{thm:2.17}. Analog zum dortigen Induktionsbeweis zeigt man $\widetilde P(\zeta)^2\to\infty$ auf $\C^n / \Lambda_*(P)$,  wobei 
\begin{equation}
\Lambda_*(P)=\{ \vartheta\in\C^n : \forall_{\zeta\in\C^n}\; P(\zeta+\vartheta)=P(\zeta)  \}
\end{equation} 
den entsprechenden komplexe Linienraum bezeichnet. Damit impliziert die untere Schranke aus {\bf (iv)} auf dem Streifen $|\Im\zeta|\le A$ die Divergenz von $|P(\zeta)|$. Da die Implikation {\bf (v) $\Rightarrow$ (i)} offensichtlich ist, ist damit die Äquivalenz der Bedingungen {\bf (i)}--{\bf(v)} gezeigt. Also ist jede dieser Bedingungen notwendig dafür, dass $P(\D)$ von lokalem Typ ist. Um zu beweisen, dass diese auch hinreichend sind, benötigen wir eine Fundamentallösung, welche im folgenden Abschnitt konstruiert wird.

\begin{exa}
Die äquivalenten Bedingungen  {\bf (i)}--{\bf(v)} liefern einfache Kriterien dafür, wann Operatoren von lokalem Typ sind. So sind alle elliptischen Operatoren von lokalem Typ, da für diese der homogene Hauptteil $p(\xi)$ des Symbols $P(\xi)$ keine (reellen) Nullstellen besitzt und damit $1/P(\xi) = \mathcal O(|\xi|^{-m})$, $\xi\to\infty$, gilt. Daraus folgt sofort {\bf (iii)}. 

Umgekehrt impliziert {\bf (i)}, dass das Symbol nur auf einem Kompaktum reelle Nullstellen besitzen darf. Damit sind die im letzten Kapitel schon erwähnten 
Schrödingeroperatoren $\i\partial_1-\Delta_2$ auf $\Omega\subset \R\times\R^n$ und (ultra-) hyperbolischen Operatoren $\Delta_1-\Delta_2$ auf $\Omega\subset \R^k\times\R^l$, $k,l\ne0$, beide nicht von lokalem Typ. Andererseits liefert eine kurze Rechnung, daß der Wärmeleitungsoperator $\partial_1-\Delta_2$ auf $\Omega\subset\R\times\R^n$ von lokalem Typ ist. Sein Symbol ist durch $P(\xi) = \i \xi_1 - \xi'\cdot\xi'$ in $\xi=(\xi_1,\xi')\in\R^n$ gegeben und erfüllt $|P(\xi)|^2 = |\xi_1|^2+|\xi'|^4$ und damit $1/P(\xi)\to0$, $\xi' / P(\xi)\to0$ und folglich {\bf (iii)}.
\end{exa}

\section{Konstruktion von Fundamentallösungen eines vollständigen Operators von lokalen Typ} % bearbeitet von Matthias Hofmann

Im folgenden konstruieren wir eine Fundamentallösung, also eine Distribution $E\in\mathscr D'(\R^n)$ mit
\begin{equation}
E*(P(\D) u) = u, \quad u \in \rmC_0^\infty(\R^n),
\end{equation}
welche zusätzliche Regularitätseigenschaften besitzt.
Um genau zu sein,  wollen wir zeigen:
\begin{thm}\label{fundamental_exist}
Sei $P$ vollständig und von lokalem Typ, dann besitzt $P(\D)$ eine Fundamentallösung $E$, welche 
\begin{enumerate}
\item im Gebiet $\R^n\setminus\{0\}$  durch eine unendlich differenzierbare Funktion $E(x)$ dargestellt wird und
\item für die $P^{(\alpha)}(\D) E*u$ für jedes $u\in \rmL^2(\R^n)$ mit kompakten Träger $\supp u\Subset\R^n$ lokal quadrat-integrierbar ist. 
\end{enumerate}
\end{thm}
\begin{rem}
Ist $P(\D)$ vollständig und von lokalem Typ so besitzt jede Fundamentallösung diese Eigenschaften, da die Differenz zweier Fundamentallösungen unendlich oft differenzierbar ist. Dies werden wir später sehen.
\end{rem}

Zunächst einige Worte zur Beweisidee.  Falls $P(\xi)\neq 0$ für alle $\xi\in \mathbb R$, so ließe sich $E$ bestimmen aus
\begin{equation}\label{distr1}
E*u(x) = (2\pi)^{-n/2}\int \e^{\i x\cdot \xi} \frac{\widehat u(\xi)}{P(\xi)}\, \mathrm d\xi, \quad u \in \rmC_0^\infty,
\end{equation}
oder äquivalent
\begin{equation}\label{distr2}
\langle E,\check u\rangle = (2\pi)^{-n/2} \int \frac{\widehat u(\xi)}{P(\xi)}\, \mathrm d\xi, \quad u \in\rmC_0^\infty,
\end{equation}
wobei $\check u$ durch $\check u(x)=u(-x)$ gegeben ist. Wir überlegen uns also eine Verallgemeinerung von \eqref{distr2}. 

\begin{proof}[Beweis von Satz \ref{fundamental_exist}]
Wir werden diesen Beweis in einigen Schritten führen.
\vspace{1mm}\\
\textbf{Schritt 1:} \emph{Konstruktion einer Fundamentallösung.}
Nach Satz~\ref{thm:3:3.3} {\bf (v)} gilt $|P(\xi)|\ge 1$ falls $\xi\in \mathbb R$ und $|\xi|\ge C$ mit geeigneter Konstante $C$.  Damit folgt $|P(\xi)\ge 1|$ falls $\xi_2^2+ \ldots + \xi_n^2 \ge C^2$.  

Wir können annehmen, dass der Koeffizient vor der höchsten Potenz von $\xi_1$ von $P(\xi)$ konstant ist (vgl. Lemma ??). 
%Da die Nullstellen des Polynoms stetig von den Parametern abhängen, die nicht im führenden Koeffizienten stehen.  
So finden wir außerdem leicht eine zweite Konstante $C'$ sodass $|P(\xi_1, \xi_2, \cdots, \xi_nu)|\ge 1$ für $\xi_2^2+\ldots + \xi_n^2< C$ and $|\xi_1|\ge C'$.  

Jetzt setzen wir für $u\in\rmC_0^\infty(\mathbb R^n)$
\begin{equation}\label{distr3}
\langle E, \check u\rangle  = (2\pi)^{-n/2} \int \, \mathrm d\xi_2 \cdots \, \mathrm d\xi_n \oint \frac{\widehat u(\xi)}{P(\xi)} \, \mathrm d\xi_1= (2\pi)^{-n/2} \oint \frac{\widehat u(\xi)}{P(\xi)} \, \mathrm d\xi.
\end{equation}
wobei das Integral über $\xi_1$ über die reelle Achse gehe, falls $\xi_2^2+ \cdots + \xi_n^2 \ge C^2$, und über die reelle Achse mit dem Intervall $(-C', C')$ ersetzt durch einen Halbkreis in der unteren Halbebene, falls $\xi_2^2+ \cdots+ \xi_n^2 < C^2$. Damit gilt $|P(\xi)|\ge 1$  in dem Integral.  Damit ergibt sich analog zu \eqref{distr1}
\begin{equation}
E*u(x) = (2\pi)^{-n/2}\oint e^{\i x\cdot \xi } \frac{\widehat u(\xi)}{P(\xi)}\, \mathrm d\xi, \quad u \in \rmC_0^\infty.
\end{equation} 
Damit folgt für $u\in\rmC_0^\infty(\mathbb R^n)$
\begin{equation}
E*(P(\D)u) = (2\pi)^{-n/2} \oint \e^{\i x\cdot\xi} P(\xi)  \frac{\widehat{u} (\xi)}{P(\xi)}\, \mathrm d\xi=(2\pi)^{-n/2} \oint \e^{\i x\cdot \xi} \widehat u(\xi) \, \mathrm d\xi
\end{equation}
und da der Integrand holomorph in $\xi_1$ ist, können wir den Integrationsweg als reale Achse wählen. Damit folgt
\begin{equation}
E*(P(\D) u)(x) = (2\pi)^{-n/2} \int \e^{\i x\cdot \xi} \widehat u(\xi) \, \mathrm d\xi = u(x),
\end{equation}
und $E$ ist somit eine Fundamentallösung.

\textbf{Schritt 2:} \emph{Beweis der Quadratintegrierbarkeit von $P^{(\alpha)}(\D) E * u$.}
Wir teilen das Integral \eqref{distr3} in zwei Teile auf. Falls $R=\sqrt{C^2 + C'^2}$, haben wir $|\xi|<R$ und der Teil des Integrals von \eqref{distr3}, wo $\xi$ nicht real ist. Schreibe also 
\begin{equation}\label{distr4}
E=E_1 + E_2,
\end{equation}
\begin{equation}\label{distr5}
\langle E_1,\check u\rangle = (2\pi)^{-n/2} \int_{|\xi|\ge R} \frac{\widehat u(\xi)}{P(\xi)}\, \mathrm d\xi, \quad E_2(\check u)= (2\pi)^{-n/2} \oint_{|\xi|\le R} \frac{\widehat u(\xi)}{P(\xi)}\, \mathrm d\xi
\end{equation}
Die Variable $\xi$ in der Definition von $E_1$ kann nur reelle Werte im Integral annehmen. Die Distribution $E_2$ ist eine ganze Funktion, da für $u\in\rmC_0^\infty$ erhält man mit der Definition von $\widehat u$
\begin{equation}
\langle E_2, \check u\rangle = (2\pi)^{-\nu} \oint_{|\xi |\le R} \frac{\mathrm d\xi}{P(\xi)} \int u(x) \e^{-\i x\cdot\xi}\, \mathrm dx= (2\pi)^{-n} \int \check u(x) \, \oint_{|\xi|\le R} \frac{e^{\i x\cdot \xi}}{P(\xi)}\, \mathrm d\xi \, \mathrm dx
\end{equation}
erhält. Damit ist $E_2$ regulär und
\begin{equation}
E_2(x) = (2\pi)^{-n} \oint_{|\xi|\le R} \frac{\e^{\i x\cdot \xi}}{P(\xi)}\, \mathrm d\xi,
\end{equation}
die analytisch ist, da das Integral gleichmäßig konvergent auf Kompakta ist. 

Sei $u\in L^2$ mit kompaktem Träger. Die Faltung $P^{\alpha}(\D)E_2 *u$ ist dann eine analytische Funktion.  Die Aussage ({\bf ii}) folgt dann, wenn wir zeigen, dass $P^{(\alpha)}(\D) E_1 * u$ quadratintegrierbar ist.  Sei $\phi \in\rmC_0^\infty$. Dann ist die Funktion $U*\phi$ auch in $C_0^\infty$ und im Sinne von \eqref{distr5} folgt
\begin{equation}
\langle P^{(\alpha)} (\D) E_1 * u, \check \phi\rangle  = P^{(\alpha)}(\D) E_1 * u * \phi(0) = \int_{|\xi| \ge R} \frac{P^{(\alpha)}(\xi)}{P(\xi)} \widehat u(\xi) \widehat \phi(\xi) \, \mathrm d\xi,
\end{equation}     
so dass die Fouriertransformierte von $P^{(\alpha)}(\D) E_1 * u$ eine Funktion ist die für $|\xi|<R$ verschwindet und gleich $\widehat u(\xi)P^{(\alpha)}(\xi)/P(\xi)$ falls $|\xi|\ge R$. Damit ist $P^{(\alpha)}(\D) E_1 * u$ auch quadratintegrierbar und ({\bf ii}) folgt.

\textbf{Schritt 3:} \emph{Beweis der Differenzierbarkeit in $\mathbb R^n\setminus{0}$.}
Im Folgenden wenden wir uns dem Beweis von ({\bf i}) zu. Da $E_2$ eine ganze Funktion ist, genügt es zu zeigen, dass $E_1$ unendlich oft differenzierbar für $x\neq 0$ ist.  Wir beginnen mit folgenden Lemma:
\begin{lem}\label{3.2lem1}
Sei $0\neq y \in \mathbb R^n$ fest und sei
\begin{equation}\label{3.2inf}
M(\tau) = \inf_{\zeta, \xi} |\zeta-\xi|,
\end{equation}
wobei $\xi\in \mathbb C^n$ und $P(\xi)=0$, und $\xi\in \mathbb R^n$ mit $|y, \xi|= \tau$. Dann existieren positive Zahlen $a$ und $b$ mit
\begin{equation}\label{3.2to}
M(\tau) \tau^{-b} \to a\quad \text{mit } \tau \to \infty.
\end{equation}
wo $b$ unc $d$ Konstanten aus den zwei vorigen Lemmas.
\end{lem}
\begin{proof}
Nach Theorem ?? ({\bf i}) ist der Restriktionsbereich für $\zeta$ kompakt und das Infimum in \ref{3.2inf} wird angenommen. Weiter folgt leicht, dass $M(\tau)$ stetig von $\tau$ abhängt.
\end{proof}
  

%so dass die Fouriertransformierte von $P^{(\alpha)}(\D) E_1 * u$ eine Funktion ist die für $|\xi|<R$ verschwindet und gleich $\hat u(\xi) P^{(\alpha)}(\xi)/P(\xi)$ wenn $|\xi|\ge R$. Da $P^{(\alpha)}(\xi)/P(\xi)$ beschränkt is für $|\xi|\ge R$ nach Theorem ?? ({\bf iii})  und dass $\hat u(\xi)$ quadratintegrierbar ist, folgern wir, dass die Fouriertransformierte von $P^{(\alpha)}(D) E_1 * u$ quadratintegrierbar ist. Damit ist auch $P^{(\alpha)}(D) E_1 * u$ quadratintegrierbar ist, was den Beweis von ({\bf ii}) abschließt.  

%Im Folgenden wollen wir (\bf{i})

% fehlt was
\begin{lem}\label{3.2lem2}
Es gibt positive Konstanten $c$ und $d$ so dass für hinreichend große $|\xi|$ wir 
\begin{equation}
|\zeta- \xi|\ge c |\xi|^{d}
\end{equation} 
für alle reelle $\xi$ und alle $\zeta$ mit $P(\zeta)=0$ haben.
\end{lem} 
\begin{proof}
Sei der Vektor $y$ von Lemma ?? als $(0,\ldots, 0, 1, 0, \ldots, 0)$, erhalten wir für große $|\xi_i|$
\begin{equation}
|\zeta-\xi|\ge a_i |\xi_i|^{b_i}
\end{equation}
wobei $a_i$ und $b_i$ positive Zahlen sind. Damit ergibt sich mit $c'=\min a_i$ und $d=\min b_i$ 
\begin{equation}
|\zeta - \xi|\ge c'(\max |\xi_i|)^d \ge c |\xi|^d.
\end{equation}
\end{proof}
\begin{lem}\label{3.2lem3}
Sei $y\in \mathbb R^n$ und $\eta \in \mathbb R^n$ zwei feste Vektoren. Dann gibt es eine Konstante $C$ so dass
\begin{equation}
\left |(\eta\cdot\nabla)^{k+j} \left ( \frac{1}{P(\xi)} \right )  \right | \le \frac{(k+j)! C^{k+j}}{|\xi|^{kd} ( 1+ | y\cdot\xi | )^{bj}},\qquad  |\xi|\ge R, \quad j, k=1,2,\ldots,
\end{equation}
wobei $b$ and $d$ aus dem vorigen Lemma seien. Dabei ist
\begin{equation}
(\eta\cdot\nabla)^{k+j} \left (\frac{1}{P(\xi)} \right ) =  \frac{\mathrm d^{k+j}}{\mathrm d t^{k+j}}  \frac{1}{P(\xi+ t\eta)} \bigg|_{t=0}.
\end{equation}
\end{lem}
\begin{proof}
Wir schreiben $P(\xi+t\eta) = A \prod_{1}^m (t-t_r) $. Da $\zeta = \xi+ t_r \eta$ eine Nullstelle von $P$ ist und $\zeta - \xi = t_r \eta$ können die Zahlen $t_r$ abgeschätzt werden durch Lemma \ref{3.2lem1} und \ref{3.2lem2} zu
\begin{equation}\label{3.2tr}
|t_r| \ge a' (1+|\langle y, \xi\rangle|)^b,\quad |t_r|\ge c' |\xi|^d, (|\xi|\ge R).
\end{equation}
Dann ist die $(k+j)$-te  Ableitung von $1/P(\xi+t\eta)$ für $t=0$ eine Summe von Terme die alle von der Form $A^{-1}$ geteilt durch das Produkt von $k+j+m$ der Nullstellen $t_r$. Die Zahl der Terme kann abgeschätzt werden durch
\begin{equation}
m(m+1) \cdots (m+k+j-1) =(k+j)! \binom{m+k+j-1}{k+j}<(k+j)! 2^{m+k+j-1}.
\end{equation}
Weiter ist $A$ unabhängig von $\xi$. Dies folgt leicht mit Theorem \ref{thm:3:3.3} ({\bf iii}). Sonst würde folgen 
\begin{equation}
\frac{P^{(\alpha)}(\xi + t\eta)}{P(\xi + t\eta)}\to \frac{A^{(\alpha)}(\xi)}{A(\xi)}\neq 0\quad(t\to \infty).
\end{equation}
für ein $\xi\in \mathbb R^n$ und einem Multiindex $\alpha\neq 0$.

% umsetzen
Dann folgt das Lemma, indem wir $j$ der Terme mit Nullstellen durch die erste Ungleichung aus \eqref{3.2tr}, $k$ der Terme mit der zweiten Ungleichung und die weiteren $m$ durch eine Konstante.  
\end{proof}
Seien $y$ und $\eta$ zwei feste Vektoren und sei $b$ und $d$ wie im vorigen Lemma.  Wir zeigen zunächst, dass
\begin{equation}
F= (x\cdot \eta)^l (y \cdot \nabla)^k E_1
\end{equation}
stetig ist, falls
\begin{equation}\label{3.2l}
l\ge \frac{k}{b}+r, % l hängt von k ab
\end{equation}
wobei $r=\lfloor n/d +1\rfloor$.  Dies schließt tatsächlich den Beweis des Theorems ab, da dann $E_1$ beliebig oft stetig differenzierbar ist außerhalb von $0$.  %Zusatz wie im Original?

Nach Definition von $F$ bzw. $E_1$ folgt 
\begin{equation}\label{3.2F}
\langle F, \check u\rangle = (2\pi)^{-n/2} \int_{|\xi|\ge R} \frac{(y\cdot \xi)^k}{P(\xi)} \left ( (\eta\cdot \D)^l \hat u(\xi)\right )\, \mathrm d\xi.
\end{equation}
Mit partieller Integration folgt $F(\check u)=G(\check u) + I(\check u)$ mit
\begin{equation}\label{3.2G}
\langle G,\check u\rangle = (2\pi)^{-n/2} \int_{|\xi|\ge R} \hat u(\xi) \left ( (-\eta \cdot \nabla)^l \left ( \frac{(y\cdot \xi)^k}{P(\xi)} \right ) \right )\, \mathrm d\xi
\end{equation}
\begin{equation}\label{3.2I}
\langle I,\check u\rangle = i^{-1} (2\pi)^{-n/2} \sum_{j=0}^{l-1} \int_{|\xi|=R} \left ( (-\eta \cdot \nabla)^j \left ( \frac{(y\cdot \xi)^k}{P(\xi)}\right ) \right ) \left ( (\eta \cdot \nabla)^{l-1-j} \hat u(\xi) \right ) (\eta \cdot \mathrm n) \, \mathrm d\sigma.
\end{equation}
Nach Lemma \ref{3.2lem3} können wir den Integranden von \eqref{3.2G} abschätzen, den wir mit $g(\xi)$ bezeichnen.
\begin{equation}
\begin{split}
|g(\xi)| &= \left | \sum_{j=0}^k \binom{l}{j} (y\cdot i\eta)^j (y \cdot \xi)^{k-j} \left ( (-D\cdot \eta)^{l-j} \frac{1}{P(\xi)} \right ) \right |\\
&\le \sum_{j=0}^k \binom{l}{j} \frac{k!}{(k-j)!} |y\cdot \eta|^j |y\cdot \xi|^{k-j}  \frac{(l-j)! C^{l-j}}{|\xi|^{rd} (1+|y\cdot \xi|^{b(l-j-r)})} 
\end{split}
\end{equation}
wo im ersten Schritt die Leibnizformel benutzt worden ist. Mit \eqref{3.2l} folgt
\begin{equation}
b(l-j-r) -(k-j)= b\left ( l-\frac{k}{b} -r \right )  + j(1-b) >0.
\end{equation}
und erhalten somit
\begin{equation}
|g(\xi)| \le |\xi|^{-rd} l! \sum_{j=0}^k \binom{k}{j} |y\cdot \eta|^j C^{l-j} = | \xi|^{-rd} l! C^l \left ( \frac{|y\cdot\eta|}{C+1} \right )^k.
\end{equation}
Die Funktion $|\xi|^{-rd}$ ist integrierbar über $|\xi|>R$, da $rd >n$.  Damit folgt, dass die Distribution $G$ gegeben ist durch die stetige Funktion
\begin{equation}
G(x) = (2\pi)^{-n} \int_{|\xi|\ge R} g(\xi) e^{i x\cdot \xi}\, \mathrm d\xi.
\end{equation}
Weiter folgt ähnlich wie zuvor mit der Definition von $\hat u$ eingesetzt in \eqref{3.2I}, dass $I$ durch die analytische Funktion 
\begin{equation}
I(x) = i^{-1}(2\pi)^{-n} \sum_{j=0}^{l-1} \int_{|\xi|=R} (x\cdot \eta)^{l-1-j} e^{ix\cdot \xi} \left ( (-\eta \cdot \nabla)^{j} \left ( \frac{(y\cdot \xi)^k}{P(\xi)} \right ) \right )\, \mathrm d\sigma
\end{equation} 
gegeben ist. Damit ist $F=G+I$ stetig.
\end{proof}

%
%
%
%

\begin{proof}[Beweis von Satz~\ref{thm:3:3.3} {\bf (i)--(v) $\Longrightarrow$ (vi)}]
Sei $P$ ein vollständiger Operator, der die Voraussetzungen I-V von Theorem \ref{thm:3:3.3} erfüllt und sei $\Omega\subset \mathbb R^n$ eine beliebige Umgebung um die $0$ und $\Omega'\subset \subset \Omega$ kompakt enthalten.  Sei $\rho(x)\in C_=^\infty$, sodass die für $|x|\ge \varepsilon$ verschwindet und $1$ in einer Umgebung um den Ursprung ist. Wir definieren die Distribution
\begin{equation}
F= \rho E.
\end{equation}
Der Träger von $F$ ist damit in der Kugel $\overline {B_{\varepsilon}(0)}$ enthalten und es ist
\begin{equation}
P(D) F= P(D) E - P(D)((1-\rho)E) =\delta_0 + \omega(x)
\end{equation}  
wobei $P(D)E=\delta_0$ und $\omega(x):= -P(D)((1-\rho)E)$ eine unendlich oft differenzierbare Funktion ist nach vorigen Theorem, die für $|x|\ge \varepsilon$ und in einer Umgebung um die $0$ verschwindet.  

Sei $u\in \mathcal D_{P}$, also insbesondere $u\in L^2(\Omega)$ und $P(\D)u\in L^2(\Omega)$.  Dann folgt in $\Omega'$
\begin{equation}
u= u*\delta_0 = u * (P(\D) F-\omega) = F*(P(\D)u)- \omega *u.
\end{equation}
Und damit
\begin{equation}\label{3.2wichtig}
P^{(\alpha)}(\D) u = (P^{(\alpha)}(D)F)*(Pu)-(P^{(\alpha)}(\D) \omega)*u.
\end{equation}
Im folgenden wollen wir zeigen, dass $P^{(\alpha)}(\D)\in L^2_{\text{loc}}(\Omega)$ gilt für alle Multiindices $\alpha\in \mathbb N_0^n$. Dann folgt mit Lemma \ref{lem : loktyp}, dass $P$ vom lokalen Typ ist.  $(P^{(\alpha)}(D)\omega)*u\in C^\infty$ ist offenbar quadrat-integrierbar in $\Omega'$. 
Bezeichne $\phi$ die Funktion die gleich $Pu$ auf $\Omega'_\varepsilon:=\{x\in \Omega| \dist(\Omega', x)<\varepsilon\}$ und $0$ sonst ist. Dann folgt in $\Omega'$
\begin{equation}
(P^{(\alpha)}(\D)F)* (Pu) = (P^{(\alpha)}(\D)E)*\phi+ (P^{(\alpha)}(\D)((\rho-1)E))*\phi.
\end{equation}
$(P^{(\alpha)}(\D)E)*\phi$ ist nach Satz \ref{fundamental_exist} ({\bf ii}) quadrat-integrierbar. Offenbar besitzt $\phi$ kompakten Träger und leicht folgt auch, dass $P^{(\alpha)}(\D)((\rho-1)E)*\phi$ quadrat-integrierbar in $\Omega'$  ist. Daraus folgt schließlich mit \eqref{3.2wichtig}, dass $P^{(\alpha)}u\in L^2_{\text{loc}}$.
\end{proof}


%
%
%
%

\begin{thm}
Sei $P(\D)$ ein Differentialausdruck und $\Omega\subset\R^n$ ein beschränktes Gebiet. Dann sind die folgenden Aussagen äquivalent:
\begin{enumerate}
\item $P(\D)$ ist vollständig und von lokalem Typ.
\item Jede (distributionelle) Lösung $u\in\rmL^2(\Omega)$ der Gleichung $P(\D)u=0$ ist glatt,
\begin{equation}
    \ker P(\D) = \{ u\in\rmL^2(\Omega) \;:\; P(\D)u=0\} \subseteq\rmC^\infty(\Omega).
\end{equation}
\end{enumerate}
\end{thm}
\begin{proof}
\end{proof}