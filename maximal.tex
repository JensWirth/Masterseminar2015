% !TEX root = main.tex
\chapter{Maximale Operatoren}

% Voraussichtliche Struktur
\section{Differentialoperatoren von lokalem Typ} % bearbeitet durch Thomas Hamm


\section{Konstruktion von Fundamentallösungen eines vollständigen Operators von lokalen Typ} % bearbeitet von Matthias Hofmann

Im folgenden konstruieren wir eine Fundamentallösung, d.h. eine Distribution $E$ so dass
\begin{equation}
E*(P(D) u) = u, \quad u \in C_0^\infty(\R^n)
\end{equation}
gewisse Regularitätseigenschaften erfüllt.
Um genauer zu sein,  wollen wir zeigen:
\begin{thm}\label{fundamental_exist}
Sei $P$ vollständig und von lokalem Typ, dann besitzt $P(D)$ eine Fundamentallösung $E$ mit 
\begin{enumerate}
\item Im Gebiet $\R^n\setminus\{0\}$ kann die Distribution $E$
dargestellt werden durch eine unendlich differenzierbare Funktion $E(x)$
\item Falls $u\in L^2(\R^n)$ mit kompakten Träger, so ist $P^{(\alpha)}(D) E*u$ eine lokal quadrat-integrierbare Funktion. 
\end{enumerate}
\end{thm}
\begin{rem}
Jede Fundamentallösung besitzt diese Eigenschaften und tatsächlich ist die Differenz zweier Fundamentallösungen unendlich oft differenzierbar wie wir später sehen werden.
\end{rem}

Zunächst einige Worte zur Beweisidee.  Falls $P(\xi)\neq 0$ für alle $\xi\in \mathbb R$, so ließe sich $E$ bestimmen aus
\begin{equation}\label{distr1}
E*u(x) = (2\pi)^{-n/2}\int e^{i\langle x, \xi \rangle} \frac{\hat u(\xi)}{P(\xi)}\, \mathrm d\xi, \quad u \in C_0^\infty,
\end{equation}
oder äquivalent
\begin{equation}\label{distr2}
E(\check u)= (2\pi)^{-n/2} \int \frac{\hat u(\xi)}{P(\xi)}\, \mathrm d\xi, \quad u \in C_0^\infty,
\end{equation}
wobei $\check u$ durch $\check u(x)=u(-x)$ gegeben ist. Wir überlegen uns also eine Verallgemeinerung von \ref{distr2}.

\begin{proof}[Beweis von Satz \ref{fundamental_exist}]
Wir werden diesen Beweis in einigen Schritten führen.
\vspace{1mm}\\
\textbf{Schritt 1:} \emph{Konstruktion einer Fundamentallösung.}
Nach Theorem ?? ({\bf v}) gilt $|P(\xi)|\ge 1$ falls $\xi\in \mathbb R$ und $|\xi|\ge C$ mit geeigneter Konstante $C$.  Damit folgt $|P(\xi)\ge 1|$ falls $\xi_2^2+ \ldots + \xi_n^2 \ge C^2$.  

Wir können annehmen, dass der Koeffizient vor der höchsten Potenz von $\xi_1$ von $P(\xi)$ konstant ist. Da die Nullstellen des Polynoms stetig von den Parametern abhängen, die nicht vor dem 
\end{proof}
